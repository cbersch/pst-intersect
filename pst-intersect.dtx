% \iffalse meta-comment
%
% Copyright (C) 2014 by Christoph Bersch <usenet@bersch.net>
%
% This work may be distributed and/or modified under the
% conditions of the LaTeX Project Public License, either version 1.3c
% of this license or (at your option) any later version.
% The latest version of this license is in
%   http://www.latex-project.org/lppl.txt
% and version 1.3c or later is part of all distributions of LaTeX
% version 2008/05/04 or later.
% \fi
%
% \iffalse
%<*driver>
\ProvidesFile{pst-intersect.dtx}
%</driver>
%<stylefile>\NeedsTeXFormat{LaTeX2e}[1999/12/01]
%<stylefile>\ProvidesPackage{pst-intersect}
%<*stylefile>
    [2014/02/25 v0.2 package wrapper for pst-intersect.tex]
%</stylefile>
%
%<*driver>
\documentclass[a4paper, DIV=9, oneside, toc=index, parskip=half-]{scrreprt}
\usepackage{doc}
\setcounter{IndexColumns}{2}
\usepackage[utf8]{inputenc} 
\usepackage[T1]{fontenc}
\usepackage{lmodern} 
\usepackage{amsmath, marvosym} 
\usepackage{bera}
\providecommand*\mainlang{}
\usepackage[ngerman, english,\mainlang]{babel}
\usepackage{prettyref}
\usepackage[dvipsnames,x11names,svgnames]{xcolor}
\usepackage{ragged2e, calc}
\newlength{\PITcaptionmargin}
\newlength{\PITcaptionheight}
\usepackage[labelfont={color=DOrange}, 
            singlelinecheck=false, 
            justification=raggedright]{caption} 
\DeclareCaptionFormat{pitnocaption}{%
  \setlength{\PITcaptionmargin}{\widthof{#1}+7pt}%
  \setlength{\PITcaptionheight}{\heightof{#1}+3pt}%
  \hspace*{-\PITcaptionmargin}#1\par\vspace*{-\PITcaptionheight}%
}%
\DeclareCaptionFormat{pitcaption}{%
  \setlength{\PITcaptionmargin}{\widthof{#1}+7pt}%
  \hspace*{-\PITcaptionmargin}#1#2#3\par
}%
\captionsetup[lstlisting]{format=pitnocaption}
\usepackage{multido}
\usepackage{pst-intersect}
\usepackage{hypdoc}
\hypersetup{%
  colorlinks=true, 
  urlcolor=DOrange, 
  linkcolor=pdflinkcolor, 
  breaklinks,
  linktocpage=true} 
\usepackage{breakurl}
\definecolor{DOrange}{rgb}{1,.4,.2}%
\definecolor{DDOrange}{rgb}{0.7, 0.23, 0.07}%
\colorlet{pdflinkcolor}{DOrange}
\usepackage{showexpl}
\makeatletter\renewcommand*\SX@Info{}\makeatother
\usepackage{etoolbox}
\undef{\cs}\undef{\cmd}
\usepackage{ltxdockit}
\definecolor{colKeys}{rgb}{0,0,0}
\definecolor{colIdentifier}{rgb}{0,0,0}
\colorlet{colComments}{green!60!black}
\definecolor{colString}{rgb}{0,0.5,0}
\newlength{\codeoverhang}
\setlength{\codeoverhang}{0.5\marginparwidth+\marginparsep}
\lstset{%
  language=[LaTeX]TeX, identifierstyle=\color{colIdentifier},
  keywordstyle=\color{colKeys},
  keywordstyle = [21]\color{DOrange},
  keywordstyle = [22]\color{DOrange},
  stringstyle=\color{colString},
  commentstyle=\color{colComments},
  alsoletter={12},
  float=hbp,
  basicstyle=\ttfamily\small,
  columns=flexible,
  tabsize=4,
  showspaces=false,
  showstringspaces=false,
  breaklines=true,
  breakautoindent=true,
  breakatwhitespace=true,
  captionpos=t,
  belowcaptionskip=0pt,
  abovecaptionskip=0pt,
  xleftmargin=1em,
  prebreak = {\raisebox{-0.5ex}[\ht\strutbox]{\kern0.5ex\large\Righttorque}},
  rulecolor=\color{black!20}, 
  texcsstyle = [20]\color{DDOrange},
  moretexcs = [20]{pssavebezier, pssavepath, pstracecurve, psintersect},
  explpreset={%
    pos=l, width=-99pt, hsep=5mm, overhang=\codeoverhang, varwidth,
    vsep=\bigskipamount, rframe={}}, extendedchars=true
}%
\lstdefinestyle{example}{explpreset={%
    escapechar=*, pos=l, width=-99pt, hsep=5mm, overhang=\codeoverhang,
    varwidth, vsep=\bigskipamount, rframe={}}}
\makeatletter
\providecommand\ON{%
  \gdef\lst@alloverstyle##1{\textcolor{black!50}{\strut##1}%
}}
\providecommand\OFF{\xdef\lst@alloverstyle##1{##1}}
\makeatother
\colorlet{sectioncolor}{DOrange}
\addtokomafont{sectioning}{\color{sectioncolor}}
\usepackage[automark,nouppercase]{scrpage2}
\pagestyle{scrheadings}
\clearscrheadings
\clearscrplain
\ohead{\pagemark}
\ihead{\headmark}
\ofoot[\pagemark]{}
\automark[subsection]{section}
\setheadsepline{.4pt}[\color{DOrange}]
\setheadwidth[0pt]{text}
\setfootwidth[0pt]{text}
\makeatletter
\patchcmd{\l@chapter}{1.5em}{2em}{}{}
\renewcommand*\l@section{\bprot@dottedtocline{1}{1.5em}{3.0em}}
\renewcommand*\l@subsection{\bprot@dottedtocline{2}{3.8em}{4.0em}}
\newrobustcmd*{\fnurl}[1][]{\hyper@normalise\ltd@fnurl{#1}}
\def\ltd@fnurl#1#2{\footnote{#1\hyper@linkurl{\Hurl{#2}}{#2}}}
\newrobustcmd*{\arxivurl}[1]{\href{http://arxiv.org/abs/#1}{arXiv:#1}}
\newrobustcmd*{\doiurl}[1]{\href{http://dx.doi.org/#1}{DOI:#1}}
\makeatother
\usepackage{csquotes}
\MakeAutoQuote{«}{»}
%^^A spot is used in ltxdockit.sty
\colorlet{spot}{sectioncolor}
%^^A Fonts definitions used in ltxdockit.sty
\renewcommand*{\verbatimfont}{\ttfamily}
\renewcommand*{\displayverbfont}{\ttfamily}
\renewcommand*{\marglistfont}{\spotcolor\sffamily\small}
\renewcommand*{\margnotefont}{\sffamily\small}
\renewcommand*{\optionlistfont}{\spotcolor\sffamily\displayverbfont}
\renewcommand*{\ltxsyntaxfont}{\ttfamily}
\renewcommand*{\ltxsyntaxlabelfont}{\spotcolor\displayverbfont}
\renewcommand*{\changelogfont}{\normalfont}
\renewcommand*{\changeloglabelfont}{\spotcolor\sffamily\bfseries}

\makeatletter
\renewenvironment*{ltxsyntax}
  {\list{}{%
     \setlength{\labelwidth}{\marglistwidth}%
     \setlength{\labelsep}{0pt}%
     \setlength{\leftmargin}{0pt}%
     \renewcommand*{\makelabel}[1]{%
       \hss\ltxsyntaxfont\ltxsyntaxlabelfont##1}}%
   \let\cmditem\PIT@cmditem}
  {\endlist}

\def\ltd@optionlist{%
  \let\optitem\PIT@optitem
  \let\valitem\PIT@valitem
  \let\choitem\PIT@choitem
  \let\boolitem\PIT@boolitem
  \let\intitem\PIT@intitem
  \let\numitem\PIT@numitem
  \let\psnumitem\PIT@psnumitem
  \let\pititem\ltd@item
  \let\typeitem\PIT@typeitem}

\def\PIT@typeitem#1#2{%
  \item[{#1}]%
  \label{prm:#1}%
  \begingroup\raggedright
  #2%
  \settowidth\@tempdimb{\prm{psstyle}}%
  \settowidth\@tempdimc{#1}%
  \@tempdimc=\dimexpr\@tempdimc+\labelsep-\labelwidth\relax
  \ifdim\@tempdimc>0pt%
    \@tempdima=\dimexpr\linewidth-\@tempdimb-\@tempdimc-1em\relax
  \else
    \@tempdima=\dimexpr\linewidth-\@tempdimb-1em\relax
  \fi
  \par\endgroup}

\def\PIT@option#1#2#3{%
  \item[#1]%
  \begingroup\raggedright
  \ltd@textverb{=}%
  \settowidth\@tempdimb{\ltd@textverb{=}}%
  \settowidth\@tempdimc{#1}%
  \@tempdimc=\dimexpr\labelwidth-\@tempdimc\relax
  \ifdim\@tempdimc<0pt
    \@tempdima=\dimexpr\linewidth-\@tempdimb+\@tempdimc-2em\relax
  \else
    \@tempdima=\dimexpr\linewidth-\@tempdimb-2em\relax
  \fi
  \ifblank{#3}
    {}
    {\settowidth\@tempdimb{default: #3}%
     \@tempdima=\dimexpr\@tempdima-\@tempdimb-2em\relax}%
  \parbox[t]{\@tempdima}{\raggedright #2}%
  \ifblank{#3}
    {}
    {\hfill default:~#3}%
  \par\endgroup
  \nobreak\vspace{\itemsep}}

\def\PIT@optitem{%
  \@ifstar
    {\boolfalse{@tempswa}\PIT@optitem@i}
    {\booltrue{@tempswa}\PIT@optitem@i}}
\newcommand*{\PIT@optitem@i}[3][]{%
  \ifbool{@tempswa}%
    {\label{prm:#2}\xdefLkeyword{#2}}%
    {\xLkeyword{#2}}%
  \ifblank{#1}
    {\PIT@option{#2}{#3}{}}
    {\PIT@option{#2}{#3}{\ltd@textverb{#1}}}}%

\def\PIT@valitem{%
  \@ifstar
    {\boolfalse{@tempswa}\PIT@valitem@i}
    {\booltrue{@tempswa}\PIT@valitem@i}}
\newcommand*{\PIT@valitem@i}[3][]{%
  \ifbool{@tempswa}%
    {\label{prm:#2}\xdefLkeyword{#2}}%
    {\xLkeyword{#2}}%
  \ifblank{#1}
    {\PIT@option{#2}{\prm{#3}}{}}
    {\PIT@option{#2}{\prm{#3}}{\ltd@textverb{#1}}}}%

\def\PIT@choitem{%
  \@ifstar
    {\boolfalse{@tempswa}\PIT@choitem@i}
    {\booltrue{@tempswa}\PIT@choitem@i}}
\newcommand*{\PIT@choitem@i}[3][]{%
  \ifbool{@tempswa}%
    {\label{prm:#2}\xdefLkeyword{#2}}%
    {\xLkeyword{#2}}%
  \ifblank{#1}
    {\PIT@option{#2}{\ltd@verblist{#3}}{}}
    {\PIT@option{#2}{\ltd@verblist{#3}}{\ltd@textverb{#1}}}}%

\def\PIT@boolitem{%
  \@ifstar
    {\boolfalse{@tempswa}\PIT@boolitem@i}
    {\booltrue{@tempswa}\PIT@boolitem@i}}
\newcommand*{\PIT@boolitem@i}[2][]{%
  \ifbool{@tempswa}%
    {\label{prm:#2}\xdefLkeyword{#2}}%
    {\xLkeyword{#2}}%
  \ifblank{#1}
    {\PIT@option{#2}{\ltd@verblist{true,false}}{}}
    {\PIT@option{#2}{\ltd@verblist{true,false}}{\ltd@textverb{#1}}}}%

\def\PIT@intitem{%
  \@ifstar
    {\boolfalse{@tempswa}\PIT@intitem@i}
    {\booltrue{@tempswa}\PIT@intitem@i}}
\newcommand*{\PIT@intitem@i}[2][]{%
  \ifbool{@tempswa}%
    {\label{prm:#2}\xdefLkeyword{#2}}%
    {\xLkeyword{#2}}%
  \ifblank{#1}
    {\PIT@option{#2}{\prm{int}}{}}
    {\PIT@option{#2}{\prm{int}}{\ltd@textverb{#1}}}}%

\def\PIT@numitem{%
  \@ifstar
    {\boolfalse{@tempswa}\PIT@numitem@i}
    {\booltrue{@tempswa}\PIT@numitem@i}}
\newcommand*{\PIT@numitem@i}[2][]{%
  \ifbool{@tempswa}%
    {\label{prm:#2}\xdefLkeyword{#2}}%
    {\xLkeyword{#2}}%
  \ifblank{#1}
    {\PIT@option{#2}{\prm{num}}{}}
    {\PIT@option{#2}{\prm{num}}{\ltd@textverb{#1}}}}%

\def\PIT@psnumitem{%
  \@ifstar
    {\boolfalse{@tempswa}\PIT@psnumitem@i}
    {\booltrue{@tempswa}\PIT@psnumitem@i}}
\newcommand*{\PIT@psnumitem@i}[2][]{%
  \ifbool{@tempswa}%
    {\label{prm:#2}\xdefLkeyword{#2}}%
    {\xLkeyword{#2}}%
  \ifblank{#1}
    {\PIT@option{#2}{\prm{psnum}}{}}
    {\PIT@option{#2}{\prm{psnum}}{\ltd@textverb{#1}}}}%

\def\ltd@csitem{%
  \@ifstar
    {\boolfalse{@tempswa}\ltd@csitem@i}
    {\booltrue{@tempswa}\ltd@csitem@i}}
\def\ltd@csitem@i#1{%
  \ifbool{@tempswa}
    {\ltd@item@ii{\textbackslash#1\hspace\marglistsep}{#1}}
    {\ltd@item@ii{\textbackslash#1\hspace\marglistsep}{}}}

\def\PIT@cmditem{%
  \@ifstar
    {\boolfalse{@tempswa}\PIT@cmditem@i}
    {\booltrue{@tempswa}\PIT@cmditem@i}}
\def\PIT@cmditem@i#1{%
  \ifbool{@tempswa}
    {\PIT@cmditem@ii{\textbackslash#1}{cs:#1}}
    {\PIT@cmditem@ii{\textbackslash#1}{}}}
\def\PIT@compitem{%
  \@ifstar
    {\boolfalse{@tempswa}\PIT@compitem@i}
    {\booltrue{@tempswa}\PIT@compitem@i}}
\def\PIT@compitem@i#1{%
  \ifbool{@tempswa}
    {\PIT@cmditem@ii{\textbackslash#1}{comp:#1}}
    {\PIT@cmditem@ii{\textbackslash#1}{}}}
\def\PIT@cmditem@ii#1#2{%
  \ltd@itemsave
  \ifhmode
    \itemsep-\topsep
  \else
    \ltd@itembreak
  \fi
  \item[#1]%
  \ltd@itemrest
  \ifblank{#2}{}{\label{#2}}%
  \begingroup
  \ltd@syntaxsetup
  \ltxsyntaxfont
  \let\@tempa\@empty
  \ltd@parseargs}

\def\ltd@csitem@ii#1#2{%
  \ltd@itemsave
  \ifhmode
    \itemsep-\topsep
  \else
    \ltd@itembreak
  \fi
  \item[#1]%
  \ltd@itemrest
  \ifblank{#2}{}{\label{cs:#2}}%
  \begingroup
  \ltd@syntaxsetup
  \ltxsyntaxfont
  \let\@tempa\@empty
  \ltd@parseargs}


\let\plainllap\llap
\newrobustcmd\macro@llap[1]{{\global\let\llap\plainllap
 \setbox0=\hbox\bgroup \macro@font\small\saved@macroname\egroup
 \ifdim\wd0>30mm
    \hbox to\z@ \bgroup\hss \hbox to30mm{\unhcopy0\hss}\egroup
    \edef\@tempa{\hskip\dimexpr\the\wd0-30mm}\global\everypar\expandafter{\the\expandafter\everypar
                                                                           \@tempa \global\everypar{}}%
 \else \llap{\unhbox0}\fi}}
 \AtBeginEnvironment{macro}{\let\llap\macro@llap}
\makeatother

\newcommand*{\PSstring}[1]{{\normalfont\small\ttfamily(#1)}}
\newcommand*{\PSarray}[1]{{\normalfont\small\ttfamily[#1]}}
\newcommand*{\PSname}[1]{{\normalfont\small\ttfamily/#1}}
\newcommand*{\PSproc}[1]{{\normalfont\small\ttfamily\textbraceleft #1\textbraceright}}
\newcommand*{\PSvar}[1]{{\normalfont\small\ttfamily #1}}
\newcommand*{\PSop}[1]{{\normalfont\small\ttfamily\color{DOrange}\hskip 3pt #1\hskip 3pt}}

\newcommand*{\compref}[1]{\ref{comp:#1}}
\newcommand*{\hyperpagedef}[1]{\textbf{\hyperpage{#1}}}
\newcommand*{\nodename}[1]{\emph{#1}}
\newcommand*{\param}[1]{\normalfont\texttt{#1}}
\newcommand*{\paramvalue}[1]{\texttt{#1}}
\newcommand*{\styleshape}[1]{\texttt{#1}}
\newcommand{\docindex}[1]{\sindex[doc]{#1|hyperpage}}
\newcommand{\docindexdef}[1]{\sindex[doc]{#1|hyperpagedef}}
\makeatletter
\def\Lcs{\@ifstar{\Lcs@nobm}{\Lcs@bm}}
\def\Lcs@nobm#1{\nxLcs{#1}\xLcs{#1}}
\def\Lcs@bm#1{\hyperref[cs:#1]{\Lcs@nobm{#1}}}
\def\xLcs#1{}
\def\nxLcs#1{\texttt{\textbackslash#1}}

\def\Lcomp{\@ifstar{\Lcomp@nobm}{\Lcomp@bm}}
\def\Lcomp@nobm#1{\nxLcomp{#1}\xLcomp{#1}}
\def\Lcomp@bm#1{\hyperref[comp:#1]{\Lcomp@nobm{#1}}}
\def\xLcomp#1{}
\def\nxLcomp#1{\texttt{\textbackslash#1}}

\def\LPack#1{\nxLPack{#1}}
\def\nxLPack#1{\texttt{#1}}

\def\Lenv{\@ifstar{\Lenv@nobm}{\Lenv@bm}}
\def\Lenv@nobm#1{\nxLenv{#1}\xLenv{#1}}
\def\Lenv@bm#1{\hyperref[env:#1]{\Lenv@nobm{#1}}}
\def\xLenv#1{}
\def\nxLenv#1{\texttt{#1}}
\let\orig@ltd@envitem\ltd@envitem
\def\ltd@envitem#1{\orig@ltd@envitem{#1}\label{env:#1}\xLenv{#1}}

\def\Lkeyword{\@ifstar{\Lkeyword@nobm}{\Lkeyword@bm}}
\def\Lkeyword@nobm#1{\nxLkeyword{#1}\xLkeyword{#1}}
\def\Lkeyword@bm#1{\hyperref[prm:#1]{\Lkeyword@nobm{#1}}}
\def\xLkeyword#1{}
\def\xdefLkeyword#1{}
\def\nxLkeyword#1{\texttt{#1}}

\def\xLoption#1{}
\def\Loption#1{\texttt{#1}\xLoption{#1}}
\def\nxLoption#1{\texttt{#1}}

\def\Lstyle{\@ifstar{\Lstyle@nobm}{\Lstyle@bm}}
\def\Lstyle@nobm#1{\nxLstyle{#1}\xLstyle{#1}}
\def\Lstyle@bm#1{\hyperref[sty:#1]{\Lstyle@nobm{#1}}}
\def\xLstyle#1{}
\def\xdefLstyle#1{}
\def\nxLstyle#1{\texttt{#1}}

\makeatother

\newcommand{\dipoledesc}[1]{%
  \xLdipole{#1}%
  \compitem{#1}[options](in)(out){label}%
}
\newcommand{\tripoledesc}[1]{%
  \xLtripole{#1}%
  \compitem{#1}[options](in)(center)(out){label}%
}

\newcommand{\fiberdipoledesc}[1]{%
  \xLfdipole{#1}%
  \compitem{#1}[options](in)(out){label}%
}

\newenvironment*{pssyntax}
  {\list{}{\small
     \setlength{\labelsep}{0pt}%
     \setlength{\leftmargin}{10pt}%
     \item[]}}
  {\endlist}

\newcommand{\psarglistfont}{\small}
\newenvironment*{psarglist}
  {\list{}{%
     \setlength{\labelwidth}{10pt}%
     \setlength{\labelsep}{0pt}%
     \setlength{\leftmargin}{0pt}%
     \setlength{\itemsep}{\parsep}%
     \setlength{\parsep}{0pt}%
     \renewcommand*{\makelabel}[1]{\hss\psarglistfont##1}}}
  {\endlist}

\makeatletter
\renewenvironment{theglossary}{\GlossaryParms \let\item\@idxitem \ignorespaces}{}
\makeatother
\def\psargitem#1{\item[#1]\hfill\par\nobreak}

\addtopsstyle{Fiber}{linecolor=DOrange,linewidth=1.5\pslinewidth}
\addtopsstyle{Beam}{linewidth=1.5\pslinewidth}
\EnableCrossrefs
\CodelineIndex
\OnlyDescription
\begin{document}
  \DocInput{pst-intersect.dtx}
\end{document}
%</driver>
% \fi
%
% \CheckSum{0}
%
% \CharacterTable
%  {Upper-case    \A\B\C\D\E\F\G\H\I\J\K\L\M\N\O\P\Q\R\S\T\U\V\W\X\Y\Z
%   Lower-case    \a\b\c\d\e\f\g\h\i\j\k\l\m\n\o\p\q\r\s\t\u\v\w\x\y\z
%   Digits        \0\1\2\3\4\5\6\7\8\9
%   Exclamation   \!     Double quote  \"     Hash (number) \#
%   Dollar        \$     Percent       \%     Ampersand     \&
%   Acute accent  \'     Left paren    \(     Right paren   \)
%   Asterisk      \*     Plus          \+     Comma         \,
%   Minus         \-     Point         \.     Solidus       \/
%   Colon         \:     Semicolon     \;     Less than     \<
%   Equals        \=     Greater than  \>     Question mark \?
%   Commercial at \@     Left bracket  \[     Backslash     \\
%   Right bracket \]     Circumflex    \^     Underscore    \_
%   Grave accent  \`     Left brace    \{     Vertical bar  \|
%   Right brace   \}     Tilde         \~}
%
% \GetFileInfo{pst-intersect.dtx}
%
% \newif\ifGERMAN  \GERMANfalse
% \newif\ifENGLISH \ENGLISHfalse
% \iflanguage{ngerman}{\GERMANtrue}{%
%   \iflanguage{german}{\GERMANtrue}{\GERMANfalse}}
% \iflanguage{english}{\ENGLISHtrue}{\ENGLISHfalse}
%
% \ifGERMAN
%   \newrefformat{chap}{Kap.~\ref{#1}}
%   \newrefformat{sec}{Kap.~\ref{#1}}
%   \newrefformat{l}{Zeile~\ref{#1}}
%   \newrefformat{ex}{Bsp.~\ref{#1}}
%   \newrefformat{tab}{Tab.~\ref{#1}}
%   \newrefformat{fig}{Abb.~\ref{#1}}
%   \newcommand{\PITindexMacro}{Makros}
%   \newcommand{\PITindexKeyword}{Parameter}
%   \newcommand{\PITindexEnv}{Umgebungen}
%   \newcommand{\PITindexPack}{Pakete}
%   \renewcommand*\lstlistingname{Bsp.}
% \fi
% \ifENGLISH
%   \newrefformat{chap}{Sec.~\ref{#1}}
%   \newrefformat{sec}{Sec.~\ref{#1}} 
%   \newrefformat{l}{Line~\ref{#1}}
%   \newrefformat{ex}{Ex.~\ref{#1}}
%   \newrefformat{tab}{Tab.~\ref{#1}}
%   \newrefformat{fig}{Fig.~\ref{#1}}
%   \newcommand{\PITindexMacro}{macros}
%   \newcommand{\PITindexKeyword}{parameters}
%   \newcommand{\PITindexEnv}{environments}
%   \newcommand{\PITindexPack}{packages}
%   \renewcommand*\lstlistingname{Ex.}
% \fi
%
% \makeatletter
% \renewcommand\maketitle{^^A
% \thispagestyle{empty}^^A
% \begin{titlepage}
% \begin{pspicture}(1.6in,0.685in)(10,21.7)
%   \psframe[fillstyle=solid,linecolor=lightgray,fillcolor=lightgray,linestyle=solid](0,-5.75)(21.5,10)
%   \psframe[fillstyle=solid,linecolor=Orange!85!Red,fillcolor=Orange!85!Red,linestyle=solid](0,10)(21.5,10.5)
%   \psframe[fillstyle=solid,linecolor=Orange!85!Red,fillcolor=Orange!85!Red,linestyle=solid](0,21.1)(21.5,21.2)
%   \rput[lb](3,22){\Huge\sffamily\color{Orange!65!Red}\psscalebox{2}{\textbf{PSTricks}}}
%   \rput[lb](3,14.1){\parbox{15cm}{\sffamily\RaggedRight\bfseries\huge\@title}}
%   \rput[lb](3,7.6){\parbox{13cm}{\sffamily\@date}}
%   \rput[lb](3,-2.6){\parbox[b]{17cm}{\sffamily\RaggedRight 
%     ~\hfill\makebox[7cm][l]{\ifGERMAN Paketautor:\fi\ifENGLISH Package author:\fi}\\
%     ~\hfill\makebox[7cm][l]{^^A
%       \bfseries\tabular[t]{@{}l@{}}\@author\endtabular}}}
%  \rput[C](11,4){\bgImage}
%  \end{pspicture}^^A
% \end{titlepage}}
% \makeatother
% 
% \ifGERMAN
%   \title{\texttt{pst-intersect}\\ Berechnen der Schnittpunkte beliebiger Kurven\\[0.5ex] \small \fileversion}
%   \hypersetup{pdftitle={Berechnen der Schnittpunkte beliebiger Kurven}}
% \fi
% \ifENGLISH
%   \title{\texttt{pst-intersect}\\ Intersecting arbitrary curves\\[0.5ex] \small \fileversion}
%   \hypersetup{pdftitle={Intersecting arbitrary curves}}
% \fi
% \author{Christoph Bersch}
% \date{\filedate}
% \def\bgImage{}
%
% \maketitle
% 
% \clearpage
% \tableofcontents
% \clearpage
% 
% \ifGERMAN
%   \chapter{Einführung}
% \fi
% \ifENGLISH
%   \chapter{Introduction}
% \fi
% 
% \ifGERMAN
%   \section{Über das Paket}
%
%   \LPack{pst-intersect} ist ein PSTricks-Paket zur
%   Berechnung der Schnittpunkte von Bézier-Kurven und beliebigen
%   Postscript-Pfaden.
%
%   Beachten Sie, dass die Paket-Versionen 0.x sich in einem
%   experimentellen Status befinden, und sich grundlegende Änderungen
%   ergeben können, die zur Vorgängerversion inkompatibel sind.
% \fi 
% \ifENGLISH
%   \section{About the package}
%   \LPack{pst-intersect} is a PSTricks package to calculate
%   the intersections of Bezier curves and arbitrary Postscript paths.
%
%   Please note, that package versions 0.x are experimental, and may be
%   subject to fundamental changes, which aren't backward compatible.
% \fi
%
% \ifGERMAN
%   \section{Anforderungen}
%   \LPack{pst-intersect} benötigt aktuelle Versionen der
%   Pakete \LPack{pstricks}, \LPack{pst-node} und \LPack{pst-func}.
%
%   Alle PSTricks-Pakete machen regen Gebrauch von der Postscript-Sprache, so
%   dass der typische Arbeitsfluss \opt{latex}, \opt{dvips} und
%   ggf. \opt{ps2pdf} umfasst. Es gibt viele alternative Methoden um die
%   Dokumente zu
%   kompilieren.\fnurl{http://tug.org/PSTricks/main.cgi?file=pdf/pdfoutput}
% \fi
% \ifENGLISH
%   \section{Requirements}
%   \LPack{pst-intersect} requires recent versions of
%   \LPack{pstricks}, \LPack{pst-node}, and \LPack{pst-func}.
%
%   All PSTricks package rely heavily on the Postscript language so that the
%   typical workflow involves \opt{latex}, \opt{dvips}, and \opt{ps2pdf}. Of
%   course there are several alternative ways to compile your
%   documents.\fnurl{http://tug.org/PSTricks/main.cgi?file=pdf/pdfoutput} 
% \fi
%
% \ifGERMAN
%   \section{Verbreitung und Installation}
%   Dieses Paket ist auf
%   CTAN\fnurl{http://mirror.ctan.org/help/Catalogue/entries/pst-intersect.html}
%   erhältlich.
% 
%   Das \LPack{pst-intersect}-Paket umfasst die zwei Hauptdateien
%   \texttt{pst-intersect.ins} und \texttt{pst-intersect.dtx}. Durch Aufrufen
%   von \texttt{tex pst-intersect.ins} werden die drei folgenden
%   Dateien erzeugt:
%   \begin{itemize}
%   \item \texttt{pst-intersect.pro}: die Postscript Prologdatei
%   \item \texttt{pst-intersect.sty}: die \LaTeX-Stildatei
%   \item \texttt{pst-intersect.tex}: die \TeX-Datei
%   \end{itemize}
%   Speichern Sie diese Dateien in einem Verzeichnis der Teil Ihres
%   lokalen \TeX-Baums ist.
% 
%   Vergessen Sie nicht \texttt{texhash} aufzurufen um den Baum zu
%   aktualisieren. MiK\TeX{}-Benutzer müssen die Dateinamen-Datenbank
%   (FNDB) aktualisieren.
% 
%   Detailliertere Information finden Sie in der Dokumentation Ihrer
%   \LaTeX-Distribution über die Installation in den lokalen
%   \TeX{}-Baum.
% \fi
% \ifENGLISH
%   \section{Distribution and installation}
%   This package is available on
%   CTAN\fnurl{http://mirror.ctan.org/help/Catalogue/entries/pst-intersect.html}.
% 
%   The \LPack{pst-intersect} package consists of the two main files
%   \texttt{pst-intersect.ins} and \texttt{pst-intersect.dtx}. By running \texttt{tex
%   pst-intersect.ins} the following derived files are generated:
%   \begin{itemize}
%   \item \texttt{pst-intersect.pro}: the Postscript prolog file
%   \item \texttt{pst-intersect.sty}: the \LaTeX{} style file
%   \item \texttt{pst-intersect.tex}: the \TeX{} file
%   \end{itemize}
%   Save the files in a directory which is part of your local \TeX{} tree.
% 
%   Do not forget to run \texttt{texhash} to update this tree. For MiK\TeX{}
%   users, do not forget to update the file name database (FNDB).
% 
%   For more detailed information see the documentation of your personal
%   \LaTeX{} distribution on installing packages to your local \TeX{}
%   system.
% \fi
%
% \ifGERMAN\section{Lizenz}\fi
% \ifENGLISH\section{License}\fi
% \ifGERMAN
% Es wird die Erlaubnis gewährt, dieses Dokument zu kopieren, zu verteilen
% und\slash oder zu modifizieren, unter den Bestimmungen der \LaTeX{} Project
% Public License, Version
% 1.3c.\fnurl{http://www.latex-project.org/lppl.txt}. Dieses
% Paket wird vom Autor betreut (author-maintained).
% \fi
% \ifENGLISH
% Permission is granted
% to copy, distribute and\slash or modify this software under the terms of the
% \LaTeX{} Project Public License, version
% 1.3c.\fnurl{http://www.latex-project.org/lppl.txt} This
% package is author-maintained.
% \fi
%
% \ifGERMAN
%   \section{Danksagung}
% \fi
% \ifENGLISH
%   \section{Acknowledgements}
% \fi
% \ifGERMAN
% Ich danke Marco Cecchetti, dessen
% \opt{lib2geom}-Bibliothek\fnurl{http://lib2geom.sourceforge.net/}
% mir als Vorlage für einen Großteil des Postscript-Kodes für den
% Bézier-Clipping-Algorithmus diente. Außerdem gilt mein Dank William
% A. Casselman, für seine Erlaubnis, den Quicksort-Kode und den Kode zur
% Berechung der konvexen Hüllen aus seinem Buch «Mathematical
% Illustration» verwenden zu
% dürfen\fnurl{http://www.math.ubc.ca/~cass/graphics/text/www/}. Der
% Dokumentationsstil ist eine Mischung aus der \opt{pst-doc} Klasse
% (Herbert Voß) und dem \opt{ltxdockit} Paket für die \opt{biblatex}
% Dokumentation (Philipp Lehmann).
% \fi
% \ifENGLISH
% I thank Marco Cecchetti, for his
% \opt{lib2geom}-library\fnurl{http://lib2geom.sourceforge.net/} from
% which I derived great parts of the Postscript code for the Bézier
% clipping algorithm. Also I want to thank William A. Casselman for the
% Postscript code of the quicksort procedure and the procedure for
% calculating the convex hull from his book «Mathematical
% Illustration»\fnurl{http://www.math.ubc.ca/~cass/graphics/text/www/},
% and the permission to use it. The documentation style is a mixture of
% the \opt{pst-doc} class (Herbert Voß) and the \opt{ltxdockit} package
% for the \opt{biblatex} documentation (Philipp Lehman).
% \fi
%
%
% \ifGERMAN
% \chapter{Benutzung}
% \fi
% \ifENGLISH
% \chapter{Usage}
% \fi
%
% \ifGERMAN
% Das \LPack{pst-intersect}-Paket kann Schnittpunkte von beliebigen
% Postscript-Pfaden berechnen. Diese setzen sich nur aus drei primitiven
% Operation zusammen: Linien (\opt{lineto}), Bézier-Kurven dritter
% Ordnung (\opt{curveto}) und Sprüngen (\opt{moveto}). Speziellere
% Konstruktionen, wie z.B. Kreise werden intern zu
% \opt{curveto}-Anweisungen umgewandelt. Über diese Kommandos hinaus
% kann \LPack{pst-intersect} auch Bézier-Kurven bis neunter Ordnung
% verwenden. Das diese keine primitiven Postscript-Pfadelemente
% darstellen, müssen sie gesondert behandelt werden.
%
% Der allgemeine Arbeitsablauf besteht darin eine oder mehrere Kurven
% oder Pfade zu speichern, und danach die Schnittpunkte zu
% berechnen. Anschließend können die Schnittpunkte als normale
% PSTricks-Knoten verwendet werden, oder Abschnitte der Kurven und Pfade
% nachgezogen werden (z.B. zwischen zwei Schnittpunkten). 
% \fi
% \ifENGLISH
% The \LPack{pst-intersect} package can compute the intersections of
% arbitrary Postscript paths. These are composed of three primitive
% operations: lines (\opt{lineto}), third order Bézier curves
% (\opt{curveto}) and jumps (\opt{moveto}). More specialized
% constructions, like circles, are converted internally to \opt{curveto}
% operations. Besides these three path operations, the
% \LPack{pst-intersect} supports Bézier curves up to nineth order. As
% these aren't primitive Postscript path elements, they require separate
% handling.
%
% The general workflow consists in defining and saving paths and curves,
% and then compute the intersections between them. Following, those
% intersection points can be used as normal PSTricks nodes, or portions
% of the curves and paths can be retraced (e.g. between two
% intersections).
% \fi
%
% \ifGERMAN
% \section{Speichern von Pfaden und Kurven}
% \fi
% \ifENGLISH
% \section{Saving paths and curves}
% \fi
%
% \ifGERMAN
% 
% \fi
%
% \begin{ltxsyntax}
%   \cmditem{pssavepath}[options]{curvename}{commands} 
%   
%   \ifGERMAN
%   Speichert den gesamten Pfad, der durch \prm{commands} erstellt wird,
%   unter Verwendung des Namens \prm{curvename}. Das Makro funktioniert
%   genauso wie \cs{pscustom}, und kann daher auch nur die darin
%   erlaubten Kommandos verarbeiten.
%
%   In den Standardeinstellungen wird der entsprechende Pfad auch gleich
%   gezeichnet, was mit \prm{options} beeinflusst werden kann. Mit
%   \opt{linestyle=none} wird das unterbunden.
%   \fi
%   \ifENGLISH
%   Saves the complete path, which is generated by \prm{commands}, under
%   the name \prm{curvename}. The macro is a modification of
%   \cs{pscustom}, and does, therefore, supports only the same commands.
%
%   By default, the path is also drawn, which can be changed over the
%   \prm{options}, e.g. with \opt{linestyle=none}.
%   \fi
% \end{ltxsyntax}
%\iffalse
%<*ignore>
%\fi
\begin{LTXexample}
\begin{pspicture}(3,2)
  \pssavepath[linecolor=DOrange]{MyPath}{%
    \pscurve(0,2)(0,0.5)(3,1)
  }%
\end{pspicture}
\end{LTXexample}
%\iffalse
%</ignore>
%\fi
%
% \begin{ltxsyntax}
%   \cmditem{pssavebezier}[options]{curvename}($X_0$)\ldots(\prm{$X_n$})
%
%   \ifGERMAN 
%   Die Postscript-Sprache unterstützt nur Bézier-Kurven dritter
%   Ordnung. Mit dem Makro \Lcs{pssavebezier} können Bézier-Kurven
%   bis neunter Ordnung definiert werden. Die angegebenen Knoten sind die
%   Kontrollpunkte der Kurve, für eine Kurve $n$-ter Ordnung werden
%   $(n+1)$ Kontrollpunkte benötigt. Die Darstellung der Kurve erfolgt
%   mit dem Makro \cs{psBezier} aus dem \LPack{pst-func}-Paket.
%   \fi
%   \ifENGLISH
%   The Postscript language supports only third-order Bézier
%   curves. With the macro \Lcs{pssavebezier} you can define Bézier
%   curves up to nineth order. The specified nodes are the control
%   points of the curve, for an $n$-th order curve $n+1$ control points
%   are required. The drawing of the curve is done with the
%   \cs{psBezier} macro from the \LPack{pst-func} package.
%   \fi
% \end{ltxsyntax}
%\iffalse
%<*ignore>
%\fi
\begin{LTXexample}
\begin{pspicture}(3,2)
  \pssavebezier[showpoints]{MyBez}(0,0)(0,1)(1,2)(3,2)(1,0)(3,0)
\end{pspicture}
\end{LTXexample}
%\iffalse
%</ignore>
%\fi
%
% \ifGERMAN
% \section{Schnittpunkte berechnen}
% \fi
% \ifENGLISH
% \section{Calculating intersections}
% \fi
%
% \begin{ltxsyntax}
%   \cmditem{psintersect}{curveA}{curveB}
% 
%   \ifGERMAN
%   Nachdem Sie nun Pfade oder Kurven gespeichert haben, können Sie
%   deren Schnittpunkte berechnen. Das geschieht mit dem Makro
%   \Lcs{psintersect}. Dieses benötigt als Argumente zwei Namen von
%   Pfaden oder Kurven (Das Argument \prm{curvename} der beiden
%   \cs{pssave*} Makros).
%   \fi
%   \ifENGLISH
%   After having saved some paths and curves, you can now calculate the
%   intersections. That is done with the \Lcs{psintersect} macro. This
%   needs as arguments two names of paths or curves (the \prm{curvename}
%   argument of the two \cs{pssave*} macros).
%   \fi
% \end{ltxsyntax}
%
%\iffalse
%<*ignore>
%\fi
\begin{LTXexample}
\begin{pspicture}(3,2)
  \pssavepath[linecolor=DOrange]{MyPath}{\pscurve(0,2)(0,0.5)(3,1)}
  \pssavebezier{MyBez}(0,0)(0,1)(1,2)(3,2)(1,0)(3,0)
  \psintersect[showpoints]{MyPath}{MyBez}
\end{pspicture}
\end{LTXexample}
%\iffalse
%</ignore>
%\fi
%
% \ifGERMAN
% Der PSTricks-Parameter \opt{showpoints} steuert dabei, ob die
% Schnittpunkte angezeigt werden.
% \fi
% \ifENGLISH
% The \opt{showpoints} PSTricks parameter determines, if the
% intersections are drawn directly. 
% \fi
%
% \begin{optionlist}
% \valitem[@tmp]{name}{string}
% \ifGERMAN
% Die berechneten Schnittpunkte können unter einem hier angegebenen
% Namen gespeichert und zu einem späteren Zeitpunkt verwendet werden
% (siehe \prettyref{sec:pstracecurve-int}). 
% \fi
% \ifENGLISH
% The calculated intersections can be saved for later use under this
% name (see \prettyref{sec:pstracecurve-int}).
% \fi
%
% \boolitem[true]{saveintersections}
% \ifGERMAN
% Ist dieser Schalter gesetzt, dann werden die Schnittpunkte als
% PSTricks-Knoten unter den Namen \prm{name}1, \prm{name}2 \ldots
% gespeichert. Die Nummerierung erfolgt aufsteigend nach dem Wert der
% $x$-Koordinate.
% \fi
% \ifENGLISH
% If this option is set, the intersections are saved as PSTricks nodes
% with the names \prm{name}1, \prm{name}2 \ldots. The numbering is
% ascending according to the value of their $x$-coordinate.
% \fi
%
%\iffalse
%<*ignore>
%\fi
\begin{LTXexample}
\begin{pspicture}(5,5)
  \pssavebezier[linecolor=DOrange]{A}%
             (0,0)(0,5)(5,5)(5,1)(1,1.5)
  \pssavebezier{B}(0,5)(0,0)(5,0)(5,5)(0,2)
  \psintersect[name=C, showpoints]{A}{B}
  \uput[150](C1){1}
  \uput[85](C2){2}
  \uput[90](C3){3}
  \uput[-20](C4){4}
\end{pspicture}
\end{LTXexample}
%\iffalse
%</ignore>
%\fi
% \end{optionlist}
%
% \ifGERMAN
% \section{Darstellung gespeicherter Pfade}
% \fi
% \ifENGLISH
% \section{Visualization of saved paths}
% \fi
%
% \begin{ltxsyntax}
%   \cmditem{pstracecurve}[options]{curvename}
%
% \ifGERMAN
% Gespeicherte Pfade und Kurven können mit diesem Makro nachträglich
% gezeichnet werden.
% \fi
% \ifENGLISH
% Saved paths and curves can be drawn again with this macro.
% \fi
% \end{ltxsyntax}
%
%\iffalse
%<*ignore>
%\fi
\begin{LTXexample}
\begin{pspicture}(2,2)
  \pssavepath{Circle}{\pscircle(1,1){1}}
  \pstracecurve[linestyle=dashed, linecolor=green]{Circle}
\end{pspicture}
\end{LTXexample}
%\iffalse
%</ignore>
%\fi
%
% \begin{optionlist}
% \numitem{tstart}
% \numitem{tstop}
%
% \ifGERMAN Unter Verwendung dieser beiden Parameter können auch
% Abschnitte von Pfaden und Kurven gezeichnet werden. Bei Bézier-Kurven
% ist der Parameterbereich $[0, 1]$, wobei $0$ dem Anfang der Kurve,
% also dem ersten bei \Lcs{pssavebezier} angegebenen Knoten entspricht.
% \fi
% \ifENGLISH
% With these parameters also parts of paths and curves can be drawn. For
% Bézier curves the allowed range is $[0, 1]$, where $0$ corresponds to
% the start of the curve, which is given by the first node given to
% \Lcs{pssavebezier}.
% \fi
%
%\iffalse
%<*ignore>
%\fi
\begin{LTXexample}
\begin{pspicture}(5,5)
  \psset{showpoints}
  \pssavebezier{B}(0,5)(0,0)(5,0)(5,5)(0,2)
  \pstracecurve[linestyle=dashed, linecolor=blue!50,
               tstart=0, tstop=0.5]{B}
\end{pspicture}
\end{LTXexample}
%\iffalse
%</ignore>
%\fi
%
% \medskip
% \ifGERMAN 
% Pfaden können aus mehr als einem Abschnitt bestehen, der Bereich ist
% also $[0, n]$, wobei $n$ die Anzahl der Pfadabschnitte ist. Dabei ist
% zu beachten, dass z.B. \cs{pscurve}-Pfade oder auch Kreise und
% Kreisbögen aus mehreren Abschnitten bestehen.
% \fi
% \ifENGLISH
% Paths can be composed of more than one segmet, and the range is $[0,
% n]$, where $n$ is the number of path segments. For this you must keep
% in mind, that also e.g \cs{pscurve} paths, circles or arcs consist of
% several segments.
% \fi
%
%\iffalse
%<*ignore>
%\fi
\begin{LTXexample}
\begin{pspicture}(2,2)
  \pssavepath[linestyle=none]{Circle}{\pscircle(1,1){1}}
  \pstracecurve[tstart=0, tstop=1, linecolor=green]{Circle}
  \pstracecurve[tstart=2, tstop=3, linecolor=red]{Circle}
  \pstracecurve[tstart=1.25, tstop=1.75, linecolor=blue]{Circle}
\end{pspicture}
\end{LTXexample}
%\iffalse
%</ignore>
%\fi
%
% \end{optionlist}
%
% \ifGERMAN
% Beachten Sie, dass die Reihenfolge von \Lkeyword{tstart} und
% \Lkeyword{tstop} eine Rolle spielt. Ist \Lkeyword{tstart}
% \textgreater{} \Lkeyword{tstop} dann wird die Pfadrichtung umgekehrt.
% \fi
% \ifENGLISH
% Please note, that the order of \Lkeyword{tstart} and \Lkeyword{tstop}
% plays a role. For \Lkeyword{tstart} \textgreater{} \Lkeyword{tstop}
% the path direction is reversed.
% \fi
%
% \ifGERMAN
% \section{Darstellung gespeicherter Schnitte}
% \fi
% \ifENGLISH
% \section{Visualization of saved intersections}
% \fi
% \label{sec:pstracecurve-int}
%
% \begin{ltxsyntax}
% \cmditem*{pstracecurve}[options]{intersection}{curvename}
% \end{ltxsyntax}
% 
% \begin{optionlist}
% \numitem{istart}
% \numitem{istop}
%
% \ifGERMAN 
% Dieser beiden Parameter können auch verwendet werden um Abschnitte von
% Pfaden und Kurven zwischen Schnittpunkten zu zeichnen. Die
% Schnittpunkte werden dabei angefangen bei $1$ in aufsteigender
% Reihenfolge entlang der Kurve durchnummeriert.
% \fi
% \ifENGLISH
% These two parameters can be used to draw path or curve segments
% between intersections. The intersections are numbered starting at $1$
% in ascending order along the curve.
% \fi
% \end{optionlist}
%
%\iffalse
%<*ignore>
%\fi
\begin{LTXexample}
\begin{pspicture}(5.2,5.2)
\pssavebezier[linewidth=0.5\pslinewidth, linestyle=dashed, arrows=->]{A}(0,0)(0,5)(5,2)(5,5)
\pssavebezier[linewidth=0.5\pslinewidth, linestyle=dashed, arrows=->]{B}(0,2.5)(2.5,2.5)(4.5, 3)(2,4)
\psintersect[linecolor=green!70!black, name=C]{A}{B}
\pstracecurve[linecolor=red, istart=1, istop=2]{C}{A}
\pstracecurve[linecolor=blue, istart=1, istop=2]{C}{B}
\end{pspicture}
\end{LTXexample}
%\iffalse
%</ignore>
%\fi
%
% \ifGERMAN 
% Wird nur ein Wert angegeben, beispielsweise \Lkeyword{istop}, dann
% wird die Kurve vom Anfang bis zum entsprechenden Schnittpunkt
% gezeichnet. Wird nur \Lkeyword{istart} angegeben, dann endet die Kurve
% am Ende. Die Parameter \Lkeyword{istart} bzw. \Lkeyword{istop} können
% mit \Lkeyword{tstart} bzw. \Lkeyword{tstop} kombiniert werden.
% \fi
% \ifENGLISH
% If only one value is specified, e.g. \Lkeyword{istop}, the curve is
% drawn from the start to the respective intersection. If only
% \Lkeyword{istart} is given, the curve is drawn from this intersection
% to the curve end. The parameters \Lkeyword{istart} and
% \Lkeyword{istop} can be combined with \Lkeyword{tstart} and
% \Lkeyword{tstop}.
% \fi
%
%\iffalse
%<*ignore>
%\fi
\begin{LTXexample}
\begin{pspicture}(5.2,5.2)
\pssavebezier[linewidth=0.5\pslinewidth, linestyle=dashed, arrows=->]{A}(0,0)(0,5)(5,2)(5,5)
\pssavebezier[linewidth=0.5\pslinewidth, linestyle=dashed, arrows=->]{B}(0,2.5)(2.5,2.5)(4.5, 3)(2,4)
\psintersect[linecolor=green!70!black, name=C]{A}{B}
\pstracecurve[linecolor=red, istop=2]{C}{A}
\pstracecurve[linecolor=blue, istart=1]{C}{B}
\end{pspicture}
\end{LTXexample}
%\iffalse
%</ignore>
%\fi
%
% \ifGERMAN
% \chapter{Beispiele}
% \fi
% \ifENGLISH
% \chapter{Examples}
% \fi
%\iffalse
%<*ignore>
%\fi
\begin{LTXexample}[caption={caption}]
\begin{pspicture}(5,5)
  \pssavebezier{A}(0,0)(0,5)(5,5)(5,1)(1,1.5)
  \multido{\i=100+-20,\r=1+-0.2}{5}{%
    \pstracecurve[linecolor=red!\i, tstop=\r, arrows=-|, showpoints]{A}
  }%
\end{pspicture}
\end{LTXexample}
%\iffalse
%</ignore>
%\fi 
%
%\iffalse
%<*ignore>
%\fi
\begingroup
\captionsetup[lstlisting]{format=pitcaption}
\begin{LTXexample}[pos=t, caption={%
\ifGERMAN Mit diesem Paket können auch die Schnittpunkte von
 Funktionen berechnet werden, die mit \cs{psplot} gezeichnet
 werden. Dabei ist zu beachten, dass beide einzelnen Kurven aus
 \opt{plotpoints} Abschnitten bestehen, von denen jeder mit jedem
 geschnitten wird, was zu langen Berechnungen führen kann.
 \fi
 \ifENGLISH
 The package can also calculate the intersections of functions which are
 drawn with \cs{psplot}. Here you must keep in mind, that such curves
 consists of \opt{plotpoints} segments, which must all be considered for
 intersections, what can result in long calculations.
 \fi}]
\begin{pspicture}(10,4.4)
  \pssavepath{A}{\psplot[plotpoints=200]{0}{10}{x 180 mul sin 1 add 2 mul}}
  \pssavepath{B}{\psplot[plotpoints=50]{0}{10}{2 x neg 0.5 mul exp 4 mul}}
  \psintersect[name=C, showpoints]{A}{B}
  \multido{\i=1+1}{5}{\uput[210](C\i){\i}}
  \multido{\i=6+2,\ii=7+2}{3}{\uput[225](C\i){\i}\uput[-45](C\ii){\ii}}
\end{pspicture}
\end{LTXexample}
\endgroup
%\iffalse
%</ignore>
%\fi 
%
% \appendix
%
% \ifGERMAN
% \chapter{Versionsgeschichte}
%
% Diese Versionsgeschichte ist eine Liste von Änderungen, die für den Nutzer des
% Pakets von Bedeutung sind. Änderungen, die eher technischer Natur sind und für
% den Nutzer des Pakets nicht relevant sind und das Verhalten des Pakets nicht
% ändern, werden nicht aufgeführt. Wenn ein Eintrag der Versionsgeschichte ein
% Feature als \emph{improved} oder \emph{extended} bekannt gibt, so bedeutet
% dies, dass eine Modifikation die Syntax und das Verhalten des Pakets nicht
% beeinflusst, oder das es für ältere Versionen kompatibel ist. Einträge, die
% als \emph{deprecated}, \emph{modified}, \emph{renamed}, oder \emph{removed}
% deklariert sind, verlangen besondere Aufmerksamkeit. Diese bedeuten, dass eine
% Modifikation Änderungen in existierenden Dokumenten mit sich ziehen kann. 
% \fi
% \ifENGLISH
% \chapter{Revision history}
%
% This revision history is a list of changes relevant to users of this
% package. Changes of a more technical nature which do not affect the
% user interface or the behavior of the package are not included in the
% list. If an entry in the revision history states that a feature has
% been \emph{improved} or \emph{extended}, this indicates a modification
% which either does not affect the syntax and behavior of the package or
% is syntactically backwards compatible (such as the addition of an
% optional argument to an existing command). Entries stating that a
% feature has been \emph{deprecated}, \emph{modified}, \emph{fixed},
% \emph{renamed}, or \emph{removed} demand attention. They indicate a
% modification which may require changes to existing documents.
% \fi
%
% \begin{changelog}
%\patchcmd{\release}{\setlength{\itemsep}{0pt}}{\setlength{\itemsep}{0pt}\setlength{\parsep}{0pt}}{}{}
%   \begin{release}{0.2}{2014-02-25}
%   \item Added support for \opt{arrows} parameter to \cs{pstracecurve}.
%   \item Modified parameters \opt{tstart}, \opt{tstop}, \opt{istart}
%     and \opt{istop} to respect different directions.
%   \item Added parameter \opt{reversepath}.
%   \item Fixed a bug in the termination of the iteration procedure.
%   \item Fixed a bug in the point order of Bézier curves, which was
%     related to a now fixed bug in \opt{pst-func}.
%   \end{release}
%   \begin{release}{0.1}{2014-02-19}
%   \item First CTAN version
%   \end{release}
% \end{changelog}
%
% \StopEventually{}
%
%   \begin{otherlanguage}{english}
%    \printindex[idx]
%  \end{otherlanguage}
%
% \chapter{The \LaTeX\ wrapper}
%<*stylefile>
%    \begin{macrocode}
\RequirePackage{pstricks}
\RequirePackage{pst-xkey}
\RequirePackage{pst-node}
\RequirePackage{pst-func}
% \iffalse meta-comment
%
% Copyright (C) 2014 by Christoph Bersch <usenet@bersch.net>
%
% This work may be distributed and/or modified under the
% conditions of the LaTeX Project Public License, either version 1.3c
% of this license or (at your option) any later version.
% The latest version of this license is in
%   http://www.latex-project.org/lppl.txt
% and version 1.3c or later is part of all distributions of LaTeX
% version 2008/05/04 or later.
% \fi
%
% \iffalse
%<*driver>
\ProvidesFile{pst-intersect.dtx}
%</driver>
%<stylefile>\NeedsTeXFormat{LaTeX2e}[1999/12/01]
%<stylefile>\ProvidesPackage{pst-intersect}
%<*stylefile>
    [2014/02/18 v0.1alpha package wrapper for pst-intersect.tex]
%</stylefile>
%
%<*driver>
\documentclass[a4paper, DIV=9, oneside, toc=index, parskip=half-]{scrreprt}
\usepackage{doc}
\setcounter{IndexColumns}{2}
\usepackage[utf8]{inputenc} 
\usepackage[T1]{fontenc}
\usepackage{lmodern} 
\usepackage{amsmath, marvosym} 
\usepackage{bera}
\providecommand*\mainlang{}
\usepackage[ngerman, english,\mainlang]{babel}
\usepackage{prettyref}
\usepackage[dvipsnames,x11names,svgnames]{xcolor}
\usepackage{array,booktabs,paralist,tabularx}
\usepackage{ragged2e, calc}
\usepackage{nicefrac, multido}
\usepackage{pst-intersect}
\usepackage{hypdoc}
\hypersetup{%
  colorlinks=true, 
  urlcolor=DOrange, 
  linkcolor=pdflinkcolor, 
  breaklinks,
  linktocpage=true} 
\usepackage{breakurl}
\definecolor{DOrange}{rgb}{1,.4,.2}%
\definecolor{DDOrange}{rgb}{0.7, 0.23, 0.07}%
\colorlet{pdflinkcolor}{DOrange}
\colorlet{DGreen}{green!90!black}
\usepackage{showexpl}
\makeatletter\renewcommand*\SX@Info{}\makeatother
\usepackage{etoolbox}
\undef{\cs}\undef{\cmd}
\usepackage{ltxdockit}
\newcommand{\poeTR}[1]{\TR{\ttfamily\color{DOrange}#1}}
\definecolor{colKeys}{rgb}{0,0,0}
\definecolor{colIdentifier}{rgb}{0,0,0}
\colorlet{colComments}{green!60!black}
\definecolor{colString}{rgb}{0,0.5,0}
\newlength{\codeoverhang}
\setlength{\codeoverhang}{0.5\marginparwidth+\marginparsep}
\lstset{%
  language=[LaTeX]TeX, identifierstyle=\color{colIdentifier},
  keywordstyle=\color{colKeys},
  keywordstyle = [21]\color{DOrange},
  keywordstyle = [22]\color{DOrange},
  stringstyle=\color{colString},
  commentstyle=\color{colComments},
  alsoletter={12},
  float=hbp,
  basicstyle=\ttfamily\small,
  columns=flexible,
  tabsize=4,
  showspaces=false,
  showstringspaces=false,
  breaklines=true,
  breakautoindent=true,
  breakatwhitespace=true,
  captionpos=t,
  belowcaptionskip=0pt,
  abovecaptionskip=0pt,
  xleftmargin=1em,
  prebreak = {\raisebox{-0.5ex}[\ht\strutbox]{\kern0.5ex\large\Righttorque}},
  rulecolor=\color{black!20}, 
  texcsstyle = [20]\color{DDOrange},
  moretexcs = [20]{savebezier, savepath, psshowcurve, psintersect},
  explpreset={%
    pos=l, width=-99pt, hsep=5mm, overhang=\codeoverhang, varwidth,
    vsep=\bigskipamount, rframe={}}, extendedchars=true
}%
\lstdefinestyle{example}{explpreset={%
    escapechar=*, pos=l, width=-99pt, hsep=5mm, overhang=\codeoverhang,
    varwidth, vsep=\bigskipamount, rframe={}}}
\makeatletter
\providecommand\ON{%
  \gdef\lst@alloverstyle##1{\textcolor{black!50}{\strut##1}%
}}
\providecommand\OFF{\xdef\lst@alloverstyle##1{##1}}
\makeatother
\colorlet{sectioncolor}{DOrange}
\addtokomafont{sectioning}{\color{sectioncolor}}
\usepackage[automark,nouppercase]{scrpage2}
\pagestyle{scrheadings}
\clearscrheadings
\clearscrplain
\ohead{\pagemark}
\ihead{\headmark}
\ofoot[\pagemark]{}
\automark[subsection]{section}
\setheadsepline{.4pt}[\color{DOrange}]
\setheadwidth[0pt]{text}
\setfootwidth[0pt]{text}
\makeatletter
\patchcmd{\l@chapter}{1.5em}{2em}{}{}
\renewcommand*\l@section{\bprot@dottedtocline{1}{1.5em}{3.0em}}
\renewcommand*\l@subsection{\bprot@dottedtocline{2}{3.8em}{4.0em}}
\newrobustcmd*{\fnurl}[1][]{\hyper@normalise\ltd@fnurl{#1}}
\def\ltd@fnurl#1#2{\footnote{#1\hyper@linkurl{\Hurl{#2}}{#2}}}
\newrobustcmd*{\arxivurl}[1]{\href{http://arxiv.org/abs/#1}{arXiv:#1}}
\newrobustcmd*{\doiurl}[1]{\href{http://dx.doi.org/#1}{DOI:#1}}
\makeatother
\usepackage{csquotes}
\MakeAutoQuote{«}{»}
%^^A spot is used in ltxdockit.sty
\colorlet{spot}{sectioncolor}
%^^A Fonts definitions used in ltxdockit.sty
\renewcommand*{\verbatimfont}{\ttfamily}
\renewcommand*{\displayverbfont}{\ttfamily}
\renewcommand*{\marglistfont}{\spotcolor\sffamily\small}
\renewcommand*{\margnotefont}{\sffamily\small}
\renewcommand*{\optionlistfont}{\spotcolor\sffamily\displayverbfont}
\renewcommand*{\ltxsyntaxfont}{\ttfamily}
\renewcommand*{\ltxsyntaxlabelfont}{\spotcolor\displayverbfont}
\renewcommand*{\changelogfont}{\normalfont}
\renewcommand*{\changeloglabelfont}{\spotcolor\sffamily\bfseries}

\makeatletter
\renewenvironment*{ltxsyntax}
  {\list{}{%
     \setlength{\labelwidth}{\marglistwidth}%
     \setlength{\labelsep}{0pt}%
     \setlength{\leftmargin}{0pt}%
     \renewcommand*{\makelabel}[1]{%
       \hss\ltxsyntaxfont\ltxsyntaxlabelfont##1}}%
   \let\cmditem\PIT@cmditem}
  {\endlist}

\def\ltd@optionlist{%
  \let\optitem\PIT@optitem
  \let\valitem\PIT@valitem
  \let\choitem\PIT@choitem
  \let\boolitem\PIT@boolitem
  \let\intitem\PIT@intitem
  \let\numitem\PIT@numitem
  \let\psnumitem\PIT@psnumitem
  \let\pititem\ltd@item
  \let\typeitem\PIT@typeitem}

\def\PIT@typeitem#1#2{%
  \item[{#1}]%
  \label{prm:#1}%\docindexdef{#1=\nxLkeyword{#1}}%
  \begingroup\raggedright
  #2%
  \settowidth\@tempdimb{\prm{psstyle}}%
  \settowidth\@tempdimc{#1}%
  \@tempdimc=\dimexpr\@tempdimc+\labelsep-\labelwidth\relax
  \ifdim\@tempdimc>0pt%
    \@tempdima=\dimexpr\linewidth-\@tempdimb-\@tempdimc-1em\relax
  \else
    \@tempdima=\dimexpr\linewidth-\@tempdimb-1em\relax
  \fi
  \par\endgroup}

\def\PIT@option#1#2#3{%
  \item[#1]%
  \begingroup\raggedright
  \ltd@textverb{=}%
  \settowidth\@tempdimb{\ltd@textverb{=}}%
  \settowidth\@tempdimc{#1}%
  \@tempdimc=\dimexpr\labelwidth-\@tempdimc\relax
  \ifdim\@tempdimc<0pt
    \@tempdima=\dimexpr\linewidth-\@tempdimb+\@tempdimc-2em\relax
  \else
    \@tempdima=\dimexpr\linewidth-\@tempdimb-2em\relax
  \fi
  \ifblank{#3}
    {}
    {\settowidth\@tempdimb{default: #3}%
     \@tempdima=\dimexpr\@tempdima-\@tempdimb-2em\relax}%
  \parbox[t]{\@tempdima}{\raggedright #2}%
  \ifblank{#3}
    {}
    {\hfill default:~#3}%
  \par\endgroup
  \nobreak\vspace{\itemsep}}

\def\PIT@optitem{%
  \@ifstar
    {\boolfalse{@tempswa}\PIT@optitem@i}
    {\booltrue{@tempswa}\PIT@optitem@i}}
\newcommand*{\PIT@optitem@i}[3][]{%
  \ifbool{@tempswa}%
    {\label{prm:#2}\xdefLkeyword{#2}}%
    {\xLkeyword{#2}}%
  \ifblank{#1}
    {\PIT@option{#2}{#3}{}}
    {\PIT@option{#2}{#3}{\ltd@textverb{#1}}}}%

\def\PIT@valitem{%
  \@ifstar
    {\boolfalse{@tempswa}\PIT@valitem@i}
    {\booltrue{@tempswa}\PIT@valitem@i}}
\newcommand*{\PIT@valitem@i}[3][]{%
  \ifbool{@tempswa}%
    {\label{prm:#2}\xdefLkeyword{#2}}%
    {\xLkeyword{#2}}%
  \ifblank{#1}
    {\PIT@option{#2}{\prm{#3}}{}}
    {\PIT@option{#2}{\prm{#3}}{\ltd@textverb{#1}}}}%

\def\PIT@choitem{%
  \@ifstar
    {\boolfalse{@tempswa}\PIT@choitem@i}
    {\booltrue{@tempswa}\PIT@choitem@i}}
\newcommand*{\PIT@choitem@i}[3][]{%
  \ifbool{@tempswa}%
    {\label{prm:#2}\xdefLkeyword{#2}}%
    {\xLkeyword{#2}}%
  \ifblank{#1}
    {\PIT@option{#2}{\ltd@verblist{#3}}{}}
    {\PIT@option{#2}{\ltd@verblist{#3}}{\ltd@textverb{#1}}}}%

\def\PIT@boolitem{%
  \@ifstar
    {\boolfalse{@tempswa}\PIT@boolitem@i}
    {\booltrue{@tempswa}\PIT@boolitem@i}}
\newcommand*{\PIT@boolitem@i}[2][]{%
  \ifbool{@tempswa}%
    {\label{prm:#2}\xdefLkeyword{#2}}%
    {\xLkeyword{#2}}%
  \ifblank{#1}
    {\PIT@option{#2}{\ltd@verblist{true,false}}{}}
    {\PIT@option{#2}{\ltd@verblist{true,false}}{\ltd@textverb{#1}}}}%

\def\PIT@intitem{%
  \@ifstar
    {\boolfalse{@tempswa}\PIT@intitem@i}
    {\booltrue{@tempswa}\PIT@intitem@i}}
\newcommand*{\PIT@intitem@i}[2][]{%
  \ifbool{@tempswa}%
    {\label{prm:#2}\xdefLkeyword{#2}}%
    {\xLkeyword{#2}}%
  \ifblank{#1}
    {\PIT@option{#2}{\prm{int}}{}}
    {\PIT@option{#2}{\prm{int}}{\ltd@textverb{#1}}}}%

\def\PIT@numitem{%
  \@ifstar
    {\boolfalse{@tempswa}\PIT@numitem@i}
    {\booltrue{@tempswa}\PIT@numitem@i}}
\newcommand*{\PIT@numitem@i}[2][]{%
  \ifbool{@tempswa}%
    {\label{prm:#2}\xdefLkeyword{#2}}%
    {\xLkeyword{#2}}%
  \ifblank{#1}
    {\PIT@option{#2}{\prm{num}}{}}
    {\PIT@option{#2}{\prm{num}}{\ltd@textverb{#1}}}}%

\def\PIT@psnumitem{%
  \@ifstar
    {\boolfalse{@tempswa}\PIT@psnumitem@i}
    {\booltrue{@tempswa}\PIT@psnumitem@i}}
\newcommand*{\PIT@psnumitem@i}[2][]{%
  \ifbool{@tempswa}%
    {\label{prm:#2}\xdefLkeyword{#2}}%
    {\xLkeyword{#2}}%
  \ifblank{#1}
    {\PIT@option{#2}{\prm{psnum}}{}}
    {\PIT@option{#2}{\prm{psnum}}{\ltd@textverb{#1}}}}%

\def\ltd@csitem{%
  \@ifstar
    {\boolfalse{@tempswa}\ltd@csitem@i}
    {\booltrue{@tempswa}\ltd@csitem@i}}
\def\ltd@csitem@i#1{%
  \ifbool{@tempswa}
    {\ltd@item@ii{\textbackslash#1\hspace\marglistsep}{#1}}
    {\ltd@item@ii{\textbackslash#1\hspace\marglistsep}{}}}

\def\PIT@cmditem{%
  \@ifstar
    {\boolfalse{@tempswa}\PIT@cmditem@i}
    {\booltrue{@tempswa}\PIT@cmditem@i}}
\def\PIT@cmditem@i#1{%
  \ifbool{@tempswa}
    {\PIT@cmditem@ii{\textbackslash#1}{cs:#1}}
    {\PIT@cmditem@ii{\textbackslash#1}{}}}
\def\PIT@compitem{%
  \@ifstar
    {\boolfalse{@tempswa}\PIT@compitem@i}
    {\booltrue{@tempswa}\PIT@compitem@i}}
\def\PIT@compitem@i#1{%
  \ifbool{@tempswa}
    {\PIT@cmditem@ii{\textbackslash#1}{comp:#1}}
    {\PIT@cmditem@ii{\textbackslash#1}{}}}
\def\PIT@cmditem@ii#1#2{%
  \ltd@itemsave
  \ifhmode
    \itemsep-\topsep
  \else
    \ltd@itembreak
  \fi
  \item[#1]%
  \ltd@itemrest
  \ifblank{#2}{}{\label{#2}}%
  \begingroup
  \ltd@syntaxsetup
  \ltxsyntaxfont
  \let\@tempa\@empty
  \ltd@parseargs}

\def\ltd@csitem@ii#1#2{%
  \ltd@itemsave
  \ifhmode
    \itemsep-\topsep
  \else
    \ltd@itembreak
  \fi
  \item[#1]%
  \ltd@itemrest
  \ifblank{#2}{}{\label{cs:#2}}%
  \begingroup
  \ltd@syntaxsetup
  \ltxsyntaxfont
  \let\@tempa\@empty
  \ltd@parseargs}


\let\plainllap\llap
\newrobustcmd\macro@llap[1]{{\global\let\llap\plainllap
 \setbox0=\hbox\bgroup \macro@font\small\saved@macroname\egroup
 \ifdim\wd0>30mm
    \hbox to\z@ \bgroup\hss \hbox to30mm{\unhcopy0\hss}\egroup
    \edef\@tempa{\hskip\dimexpr\the\wd0-30mm}\global\everypar\expandafter{\the\expandafter\everypar
                                                                           \@tempa \global\everypar{}}%
 \else \llap{\unhbox0}\fi}}
 \AtBeginEnvironment{macro}{\let\llap\macro@llap}
\makeatother

\newcommand*{\PSstring}[1]{{\normalfont\small\ttfamily(#1)}}
\newcommand*{\PSarray}[1]{{\normalfont\small\ttfamily[#1]}}
\newcommand*{\PSname}[1]{{\normalfont\small\ttfamily/#1}}
\newcommand*{\PSproc}[1]{{\normalfont\small\ttfamily\textbraceleft #1\textbraceright}}
\newcommand*{\PSvar}[1]{{\normalfont\small\ttfamily #1}}
\newcommand*{\PSop}[1]{{\normalfont\small\ttfamily\color{DOrange}\hskip 3pt #1\hskip 3pt}}

\newcommand*{\compref}[1]{\ref{comp:#1}}
\newcommand*{\hyperpagedef}[1]{\textbf{\hyperpage{#1}}}
\newcommand*{\nodename}[1]{\emph{#1}}
\newcommand*{\param}[1]{\normalfont\texttt{#1}}
\newcommand*{\paramvalue}[1]{\texttt{#1}}
\newcommand*{\styleshape}[1]{\texttt{#1}}
\newcommand{\docindex}[1]{\sindex[doc]{#1|hyperpage}}
\newcommand{\docindexdef}[1]{\sindex[doc]{#1|hyperpagedef}}
\makeatletter
\def\Lcs{\@ifstar{\Lcs@nobm}{\Lcs@bm}}
\def\Lcs@nobm#1{\nxLcs{#1}\xLcs{#1}}
\def\Lcs@bm#1{\hyperref[cs:#1]{\Lcs@nobm{#1}}}
\def\xLcs#1{}
\def\nxLcs#1{\texttt{\textbackslash#1}}

\def\Lcomp{\@ifstar{\Lcomp@nobm}{\Lcomp@bm}}
\def\Lcomp@nobm#1{\nxLcomp{#1}\xLcomp{#1}}
\def\Lcomp@bm#1{\hyperref[comp:#1]{\Lcomp@nobm{#1}}}
\def\xLcomp#1{}
\def\nxLcomp#1{\texttt{\textbackslash#1}}

\def\LPack#1{\nxLPack{#1}}
\def\nxLPack#1{\texttt{#1}}

\def\Lenv{\@ifstar{\Lenv@nobm}{\Lenv@bm}}
\def\Lenv@nobm#1{\nxLenv{#1}\xLenv{#1}}
\def\Lenv@bm#1{\hyperref[env:#1]{\Lenv@nobm{#1}}}
\def\xLenv#1{}
\def\nxLenv#1{\texttt{#1}}
\let\orig@ltd@envitem\ltd@envitem
\def\ltd@envitem#1{\orig@ltd@envitem{#1}\label{env:#1}\xLenv{#1}}

\def\Lkeyword{\@ifstar{\Lkeyword@nobm}{\Lkeyword@bm}}
\def\Lkeyword@nobm#1{\nxLkeyword{#1}\xLkeyword{#1}}
\def\Lkeyword@bm#1{\hyperref[prm:#1]{\Lkeyword@nobm{#1}}}
\def\xLkeyword#1{}
\def\xdefLkeyword#1{}
\def\nxLkeyword#1{\texttt{#1}}

\def\xLoption#1{}
\def\Loption#1{\texttt{#1}\xLoption{#1}}
\def\nxLoption#1{\texttt{#1}}

\def\Lstyle{\@ifstar{\Lstyle@nobm}{\Lstyle@bm}}
\def\Lstyle@nobm#1{\nxLstyle{#1}\xLstyle{#1}}
\def\Lstyle@bm#1{\hyperref[sty:#1]{\Lstyle@nobm{#1}}}
\def\xLstyle#1{}
\def\xdefLstyle#1{}
\def\nxLstyle#1{\texttt{#1}}

\makeatother

\newcommand{\dipoledesc}[1]{%
  \xLdipole{#1}%
  \compitem{#1}[options](in)(out){label}%
}
\newcommand{\tripoledesc}[1]{%
  \xLtripole{#1}%
  \compitem{#1}[options](in)(center)(out){label}%
}

\newcommand{\fiberdipoledesc}[1]{%
  \xLfdipole{#1}%
  \compitem{#1}[options](in)(out){label}%
}

\newenvironment*{pssyntax}
  {\list{}{\small
     \setlength{\labelsep}{0pt}%
     \setlength{\leftmargin}{10pt}%
     \item[]}}
  {\endlist}

\newcommand{\psarglistfont}{\small}
\newenvironment*{psarglist}
  {\list{}{%
     \setlength{\labelwidth}{10pt}%
     \setlength{\labelsep}{0pt}%
     \setlength{\leftmargin}{0pt}%
     \setlength{\itemsep}{\parsep}%
     \setlength{\parsep}{0pt}%
     \renewcommand*{\makelabel}[1]{\hss\psarglistfont##1}}}
  {\endlist}

\makeatletter
\renewenvironment{theglossary}{\GlossaryParms \let\item\@idxitem \ignorespaces}{}
\makeatother
\def\psargitem#1{\item[#1]\hfill\par\nobreak}

\addtopsstyle{Fiber}{linecolor=DOrange,linewidth=1.5\pslinewidth}
\addtopsstyle{Beam}{linewidth=1.5\pslinewidth}
\EnableCrossrefs
\CodelineIndex
\OnlyDescription
\begin{document}
  \DocInput{pst-intersect.dtx}
\end{document}
%</driver>
% \fi
%
% \CheckSum{0}
%
% \CharacterTable
%  {Upper-case    \A\B\C\D\E\F\G\H\I\J\K\L\M\N\O\P\Q\R\S\T\U\V\W\X\Y\Z
%   Lower-case    \a\b\c\d\e\f\g\h\i\j\k\l\m\n\o\p\q\r\s\t\u\v\w\x\y\z
%   Digits        \0\1\2\3\4\5\6\7\8\9
%   Exclamation   \!     Double quote  \"     Hash (number) \#
%   Dollar        \$     Percent       \%     Ampersand     \&
%   Acute accent  \'     Left paren    \(     Right paren   \)
%   Asterisk      \*     Plus          \+     Comma         \,
%   Minus         \-     Point         \.     Solidus       \/
%   Colon         \:     Semicolon     \;     Less than     \<
%   Equals        \=     Greater than  \>     Question mark \?
%   Commercial at \@     Left bracket  \[     Backslash     \\
%   Right bracket \]     Circumflex    \^     Underscore    \_
%   Grave accent  \`     Left brace    \{     Vertical bar  \|
%   Right brace   \}     Tilde         \~}
%
% \GetFileInfo{pst-intersect.dtx}
%
% \newif\ifGERMAN  \GERMANfalse
% \newif\ifENGLISH \ENGLISHfalse
% \iflanguage{ngerman}{\GERMANtrue}{%
%   \iflanguage{german}{\GERMANtrue}{\GERMANfalse}}
% \iflanguage{english}{\ENGLISHtrue}{\ENGLISHfalse}
%
% \ifGERMAN
%   \newrefformat{chap}{Kap.~\ref{#1}}
%   \newrefformat{sec}{Kap.~\ref{#1}}
%   \newrefformat{l}{Zeile~\ref{#1}}
%   \newrefformat{ex}{Bsp.~\ref{#1}}
%   \newrefformat{tab}{Tab.~\ref{#1}}
%   \newrefformat{fig}{Abb.~\ref{#1}}
%   \newcommand{\PITindexMacro}{Makros}
%   \newcommand{\PITindexKeyword}{Parameter}
%   \newcommand{\PITindexEnv}{Umgebungen}
%   \newcommand{\PITindexPack}{Pakete}
%   \renewcommand*\lstlistingname{Bsp.}
% \fi
% \ifENGLISH
%   \newrefformat{chap}{Sec.~\ref{#1}}
%   \newrefformat{sec}{Sec.~\ref{#1}} 
%   \newrefformat{l}{Line~\ref{#1}}
%   \newrefformat{ex}{Ex.~\ref{#1}}
%   \newrefformat{tab}{Tab.~\ref{#1}}
%   \newrefformat{fig}{Fig.~\ref{#1}}
%   \newcommand{\PITindexMacro}{macros}
%   \newcommand{\PITindexKeyword}{parameters}
%   \newcommand{\PITindexEnv}{environments}
%   \newcommand{\PITindexPack}{packages}
%   \renewcommand*\lstlistingname{Ex.}
% \fi
%
% \makeatletter
% \renewcommand\maketitle{^^A
% \thispagestyle{empty}^^A
% \begin{titlepage}
% \begin{pspicture}(1.6in,0.685in)(10,21.7)
%   \psframe[fillstyle=solid,linecolor=lightgray,fillcolor=lightgray,linestyle=solid](0,-5.75)(21.5,10)
%   \psframe[fillstyle=solid,linecolor=Orange!85!Red,fillcolor=Orange!85!Red,linestyle=solid](0,10)(21.5,10.5)
%   \psframe[fillstyle=solid,linecolor=Orange!85!Red,fillcolor=Orange!85!Red,linestyle=solid](0,21.1)(21.5,21.2)
%   \rput[lb](3,22){\Huge\sffamily\color{Orange!65!Red}\psscalebox{2}{\textbf{PSTricks}}}
%   \rput[lb](3,14.1){\parbox{15cm}{\sffamily\RaggedRight\bfseries\huge\@title}}
%   \rput[lb](3,7.6){\parbox{13cm}{\sffamily\@date}}
%   \rput[lb](3,-2.6){\parbox[b]{17cm}{\sffamily\RaggedRight 
%     ~\hfill\makebox[7cm][l]{\ifGERMAN Paketautor:\fi\ifENGLISH Package author:\fi}\\
%     ~\hfill\makebox[7cm][l]{^^A
%       \bfseries\tabular[t]{@{}l@{}}\@author\endtabular}}}
%  \rput[C](11,4){\bgImage}
%  \end{pspicture}^^A
% \end{titlepage}}
% \makeatother
% 
% \ifGERMAN
%   \title{\texttt{pst-intersect}\\ Berechnen der Schnittpunkte beliebiger Kurven\\[0.5ex] \small \fileversion}
%   \hypersetup{pdftitle={Berechnen der Schnittpunkte beliebiger Kurven}}
% \fi
% \ifENGLISH
%   \title{\texttt{pst-optexp}\\ Intersecting arbitrary curves\\[0.5ex] \small \fileversion}
%   \hypersetup{pdftitle={Intersecting arbitrary curves}}
% \fi
% \author{Christoph Bersch}
% \date{\filedate}
% \def\bgImage{}
%
% \maketitle
% 
% \clearpage
% \tableofcontents
% \clearpage
% 
% \ifGERMAN
%   \chapter{Einführung}
% \fi
% \ifENGLISH
%   \chapter{Introduction}
% \fi
% 
% \ifGERMAN
%   \section{Über das Paket}
%
%   \LPack{pst-intersect} ist ein PSTricks-Paket zur
%   Berechnung der Schnittpunkte von Bézier-Kurven und beliebigen
%   Postscript-Pfaden.
% \fi 
% \ifENGLISH
%   \section{About the package}
%   \LPack{pst-intersect} is a PSTricks package to calculate
%   the intersections of Bezier curves and arbitrary Postscript paths.
% \fi
%
% \ifGERMAN
%   \section{Anforderungen}
%   \LPack{pst-intersect} benötigt aktuelle Versionen der
%   Pakete \LPack{pstricks}, \LPack{pst-node} und \LPack{pst-func}.
%
%   Alle PSTricks-Pakete machen regen Gebrauch von der Postscript-Sprache, so
%   dass der typische Arbeitsfluss \opt{latex}, \opt{dvips} und
%   ggf. \opt{ps2pdf} umfasst. Es gibt viele alternative Methoden um die
%   Dokumente zu
%   kompilieren.\fnurl{http://tug.org/PSTricks/main.cgi?file=pdf/pdfoutput}
% \fi
% \ifENGLISH
%   \section{Requirements}
%   \LPack{pst-intersect} requires recent versions of
%   \LPack{pstricks}, \LPack{pst-node}, and \LPack{pst-func}.
%
%   All PSTricks package rely heavily on the Postscript language so that the
%   typical workflow involves \opt{latex}, \opt{dvips}, and \opt{ps2pdf}. Of
%   course there are several alternative ways to compile your
%   documents.\fnurl{http://tug.org/PSTricks/main.cgi?file=pdf/pdfoutput} 
% \fi
%
% \ifGERMAN
%   \section{Verbreitung und Installation}
%   %Dieses Paket ist auf
%   %CTAN\fnurl{http://mirror.ctan.org/help/Catalogue/entries/pst-intersect.html}
%   %erhältlich.^^A und in \TeX Live and MiK\TeX{} enthalten.
% 
%   Das \LPack{pst-intersect}-Paket umfasst die zwei Hauptdateien
%   \texttt{pst-intersect.ins} und \texttt{pst-intersect.dtx}. Durch Aufrufen
%   von \texttt{tex pst-intersect.ins} werden die drei folgenden
%   Dateien erzeugt:
%   \begin{itemize}
%   \item \texttt{pst-intersect.pro}: die Postscript Prologdatei
%   \item \texttt{pst-intersect.sty}: die \LaTeX-Stildatei
%   \item \texttt{pst-intersect.tex}: die \TeX-Datei
%   \end{itemize}
%   Speichern Sie diese Dateien in einem Verzeichnis der Teil Ihres
%   lokalen \TeX-Baums ist.
% 
%   Vergessen Sie nicht \texttt{texhash} aufzurufen um den Baum zu
%   aktualisieren. MiK\TeX{}-Benutzer müssen die Dateinamen-Datenbank
%   (FNDB) aktualisieren.
% 
%   Detailliertere Information finden Sie in der Dokumentation Ihrer
%   \LaTeX-Distribution über die Installation in den lokalen
%   \TeX{}-Baum.
% \fi
% \ifENGLISH
%   \section{Distribution and installation}
%   This package is available on
%   CTAN\fnurl{http://mirror.ctan.org/help/Catalogue/entries/pst-intersect.html}.
%   ^^A and is included in \TeX Live and MiK\TeX.
% 
%   The \LPack{pst-intersect} package consists of the two main files
%   \texttt{pst-intersect.ins} and \texttt{pst-intersect.dtx}. By running \texttt{tex
%   pst-intersect.ins} the following derived files are generated:
%   \begin{itemize}
%   \item \texttt{pst-intersect.pro}: the Postscript prolog file
%   \item \texttt{pst-intersect.sty}: the \LaTeX{} style file
%   \item \texttt{pst-intersect.tex}: the \TeX{} file
%   \end{itemize}
%   Save the files in a directory which is part of your local \TeX{} tree.
% 
%   Do not forget to run \texttt{texhash} to update this tree. For MiK\TeX{}
%   users, do not forget to update the file name database (FNDB).
% 
%   For more detailed information see the documentation of your personal
%   \LaTeX{} distribution on installing packages to your local \TeX{}
%   system.
% \fi
%
% \ifGERMAN\section{Lizenz}\fi
% \ifENGLISH\section{License}\fi
% \ifGERMAN
% Es wird die Erlaubnis gewährt, dieses Dokument zu kopieren, zu verteilen
% und\slash oder zu modifizieren, unter den Bestimmungen der \LaTeX{} Project
% Public License, Version
% 1.3c.\fnurl{http://www.latex-project.org/lppl.txt}. Dieses
% Paket wird vom Autor betreut (author-maintained).
% \fi
% \ifENGLISH
% Permission is granted
% to copy, distribute and\slash or modify this software under the terms of the
% \LaTeX{} Project Public License, version
% 1.3c.\fnurl{http://www.latex-project.org/lppl.txt} This
% package is author-maintained.
% \fi
%
% \ifGERMAN
%   \section{Danksagung}
% \fi
% \ifENGLISH
%   \section{Acknowledgements}
% \fi
% \ifGERMAN
% Ich danke Marco Cecchetti, dessen
% \opt{lib2geom}-Bibliothek\fnurl{http://lib2geom.sourceforge.net/}
% mir als Vorlage für einen Großteil des Postscript-Kodes für den
% Bézier-Clipping-Algorithmus diente. Außerdem gilt mein Dank William
% A. Casselman, für seine Erlaubnis, den Quicksort-Kode und den Kode zur
% Berechung der konvexen Hüllen aus seinem Buch «Mathematical
% Illustration» verwenden zu
% dürfen\fnurl{http://www.math.ubc.ca/~cass/graphics/text/www/}. Der
% Dokumentationsstil ist eine Mischung aus der \opt{pst-doc} Klasse
% (Herbert Voß) und dem \opt{ltxdockit} Paket für die \opt{biblatex}
% Dokumentation (Philipp Lehmann).
% \fi
% \ifENGLISH
% I thank Marco Cecchetti, for his
% \opt{lib2geom}-library\fnurl{http://lib2geom.sourceforge.net/} from
% which I derived great parts of the Postscript code for the Bézier
% clipping algorithm. Also I want to thank William A. Casselman for the
% Postscript code of the quicksort procedure and the procedure for
% calculating the convex hull from his book «Mathematical
% Illustration»\fnurl{http://www.math.ubc.ca/~cass/graphics/text/www/},
% and the permission to use it. The documentation style is a mixture of
% the \opt{pst-doc} class (Herbert Voß) and the \opt{ltxdockit} package
% for the \opt{biblatex} documentation (Philipp Lehman).
% \fi
%
%
% \ifGERMAN
% \chapter{Benutzung}
% \fi
% \ifENGLISH
% \chapter{Usage}
% \fi
%
% \ifGERMAN
% Das \LPack{pst-intersect}-Paket kann Schnittpunkte von beliebigen
% Postscript-Pfaden berechnen. Diese setzen sich nur aus drei primitiven
% Operation zusammen: Linien (\opt{lineto}), Bézier-Kurven dritter
% Ordnung (\opt{curveto}) und Sprüngen (\opt{moveto}). Speziellere
% Konstruktionen, wie z.B. Kreise werden intern zu
% \opt{curveto}-Anweisungen umgewandelt. Über diese Kommandos hinaus
% kann \LPack{pst-intersect} auch Bézier-Kurven beliebiger Ordnung
% verwenden. Das diese keine primitiven Postscript-Pfadelemente
% darstellen, müssen sie gesondert behandelt werden.
%
% Der allgemeine Arbeitsablauf besteht darin eine oder mehrere Kurven
% oder Pfade zu speichern, und danach die Schnittpunkte zu
% berechnen. Anschließend können die Schnittpunkte als normale
% PSTricks-Knoten verwendet werden, oder Abschnitte der Kurven und Pfade
% nachgezogen werden (z.B. zwischen zwei Schnittpunkten). 
% \fi
%
% \ifGERMAN
% \section{Speichern von Pfaden und Kurven}
% \fi
% \ifENGLISH
% \section{Saving paths and curves}
% \fi
%
% \ifGERMAN
% 
% \fi
%
% \begin{ltxsyntax}
%   \cmditem{savepath}[options]{curvename}{commands} 
%   
%   \ifGERMAN
%   Speichert den gesamten Pfad, der durch \prm{commands} erstellt wird,
%   unter Verwendung des Namens \prm{curvename}. Das Makro funktioniert
%   genauso wie \cs{pscustom}, und kann daher auch nur die darin
%   erlaubten Kommandos verarbeiten.
%
%   In den Standardeinstellungen wird der entsprechende Pfad auch gleich
%   gezeichnet, was mit \prm{options} beeinflusst werden kann. Mit
%   \opt{linestyle=none} wird das unterbunden.
%   \fi
% \end{ltxsyntax}
%\iffalse
%<*ignore>
%\fi
\begin{LTXexample}
\begin{pspicture}(3,2)
  \savepath[linecolor=DOrange]{MyPath}{%
    \pscurve(0,2)(0,0.5)(3,1)
  }%
\end{pspicture}
\end{LTXexample}
%\iffalse
%</ignore>
%\fi
%
% \begin{ltxsyntax}
%   \cmditem{savebezier}[options]{curvename}($X_0$)\ldots(\prm{$X_n$})
%
%   \ifGERMAN 
%   Die Postscript-Sprache unterstützt nur Bézier-Kurven dritter
%   Ordnung. Mit dem Makro \Lcs{savebezier} können Bézier-Kurven
%   beliebiger Ordnung definiert werden. Die angegebenen Knoten sind die
%   Kontrollpunkte der Kurve, für eine Kurve $n$-ter Ordnung werden
%   $(n+1)$ Kontrollpunkte benötigt. Die Darstellung der Kurve erfolgt
%   mit dem Makro \cs{psBezier} aus dem \LPack{pst-func}-Paket, welches
%   nur Kurven bis neunter Ordnung darstellen kann.
%   \fi
% \end{ltxsyntax}
%\iffalse
%<*ignore>
%\fi
\begin{LTXexample}
\begin{pspicture}(3,2)
  \savebezier[showpoints]{MyBez}(0,0)(0,1)(1,2)(3,2)(1,0)(3,0)
\end{pspicture}
\end{LTXexample}
%\iffalse
%</ignore>
%\fi
%
% \ifGERMAN
% \section{Schnittpunkte berechnen}
% \fi
% \ifENGLISH
% \section{Calculating intersections}
% \fi
%
% \begin{ltxsyntax}
%   \cmditem{psintersect}{curveA}{curveB}
% \end{ltxsyntax}
% 
% \ifGERMAN
% Nachdem Sie nun Pfade oder Kurven gespeichert haben, können Sie deren
% Schnittpunkte berechnen. Das geschieht mit dem Makro
% \Lcs{psintersect}. Dieses benötigt als Argumente zwei Namen von Pfaden
% oder Kurven (Das Argument \prm{curvename} der beiden \cs{save*}
% Makros).
% \fi
%
%\iffalse
%<*ignore>
%\fi
\begin{LTXexample}
\begin{pspicture}(3,2)
  \savepath[linecolor=DOrange]{MyPath}{\pscurve(0,2)(0,0.5)(3,1)}
  \savebezier{MyBez}(0,0)(0,1)(1,2)(3,2)(1,0)(3,0)
  \psintersect[showpoints]{MyPath}{MyBez}
\end{pspicture}
\end{LTXexample}
%\iffalse
%</ignore>
%\fi
%
% \ifGERMAN
% Der PSTricks-Parameter \opt{showpoints} steuert dabei, ob die Schnittpunkte angezeigt werden.
% \fi
%
% \begin{optionlist}
% \valitem[@tmp]{name}{string}
% \ifGERMAN
% Die berechneten Schnittpunkte können unter einem hier angegebenen
% Namen gespeichert und zu einem späteren Zeitpunkt verwendet werden
% (siehe \prettyref{sec:psshowcurve}). 
% \fi
%
% \boolitem[true]{saveintersections}
% \ifGERMAN
% Ist dieser Schalter gesetzt, dann werden die Schnittpunkte als
% PSTricks-Knoten unter den Namen \prm{name}1, \prm{name}2 \ldots
% gespeichert. Die Nummerierung erfolgt aufsteigend nach dem Wert der
% $x$-Koordinate.
% \fi
%
%\iffalse
%<*ignore>
%\fi
\begin{LTXexample}
\begin{pspicture}(5,5)
  \savebezier[linecolor=DOrange]{A}%
             (0,0)(0,5)(5,5)(5,1)(1,1.5)
  \savebezier{B}(0,5)(0,0)(5,0)(5,5)(0,2)
  \psintersect[name=C, showpoints]{A}{B}
  \uput[150](C1){1}
  \uput[85](C2){2}
  \uput[90](C3){3}
  \uput[-20](C4){4}
\end{pspicture}
\end{LTXexample}
%\iffalse
%</ignore>
%\fi
% \end{optionlist}
%
% \ifGERMAN
% \section{Darstellung gespeicherter Pfade}
% \fi
% \ifENGLISH
% \section{Visualization of saved paths}
% \fi
%
% \begin{ltxsyntax}
%   \cmditem{psshowcurve}[options]{curvename}
%
% \ifGERMAN
% Gespeicherte Pfade und Kurven können mit diesem Makro nachträglich gezeichnet werden.
% \fi
% \end{ltxsyntax}
%
%\iffalse
%<*ignore>
%\fi
\begin{LTXexample}
\begin{pspicture}(2,2)
  \savepath{Circle}{\pscircle(1,1){1}}
  \psshowcurve[linestyle=dashed, linecolor=green]{Circle}
\end{pspicture}
\end{LTXexample}
%\iffalse
%</ignore>
%\fi
%
% \begin{optionlist}
% \numitem{tstart}
% \numitem{tstop}
%
% \ifGERMAN Unter Verwendung dieser beiden Parameter können auch
% Abschnitte von Pfaden und Kurven gezeichnet werden. Bei Bézier-Kurven
% ist der Parameterbereich $[0, 1]$, wobei $0$ dem Anfang der Kurve,
% also dem ersten bei \Lcs{savebezier} angegebenen Knoten entspricht.
% \fi
%
%\iffalse
%<*ignore>
%\fi
\begin{LTXexample}
\begin{pspicture}(5,5)
  \psset{showpoints}
  \savebezier{B}(0,5)(0,0)(5,0)(5,5)(0,2)
  \psshowcurve[linestyle=dashed, linecolor=blue!50,
               tstart=0, tstop=0.5]{B}
\end{pspicture}
\end{LTXexample}
%\iffalse
%</ignore>
%\fi
%
% \medskip
% \ifGERMAN 
% Pfaden können aus mehr als einem Abschnitt bestehen, der Bereich ist
% also $[0, n]$, wobei $n$ die Anzahl der Pfadabschnitte ist. Dabei ist
% zu beachten, dass z.B. \cs{pscurve}-Pfade oder auch Kreise und
% Kreisbögen aus mehreren Abschnitten bestehen.
% \fi
%
%\iffalse
%<*ignore>
%\fi
\begin{LTXexample}
\begin{pspicture}(2,2)
  \savepath[linestyle=none]{Circle}{\pscircle(1,1){1}}
  \psshowcurve[tstart=0, tstop=1, linecolor=green]{Circle}
  \psshowcurve[tstart=2, tstop=3, linecolor=red]{Circle}
  \psshowcurve[tstart=1.25, tstop=1.75, linecolor=blue]{Circle}
\end{pspicture}
\end{LTXexample}
%\iffalse
%</ignore>
%\fi
%
% \end{optionlist}
%
% \ifGERMAN
% Beachten Sie, dass in der Version 0.1 die Reihenfolge von
% \Lkeyword{tstart} und \Lkeyword{tstop} noch keine Rolle spielt. Für
% kommende Versionen ist geplant, dass bei \Lkeyword{tstart}
% \textgreater{} \Lkeyword{tstop} die Pfadrichtung umgekehrt wird. Zur
% Zeit werden auch noch keine Pfeile für \Lcs{psshowcurve} unterstützt.
% \fi
%
% \ifGERMAN
% \section{Darstellung gespeicherter Schnitte}
% \fi
% \ifENGLISH
% \section{Visualization of saved intersections}
% \fi
%
% \begin{ltxsyntax}
% \cmditem{psshowcurve}[options]{intersection}{curvename}
% \end{ltxsyntax}
% 
% \begin{optionlist}
% \numitem{istart}
% \numitem{istop}
% \end{optionlist}
% 
% \ifGERMAN
%
% \fi
%\iffalse
%<*ignore>
%\fi
\begin{LTXexample}
\begin{pspicture}(5.2,5.2)
\savebezier[linewidth=0.5\pslinewidth, linestyle=dashed]{A}(0,0)(0,5)(5,2)(5,5)
\savebezier[linewidth=0.5\pslinewidth, linestyle=dashed]{B}(0,2.5)(2.5,2.5)(4.5, 3)(2,4)
\psintersect[linecolor=green!70!black, name=C]{A}{B}
\psshowcurve[linecolor=red, istart=1, istop=2]{C}{A}
\psshowcurve[linecolor=blue, istart=1, istop=2]{C}{B}
\end{pspicture}
\end{LTXexample}
%\iffalse
%</ignore>
%\fi
%
% \ifGERMAN
% \chapter{Beispiele}
% \fi
% \ifENGLISH
% \chapter{Examples}
% \fi
%\iffalse
%<*ignore>
%\fi
\begin{LTXexample}
\begin{pspicture}(5,5)
  \savebezier{A}(0,0)(0,5)(5,5)(5,1)(1,1.5)
  \multido{\i=100+-20,\r=1+-0.2}{5}{%
    \psshowcurve[linecolor=red!\i, tstop=\r, arrows=-|, showpoints]{A}
  }%
\end{pspicture}
\end{LTXexample}
%\iffalse
%</ignore>
%\fi 
%
% \appendix
%
% \ifGERMAN
% \chapter{Versionsgeschichte}
%
% Diese Versionsgeschichte ist eine Liste von Änderungen, die für den Nutzer des
% Pakets von Bedeutung sind. Änderungen, die eher technischer Natur sind und für
% den Nutzer des Pakets nicht relevant sind und das Verhalten des Pakets nicht
% ändern, werden nicht aufgeführt. Wenn ein Eintrag der Versionsgeschichte ein
% Feature als \emph{improved} oder \emph{extended} bekannt gibt, so bedeutet
% dies, dass eine Modifikation die Syntax und das Verhalten des Pakets nicht
% beeinflusst, oder das es für ältere Versionen kompatibel ist. Einträge, die
% als \emph{deprecated}, \emph{modified}, \emph{renamed}, oder \emph{removed}
% deklariert sind, verlangen besondere Aufmerksamkeit. Diese bedeuten, dass eine
% Modifikation Änderungen in existierenden Dokumenten mit sich ziehen kann. 
% \fi
% \ifENGLISH
% \chapter{Revision history}
%
% This revision history is a list of changes relevant to users of this
% package. Changes of a more technical nature which do not affect the
% user interface or the behavior of the package are not included in the
% list. If an entry in the revision history states that a feature has
% been \emph{improved} or \emph{extended}, this indicates a modification
% which either does not affect the syntax and behavior of the package or
% is syntactically backwards compatible (such as the addition of an
% optional argument to an existing command). Entries stating that a
% feature has been \emph{deprecated}, \emph{modified}, \emph{fixed},
% \emph{renamed}, or \emph{removed} demand attention. They indicate a
% modification which may require changes to existing documents.
% \fi
%
% \begin{changelog}
%\patchcmd{\release}{\setlength{\itemsep}{0pt}}{\setlength{\itemsep}{0pt}\setlength{\parsep}{0pt}}{}{}
%   \begin{release}{1.0}{2014-xx-xx}
%   \item First CTAN version
%   \end{release}
% \end{changelog}
%
% \StopEventually{}
%
%   \begin{otherlanguage}{english}
%    \printindex[idx]
%  \end{otherlanguage}
%
% \chapter{The \LaTeX\ wrapper}
%<*stylefile>
%    \begin{macrocode}
\NeedsTeXFormat{LaTeX2e}[1999/12/01]
\ProvidesPackage{pst-intersect}%
   [2014/02/06 v0.1alpha package wrapper for pst-intersect.tex]
\RequirePackage{pstricks}
\RequirePackage{pst-xkey}
\RequirePackage{pst-node}
\RequirePackage{multido}
\RequirePackage{pst-func}
% \iffalse meta-comment
%
% Copyright (C) 2014 by Christoph Bersch <usenet@bersch.net>
%
% This work may be distributed and/or modified under the
% conditions of the LaTeX Project Public License, either version 1.3c
% of this license or (at your option) any later version.
% The latest version of this license is in
%   http://www.latex-project.org/lppl.txt
% and version 1.3c or later is part of all distributions of LaTeX
% version 2008/05/04 or later.
% \fi
%
% \iffalse
%<*driver>
\ProvidesFile{pst-intersect.dtx}
%</driver>
%<stylefile>\NeedsTeXFormat{LaTeX2e}[1999/12/01]
%<stylefile>\ProvidesPackage{pst-intersect}
%<*stylefile>
    [2014/02/18 v0.1alpha package wrapper for pst-intersect.tex]
%</stylefile>
%
%<*driver>
\documentclass[a4paper, DIV=9, oneside, toc=index, parskip=half-]{scrreprt}
\usepackage{doc}
\setcounter{IndexColumns}{2}
\usepackage[utf8]{inputenc} 
\usepackage[T1]{fontenc}
\usepackage{lmodern} 
\usepackage{amsmath, marvosym} 
\usepackage{bera}
\providecommand*\mainlang{}
\usepackage[ngerman, english,\mainlang]{babel}
\usepackage{prettyref}
\usepackage[dvipsnames,x11names,svgnames]{xcolor}
\usepackage{array,booktabs,paralist,tabularx}
\usepackage{ragged2e, calc}
\usepackage{nicefrac, multido}
\usepackage{pst-intersect}
\usepackage{hypdoc}
\hypersetup{%
  colorlinks=true, 
  urlcolor=DOrange, 
  linkcolor=pdflinkcolor, 
  breaklinks,
  linktocpage=true} 
\usepackage{breakurl}
\definecolor{DOrange}{rgb}{1,.4,.2}%
\definecolor{DDOrange}{rgb}{0.7, 0.23, 0.07}%
\colorlet{pdflinkcolor}{DOrange}
\colorlet{DGreen}{green!90!black}
\usepackage{showexpl}
\makeatletter\renewcommand*\SX@Info{}\makeatother
\usepackage{etoolbox}
\undef{\cs}\undef{\cmd}
\usepackage{ltxdockit}
\newcommand{\poeTR}[1]{\TR{\ttfamily\color{DOrange}#1}}
\definecolor{colKeys}{rgb}{0,0,0}
\definecolor{colIdentifier}{rgb}{0,0,0}
\colorlet{colComments}{green!60!black}
\definecolor{colString}{rgb}{0,0.5,0}
\newlength{\codeoverhang}
\setlength{\codeoverhang}{0.5\marginparwidth+\marginparsep}
\lstset{%
  language=[LaTeX]TeX, identifierstyle=\color{colIdentifier},
  keywordstyle=\color{colKeys},
  keywordstyle = [21]\color{DOrange},
  keywordstyle = [22]\color{DOrange},
  stringstyle=\color{colString},
  commentstyle=\color{colComments},
  alsoletter={12},
  float=hbp,
  basicstyle=\ttfamily\small,
  columns=flexible,
  tabsize=4,
  showspaces=false,
  showstringspaces=false,
  breaklines=true,
  breakautoindent=true,
  breakatwhitespace=true,
  captionpos=t,
  belowcaptionskip=0pt,
  abovecaptionskip=0pt,
  xleftmargin=1em,
  prebreak = {\raisebox{-0.5ex}[\ht\strutbox]{\kern0.5ex\large\Righttorque}},
  rulecolor=\color{black!20}, 
  texcsstyle = [20]\color{DDOrange},
  moretexcs = [20]{savebezier, savepath, psshowcurve, psintersect},
  explpreset={%
    pos=l, width=-99pt, hsep=5mm, overhang=\codeoverhang, varwidth,
    vsep=\bigskipamount, rframe={}}, extendedchars=true
}%
\lstdefinestyle{example}{explpreset={%
    escapechar=*, pos=l, width=-99pt, hsep=5mm, overhang=\codeoverhang,
    varwidth, vsep=\bigskipamount, rframe={}}}
\makeatletter
\providecommand\ON{%
  \gdef\lst@alloverstyle##1{\textcolor{black!50}{\strut##1}%
}}
\providecommand\OFF{\xdef\lst@alloverstyle##1{##1}}
\makeatother
\colorlet{sectioncolor}{DOrange}
\addtokomafont{sectioning}{\color{sectioncolor}}
\usepackage[automark,nouppercase]{scrpage2}
\pagestyle{scrheadings}
\clearscrheadings
\clearscrplain
\ohead{\pagemark}
\ihead{\headmark}
\ofoot[\pagemark]{}
\automark[subsection]{section}
\setheadsepline{.4pt}[\color{DOrange}]
\setheadwidth[0pt]{text}
\setfootwidth[0pt]{text}
\makeatletter
\patchcmd{\l@chapter}{1.5em}{2em}{}{}
\renewcommand*\l@section{\bprot@dottedtocline{1}{1.5em}{3.0em}}
\renewcommand*\l@subsection{\bprot@dottedtocline{2}{3.8em}{4.0em}}
\newrobustcmd*{\fnurl}[1][]{\hyper@normalise\ltd@fnurl{#1}}
\def\ltd@fnurl#1#2{\footnote{#1\hyper@linkurl{\Hurl{#2}}{#2}}}
\newrobustcmd*{\arxivurl}[1]{\href{http://arxiv.org/abs/#1}{arXiv:#1}}
\newrobustcmd*{\doiurl}[1]{\href{http://dx.doi.org/#1}{DOI:#1}}
\makeatother
\usepackage{csquotes}
\MakeAutoQuote{«}{»}
%^^A spot is used in ltxdockit.sty
\colorlet{spot}{sectioncolor}
%^^A Fonts definitions used in ltxdockit.sty
\renewcommand*{\verbatimfont}{\ttfamily}
\renewcommand*{\displayverbfont}{\ttfamily}
\renewcommand*{\marglistfont}{\spotcolor\sffamily\small}
\renewcommand*{\margnotefont}{\sffamily\small}
\renewcommand*{\optionlistfont}{\spotcolor\sffamily\displayverbfont}
\renewcommand*{\ltxsyntaxfont}{\ttfamily}
\renewcommand*{\ltxsyntaxlabelfont}{\spotcolor\displayverbfont}
\renewcommand*{\changelogfont}{\normalfont}
\renewcommand*{\changeloglabelfont}{\spotcolor\sffamily\bfseries}

\makeatletter
\renewenvironment*{ltxsyntax}
  {\list{}{%
     \setlength{\labelwidth}{\marglistwidth}%
     \setlength{\labelsep}{0pt}%
     \setlength{\leftmargin}{0pt}%
     \renewcommand*{\makelabel}[1]{%
       \hss\ltxsyntaxfont\ltxsyntaxlabelfont##1}}%
   \let\cmditem\PIT@cmditem}
  {\endlist}

\def\ltd@optionlist{%
  \let\optitem\PIT@optitem
  \let\valitem\PIT@valitem
  \let\choitem\PIT@choitem
  \let\boolitem\PIT@boolitem
  \let\intitem\PIT@intitem
  \let\numitem\PIT@numitem
  \let\psnumitem\PIT@psnumitem
  \let\pititem\ltd@item
  \let\typeitem\PIT@typeitem}

\def\PIT@typeitem#1#2{%
  \item[{#1}]%
  \label{prm:#1}%\docindexdef{#1=\nxLkeyword{#1}}%
  \begingroup\raggedright
  #2%
  \settowidth\@tempdimb{\prm{psstyle}}%
  \settowidth\@tempdimc{#1}%
  \@tempdimc=\dimexpr\@tempdimc+\labelsep-\labelwidth\relax
  \ifdim\@tempdimc>0pt%
    \@tempdima=\dimexpr\linewidth-\@tempdimb-\@tempdimc-1em\relax
  \else
    \@tempdima=\dimexpr\linewidth-\@tempdimb-1em\relax
  \fi
  \par\endgroup}

\def\PIT@option#1#2#3{%
  \item[#1]%
  \begingroup\raggedright
  \ltd@textverb{=}%
  \settowidth\@tempdimb{\ltd@textverb{=}}%
  \settowidth\@tempdimc{#1}%
  \@tempdimc=\dimexpr\labelwidth-\@tempdimc\relax
  \ifdim\@tempdimc<0pt
    \@tempdima=\dimexpr\linewidth-\@tempdimb+\@tempdimc-2em\relax
  \else
    \@tempdima=\dimexpr\linewidth-\@tempdimb-2em\relax
  \fi
  \ifblank{#3}
    {}
    {\settowidth\@tempdimb{default: #3}%
     \@tempdima=\dimexpr\@tempdima-\@tempdimb-2em\relax}%
  \parbox[t]{\@tempdima}{\raggedright #2}%
  \ifblank{#3}
    {}
    {\hfill default:~#3}%
  \par\endgroup
  \nobreak\vspace{\itemsep}}

\def\PIT@optitem{%
  \@ifstar
    {\boolfalse{@tempswa}\PIT@optitem@i}
    {\booltrue{@tempswa}\PIT@optitem@i}}
\newcommand*{\PIT@optitem@i}[3][]{%
  \ifbool{@tempswa}%
    {\label{prm:#2}\xdefLkeyword{#2}}%
    {\xLkeyword{#2}}%
  \ifblank{#1}
    {\PIT@option{#2}{#3}{}}
    {\PIT@option{#2}{#3}{\ltd@textverb{#1}}}}%

\def\PIT@valitem{%
  \@ifstar
    {\boolfalse{@tempswa}\PIT@valitem@i}
    {\booltrue{@tempswa}\PIT@valitem@i}}
\newcommand*{\PIT@valitem@i}[3][]{%
  \ifbool{@tempswa}%
    {\label{prm:#2}\xdefLkeyword{#2}}%
    {\xLkeyword{#2}}%
  \ifblank{#1}
    {\PIT@option{#2}{\prm{#3}}{}}
    {\PIT@option{#2}{\prm{#3}}{\ltd@textverb{#1}}}}%

\def\PIT@choitem{%
  \@ifstar
    {\boolfalse{@tempswa}\PIT@choitem@i}
    {\booltrue{@tempswa}\PIT@choitem@i}}
\newcommand*{\PIT@choitem@i}[3][]{%
  \ifbool{@tempswa}%
    {\label{prm:#2}\xdefLkeyword{#2}}%
    {\xLkeyword{#2}}%
  \ifblank{#1}
    {\PIT@option{#2}{\ltd@verblist{#3}}{}}
    {\PIT@option{#2}{\ltd@verblist{#3}}{\ltd@textverb{#1}}}}%

\def\PIT@boolitem{%
  \@ifstar
    {\boolfalse{@tempswa}\PIT@boolitem@i}
    {\booltrue{@tempswa}\PIT@boolitem@i}}
\newcommand*{\PIT@boolitem@i}[2][]{%
  \ifbool{@tempswa}%
    {\label{prm:#2}\xdefLkeyword{#2}}%
    {\xLkeyword{#2}}%
  \ifblank{#1}
    {\PIT@option{#2}{\ltd@verblist{true,false}}{}}
    {\PIT@option{#2}{\ltd@verblist{true,false}}{\ltd@textverb{#1}}}}%

\def\PIT@intitem{%
  \@ifstar
    {\boolfalse{@tempswa}\PIT@intitem@i}
    {\booltrue{@tempswa}\PIT@intitem@i}}
\newcommand*{\PIT@intitem@i}[2][]{%
  \ifbool{@tempswa}%
    {\label{prm:#2}\xdefLkeyword{#2}}%
    {\xLkeyword{#2}}%
  \ifblank{#1}
    {\PIT@option{#2}{\prm{int}}{}}
    {\PIT@option{#2}{\prm{int}}{\ltd@textverb{#1}}}}%

\def\PIT@numitem{%
  \@ifstar
    {\boolfalse{@tempswa}\PIT@numitem@i}
    {\booltrue{@tempswa}\PIT@numitem@i}}
\newcommand*{\PIT@numitem@i}[2][]{%
  \ifbool{@tempswa}%
    {\label{prm:#2}\xdefLkeyword{#2}}%
    {\xLkeyword{#2}}%
  \ifblank{#1}
    {\PIT@option{#2}{\prm{num}}{}}
    {\PIT@option{#2}{\prm{num}}{\ltd@textverb{#1}}}}%

\def\PIT@psnumitem{%
  \@ifstar
    {\boolfalse{@tempswa}\PIT@psnumitem@i}
    {\booltrue{@tempswa}\PIT@psnumitem@i}}
\newcommand*{\PIT@psnumitem@i}[2][]{%
  \ifbool{@tempswa}%
    {\label{prm:#2}\xdefLkeyword{#2}}%
    {\xLkeyword{#2}}%
  \ifblank{#1}
    {\PIT@option{#2}{\prm{psnum}}{}}
    {\PIT@option{#2}{\prm{psnum}}{\ltd@textverb{#1}}}}%

\def\ltd@csitem{%
  \@ifstar
    {\boolfalse{@tempswa}\ltd@csitem@i}
    {\booltrue{@tempswa}\ltd@csitem@i}}
\def\ltd@csitem@i#1{%
  \ifbool{@tempswa}
    {\ltd@item@ii{\textbackslash#1\hspace\marglistsep}{#1}}
    {\ltd@item@ii{\textbackslash#1\hspace\marglistsep}{}}}

\def\PIT@cmditem{%
  \@ifstar
    {\boolfalse{@tempswa}\PIT@cmditem@i}
    {\booltrue{@tempswa}\PIT@cmditem@i}}
\def\PIT@cmditem@i#1{%
  \ifbool{@tempswa}
    {\PIT@cmditem@ii{\textbackslash#1}{cs:#1}}
    {\PIT@cmditem@ii{\textbackslash#1}{}}}
\def\PIT@compitem{%
  \@ifstar
    {\boolfalse{@tempswa}\PIT@compitem@i}
    {\booltrue{@tempswa}\PIT@compitem@i}}
\def\PIT@compitem@i#1{%
  \ifbool{@tempswa}
    {\PIT@cmditem@ii{\textbackslash#1}{comp:#1}}
    {\PIT@cmditem@ii{\textbackslash#1}{}}}
\def\PIT@cmditem@ii#1#2{%
  \ltd@itemsave
  \ifhmode
    \itemsep-\topsep
  \else
    \ltd@itembreak
  \fi
  \item[#1]%
  \ltd@itemrest
  \ifblank{#2}{}{\label{#2}}%
  \begingroup
  \ltd@syntaxsetup
  \ltxsyntaxfont
  \let\@tempa\@empty
  \ltd@parseargs}

\def\ltd@csitem@ii#1#2{%
  \ltd@itemsave
  \ifhmode
    \itemsep-\topsep
  \else
    \ltd@itembreak
  \fi
  \item[#1]%
  \ltd@itemrest
  \ifblank{#2}{}{\label{cs:#2}}%
  \begingroup
  \ltd@syntaxsetup
  \ltxsyntaxfont
  \let\@tempa\@empty
  \ltd@parseargs}


\let\plainllap\llap
\newrobustcmd\macro@llap[1]{{\global\let\llap\plainllap
 \setbox0=\hbox\bgroup \macro@font\small\saved@macroname\egroup
 \ifdim\wd0>30mm
    \hbox to\z@ \bgroup\hss \hbox to30mm{\unhcopy0\hss}\egroup
    \edef\@tempa{\hskip\dimexpr\the\wd0-30mm}\global\everypar\expandafter{\the\expandafter\everypar
                                                                           \@tempa \global\everypar{}}%
 \else \llap{\unhbox0}\fi}}
 \AtBeginEnvironment{macro}{\let\llap\macro@llap}
\makeatother

\newcommand*{\PSstring}[1]{{\normalfont\small\ttfamily(#1)}}
\newcommand*{\PSarray}[1]{{\normalfont\small\ttfamily[#1]}}
\newcommand*{\PSname}[1]{{\normalfont\small\ttfamily/#1}}
\newcommand*{\PSproc}[1]{{\normalfont\small\ttfamily\textbraceleft #1\textbraceright}}
\newcommand*{\PSvar}[1]{{\normalfont\small\ttfamily #1}}
\newcommand*{\PSop}[1]{{\normalfont\small\ttfamily\color{DOrange}\hskip 3pt #1\hskip 3pt}}

\newcommand*{\compref}[1]{\ref{comp:#1}}
\newcommand*{\hyperpagedef}[1]{\textbf{\hyperpage{#1}}}
\newcommand*{\nodename}[1]{\emph{#1}}
\newcommand*{\param}[1]{\normalfont\texttt{#1}}
\newcommand*{\paramvalue}[1]{\texttt{#1}}
\newcommand*{\styleshape}[1]{\texttt{#1}}
\newcommand{\docindex}[1]{\sindex[doc]{#1|hyperpage}}
\newcommand{\docindexdef}[1]{\sindex[doc]{#1|hyperpagedef}}
\makeatletter
\def\Lcs{\@ifstar{\Lcs@nobm}{\Lcs@bm}}
\def\Lcs@nobm#1{\nxLcs{#1}\xLcs{#1}}
\def\Lcs@bm#1{\hyperref[cs:#1]{\Lcs@nobm{#1}}}
\def\xLcs#1{}
\def\nxLcs#1{\texttt{\textbackslash#1}}

\def\Lcomp{\@ifstar{\Lcomp@nobm}{\Lcomp@bm}}
\def\Lcomp@nobm#1{\nxLcomp{#1}\xLcomp{#1}}
\def\Lcomp@bm#1{\hyperref[comp:#1]{\Lcomp@nobm{#1}}}
\def\xLcomp#1{}
\def\nxLcomp#1{\texttt{\textbackslash#1}}

\def\LPack#1{\nxLPack{#1}}
\def\nxLPack#1{\texttt{#1}}

\def\Lenv{\@ifstar{\Lenv@nobm}{\Lenv@bm}}
\def\Lenv@nobm#1{\nxLenv{#1}\xLenv{#1}}
\def\Lenv@bm#1{\hyperref[env:#1]{\Lenv@nobm{#1}}}
\def\xLenv#1{}
\def\nxLenv#1{\texttt{#1}}
\let\orig@ltd@envitem\ltd@envitem
\def\ltd@envitem#1{\orig@ltd@envitem{#1}\label{env:#1}\xLenv{#1}}

\def\Lkeyword{\@ifstar{\Lkeyword@nobm}{\Lkeyword@bm}}
\def\Lkeyword@nobm#1{\nxLkeyword{#1}\xLkeyword{#1}}
\def\Lkeyword@bm#1{\hyperref[prm:#1]{\Lkeyword@nobm{#1}}}
\def\xLkeyword#1{}
\def\xdefLkeyword#1{}
\def\nxLkeyword#1{\texttt{#1}}

\def\xLoption#1{}
\def\Loption#1{\texttt{#1}\xLoption{#1}}
\def\nxLoption#1{\texttt{#1}}

\def\Lstyle{\@ifstar{\Lstyle@nobm}{\Lstyle@bm}}
\def\Lstyle@nobm#1{\nxLstyle{#1}\xLstyle{#1}}
\def\Lstyle@bm#1{\hyperref[sty:#1]{\Lstyle@nobm{#1}}}
\def\xLstyle#1{}
\def\xdefLstyle#1{}
\def\nxLstyle#1{\texttt{#1}}

\makeatother

\newcommand{\dipoledesc}[1]{%
  \xLdipole{#1}%
  \compitem{#1}[options](in)(out){label}%
}
\newcommand{\tripoledesc}[1]{%
  \xLtripole{#1}%
  \compitem{#1}[options](in)(center)(out){label}%
}

\newcommand{\fiberdipoledesc}[1]{%
  \xLfdipole{#1}%
  \compitem{#1}[options](in)(out){label}%
}

\newenvironment*{pssyntax}
  {\list{}{\small
     \setlength{\labelsep}{0pt}%
     \setlength{\leftmargin}{10pt}%
     \item[]}}
  {\endlist}

\newcommand{\psarglistfont}{\small}
\newenvironment*{psarglist}
  {\list{}{%
     \setlength{\labelwidth}{10pt}%
     \setlength{\labelsep}{0pt}%
     \setlength{\leftmargin}{0pt}%
     \setlength{\itemsep}{\parsep}%
     \setlength{\parsep}{0pt}%
     \renewcommand*{\makelabel}[1]{\hss\psarglistfont##1}}}
  {\endlist}

\makeatletter
\renewenvironment{theglossary}{\GlossaryParms \let\item\@idxitem \ignorespaces}{}
\makeatother
\def\psargitem#1{\item[#1]\hfill\par\nobreak}

\addtopsstyle{Fiber}{linecolor=DOrange,linewidth=1.5\pslinewidth}
\addtopsstyle{Beam}{linewidth=1.5\pslinewidth}
\EnableCrossrefs
\CodelineIndex
\OnlyDescription
\begin{document}
  \DocInput{pst-intersect.dtx}
\end{document}
%</driver>
% \fi
%
% \CheckSum{0}
%
% \CharacterTable
%  {Upper-case    \A\B\C\D\E\F\G\H\I\J\K\L\M\N\O\P\Q\R\S\T\U\V\W\X\Y\Z
%   Lower-case    \a\b\c\d\e\f\g\h\i\j\k\l\m\n\o\p\q\r\s\t\u\v\w\x\y\z
%   Digits        \0\1\2\3\4\5\6\7\8\9
%   Exclamation   \!     Double quote  \"     Hash (number) \#
%   Dollar        \$     Percent       \%     Ampersand     \&
%   Acute accent  \'     Left paren    \(     Right paren   \)
%   Asterisk      \*     Plus          \+     Comma         \,
%   Minus         \-     Point         \.     Solidus       \/
%   Colon         \:     Semicolon     \;     Less than     \<
%   Equals        \=     Greater than  \>     Question mark \?
%   Commercial at \@     Left bracket  \[     Backslash     \\
%   Right bracket \]     Circumflex    \^     Underscore    \_
%   Grave accent  \`     Left brace    \{     Vertical bar  \|
%   Right brace   \}     Tilde         \~}
%
% \GetFileInfo{pst-intersect.dtx}
%
% \newif\ifGERMAN  \GERMANfalse
% \newif\ifENGLISH \ENGLISHfalse
% \iflanguage{ngerman}{\GERMANtrue}{%
%   \iflanguage{german}{\GERMANtrue}{\GERMANfalse}}
% \iflanguage{english}{\ENGLISHtrue}{\ENGLISHfalse}
%
% \ifGERMAN
%   \newrefformat{chap}{Kap.~\ref{#1}}
%   \newrefformat{sec}{Kap.~\ref{#1}}
%   \newrefformat{l}{Zeile~\ref{#1}}
%   \newrefformat{ex}{Bsp.~\ref{#1}}
%   \newrefformat{tab}{Tab.~\ref{#1}}
%   \newrefformat{fig}{Abb.~\ref{#1}}
%   \newcommand{\PITindexMacro}{Makros}
%   \newcommand{\PITindexKeyword}{Parameter}
%   \newcommand{\PITindexEnv}{Umgebungen}
%   \newcommand{\PITindexPack}{Pakete}
%   \renewcommand*\lstlistingname{Bsp.}
% \fi
% \ifENGLISH
%   \newrefformat{chap}{Sec.~\ref{#1}}
%   \newrefformat{sec}{Sec.~\ref{#1}} 
%   \newrefformat{l}{Line~\ref{#1}}
%   \newrefformat{ex}{Ex.~\ref{#1}}
%   \newrefformat{tab}{Tab.~\ref{#1}}
%   \newrefformat{fig}{Fig.~\ref{#1}}
%   \newcommand{\PITindexMacro}{macros}
%   \newcommand{\PITindexKeyword}{parameters}
%   \newcommand{\PITindexEnv}{environments}
%   \newcommand{\PITindexPack}{packages}
%   \renewcommand*\lstlistingname{Ex.}
% \fi
%
% \makeatletter
% \renewcommand\maketitle{^^A
% \thispagestyle{empty}^^A
% \begin{titlepage}
% \begin{pspicture}(1.6in,0.685in)(10,21.7)
%   \psframe[fillstyle=solid,linecolor=lightgray,fillcolor=lightgray,linestyle=solid](0,-5.75)(21.5,10)
%   \psframe[fillstyle=solid,linecolor=Orange!85!Red,fillcolor=Orange!85!Red,linestyle=solid](0,10)(21.5,10.5)
%   \psframe[fillstyle=solid,linecolor=Orange!85!Red,fillcolor=Orange!85!Red,linestyle=solid](0,21.1)(21.5,21.2)
%   \rput[lb](3,22){\Huge\sffamily\color{Orange!65!Red}\psscalebox{2}{\textbf{PSTricks}}}
%   \rput[lb](3,14.1){\parbox{15cm}{\sffamily\RaggedRight\bfseries\huge\@title}}
%   \rput[lb](3,7.6){\parbox{13cm}{\sffamily\@date}}
%   \rput[lb](3,-2.6){\parbox[b]{17cm}{\sffamily\RaggedRight 
%     ~\hfill\makebox[7cm][l]{\ifGERMAN Paketautor:\fi\ifENGLISH Package author:\fi}\\
%     ~\hfill\makebox[7cm][l]{^^A
%       \bfseries\tabular[t]{@{}l@{}}\@author\endtabular}}}
%  \rput[C](11,4){\bgImage}
%  \end{pspicture}^^A
% \end{titlepage}}
% \makeatother
% 
% \ifGERMAN
%   \title{\texttt{pst-intersect}\\ Berechnen der Schnittpunkte beliebiger Kurven\\[0.5ex] \small \fileversion}
%   \hypersetup{pdftitle={Berechnen der Schnittpunkte beliebiger Kurven}}
% \fi
% \ifENGLISH
%   \title{\texttt{pst-optexp}\\ Intersecting arbitrary curves\\[0.5ex] \small \fileversion}
%   \hypersetup{pdftitle={Intersecting arbitrary curves}}
% \fi
% \author{Christoph Bersch}
% \date{\filedate}
% \def\bgImage{}
%
% \maketitle
% 
% \clearpage
% \tableofcontents
% \clearpage
% 
% \ifGERMAN
%   \chapter{Einführung}
% \fi
% \ifENGLISH
%   \chapter{Introduction}
% \fi
% 
% \ifGERMAN
%   \section{Über das Paket}
%
%   \LPack{pst-intersect} ist ein PSTricks-Paket zur
%   Berechnung der Schnittpunkte von Bézier-Kurven und beliebigen
%   Postscript-Pfaden.
% \fi 
% \ifENGLISH
%   \section{About the package}
%   \LPack{pst-intersect} is a PSTricks package to calculate
%   the intersections of Bezier curves and arbitrary Postscript paths.
% \fi
%
% \ifGERMAN
%   \section{Anforderungen}
%   \LPack{pst-intersect} benötigt aktuelle Versionen der
%   Pakete \LPack{pstricks}, \LPack{pst-node} und \LPack{pst-func}.
%
%   Alle PSTricks-Pakete machen regen Gebrauch von der Postscript-Sprache, so
%   dass der typische Arbeitsfluss \opt{latex}, \opt{dvips} und
%   ggf. \opt{ps2pdf} umfasst. Es gibt viele alternative Methoden um die
%   Dokumente zu
%   kompilieren.\fnurl{http://tug.org/PSTricks/main.cgi?file=pdf/pdfoutput}
% \fi
% \ifENGLISH
%   \section{Requirements}
%   \LPack{pst-intersect} requires recent versions of
%   \LPack{pstricks}, \LPack{pst-node}, and \LPack{pst-func}.
%
%   All PSTricks package rely heavily on the Postscript language so that the
%   typical workflow involves \opt{latex}, \opt{dvips}, and \opt{ps2pdf}. Of
%   course there are several alternative ways to compile your
%   documents.\fnurl{http://tug.org/PSTricks/main.cgi?file=pdf/pdfoutput} 
% \fi
%
% \ifGERMAN
%   \section{Verbreitung und Installation}
%   %Dieses Paket ist auf
%   %CTAN\fnurl{http://mirror.ctan.org/help/Catalogue/entries/pst-intersect.html}
%   %erhältlich.^^A und in \TeX Live and MiK\TeX{} enthalten.
% 
%   Das \LPack{pst-intersect}-Paket umfasst die zwei Hauptdateien
%   \texttt{pst-intersect.ins} und \texttt{pst-intersect.dtx}. Durch Aufrufen
%   von \texttt{tex pst-intersect.ins} werden die drei folgenden
%   Dateien erzeugt:
%   \begin{itemize}
%   \item \texttt{pst-intersect.pro}: die Postscript Prologdatei
%   \item \texttt{pst-intersect.sty}: die \LaTeX-Stildatei
%   \item \texttt{pst-intersect.tex}: die \TeX-Datei
%   \end{itemize}
%   Speichern Sie diese Dateien in einem Verzeichnis der Teil Ihres
%   lokalen \TeX-Baums ist.
% 
%   Vergessen Sie nicht \texttt{texhash} aufzurufen um den Baum zu
%   aktualisieren. MiK\TeX{}-Benutzer müssen die Dateinamen-Datenbank
%   (FNDB) aktualisieren.
% 
%   Detailliertere Information finden Sie in der Dokumentation Ihrer
%   \LaTeX-Distribution über die Installation in den lokalen
%   \TeX{}-Baum.
% \fi
% \ifENGLISH
%   \section{Distribution and installation}
%   This package is available on
%   CTAN\fnurl{http://mirror.ctan.org/help/Catalogue/entries/pst-intersect.html}.
%   ^^A and is included in \TeX Live and MiK\TeX.
% 
%   The \LPack{pst-intersect} package consists of the two main files
%   \texttt{pst-intersect.ins} and \texttt{pst-intersect.dtx}. By running \texttt{tex
%   pst-intersect.ins} the following derived files are generated:
%   \begin{itemize}
%   \item \texttt{pst-intersect.pro}: the Postscript prolog file
%   \item \texttt{pst-intersect.sty}: the \LaTeX{} style file
%   \item \texttt{pst-intersect.tex}: the \TeX{} file
%   \end{itemize}
%   Save the files in a directory which is part of your local \TeX{} tree.
% 
%   Do not forget to run \texttt{texhash} to update this tree. For MiK\TeX{}
%   users, do not forget to update the file name database (FNDB).
% 
%   For more detailed information see the documentation of your personal
%   \LaTeX{} distribution on installing packages to your local \TeX{}
%   system.
% \fi
%
% \ifGERMAN\section{Lizenz}\fi
% \ifENGLISH\section{License}\fi
% \ifGERMAN
% Es wird die Erlaubnis gewährt, dieses Dokument zu kopieren, zu verteilen
% und\slash oder zu modifizieren, unter den Bestimmungen der \LaTeX{} Project
% Public License, Version
% 1.3c.\fnurl{http://www.latex-project.org/lppl.txt}. Dieses
% Paket wird vom Autor betreut (author-maintained).
% \fi
% \ifENGLISH
% Permission is granted
% to copy, distribute and\slash or modify this software under the terms of the
% \LaTeX{} Project Public License, version
% 1.3c.\fnurl{http://www.latex-project.org/lppl.txt} This
% package is author-maintained.
% \fi
%
% \ifGERMAN
%   \section{Danksagung}
% \fi
% \ifENGLISH
%   \section{Acknowledgements}
% \fi
% \ifGERMAN
% Ich danke Marco Cecchetti, dessen
% \opt{lib2geom}-Bibliothek\fnurl{http://lib2geom.sourceforge.net/}
% mir als Vorlage für einen Großteil des Postscript-Kodes für den
% Bézier-Clipping-Algorithmus diente. Außerdem gilt mein Dank William
% A. Casselman, für seine Erlaubnis, den Quicksort-Kode und den Kode zur
% Berechung der konvexen Hüllen aus seinem Buch «Mathematical
% Illustration» verwenden zu
% dürfen\fnurl{http://www.math.ubc.ca/~cass/graphics/text/www/}. Der
% Dokumentationsstil ist eine Mischung aus der \opt{pst-doc} Klasse
% (Herbert Voß) und dem \opt{ltxdockit} Paket für die \opt{biblatex}
% Dokumentation (Philipp Lehmann).
% \fi
% \ifENGLISH
% I thank Marco Cecchetti, for his
% \opt{lib2geom}-library\fnurl{http://lib2geom.sourceforge.net/} from
% which I derived great parts of the Postscript code for the Bézier
% clipping algorithm. Also I want to thank William A. Casselman for the
% Postscript code of the quicksort procedure and the procedure for
% calculating the convex hull from his book «Mathematical
% Illustration»\fnurl{http://www.math.ubc.ca/~cass/graphics/text/www/},
% and the permission to use it. The documentation style is a mixture of
% the \opt{pst-doc} class (Herbert Voß) and the \opt{ltxdockit} package
% for the \opt{biblatex} documentation (Philipp Lehman).
% \fi
%
%
% \ifGERMAN
% \chapter{Benutzung}
% \fi
% \ifENGLISH
% \chapter{Usage}
% \fi
%
% \ifGERMAN
% Das \LPack{pst-intersect}-Paket kann Schnittpunkte von beliebigen
% Postscript-Pfaden berechnen. Diese setzen sich nur aus drei primitiven
% Operation zusammen: Linien (\opt{lineto}), Bézier-Kurven dritter
% Ordnung (\opt{curveto}) und Sprüngen (\opt{moveto}). Speziellere
% Konstruktionen, wie z.B. Kreise werden intern zu
% \opt{curveto}-Anweisungen umgewandelt. Über diese Kommandos hinaus
% kann \LPack{pst-intersect} auch Bézier-Kurven beliebiger Ordnung
% verwenden. Das diese keine primitiven Postscript-Pfadelemente
% darstellen, müssen sie gesondert behandelt werden.
%
% Der allgemeine Arbeitsablauf besteht darin eine oder mehrere Kurven
% oder Pfade zu speichern, und danach die Schnittpunkte zu
% berechnen. Anschließend können die Schnittpunkte als normale
% PSTricks-Knoten verwendet werden, oder Abschnitte der Kurven und Pfade
% nachgezogen werden (z.B. zwischen zwei Schnittpunkten). 
% \fi
%
% \ifGERMAN
% \section{Speichern von Pfaden und Kurven}
% \fi
% \ifENGLISH
% \section{Saving paths and curves}
% \fi
%
% \ifGERMAN
% 
% \fi
%
% \begin{ltxsyntax}
%   \cmditem{savepath}[options]{curvename}{commands} 
%   
%   \ifGERMAN
%   Speichert den gesamten Pfad, der durch \prm{commands} erstellt wird,
%   unter Verwendung des Namens \prm{curvename}. Das Makro funktioniert
%   genauso wie \cs{pscustom}, und kann daher auch nur die darin
%   erlaubten Kommandos verarbeiten.
%
%   In den Standardeinstellungen wird der entsprechende Pfad auch gleich
%   gezeichnet, was mit \prm{options} beeinflusst werden kann. Mit
%   \opt{linestyle=none} wird das unterbunden.
%   \fi
% \end{ltxsyntax}
%\iffalse
%<*ignore>
%\fi
\begin{LTXexample}
\begin{pspicture}(3,2)
  \savepath[linecolor=DOrange]{MyPath}{%
    \pscurve(0,2)(0,0.5)(3,1)
  }%
\end{pspicture}
\end{LTXexample}
%\iffalse
%</ignore>
%\fi
%
% \begin{ltxsyntax}
%   \cmditem{savebezier}[options]{curvename}($X_0$)\ldots(\prm{$X_n$})
%
%   \ifGERMAN 
%   Die Postscript-Sprache unterstützt nur Bézier-Kurven dritter
%   Ordnung. Mit dem Makro \Lcs{savebezier} können Bézier-Kurven
%   beliebiger Ordnung definiert werden. Die angegebenen Knoten sind die
%   Kontrollpunkte der Kurve, für eine Kurve $n$-ter Ordnung werden
%   $(n+1)$ Kontrollpunkte benötigt. Die Darstellung der Kurve erfolgt
%   mit dem Makro \cs{psBezier} aus dem \LPack{pst-func}-Paket, welches
%   nur Kurven bis neunter Ordnung darstellen kann.
%   \fi
% \end{ltxsyntax}
%\iffalse
%<*ignore>
%\fi
\begin{LTXexample}
\begin{pspicture}(3,2)
  \savebezier[showpoints]{MyBez}(0,0)(0,1)(1,2)(3,2)(1,0)(3,0)
\end{pspicture}
\end{LTXexample}
%\iffalse
%</ignore>
%\fi
%
% \ifGERMAN
% \section{Schnittpunkte berechnen}
% \fi
% \ifENGLISH
% \section{Calculating intersections}
% \fi
%
% \begin{ltxsyntax}
%   \cmditem{psintersect}{curveA}{curveB}
% \end{ltxsyntax}
% 
% \ifGERMAN
% Nachdem Sie nun Pfade oder Kurven gespeichert haben, können Sie deren
% Schnittpunkte berechnen. Das geschieht mit dem Makro
% \Lcs{psintersect}. Dieses benötigt als Argumente zwei Namen von Pfaden
% oder Kurven (Das Argument \prm{curvename} der beiden \cs{save*}
% Makros).
% \fi
%
%\iffalse
%<*ignore>
%\fi
\begin{LTXexample}
\begin{pspicture}(3,2)
  \savepath[linecolor=DOrange]{MyPath}{\pscurve(0,2)(0,0.5)(3,1)}
  \savebezier{MyBez}(0,0)(0,1)(1,2)(3,2)(1,0)(3,0)
  \psintersect[showpoints]{MyPath}{MyBez}
\end{pspicture}
\end{LTXexample}
%\iffalse
%</ignore>
%\fi
%
% \ifGERMAN
% Der PSTricks-Parameter \opt{showpoints} steuert dabei, ob die Schnittpunkte angezeigt werden.
% \fi
%
% \begin{optionlist}
% \valitem[@tmp]{name}{string}
% \ifGERMAN
% Die berechneten Schnittpunkte können unter einem hier angegebenen
% Namen gespeichert und zu einem späteren Zeitpunkt verwendet werden
% (siehe \prettyref{sec:psshowcurve}). 
% \fi
%
% \boolitem[true]{saveintersections}
% \ifGERMAN
% Ist dieser Schalter gesetzt, dann werden die Schnittpunkte als
% PSTricks-Knoten unter den Namen \prm{name}1, \prm{name}2 \ldots
% gespeichert. Die Nummerierung erfolgt aufsteigend nach dem Wert der
% $x$-Koordinate.
% \fi
%
%\iffalse
%<*ignore>
%\fi
\begin{LTXexample}
\begin{pspicture}(5,5)
  \savebezier[linecolor=DOrange]{A}%
             (0,0)(0,5)(5,5)(5,1)(1,1.5)
  \savebezier{B}(0,5)(0,0)(5,0)(5,5)(0,2)
  \psintersect[name=C, showpoints]{A}{B}
  \uput[150](C1){1}
  \uput[85](C2){2}
  \uput[90](C3){3}
  \uput[-20](C4){4}
\end{pspicture}
\end{LTXexample}
%\iffalse
%</ignore>
%\fi
% \end{optionlist}
%
% \ifGERMAN
% \section{Darstellung gespeicherter Pfade}
% \fi
% \ifENGLISH
% \section{Visualization of saved paths}
% \fi
%
% \begin{ltxsyntax}
%   \cmditem{psshowcurve}[options]{curvename}
%
% \ifGERMAN
% Gespeicherte Pfade und Kurven können mit diesem Makro nachträglich gezeichnet werden.
% \fi
% \end{ltxsyntax}
%
%\iffalse
%<*ignore>
%\fi
\begin{LTXexample}
\begin{pspicture}(2,2)
  \savepath{Circle}{\pscircle(1,1){1}}
  \psshowcurve[linestyle=dashed, linecolor=green]{Circle}
\end{pspicture}
\end{LTXexample}
%\iffalse
%</ignore>
%\fi
%
% \begin{optionlist}
% \numitem{tstart}
% \numitem{tstop}
%
% \ifGERMAN Unter Verwendung dieser beiden Parameter können auch
% Abschnitte von Pfaden und Kurven gezeichnet werden. Bei Bézier-Kurven
% ist der Parameterbereich $[0, 1]$, wobei $0$ dem Anfang der Kurve,
% also dem ersten bei \Lcs{savebezier} angegebenen Knoten entspricht.
% \fi
%
%\iffalse
%<*ignore>
%\fi
\begin{LTXexample}
\begin{pspicture}(5,5)
  \psset{showpoints}
  \savebezier{B}(0,5)(0,0)(5,0)(5,5)(0,2)
  \psshowcurve[linestyle=dashed, linecolor=blue!50,
               tstart=0, tstop=0.5]{B}
\end{pspicture}
\end{LTXexample}
%\iffalse
%</ignore>
%\fi
%
% \medskip
% \ifGERMAN 
% Pfaden können aus mehr als einem Abschnitt bestehen, der Bereich ist
% also $[0, n]$, wobei $n$ die Anzahl der Pfadabschnitte ist. Dabei ist
% zu beachten, dass z.B. \cs{pscurve}-Pfade oder auch Kreise und
% Kreisbögen aus mehreren Abschnitten bestehen.
% \fi
%
%\iffalse
%<*ignore>
%\fi
\begin{LTXexample}
\begin{pspicture}(2,2)
  \savepath[linestyle=none]{Circle}{\pscircle(1,1){1}}
  \psshowcurve[tstart=0, tstop=1, linecolor=green]{Circle}
  \psshowcurve[tstart=2, tstop=3, linecolor=red]{Circle}
  \psshowcurve[tstart=1.25, tstop=1.75, linecolor=blue]{Circle}
\end{pspicture}
\end{LTXexample}
%\iffalse
%</ignore>
%\fi
%
% \end{optionlist}
%
% \ifGERMAN
% Beachten Sie, dass in der Version 0.1 die Reihenfolge von
% \Lkeyword{tstart} und \Lkeyword{tstop} noch keine Rolle spielt. Für
% kommende Versionen ist geplant, dass bei \Lkeyword{tstart}
% \textgreater{} \Lkeyword{tstop} die Pfadrichtung umgekehrt wird. Zur
% Zeit werden auch noch keine Pfeile für \Lcs{psshowcurve} unterstützt.
% \fi
%
% \ifGERMAN
% \section{Darstellung gespeicherter Schnitte}
% \fi
% \ifENGLISH
% \section{Visualization of saved intersections}
% \fi
%
% \begin{ltxsyntax}
% \cmditem{psshowcurve}[options]{intersection}{curvename}
% \end{ltxsyntax}
% 
% \begin{optionlist}
% \numitem{istart}
% \numitem{istop}
% \end{optionlist}
% 
% \ifGERMAN
%
% \fi
%\iffalse
%<*ignore>
%\fi
\begin{LTXexample}
\begin{pspicture}(5.2,5.2)
\savebezier[linewidth=0.5\pslinewidth, linestyle=dashed]{A}(0,0)(0,5)(5,2)(5,5)
\savebezier[linewidth=0.5\pslinewidth, linestyle=dashed]{B}(0,2.5)(2.5,2.5)(4.5, 3)(2,4)
\psintersect[linecolor=green!70!black, name=C]{A}{B}
\psshowcurve[linecolor=red, istart=1, istop=2]{C}{A}
\psshowcurve[linecolor=blue, istart=1, istop=2]{C}{B}
\end{pspicture}
\end{LTXexample}
%\iffalse
%</ignore>
%\fi
%
% \ifGERMAN
% \chapter{Beispiele}
% \fi
% \ifENGLISH
% \chapter{Examples}
% \fi
%\iffalse
%<*ignore>
%\fi
\begin{LTXexample}
\begin{pspicture}(5,5)
  \savebezier{A}(0,0)(0,5)(5,5)(5,1)(1,1.5)
  \multido{\i=100+-20,\r=1+-0.2}{5}{%
    \psshowcurve[linecolor=red!\i, tstop=\r, arrows=-|, showpoints]{A}
  }%
\end{pspicture}
\end{LTXexample}
%\iffalse
%</ignore>
%\fi 
%
% \appendix
%
% \ifGERMAN
% \chapter{Versionsgeschichte}
%
% Diese Versionsgeschichte ist eine Liste von Änderungen, die für den Nutzer des
% Pakets von Bedeutung sind. Änderungen, die eher technischer Natur sind und für
% den Nutzer des Pakets nicht relevant sind und das Verhalten des Pakets nicht
% ändern, werden nicht aufgeführt. Wenn ein Eintrag der Versionsgeschichte ein
% Feature als \emph{improved} oder \emph{extended} bekannt gibt, so bedeutet
% dies, dass eine Modifikation die Syntax und das Verhalten des Pakets nicht
% beeinflusst, oder das es für ältere Versionen kompatibel ist. Einträge, die
% als \emph{deprecated}, \emph{modified}, \emph{renamed}, oder \emph{removed}
% deklariert sind, verlangen besondere Aufmerksamkeit. Diese bedeuten, dass eine
% Modifikation Änderungen in existierenden Dokumenten mit sich ziehen kann. 
% \fi
% \ifENGLISH
% \chapter{Revision history}
%
% This revision history is a list of changes relevant to users of this
% package. Changes of a more technical nature which do not affect the
% user interface or the behavior of the package are not included in the
% list. If an entry in the revision history states that a feature has
% been \emph{improved} or \emph{extended}, this indicates a modification
% which either does not affect the syntax and behavior of the package or
% is syntactically backwards compatible (such as the addition of an
% optional argument to an existing command). Entries stating that a
% feature has been \emph{deprecated}, \emph{modified}, \emph{fixed},
% \emph{renamed}, or \emph{removed} demand attention. They indicate a
% modification which may require changes to existing documents.
% \fi
%
% \begin{changelog}
%\patchcmd{\release}{\setlength{\itemsep}{0pt}}{\setlength{\itemsep}{0pt}\setlength{\parsep}{0pt}}{}{}
%   \begin{release}{1.0}{2014-xx-xx}
%   \item First CTAN version
%   \end{release}
% \end{changelog}
%
% \StopEventually{}
%
%   \begin{otherlanguage}{english}
%    \printindex[idx]
%  \end{otherlanguage}
%
% \chapter{The \LaTeX\ wrapper}
%<*stylefile>
%    \begin{macrocode}
\NeedsTeXFormat{LaTeX2e}[1999/12/01]
\ProvidesPackage{pst-intersect}%
   [2014/02/06 v0.1alpha package wrapper for pst-intersect.tex]
\RequirePackage{pstricks}
\RequirePackage{pst-xkey}
\RequirePackage{pst-node}
\RequirePackage{multido}
\RequirePackage{pst-func}
% \iffalse meta-comment
%
% Copyright (C) 2014 by Christoph Bersch <usenet@bersch.net>
%
% This work may be distributed and/or modified under the
% conditions of the LaTeX Project Public License, either version 1.3c
% of this license or (at your option) any later version.
% The latest version of this license is in
%   http://www.latex-project.org/lppl.txt
% and version 1.3c or later is part of all distributions of LaTeX
% version 2008/05/04 or later.
% \fi
%
% \iffalse
%<*driver>
\ProvidesFile{pst-intersect.dtx}
%</driver>
%<stylefile>\NeedsTeXFormat{LaTeX2e}[1999/12/01]
%<stylefile>\ProvidesPackage{pst-intersect}
%<*stylefile>
    [2014/02/18 v0.1alpha package wrapper for pst-intersect.tex]
%</stylefile>
%
%<*driver>
\documentclass[a4paper, DIV=9, oneside, toc=index, parskip=half-]{scrreprt}
\usepackage{doc}
\setcounter{IndexColumns}{2}
\usepackage[utf8]{inputenc} 
\usepackage[T1]{fontenc}
\usepackage{lmodern} 
\usepackage{amsmath, marvosym} 
\usepackage{bera}
\providecommand*\mainlang{}
\usepackage[ngerman, english,\mainlang]{babel}
\usepackage{prettyref}
\usepackage[dvipsnames,x11names,svgnames]{xcolor}
\usepackage{array,booktabs,paralist,tabularx}
\usepackage{ragged2e, calc}
\usepackage{nicefrac, multido}
\usepackage{pst-intersect}
\usepackage{hypdoc}
\hypersetup{%
  colorlinks=true, 
  urlcolor=DOrange, 
  linkcolor=pdflinkcolor, 
  breaklinks,
  linktocpage=true} 
\usepackage{breakurl}
\definecolor{DOrange}{rgb}{1,.4,.2}%
\definecolor{DDOrange}{rgb}{0.7, 0.23, 0.07}%
\colorlet{pdflinkcolor}{DOrange}
\colorlet{DGreen}{green!90!black}
\usepackage{showexpl}
\makeatletter\renewcommand*\SX@Info{}\makeatother
\usepackage{etoolbox}
\undef{\cs}\undef{\cmd}
\usepackage{ltxdockit}
\newcommand{\poeTR}[1]{\TR{\ttfamily\color{DOrange}#1}}
\definecolor{colKeys}{rgb}{0,0,0}
\definecolor{colIdentifier}{rgb}{0,0,0}
\colorlet{colComments}{green!60!black}
\definecolor{colString}{rgb}{0,0.5,0}
\newlength{\codeoverhang}
\setlength{\codeoverhang}{0.5\marginparwidth+\marginparsep}
\lstset{%
  language=[LaTeX]TeX, identifierstyle=\color{colIdentifier},
  keywordstyle=\color{colKeys},
  keywordstyle = [21]\color{DOrange},
  keywordstyle = [22]\color{DOrange},
  stringstyle=\color{colString},
  commentstyle=\color{colComments},
  alsoletter={12},
  float=hbp,
  basicstyle=\ttfamily\small,
  columns=flexible,
  tabsize=4,
  showspaces=false,
  showstringspaces=false,
  breaklines=true,
  breakautoindent=true,
  breakatwhitespace=true,
  captionpos=t,
  belowcaptionskip=0pt,
  abovecaptionskip=0pt,
  xleftmargin=1em,
  prebreak = {\raisebox{-0.5ex}[\ht\strutbox]{\kern0.5ex\large\Righttorque}},
  rulecolor=\color{black!20}, 
  texcsstyle = [20]\color{DDOrange},
  moretexcs = [20]{savebezier, savepath, psshowcurve, psintersect},
  explpreset={%
    pos=l, width=-99pt, hsep=5mm, overhang=\codeoverhang, varwidth,
    vsep=\bigskipamount, rframe={}}, extendedchars=true
}%
\lstdefinestyle{example}{explpreset={%
    escapechar=*, pos=l, width=-99pt, hsep=5mm, overhang=\codeoverhang,
    varwidth, vsep=\bigskipamount, rframe={}}}
\makeatletter
\providecommand\ON{%
  \gdef\lst@alloverstyle##1{\textcolor{black!50}{\strut##1}%
}}
\providecommand\OFF{\xdef\lst@alloverstyle##1{##1}}
\makeatother
\colorlet{sectioncolor}{DOrange}
\addtokomafont{sectioning}{\color{sectioncolor}}
\usepackage[automark,nouppercase]{scrpage2}
\pagestyle{scrheadings}
\clearscrheadings
\clearscrplain
\ohead{\pagemark}
\ihead{\headmark}
\ofoot[\pagemark]{}
\automark[subsection]{section}
\setheadsepline{.4pt}[\color{DOrange}]
\setheadwidth[0pt]{text}
\setfootwidth[0pt]{text}
\makeatletter
\patchcmd{\l@chapter}{1.5em}{2em}{}{}
\renewcommand*\l@section{\bprot@dottedtocline{1}{1.5em}{3.0em}}
\renewcommand*\l@subsection{\bprot@dottedtocline{2}{3.8em}{4.0em}}
\newrobustcmd*{\fnurl}[1][]{\hyper@normalise\ltd@fnurl{#1}}
\def\ltd@fnurl#1#2{\footnote{#1\hyper@linkurl{\Hurl{#2}}{#2}}}
\newrobustcmd*{\arxivurl}[1]{\href{http://arxiv.org/abs/#1}{arXiv:#1}}
\newrobustcmd*{\doiurl}[1]{\href{http://dx.doi.org/#1}{DOI:#1}}
\makeatother
\usepackage{csquotes}
\MakeAutoQuote{«}{»}
%^^A spot is used in ltxdockit.sty
\colorlet{spot}{sectioncolor}
%^^A Fonts definitions used in ltxdockit.sty
\renewcommand*{\verbatimfont}{\ttfamily}
\renewcommand*{\displayverbfont}{\ttfamily}
\renewcommand*{\marglistfont}{\spotcolor\sffamily\small}
\renewcommand*{\margnotefont}{\sffamily\small}
\renewcommand*{\optionlistfont}{\spotcolor\sffamily\displayverbfont}
\renewcommand*{\ltxsyntaxfont}{\ttfamily}
\renewcommand*{\ltxsyntaxlabelfont}{\spotcolor\displayverbfont}
\renewcommand*{\changelogfont}{\normalfont}
\renewcommand*{\changeloglabelfont}{\spotcolor\sffamily\bfseries}

\makeatletter
\renewenvironment*{ltxsyntax}
  {\list{}{%
     \setlength{\labelwidth}{\marglistwidth}%
     \setlength{\labelsep}{0pt}%
     \setlength{\leftmargin}{0pt}%
     \renewcommand*{\makelabel}[1]{%
       \hss\ltxsyntaxfont\ltxsyntaxlabelfont##1}}%
   \let\cmditem\PIT@cmditem}
  {\endlist}

\def\ltd@optionlist{%
  \let\optitem\PIT@optitem
  \let\valitem\PIT@valitem
  \let\choitem\PIT@choitem
  \let\boolitem\PIT@boolitem
  \let\intitem\PIT@intitem
  \let\numitem\PIT@numitem
  \let\psnumitem\PIT@psnumitem
  \let\pititem\ltd@item
  \let\typeitem\PIT@typeitem}

\def\PIT@typeitem#1#2{%
  \item[{#1}]%
  \label{prm:#1}%\docindexdef{#1=\nxLkeyword{#1}}%
  \begingroup\raggedright
  #2%
  \settowidth\@tempdimb{\prm{psstyle}}%
  \settowidth\@tempdimc{#1}%
  \@tempdimc=\dimexpr\@tempdimc+\labelsep-\labelwidth\relax
  \ifdim\@tempdimc>0pt%
    \@tempdima=\dimexpr\linewidth-\@tempdimb-\@tempdimc-1em\relax
  \else
    \@tempdima=\dimexpr\linewidth-\@tempdimb-1em\relax
  \fi
  \par\endgroup}

\def\PIT@option#1#2#3{%
  \item[#1]%
  \begingroup\raggedright
  \ltd@textverb{=}%
  \settowidth\@tempdimb{\ltd@textverb{=}}%
  \settowidth\@tempdimc{#1}%
  \@tempdimc=\dimexpr\labelwidth-\@tempdimc\relax
  \ifdim\@tempdimc<0pt
    \@tempdima=\dimexpr\linewidth-\@tempdimb+\@tempdimc-2em\relax
  \else
    \@tempdima=\dimexpr\linewidth-\@tempdimb-2em\relax
  \fi
  \ifblank{#3}
    {}
    {\settowidth\@tempdimb{default: #3}%
     \@tempdima=\dimexpr\@tempdima-\@tempdimb-2em\relax}%
  \parbox[t]{\@tempdima}{\raggedright #2}%
  \ifblank{#3}
    {}
    {\hfill default:~#3}%
  \par\endgroup
  \nobreak\vspace{\itemsep}}

\def\PIT@optitem{%
  \@ifstar
    {\boolfalse{@tempswa}\PIT@optitem@i}
    {\booltrue{@tempswa}\PIT@optitem@i}}
\newcommand*{\PIT@optitem@i}[3][]{%
  \ifbool{@tempswa}%
    {\label{prm:#2}\xdefLkeyword{#2}}%
    {\xLkeyword{#2}}%
  \ifblank{#1}
    {\PIT@option{#2}{#3}{}}
    {\PIT@option{#2}{#3}{\ltd@textverb{#1}}}}%

\def\PIT@valitem{%
  \@ifstar
    {\boolfalse{@tempswa}\PIT@valitem@i}
    {\booltrue{@tempswa}\PIT@valitem@i}}
\newcommand*{\PIT@valitem@i}[3][]{%
  \ifbool{@tempswa}%
    {\label{prm:#2}\xdefLkeyword{#2}}%
    {\xLkeyword{#2}}%
  \ifblank{#1}
    {\PIT@option{#2}{\prm{#3}}{}}
    {\PIT@option{#2}{\prm{#3}}{\ltd@textverb{#1}}}}%

\def\PIT@choitem{%
  \@ifstar
    {\boolfalse{@tempswa}\PIT@choitem@i}
    {\booltrue{@tempswa}\PIT@choitem@i}}
\newcommand*{\PIT@choitem@i}[3][]{%
  \ifbool{@tempswa}%
    {\label{prm:#2}\xdefLkeyword{#2}}%
    {\xLkeyword{#2}}%
  \ifblank{#1}
    {\PIT@option{#2}{\ltd@verblist{#3}}{}}
    {\PIT@option{#2}{\ltd@verblist{#3}}{\ltd@textverb{#1}}}}%

\def\PIT@boolitem{%
  \@ifstar
    {\boolfalse{@tempswa}\PIT@boolitem@i}
    {\booltrue{@tempswa}\PIT@boolitem@i}}
\newcommand*{\PIT@boolitem@i}[2][]{%
  \ifbool{@tempswa}%
    {\label{prm:#2}\xdefLkeyword{#2}}%
    {\xLkeyword{#2}}%
  \ifblank{#1}
    {\PIT@option{#2}{\ltd@verblist{true,false}}{}}
    {\PIT@option{#2}{\ltd@verblist{true,false}}{\ltd@textverb{#1}}}}%

\def\PIT@intitem{%
  \@ifstar
    {\boolfalse{@tempswa}\PIT@intitem@i}
    {\booltrue{@tempswa}\PIT@intitem@i}}
\newcommand*{\PIT@intitem@i}[2][]{%
  \ifbool{@tempswa}%
    {\label{prm:#2}\xdefLkeyword{#2}}%
    {\xLkeyword{#2}}%
  \ifblank{#1}
    {\PIT@option{#2}{\prm{int}}{}}
    {\PIT@option{#2}{\prm{int}}{\ltd@textverb{#1}}}}%

\def\PIT@numitem{%
  \@ifstar
    {\boolfalse{@tempswa}\PIT@numitem@i}
    {\booltrue{@tempswa}\PIT@numitem@i}}
\newcommand*{\PIT@numitem@i}[2][]{%
  \ifbool{@tempswa}%
    {\label{prm:#2}\xdefLkeyword{#2}}%
    {\xLkeyword{#2}}%
  \ifblank{#1}
    {\PIT@option{#2}{\prm{num}}{}}
    {\PIT@option{#2}{\prm{num}}{\ltd@textverb{#1}}}}%

\def\PIT@psnumitem{%
  \@ifstar
    {\boolfalse{@tempswa}\PIT@psnumitem@i}
    {\booltrue{@tempswa}\PIT@psnumitem@i}}
\newcommand*{\PIT@psnumitem@i}[2][]{%
  \ifbool{@tempswa}%
    {\label{prm:#2}\xdefLkeyword{#2}}%
    {\xLkeyword{#2}}%
  \ifblank{#1}
    {\PIT@option{#2}{\prm{psnum}}{}}
    {\PIT@option{#2}{\prm{psnum}}{\ltd@textverb{#1}}}}%

\def\ltd@csitem{%
  \@ifstar
    {\boolfalse{@tempswa}\ltd@csitem@i}
    {\booltrue{@tempswa}\ltd@csitem@i}}
\def\ltd@csitem@i#1{%
  \ifbool{@tempswa}
    {\ltd@item@ii{\textbackslash#1\hspace\marglistsep}{#1}}
    {\ltd@item@ii{\textbackslash#1\hspace\marglistsep}{}}}

\def\PIT@cmditem{%
  \@ifstar
    {\boolfalse{@tempswa}\PIT@cmditem@i}
    {\booltrue{@tempswa}\PIT@cmditem@i}}
\def\PIT@cmditem@i#1{%
  \ifbool{@tempswa}
    {\PIT@cmditem@ii{\textbackslash#1}{cs:#1}}
    {\PIT@cmditem@ii{\textbackslash#1}{}}}
\def\PIT@compitem{%
  \@ifstar
    {\boolfalse{@tempswa}\PIT@compitem@i}
    {\booltrue{@tempswa}\PIT@compitem@i}}
\def\PIT@compitem@i#1{%
  \ifbool{@tempswa}
    {\PIT@cmditem@ii{\textbackslash#1}{comp:#1}}
    {\PIT@cmditem@ii{\textbackslash#1}{}}}
\def\PIT@cmditem@ii#1#2{%
  \ltd@itemsave
  \ifhmode
    \itemsep-\topsep
  \else
    \ltd@itembreak
  \fi
  \item[#1]%
  \ltd@itemrest
  \ifblank{#2}{}{\label{#2}}%
  \begingroup
  \ltd@syntaxsetup
  \ltxsyntaxfont
  \let\@tempa\@empty
  \ltd@parseargs}

\def\ltd@csitem@ii#1#2{%
  \ltd@itemsave
  \ifhmode
    \itemsep-\topsep
  \else
    \ltd@itembreak
  \fi
  \item[#1]%
  \ltd@itemrest
  \ifblank{#2}{}{\label{cs:#2}}%
  \begingroup
  \ltd@syntaxsetup
  \ltxsyntaxfont
  \let\@tempa\@empty
  \ltd@parseargs}


\let\plainllap\llap
\newrobustcmd\macro@llap[1]{{\global\let\llap\plainllap
 \setbox0=\hbox\bgroup \macro@font\small\saved@macroname\egroup
 \ifdim\wd0>30mm
    \hbox to\z@ \bgroup\hss \hbox to30mm{\unhcopy0\hss}\egroup
    \edef\@tempa{\hskip\dimexpr\the\wd0-30mm}\global\everypar\expandafter{\the\expandafter\everypar
                                                                           \@tempa \global\everypar{}}%
 \else \llap{\unhbox0}\fi}}
 \AtBeginEnvironment{macro}{\let\llap\macro@llap}
\makeatother

\newcommand*{\PSstring}[1]{{\normalfont\small\ttfamily(#1)}}
\newcommand*{\PSarray}[1]{{\normalfont\small\ttfamily[#1]}}
\newcommand*{\PSname}[1]{{\normalfont\small\ttfamily/#1}}
\newcommand*{\PSproc}[1]{{\normalfont\small\ttfamily\textbraceleft #1\textbraceright}}
\newcommand*{\PSvar}[1]{{\normalfont\small\ttfamily #1}}
\newcommand*{\PSop}[1]{{\normalfont\small\ttfamily\color{DOrange}\hskip 3pt #1\hskip 3pt}}

\newcommand*{\compref}[1]{\ref{comp:#1}}
\newcommand*{\hyperpagedef}[1]{\textbf{\hyperpage{#1}}}
\newcommand*{\nodename}[1]{\emph{#1}}
\newcommand*{\param}[1]{\normalfont\texttt{#1}}
\newcommand*{\paramvalue}[1]{\texttt{#1}}
\newcommand*{\styleshape}[1]{\texttt{#1}}
\newcommand{\docindex}[1]{\sindex[doc]{#1|hyperpage}}
\newcommand{\docindexdef}[1]{\sindex[doc]{#1|hyperpagedef}}
\makeatletter
\def\Lcs{\@ifstar{\Lcs@nobm}{\Lcs@bm}}
\def\Lcs@nobm#1{\nxLcs{#1}\xLcs{#1}}
\def\Lcs@bm#1{\hyperref[cs:#1]{\Lcs@nobm{#1}}}
\def\xLcs#1{}
\def\nxLcs#1{\texttt{\textbackslash#1}}

\def\Lcomp{\@ifstar{\Lcomp@nobm}{\Lcomp@bm}}
\def\Lcomp@nobm#1{\nxLcomp{#1}\xLcomp{#1}}
\def\Lcomp@bm#1{\hyperref[comp:#1]{\Lcomp@nobm{#1}}}
\def\xLcomp#1{}
\def\nxLcomp#1{\texttt{\textbackslash#1}}

\def\LPack#1{\nxLPack{#1}}
\def\nxLPack#1{\texttt{#1}}

\def\Lenv{\@ifstar{\Lenv@nobm}{\Lenv@bm}}
\def\Lenv@nobm#1{\nxLenv{#1}\xLenv{#1}}
\def\Lenv@bm#1{\hyperref[env:#1]{\Lenv@nobm{#1}}}
\def\xLenv#1{}
\def\nxLenv#1{\texttt{#1}}
\let\orig@ltd@envitem\ltd@envitem
\def\ltd@envitem#1{\orig@ltd@envitem{#1}\label{env:#1}\xLenv{#1}}

\def\Lkeyword{\@ifstar{\Lkeyword@nobm}{\Lkeyword@bm}}
\def\Lkeyword@nobm#1{\nxLkeyword{#1}\xLkeyword{#1}}
\def\Lkeyword@bm#1{\hyperref[prm:#1]{\Lkeyword@nobm{#1}}}
\def\xLkeyword#1{}
\def\xdefLkeyword#1{}
\def\nxLkeyword#1{\texttt{#1}}

\def\xLoption#1{}
\def\Loption#1{\texttt{#1}\xLoption{#1}}
\def\nxLoption#1{\texttt{#1}}

\def\Lstyle{\@ifstar{\Lstyle@nobm}{\Lstyle@bm}}
\def\Lstyle@nobm#1{\nxLstyle{#1}\xLstyle{#1}}
\def\Lstyle@bm#1{\hyperref[sty:#1]{\Lstyle@nobm{#1}}}
\def\xLstyle#1{}
\def\xdefLstyle#1{}
\def\nxLstyle#1{\texttt{#1}}

\makeatother

\newcommand{\dipoledesc}[1]{%
  \xLdipole{#1}%
  \compitem{#1}[options](in)(out){label}%
}
\newcommand{\tripoledesc}[1]{%
  \xLtripole{#1}%
  \compitem{#1}[options](in)(center)(out){label}%
}

\newcommand{\fiberdipoledesc}[1]{%
  \xLfdipole{#1}%
  \compitem{#1}[options](in)(out){label}%
}

\newenvironment*{pssyntax}
  {\list{}{\small
     \setlength{\labelsep}{0pt}%
     \setlength{\leftmargin}{10pt}%
     \item[]}}
  {\endlist}

\newcommand{\psarglistfont}{\small}
\newenvironment*{psarglist}
  {\list{}{%
     \setlength{\labelwidth}{10pt}%
     \setlength{\labelsep}{0pt}%
     \setlength{\leftmargin}{0pt}%
     \setlength{\itemsep}{\parsep}%
     \setlength{\parsep}{0pt}%
     \renewcommand*{\makelabel}[1]{\hss\psarglistfont##1}}}
  {\endlist}

\makeatletter
\renewenvironment{theglossary}{\GlossaryParms \let\item\@idxitem \ignorespaces}{}
\makeatother
\def\psargitem#1{\item[#1]\hfill\par\nobreak}

\addtopsstyle{Fiber}{linecolor=DOrange,linewidth=1.5\pslinewidth}
\addtopsstyle{Beam}{linewidth=1.5\pslinewidth}
\EnableCrossrefs
\CodelineIndex
\OnlyDescription
\begin{document}
  \DocInput{pst-intersect.dtx}
\end{document}
%</driver>
% \fi
%
% \CheckSum{0}
%
% \CharacterTable
%  {Upper-case    \A\B\C\D\E\F\G\H\I\J\K\L\M\N\O\P\Q\R\S\T\U\V\W\X\Y\Z
%   Lower-case    \a\b\c\d\e\f\g\h\i\j\k\l\m\n\o\p\q\r\s\t\u\v\w\x\y\z
%   Digits        \0\1\2\3\4\5\6\7\8\9
%   Exclamation   \!     Double quote  \"     Hash (number) \#
%   Dollar        \$     Percent       \%     Ampersand     \&
%   Acute accent  \'     Left paren    \(     Right paren   \)
%   Asterisk      \*     Plus          \+     Comma         \,
%   Minus         \-     Point         \.     Solidus       \/
%   Colon         \:     Semicolon     \;     Less than     \<
%   Equals        \=     Greater than  \>     Question mark \?
%   Commercial at \@     Left bracket  \[     Backslash     \\
%   Right bracket \]     Circumflex    \^     Underscore    \_
%   Grave accent  \`     Left brace    \{     Vertical bar  \|
%   Right brace   \}     Tilde         \~}
%
% \GetFileInfo{pst-intersect.dtx}
%
% \newif\ifGERMAN  \GERMANfalse
% \newif\ifENGLISH \ENGLISHfalse
% \iflanguage{ngerman}{\GERMANtrue}{%
%   \iflanguage{german}{\GERMANtrue}{\GERMANfalse}}
% \iflanguage{english}{\ENGLISHtrue}{\ENGLISHfalse}
%
% \ifGERMAN
%   \newrefformat{chap}{Kap.~\ref{#1}}
%   \newrefformat{sec}{Kap.~\ref{#1}}
%   \newrefformat{l}{Zeile~\ref{#1}}
%   \newrefformat{ex}{Bsp.~\ref{#1}}
%   \newrefformat{tab}{Tab.~\ref{#1}}
%   \newrefformat{fig}{Abb.~\ref{#1}}
%   \newcommand{\PITindexMacro}{Makros}
%   \newcommand{\PITindexKeyword}{Parameter}
%   \newcommand{\PITindexEnv}{Umgebungen}
%   \newcommand{\PITindexPack}{Pakete}
%   \renewcommand*\lstlistingname{Bsp.}
% \fi
% \ifENGLISH
%   \newrefformat{chap}{Sec.~\ref{#1}}
%   \newrefformat{sec}{Sec.~\ref{#1}} 
%   \newrefformat{l}{Line~\ref{#1}}
%   \newrefformat{ex}{Ex.~\ref{#1}}
%   \newrefformat{tab}{Tab.~\ref{#1}}
%   \newrefformat{fig}{Fig.~\ref{#1}}
%   \newcommand{\PITindexMacro}{macros}
%   \newcommand{\PITindexKeyword}{parameters}
%   \newcommand{\PITindexEnv}{environments}
%   \newcommand{\PITindexPack}{packages}
%   \renewcommand*\lstlistingname{Ex.}
% \fi
%
% \makeatletter
% \renewcommand\maketitle{^^A
% \thispagestyle{empty}^^A
% \begin{titlepage}
% \begin{pspicture}(1.6in,0.685in)(10,21.7)
%   \psframe[fillstyle=solid,linecolor=lightgray,fillcolor=lightgray,linestyle=solid](0,-5.75)(21.5,10)
%   \psframe[fillstyle=solid,linecolor=Orange!85!Red,fillcolor=Orange!85!Red,linestyle=solid](0,10)(21.5,10.5)
%   \psframe[fillstyle=solid,linecolor=Orange!85!Red,fillcolor=Orange!85!Red,linestyle=solid](0,21.1)(21.5,21.2)
%   \rput[lb](3,22){\Huge\sffamily\color{Orange!65!Red}\psscalebox{2}{\textbf{PSTricks}}}
%   \rput[lb](3,14.1){\parbox{15cm}{\sffamily\RaggedRight\bfseries\huge\@title}}
%   \rput[lb](3,7.6){\parbox{13cm}{\sffamily\@date}}
%   \rput[lb](3,-2.6){\parbox[b]{17cm}{\sffamily\RaggedRight 
%     ~\hfill\makebox[7cm][l]{\ifGERMAN Paketautor:\fi\ifENGLISH Package author:\fi}\\
%     ~\hfill\makebox[7cm][l]{^^A
%       \bfseries\tabular[t]{@{}l@{}}\@author\endtabular}}}
%  \rput[C](11,4){\bgImage}
%  \end{pspicture}^^A
% \end{titlepage}}
% \makeatother
% 
% \ifGERMAN
%   \title{\texttt{pst-intersect}\\ Berechnen der Schnittpunkte beliebiger Kurven\\[0.5ex] \small \fileversion}
%   \hypersetup{pdftitle={Berechnen der Schnittpunkte beliebiger Kurven}}
% \fi
% \ifENGLISH
%   \title{\texttt{pst-optexp}\\ Intersecting arbitrary curves\\[0.5ex] \small \fileversion}
%   \hypersetup{pdftitle={Intersecting arbitrary curves}}
% \fi
% \author{Christoph Bersch}
% \date{\filedate}
% \def\bgImage{}
%
% \maketitle
% 
% \clearpage
% \tableofcontents
% \clearpage
% 
% \ifGERMAN
%   \chapter{Einführung}
% \fi
% \ifENGLISH
%   \chapter{Introduction}
% \fi
% 
% \ifGERMAN
%   \section{Über das Paket}
%
%   \LPack{pst-intersect} ist ein PSTricks-Paket zur
%   Berechnung der Schnittpunkte von Bézier-Kurven und beliebigen
%   Postscript-Pfaden.
% \fi 
% \ifENGLISH
%   \section{About the package}
%   \LPack{pst-intersect} is a PSTricks package to calculate
%   the intersections of Bezier curves and arbitrary Postscript paths.
% \fi
%
% \ifGERMAN
%   \section{Anforderungen}
%   \LPack{pst-intersect} benötigt aktuelle Versionen der
%   Pakete \LPack{pstricks}, \LPack{pst-node} und \LPack{pst-func}.
%
%   Alle PSTricks-Pakete machen regen Gebrauch von der Postscript-Sprache, so
%   dass der typische Arbeitsfluss \opt{latex}, \opt{dvips} und
%   ggf. \opt{ps2pdf} umfasst. Es gibt viele alternative Methoden um die
%   Dokumente zu
%   kompilieren.\fnurl{http://tug.org/PSTricks/main.cgi?file=pdf/pdfoutput}
% \fi
% \ifENGLISH
%   \section{Requirements}
%   \LPack{pst-intersect} requires recent versions of
%   \LPack{pstricks}, \LPack{pst-node}, and \LPack{pst-func}.
%
%   All PSTricks package rely heavily on the Postscript language so that the
%   typical workflow involves \opt{latex}, \opt{dvips}, and \opt{ps2pdf}. Of
%   course there are several alternative ways to compile your
%   documents.\fnurl{http://tug.org/PSTricks/main.cgi?file=pdf/pdfoutput} 
% \fi
%
% \ifGERMAN
%   \section{Verbreitung und Installation}
%   %Dieses Paket ist auf
%   %CTAN\fnurl{http://mirror.ctan.org/help/Catalogue/entries/pst-intersect.html}
%   %erhältlich.^^A und in \TeX Live and MiK\TeX{} enthalten.
% 
%   Das \LPack{pst-intersect}-Paket umfasst die zwei Hauptdateien
%   \texttt{pst-intersect.ins} und \texttt{pst-intersect.dtx}. Durch Aufrufen
%   von \texttt{tex pst-intersect.ins} werden die drei folgenden
%   Dateien erzeugt:
%   \begin{itemize}
%   \item \texttt{pst-intersect.pro}: die Postscript Prologdatei
%   \item \texttt{pst-intersect.sty}: die \LaTeX-Stildatei
%   \item \texttt{pst-intersect.tex}: die \TeX-Datei
%   \end{itemize}
%   Speichern Sie diese Dateien in einem Verzeichnis der Teil Ihres
%   lokalen \TeX-Baums ist.
% 
%   Vergessen Sie nicht \texttt{texhash} aufzurufen um den Baum zu
%   aktualisieren. MiK\TeX{}-Benutzer müssen die Dateinamen-Datenbank
%   (FNDB) aktualisieren.
% 
%   Detailliertere Information finden Sie in der Dokumentation Ihrer
%   \LaTeX-Distribution über die Installation in den lokalen
%   \TeX{}-Baum.
% \fi
% \ifENGLISH
%   \section{Distribution and installation}
%   This package is available on
%   CTAN\fnurl{http://mirror.ctan.org/help/Catalogue/entries/pst-intersect.html}.
%   ^^A and is included in \TeX Live and MiK\TeX.
% 
%   The \LPack{pst-intersect} package consists of the two main files
%   \texttt{pst-intersect.ins} and \texttt{pst-intersect.dtx}. By running \texttt{tex
%   pst-intersect.ins} the following derived files are generated:
%   \begin{itemize}
%   \item \texttt{pst-intersect.pro}: the Postscript prolog file
%   \item \texttt{pst-intersect.sty}: the \LaTeX{} style file
%   \item \texttt{pst-intersect.tex}: the \TeX{} file
%   \end{itemize}
%   Save the files in a directory which is part of your local \TeX{} tree.
% 
%   Do not forget to run \texttt{texhash} to update this tree. For MiK\TeX{}
%   users, do not forget to update the file name database (FNDB).
% 
%   For more detailed information see the documentation of your personal
%   \LaTeX{} distribution on installing packages to your local \TeX{}
%   system.
% \fi
%
% \ifGERMAN\section{Lizenz}\fi
% \ifENGLISH\section{License}\fi
% \ifGERMAN
% Es wird die Erlaubnis gewährt, dieses Dokument zu kopieren, zu verteilen
% und\slash oder zu modifizieren, unter den Bestimmungen der \LaTeX{} Project
% Public License, Version
% 1.3c.\fnurl{http://www.latex-project.org/lppl.txt}. Dieses
% Paket wird vom Autor betreut (author-maintained).
% \fi
% \ifENGLISH
% Permission is granted
% to copy, distribute and\slash or modify this software under the terms of the
% \LaTeX{} Project Public License, version
% 1.3c.\fnurl{http://www.latex-project.org/lppl.txt} This
% package is author-maintained.
% \fi
%
% \ifGERMAN
%   \section{Danksagung}
% \fi
% \ifENGLISH
%   \section{Acknowledgements}
% \fi
% \ifGERMAN
% Ich danke Marco Cecchetti, dessen
% \opt{lib2geom}-Bibliothek\fnurl{http://lib2geom.sourceforge.net/}
% mir als Vorlage für einen Großteil des Postscript-Kodes für den
% Bézier-Clipping-Algorithmus diente. Außerdem gilt mein Dank William
% A. Casselman, für seine Erlaubnis, den Quicksort-Kode und den Kode zur
% Berechung der konvexen Hüllen aus seinem Buch «Mathematical
% Illustration» verwenden zu
% dürfen\fnurl{http://www.math.ubc.ca/~cass/graphics/text/www/}. Der
% Dokumentationsstil ist eine Mischung aus der \opt{pst-doc} Klasse
% (Herbert Voß) und dem \opt{ltxdockit} Paket für die \opt{biblatex}
% Dokumentation (Philipp Lehmann).
% \fi
% \ifENGLISH
% I thank Marco Cecchetti, for his
% \opt{lib2geom}-library\fnurl{http://lib2geom.sourceforge.net/} from
% which I derived great parts of the Postscript code for the Bézier
% clipping algorithm. Also I want to thank William A. Casselman for the
% Postscript code of the quicksort procedure and the procedure for
% calculating the convex hull from his book «Mathematical
% Illustration»\fnurl{http://www.math.ubc.ca/~cass/graphics/text/www/},
% and the permission to use it. The documentation style is a mixture of
% the \opt{pst-doc} class (Herbert Voß) and the \opt{ltxdockit} package
% for the \opt{biblatex} documentation (Philipp Lehman).
% \fi
%
%
% \ifGERMAN
% \chapter{Benutzung}
% \fi
% \ifENGLISH
% \chapter{Usage}
% \fi
%
% \ifGERMAN
% Das \LPack{pst-intersect}-Paket kann Schnittpunkte von beliebigen
% Postscript-Pfaden berechnen. Diese setzen sich nur aus drei primitiven
% Operation zusammen: Linien (\opt{lineto}), Bézier-Kurven dritter
% Ordnung (\opt{curveto}) und Sprüngen (\opt{moveto}). Speziellere
% Konstruktionen, wie z.B. Kreise werden intern zu
% \opt{curveto}-Anweisungen umgewandelt. Über diese Kommandos hinaus
% kann \LPack{pst-intersect} auch Bézier-Kurven beliebiger Ordnung
% verwenden. Das diese keine primitiven Postscript-Pfadelemente
% darstellen, müssen sie gesondert behandelt werden.
%
% Der allgemeine Arbeitsablauf besteht darin eine oder mehrere Kurven
% oder Pfade zu speichern, und danach die Schnittpunkte zu
% berechnen. Anschließend können die Schnittpunkte als normale
% PSTricks-Knoten verwendet werden, oder Abschnitte der Kurven und Pfade
% nachgezogen werden (z.B. zwischen zwei Schnittpunkten). 
% \fi
%
% \ifGERMAN
% \section{Speichern von Pfaden und Kurven}
% \fi
% \ifENGLISH
% \section{Saving paths and curves}
% \fi
%
% \ifGERMAN
% 
% \fi
%
% \begin{ltxsyntax}
%   \cmditem{savepath}[options]{curvename}{commands} 
%   
%   \ifGERMAN
%   Speichert den gesamten Pfad, der durch \prm{commands} erstellt wird,
%   unter Verwendung des Namens \prm{curvename}. Das Makro funktioniert
%   genauso wie \cs{pscustom}, und kann daher auch nur die darin
%   erlaubten Kommandos verarbeiten.
%
%   In den Standardeinstellungen wird der entsprechende Pfad auch gleich
%   gezeichnet, was mit \prm{options} beeinflusst werden kann. Mit
%   \opt{linestyle=none} wird das unterbunden.
%   \fi
% \end{ltxsyntax}
%\iffalse
%<*ignore>
%\fi
\begin{LTXexample}
\begin{pspicture}(3,2)
  \savepath[linecolor=DOrange]{MyPath}{%
    \pscurve(0,2)(0,0.5)(3,1)
  }%
\end{pspicture}
\end{LTXexample}
%\iffalse
%</ignore>
%\fi
%
% \begin{ltxsyntax}
%   \cmditem{savebezier}[options]{curvename}($X_0$)\ldots(\prm{$X_n$})
%
%   \ifGERMAN 
%   Die Postscript-Sprache unterstützt nur Bézier-Kurven dritter
%   Ordnung. Mit dem Makro \Lcs{savebezier} können Bézier-Kurven
%   beliebiger Ordnung definiert werden. Die angegebenen Knoten sind die
%   Kontrollpunkte der Kurve, für eine Kurve $n$-ter Ordnung werden
%   $(n+1)$ Kontrollpunkte benötigt. Die Darstellung der Kurve erfolgt
%   mit dem Makro \cs{psBezier} aus dem \LPack{pst-func}-Paket, welches
%   nur Kurven bis neunter Ordnung darstellen kann.
%   \fi
% \end{ltxsyntax}
%\iffalse
%<*ignore>
%\fi
\begin{LTXexample}
\begin{pspicture}(3,2)
  \savebezier[showpoints]{MyBez}(0,0)(0,1)(1,2)(3,2)(1,0)(3,0)
\end{pspicture}
\end{LTXexample}
%\iffalse
%</ignore>
%\fi
%
% \ifGERMAN
% \section{Schnittpunkte berechnen}
% \fi
% \ifENGLISH
% \section{Calculating intersections}
% \fi
%
% \begin{ltxsyntax}
%   \cmditem{psintersect}{curveA}{curveB}
% \end{ltxsyntax}
% 
% \ifGERMAN
% Nachdem Sie nun Pfade oder Kurven gespeichert haben, können Sie deren
% Schnittpunkte berechnen. Das geschieht mit dem Makro
% \Lcs{psintersect}. Dieses benötigt als Argumente zwei Namen von Pfaden
% oder Kurven (Das Argument \prm{curvename} der beiden \cs{save*}
% Makros).
% \fi
%
%\iffalse
%<*ignore>
%\fi
\begin{LTXexample}
\begin{pspicture}(3,2)
  \savepath[linecolor=DOrange]{MyPath}{\pscurve(0,2)(0,0.5)(3,1)}
  \savebezier{MyBez}(0,0)(0,1)(1,2)(3,2)(1,0)(3,0)
  \psintersect[showpoints]{MyPath}{MyBez}
\end{pspicture}
\end{LTXexample}
%\iffalse
%</ignore>
%\fi
%
% \ifGERMAN
% Der PSTricks-Parameter \opt{showpoints} steuert dabei, ob die Schnittpunkte angezeigt werden.
% \fi
%
% \begin{optionlist}
% \valitem[@tmp]{name}{string}
% \ifGERMAN
% Die berechneten Schnittpunkte können unter einem hier angegebenen
% Namen gespeichert und zu einem späteren Zeitpunkt verwendet werden
% (siehe \prettyref{sec:psshowcurve}). 
% \fi
%
% \boolitem[true]{saveintersections}
% \ifGERMAN
% Ist dieser Schalter gesetzt, dann werden die Schnittpunkte als
% PSTricks-Knoten unter den Namen \prm{name}1, \prm{name}2 \ldots
% gespeichert. Die Nummerierung erfolgt aufsteigend nach dem Wert der
% $x$-Koordinate.
% \fi
%
%\iffalse
%<*ignore>
%\fi
\begin{LTXexample}
\begin{pspicture}(5,5)
  \savebezier[linecolor=DOrange]{A}%
             (0,0)(0,5)(5,5)(5,1)(1,1.5)
  \savebezier{B}(0,5)(0,0)(5,0)(5,5)(0,2)
  \psintersect[name=C, showpoints]{A}{B}
  \uput[150](C1){1}
  \uput[85](C2){2}
  \uput[90](C3){3}
  \uput[-20](C4){4}
\end{pspicture}
\end{LTXexample}
%\iffalse
%</ignore>
%\fi
% \end{optionlist}
%
% \ifGERMAN
% \section{Darstellung gespeicherter Pfade}
% \fi
% \ifENGLISH
% \section{Visualization of saved paths}
% \fi
%
% \begin{ltxsyntax}
%   \cmditem{psshowcurve}[options]{curvename}
%
% \ifGERMAN
% Gespeicherte Pfade und Kurven können mit diesem Makro nachträglich gezeichnet werden.
% \fi
% \end{ltxsyntax}
%
%\iffalse
%<*ignore>
%\fi
\begin{LTXexample}
\begin{pspicture}(2,2)
  \savepath{Circle}{\pscircle(1,1){1}}
  \psshowcurve[linestyle=dashed, linecolor=green]{Circle}
\end{pspicture}
\end{LTXexample}
%\iffalse
%</ignore>
%\fi
%
% \begin{optionlist}
% \numitem{tstart}
% \numitem{tstop}
%
% \ifGERMAN Unter Verwendung dieser beiden Parameter können auch
% Abschnitte von Pfaden und Kurven gezeichnet werden. Bei Bézier-Kurven
% ist der Parameterbereich $[0, 1]$, wobei $0$ dem Anfang der Kurve,
% also dem ersten bei \Lcs{savebezier} angegebenen Knoten entspricht.
% \fi
%
%\iffalse
%<*ignore>
%\fi
\begin{LTXexample}
\begin{pspicture}(5,5)
  \psset{showpoints}
  \savebezier{B}(0,5)(0,0)(5,0)(5,5)(0,2)
  \psshowcurve[linestyle=dashed, linecolor=blue!50,
               tstart=0, tstop=0.5]{B}
\end{pspicture}
\end{LTXexample}
%\iffalse
%</ignore>
%\fi
%
% \medskip
% \ifGERMAN 
% Pfaden können aus mehr als einem Abschnitt bestehen, der Bereich ist
% also $[0, n]$, wobei $n$ die Anzahl der Pfadabschnitte ist. Dabei ist
% zu beachten, dass z.B. \cs{pscurve}-Pfade oder auch Kreise und
% Kreisbögen aus mehreren Abschnitten bestehen.
% \fi
%
%\iffalse
%<*ignore>
%\fi
\begin{LTXexample}
\begin{pspicture}(2,2)
  \savepath[linestyle=none]{Circle}{\pscircle(1,1){1}}
  \psshowcurve[tstart=0, tstop=1, linecolor=green]{Circle}
  \psshowcurve[tstart=2, tstop=3, linecolor=red]{Circle}
  \psshowcurve[tstart=1.25, tstop=1.75, linecolor=blue]{Circle}
\end{pspicture}
\end{LTXexample}
%\iffalse
%</ignore>
%\fi
%
% \end{optionlist}
%
% \ifGERMAN
% Beachten Sie, dass in der Version 0.1 die Reihenfolge von
% \Lkeyword{tstart} und \Lkeyword{tstop} noch keine Rolle spielt. Für
% kommende Versionen ist geplant, dass bei \Lkeyword{tstart}
% \textgreater{} \Lkeyword{tstop} die Pfadrichtung umgekehrt wird. Zur
% Zeit werden auch noch keine Pfeile für \Lcs{psshowcurve} unterstützt.
% \fi
%
% \ifGERMAN
% \section{Darstellung gespeicherter Schnitte}
% \fi
% \ifENGLISH
% \section{Visualization of saved intersections}
% \fi
%
% \begin{ltxsyntax}
% \cmditem{psshowcurve}[options]{intersection}{curvename}
% \end{ltxsyntax}
% 
% \begin{optionlist}
% \numitem{istart}
% \numitem{istop}
% \end{optionlist}
% 
% \ifGERMAN
%
% \fi
%\iffalse
%<*ignore>
%\fi
\begin{LTXexample}
\begin{pspicture}(5.2,5.2)
\savebezier[linewidth=0.5\pslinewidth, linestyle=dashed]{A}(0,0)(0,5)(5,2)(5,5)
\savebezier[linewidth=0.5\pslinewidth, linestyle=dashed]{B}(0,2.5)(2.5,2.5)(4.5, 3)(2,4)
\psintersect[linecolor=green!70!black, name=C]{A}{B}
\psshowcurve[linecolor=red, istart=1, istop=2]{C}{A}
\psshowcurve[linecolor=blue, istart=1, istop=2]{C}{B}
\end{pspicture}
\end{LTXexample}
%\iffalse
%</ignore>
%\fi
%
% \ifGERMAN
% \chapter{Beispiele}
% \fi
% \ifENGLISH
% \chapter{Examples}
% \fi
%\iffalse
%<*ignore>
%\fi
\begin{LTXexample}
\begin{pspicture}(5,5)
  \savebezier{A}(0,0)(0,5)(5,5)(5,1)(1,1.5)
  \multido{\i=100+-20,\r=1+-0.2}{5}{%
    \psshowcurve[linecolor=red!\i, tstop=\r, arrows=-|, showpoints]{A}
  }%
\end{pspicture}
\end{LTXexample}
%\iffalse
%</ignore>
%\fi 
%
% \appendix
%
% \ifGERMAN
% \chapter{Versionsgeschichte}
%
% Diese Versionsgeschichte ist eine Liste von Änderungen, die für den Nutzer des
% Pakets von Bedeutung sind. Änderungen, die eher technischer Natur sind und für
% den Nutzer des Pakets nicht relevant sind und das Verhalten des Pakets nicht
% ändern, werden nicht aufgeführt. Wenn ein Eintrag der Versionsgeschichte ein
% Feature als \emph{improved} oder \emph{extended} bekannt gibt, so bedeutet
% dies, dass eine Modifikation die Syntax und das Verhalten des Pakets nicht
% beeinflusst, oder das es für ältere Versionen kompatibel ist. Einträge, die
% als \emph{deprecated}, \emph{modified}, \emph{renamed}, oder \emph{removed}
% deklariert sind, verlangen besondere Aufmerksamkeit. Diese bedeuten, dass eine
% Modifikation Änderungen in existierenden Dokumenten mit sich ziehen kann. 
% \fi
% \ifENGLISH
% \chapter{Revision history}
%
% This revision history is a list of changes relevant to users of this
% package. Changes of a more technical nature which do not affect the
% user interface or the behavior of the package are not included in the
% list. If an entry in the revision history states that a feature has
% been \emph{improved} or \emph{extended}, this indicates a modification
% which either does not affect the syntax and behavior of the package or
% is syntactically backwards compatible (such as the addition of an
% optional argument to an existing command). Entries stating that a
% feature has been \emph{deprecated}, \emph{modified}, \emph{fixed},
% \emph{renamed}, or \emph{removed} demand attention. They indicate a
% modification which may require changes to existing documents.
% \fi
%
% \begin{changelog}
%\patchcmd{\release}{\setlength{\itemsep}{0pt}}{\setlength{\itemsep}{0pt}\setlength{\parsep}{0pt}}{}{}
%   \begin{release}{1.0}{2014-xx-xx}
%   \item First CTAN version
%   \end{release}
% \end{changelog}
%
% \StopEventually{}
%
%   \begin{otherlanguage}{english}
%    \printindex[idx]
%  \end{otherlanguage}
%
% \chapter{The \LaTeX\ wrapper}
%<*stylefile>
%    \begin{macrocode}
\NeedsTeXFormat{LaTeX2e}[1999/12/01]
\ProvidesPackage{pst-intersect}%
   [2014/02/06 v0.1alpha package wrapper for pst-intersect.tex]
\RequirePackage{pstricks}
\RequirePackage{pst-xkey}
\RequirePackage{pst-node}
\RequirePackage{multido}
\RequirePackage{pst-func}
\input{pst-intersect.tex}
\IfFileExists{pst-intersect.pro}{%
    \ProvidesFile{pst-intersect.pro}
      [2014/02/06 PostScript prologue file]
      \@addtofilelist{pst-intersect.pro}}{}%
%    \end{macrocode}
%</stylefile>
%
% \chapter{The \TeX\ implementation}
%
%<*texfile>
%    \begin{macrocode}
\csname PSTintersectLoaded\endcsname
\let\PSTintersectLoaded\endinput

\ifx\PSTricksLoaded\endinput\else\input pstricks.tex \fi
\ifx\PSTXKeyLoaded\endinput\else \input pst-xkey.tex \fi
\ifx\PSTnodesLoaded\endinput\else\input pst-node.tex \fi
\ifx\PSTfuncLoaded\endinput\else \input pst-func.tex \fi

\edef\PstAtCode{\the\catcode`\@} \catcode`\@=11\relax

\pst@addfams{intersect}
\pstheader{pst-intersect.pro}

\def\pst@intersectdict{tx@IntersectDict begin }
\def\PIT@dict#1{\pst@intersectdict #1 end}
\def\PIT@Verb#1{\pst@Verb{\PIT@dict{#1} }}%

\def\savebezier{\pst@object{savebezier}}
\def\savebezier@i#1{%
  \begin@OpenObj
    \PIT@checkname{#1}%
    \addto@pscode{ /\PIT@name{#1} }%
    \pst@getcoors[\savebezier@ii%]
}%
\def\savebezier@ii{%
  \addto@pscode{%
    % only 10 points allowed, remove the rest
    counttomark 20 gt { counttomark 20 sub { pop } repeat } if
    % reverse the point order
    counttomark -2 4 { 2 roll } for
    %\addto@pscode{%
    %counttomark 2 add -1 roll pop % remove the path name
    counttomark 2 idiv 1 sub dup 9 gt { pop 9 } if
    \psk@plotpoints\space exch
    \txFunc@BezierCurve
    \ifshowpoints \txFunc@BezierShowPoints \else pop \fi
    tx@FuncDict begin Points aload pop end
  }%
  \let\use@pscode\PIT@use@pscode
  \end@OpenObj
  \PIT@Verb{%
    [ count 1 sub 1 roll ] ArrayToPointArray def 
  }%
\ignorespaces}%
\define@key[psset]{intersect}{tstart}{%
  \pst@checknum{#1}\PIT@key@tstart
}
\define@key[psset]{intersect}{tstop}{%
  \pst@checknum{#1}\PIT@key@tstop
}
\define@key[psset]{intersect}{istart}{%
  \pst@checknum{#1}\PIT@key@istart
}
\define@key[psset]{intersect}{istop}{%
  \pst@checknum{#1}\PIT@key@istop
}
\define@key[psset]{intersect}{name}{%
  \def\PIT@key@name{#1}%
}%
\newif\PIT@saveintersections
\define@boolkey[psset]{intersect}[PIT@]{saveintersections}[true]{}
\psset[intersect]{%
  tstart=-1,
  tstop=-1,
  istart=-1,
  istop=-1,
  name={},
  saveintersections
}%
\def\PIT@use@pscode{%
  \pstverb{%
    \pst@dict
    \tx@STP
    \pst@newpath
    \psk@origin
    \psk@swapaxes
    \pst@code
    end
    count /ocount exch def
  }%
  \gdef\pst@code{}%
}%
\let\PIT@pst@stroke@orig\pst@stroke
\def\PIT@save@path{%
  \PIT@pst@stroke@orig
  \addto@pscode{%
    clear mark
    { /movetype counttomark 3 roll }
    { /linetype counttomark 3 roll }
    { /curvetype counttomark 7 roll }{} pathforall 
    counttomark 1 add -1 roll pop count }%
}%
\def\PIT@name@default{@tmp}%
\def\PIT@name#1{PIT@#1}%
\def\PIT@checkname#1{%
  \ifx\@empty#1\@empty
    \@pstrickserr{Unexpected empty argument!}\@ehpb
  \fi
}%
\def\savepath{\pst@object{savepath}}%
\long\def\savepath@i#1#2{%
  \begin@SpecialObj
    \PIT@checkname{#1}%
    \let\pst@stroke\PIT@save@path
    \let\use@pscode\PIT@use@pscode
    \pscustom{#2}%
    \PIT@Verb{%
      /\PIT@name{#1}
      [ 3 -1 roll 2 add 2 roll ] def }%
  \end@SpecialObj
}%
\def\psshowcurve{\pst@object{psshowcurve}}%
\def\psshowcurve@i#1{%
  \addbefore@par{plotpoints=200}%
   \@ifnextchar\bgroup
     {\PIT@traceintersection{#1}}%
     {\PIT@tracecurve{#1}}%
}%
\def\PIT@tracecurve#1{%
  \PIT@checkname{#1}%
  \begin@OpenObj
    \addto@pscode{%
      \pst@intersectdict
        \PIT@name{#1} dup IsPath {
          \PIT@key@tstart\space\PIT@key@tstop\space
          ShowPathPortion
        }{
          [exch dup
          \PIT@key@tstart\space\PIT@key@tstop\space 
          dup 0 lt { pop 1 } if
          ToUnitInterval Portion 
          { aload pop } forall
          counttomark 2 sub 2 idiv
          \psk@plotpoints
          exch
          \txFunc@BezierCurve
          \ifshowpoints \txFunc@BezierShowPoints \else pop \fi
        } ifelse
      end
    }%
  \end@OpenObj
}%
\def\PIT@traceintersection#1#2{%
  \PIT@checkname{#2}%
  \begin@OpenObj
    \addto@pscode{%
      \pst@intersectdict
        \ifx\\#1\\%
          /\PIT@name{\PIT@name@default}
        \else
          /\PIT@name{#1}
        \fi 
        dup currentdict exch known not {
          \ifx\\#1\\%
            (You haven't defined an intersection!) ==
          \else
            (You haven't defined the intersection '#1') ==
          \fi
        } if 
        load
        dup dup type /dicttype eq exch /\PIT@name{#2} known and not {
          (You haven't defined the intersection '#2') ==
        } if
        dup /\PIT@name{#2} get
        exch /\PIT@name{#2}@t get
        dup length \PIT@key@istart\space ge 0 \PIT@key@istart\space lt and {
          dup \PIT@key@istart\space cvi 1 sub get
        } {
          \PIT@key@tstart
        } ifelse
        exch % [curve] t_istart|tstart [ts]
        %
        dup length \PIT@key@istop\space ge 0 \PIT@key@istop\space lt and {
          \PIT@key@istop\space cvi 1 sub get
        } {
          pop \PIT@key@tstop
        } ifelse
        2 copy gt { exch } if 
        3 -1 roll dup
        IsPath {
          3 1 roll
          ShowPathPortion
        }{
          [ exch dup 5 -2 roll
          dup 0 lt { pop 1 } if
          ToUnitInterval Portion 
          { aload pop } forall
          counttomark 2 sub 2 idiv
          \psk@plotpoints
          exch
          \txFunc@BezierCurve
          \ifshowpoints \txFunc@BezierShowPoints \else pop \fi
        } ifelse
      end
    }%
  \end@OpenObj
}%
%
% \begin{macro}{\psintersect}
%    \begin{macrocode}
\def\psintersect{\pst@object{psintersect}}
\def\psintersect@i#1#2{%
  \PIT@checkname{#1}%
  \PIT@checkname{#2}%
  \begin@SpecialObj
  \def\PIT@@name{%
    \ifx\PIT@key@name\@empty
      \PIT@name{\PIT@name@default}
    \else
      \PIT@name{\PIT@key@name} 
    \fi}%
  \PIT@Verb{%
    currentdict /\PIT@name{#1} known not {
      (You haven't defined the curve or path '#1') ==
    } if
    currentdict /\PIT@name{#2} known not {
      (You haven't defined the curve or path '#2') ==
    } if
    \PIT@name{#1} \PIT@name{#2}
    \PIT@name{#1} IsPath {
      \PIT@name{#2} IsPath {
        IntersectPaths
      }{
        IntersectPathCurve
      } ifelse
    }{
      \PIT@name{#2} IsPath {
        IntersectCurvePath
      }{
        IntersectBeziers
        4 copy LoadIntersectionPoints 5 1 roll
      } ifelse
    } ifelse
    /\PIT@@name\space /\PIT@name{#1} /\PIT@name{#2} 8 3 roll 
    SaveIntersection
  }%
  \ifPIT@saveintersections
    \pst@Verb{%
      \pst@intersectdict 
        \PIT@@name\space /Points get 
        ArrayToPointArray
      end
      tx@NodeDict begin 
        dup length 1 1 3 -1 roll {
          2 copy 1 sub get cvx
          false 3 -1 roll (N@\PIT@key@name) exch 20 string cvs 
          \pst@intersectdict strcat end cvn
          10 {InitPnode} /NodeScale {} def NewNode
        } for
      end
      pop
    }%
  \fi
  \ifshowpoints
    \addto@pscode{%
      \pst@intersectdict
        [ \PIT@@name\space /Points get aload pop 
      end 
    }%
    \psdots@ii
  \else
    \end@SpecialObj
  \fi
}%
%    \end{macrocode}
% \end{macro}
% 
%    \begin{macrocode}
\catcode`\@=\PstAtCode\relax
%    \end{macrocode}
%</texfile> 
%
% \chapter{The Postscript header file}
% \makeatletter
%^^A Copied this definition from doc.sty and changed it not to add a
%^^A backslash to the Postscript procedure name in the index.
% \def\SpecialIndex@#1#2{%
%    \@SpecialIndexHelper@#1\@nil
%    \def\@tempb{ }%
%    \ifcat \@tempb\@gtempa
%       \special@index{\quotechar\space\actualchar
%                      \string\verb\quotechar*\verbatimchar
%                      \quotechar\space\verbatimchar#2}%
%    \else
%      \def\@tempb##1##2\relax{\ifx\relax##2\relax
%           \def\@tempc{\special@index{\quotechar##1\actualchar
%                       \string\verb\quotechar*\verbatimchar
%                       \quotechar##1\verbatimchar#2}}%
%         \else
%           \def\@tempc{\special@index{##1##2\actualchar
%                        \string\verb\quotechar*\verbatimchar##1##2\verbatimchar#2}}%
%         \fi}%
%      \expandafter\@tempb\@gtempa\relax
%      \@tempc
%    \fi}
% \makeatother
%
%<*prolog>
%    \begin{macrocode}
/tx@IntersectDict 200 dict def
tx@IntersectDict begin
%    \end{macrocode}
% These are some helper procedures for vector operations.
%
% \begin{macro}{VecAdd}
% Addition of two vectors.
% \begin{pssyntax}
%   \PSvar{Xa Ya Xb Yb} \PSop{VecAdd} \PSvar{Xa+Xb Ya+Yb}
% \end{pssyntax}
%    \begin{macrocode}
/VecAdd {
    3 -1 roll add 3 1 roll add exch
} bind def
%    \end{macrocode}
% \end{macro}
%
% \begin{macro}{VecSub}
% Subtraction of two vectors.
% \begin{pssyntax}
%   \PSvar{Xa Ya Xb Yb} \PSop{VecSub} \PSvar{Xa-Xb Ya-Yb}
% \end{pssyntax}
%    \begin{macrocode}
/VecSub {
    neg 3 -1 roll add 3 1 roll neg add exch
} bind def
%    \end{macrocode}
% \end{macro}
% 
% \begin{macro}{VecScale}
% Scale a vector by a factor \PSvar{fac}.
% \begin{pssyntax}
%   \PSvar{Xa Ya fac} \PSop{VecScale} \PSvar{fac}$\cdot$\PSvar{Xa} \PSvar{fac}$\cdot$\PSvar{Ya}
% \end{pssyntax}
%    \begin{macrocode}
/VecScale {
  dup 4 -1 roll mul 3 1 roll mul
} bind def
%    \end{macrocode}
% \end{macro}
%
% \begin{macro}{ToVec}
%   Convert two numbers to a procedure holding the two values. This
%   representation is used to save coordinate values of nodes and vectors.
%   \begin{pssyntax}
%     \PSvar{X Y} \PSop{ToVec} \PSarray{X Y}
%   \end{pssyntax}
%    \begin{macrocode}
/ToVec {
    [ 3 1 roll ]
} bind def
%    \end{macrocode}
% \end{macro}
%
% \PSvar{MaxPrecision} gives the precision of the curve parameter t for the
% intersection. This shouldn't be lower than $10^{-6}$, because
% PostScript uses single precision.
%    \begin{macrocode}
/MaxPrecision 1e-6 def
%    \end{macrocode}
%
% \PSvar{Epsilon} gives the allowed relative error of the intersection point. 
%    \begin{macrocode}
/Epsilon 1e-4 def
%    \end{macrocode}
% 
% The threshold for curve subdivision, see below.
%    \begin{macrocode}
/MinClippedSizeThreshold 0.8 def
%    \end{macrocode}
%
% The predefined intervals for the subdivision of the curves.
%    \begin{macrocode}
/H1Interval [0 0.5] def
/H2Interval [0.5 MaxPrecision add 1] def
%    \end{macrocode}
% 
% \begin{macro}{IntersectBeziers}
%   The main procedure, which computes the intersection of two bezier
%   curves of arbitrary order.  This, and most of the following
%   procedures operate on curves, which are stored as arrays of points,
%   the points are also arrays with two elements -- \PSvar{X} and
%   \PSvar{Y}. A Bezier curve of $n$-th order is then givesn by
%   \PSarray{\PSarray{X0 Y0} \PSarray{X1 Y1} \ldots \PSarray{XN YN}}.
%
% \begin{pssyntax}
%   \PSarray{curveA} \PSarray{curveB} \PSop{IntersectBeziers} 
%   \PSarray{curveA} \PSarray{tA} \PSarray{curveB} \PSarray{tB}
% \end{pssyntax}
%    \begin{macrocode}
/IntersectBeziers {
  2 copy length 2 eq exch length 2 eq and {
    IntersectLines
  }{ 
    2 copy [0 1] [0 1] IterateIntersection
  } ifelse
  3 -1 roll exch
} bind def
%    \end{macrocode}
% \end{macro}
%
% \begin{macro}{IntersectLines}
% 
%   \begin{pssyntax}
%     \PSarray{lineA} \PSarray{lineB} \PSop{IntersectLines}
%     \PSarray{lineA} \PSarray{tA} \PSarray{lineB} \PSarray{tB}
%   \end{pssyntax}
%    \begin{macrocode}
/IntersectLines {
  (IntersectLines) DebugBegin
  2 copy
  exch { aload pop } forall 5 -1 roll { aload pop } forall
  8 -2 roll 2 copy 10 4 roll 4 2 roll 2 copy 6 2 roll 10 2 roll
  VecSub
  6 2 roll 4 2 roll VecSub
  8 4 roll 4 2 roll VecSub % X3-X4 Y3-Y4 X2-X1 Y2-Y1 X3-X1 Y3-Y1 % b1 b2 a1 a2 c1 c2
  6 copy 12 -4 roll 
  neg 4 -1 roll mul 3 1 roll mul add
  dup 0 eq {
    % no intersections
    9 { pop } repeat [] []
  } {
    dup 10 1 roll 5 1 roll
     4 -1 roll mul 3 1 roll mul sub exch div
     6 1 roll 4 -1 roll mul 3 1 roll mul sub exch div
     2 copy 2 copy 0 ge exch 0 ge and 3 1 roll 1 le exch 1 le and and {
       [ exch ] exch [ exch ]
     } {
       pop pop [] []
     } ifelse
  } ifelse
  DebugEnd
} bind def
%    \end{macrocode}
% \end{macro}
% \begin{macro}{IntersectPaths}
%   \begin{pssyntax}
%     \PSarray{pathA} \PSarray{pathB} \PSop{IntersectPaths}
%     \PSarray{intersections} \PSarray{pathA} \PSarray{tA} \PSarray{pathB} \PSarray{tB}
%   \end{pssyntax}
%    \begin{macrocode}
/IntersectPaths {
  (IntersectPaths) DebugBegin
  6 dict begin 
    2 copy exch PreparePath dup length /nA exch def 
    exch PreparePath dup length /nB exch def
    /isect [] def
    /tA [] def /tB [] def
    { % [pathA] [Bi]
      /nB nB 1 sub def
      exch dup 3 1 roll % [pathA] [Bi] [pathA]
      {
        /nA nA 1 sub def
        exch dup 3 1 roll % [pathA] [Bi] [Aj] [Bi]
        IntersectBeziers % [curveA] [tA] [curveB] [tB]
        4 copy LoadIntersectionPoints
        [ exch isect aload pop ] /isect exch def
        exch pop 3 -1 roll pop
        [ tB aload length 2 add -1 roll TArray { nB add } forall ] /tB exch def
        [ tA aload length 2 add -1 roll TArray { nA add } forall ] /tA exch def
      } forall
      pop % remove [Bi]
      dup length /nA exch def
    } forall
    pop % remove [pathA]
    [ isect { aload pop } forall ] 3 1 roll tA exch tB
    % [intersections] [pathA] [tA] [pathB] [tB]
  end
  DebugEnd
} bind def
%    \end{macrocode}
% \end{macro}
%
% \begin{macro}{IntersectCurvePath}
%   \begin{pssyntax}
%     \PSarray{curveA} \PSarray{pathB} \PSop{IntersectCurvePath}
%     \PSarray{intersections} \PSarray{curveA} \PSarray{tA} \PSarray{pathB} \PSarray{tB}
%   \end{pssyntax}
%    \begin{macrocode}
/IntersectCurvePath {
  (IntersectCurvePath) DebugBegin
  6 dict begin 
    2 copy PreparePath dup length /n exch def
    /isect [] def
    /tA [] def /tB [] def
    { % [curveA] [Bi]
      /n n 1 sub def
      exch dup 3 -1 roll % [curveA] [curveA] [Bi] 
      IntersectBeziers 
      4 copy LoadIntersectionPoints % [curveA] [tA] [curveB] [tB]
      [ exch isect aload pop ] /isect exch def
      pop 3 -1 roll pop
      [ tB aload length 2 add -1 roll TArray { n add } forall ] /tB exch def
      [ tA aload length 2 add -1 roll TArray aload pop ] /tA exch def
    } forall
    pop % remove [curveA]
    [ isect { aload pop } forall ] 3 1 roll tA exch tB
    % [intersections] [curveA] [tA] [pathB] [tB]
  end
  DebugEnd
} bind def
/IntersectPathCurve {
  exch IntersectCurvePath 4 2 roll
} bind def
%    \end{macrocode}
% \end{macro}
% 
% \begin{macro}{SaveIntersection}
%   \begin{pssyntax}
%     \PSname{isectname} \PSname{nameA} \PSname{nameB} 
%     \PSarray{intersectionpoints} \PSarray{A} \PSarray{tA} \PSarray{B} \PSarray{tB} 
%     \PSop{SaveIntersection}
%   \end{pssyntax}
%    \begin{macrocode}
/SaveIntersection {
  (SaveIntersection) DebugBegin
  4 dict dup 10 -1 roll exch def
  begin %
    /Points 6 -1 roll def
    5 -1 roll dup 4 -1 roll def % /curveA [curveA] [tA] [tB] /curveB /curveB [curveB] def
    nametostr (@t) strcat cvn exch TArray def % /curveA [curveA] [tA] /curveB@t [tB] def
    3 -1 roll dup 4 -1 roll def
    nametostr (@t) strcat cvn exch TArray def
  end
  DebugEnd
} bind def
%    \end{macrocode}
% \end{macro}
% 
% \begin{macro}{TArray}
%   The curve parameters \PSvar{t} as determined by
%   \PSvar{IntersectBeziers} are given in a special array
%   construct. \PSvar{TArray} creates a simple array with the
%   \PSvar{t}-values given in ascending order.
%
% \begin{pssyntax}
% \PSarray{\PSarray{t0a t0b} \ldots \PSvar{null}\ldots \PSvar{integer}}
% \PSop{TArray} \PSarray{t0 t1 \ldots tN}
% \end{pssyntax}
%    \begin{macrocode}
/TArray {
  (TArray) DebugBegin
  dup length 0 gt {
    dup 0 get type /arraytype eq {
      [ exch
      { %dup type /nulltype eq { pop exit } if
  	aload pop add 0.5 mul
      } forall ]
    } if
    dup /lt exch quicksort
  } if
  DebugEnd %1 debug
} bind def
%    \end{macrocode}
% \end{macro} 
% 
% We can save arbitrary paths using \PSvar{pathforall}. The saved path
% contains the commands \PSname{movetype}, \PSname{linetype} and
% \PSname{curvetype}. By default, these are defined as the respective
% original procedures.
%    \begin{macrocode}
/movetype { /moveto load } bind def
/linetype { /lineto load } bind def
/curvetype { /curveto load } bind def
%    \end{macrocode}
%
% [ ... /movetype ... /linetype .../curvetype ]
%    \begin{macrocode}
/PreparePath {
    [ exch aload pop
    {
	dup type /nametype eq not { exit } if
	dup /movetype eq {
	    pop ToVec /@mycp exch def
	} {
	    dup /linetype eq {
		pop [ @mycp 4 2 roll 2 copy ToVec /@mycp exch def ToVec ]
	    } {
		pop [ @mycp 8 2 roll 2 copy ToVec /@mycp exch def
		ToVec 5 1 roll ToVec 4 1 roll ToVec 3 1 roll ]
	    } ifelse
	    counttomark 1 roll	
	} ifelse
    } loop ]
} bind def
%    \end{macrocode}
%
% \begin{macro}{LoadLineIntersectionPoints}
% Prepare \PSarray{Curve} for use with tx@Func
% \begin{pssyntax}
% \PSarray{curve} \PSarray{t} \PSop{LoadLineIntersectionPoints}
% \PSarray{I0.x I0.y \ldots IN.x YN.x}
% \end{pssyntax}
%    \begin{macrocode}
/LoadLineIntersectionPoints {
  (LoadLineIntersectionPoints) DebugBegin
  exch [ exch { aload pop } forall ]
  tx@Dict begin tx@FuncDict begin 2 dict begin
    dup length 2 idiv 1 sub /BezierType exch def /Points exch def
    [ exch { GetBezierCoor } forall ]
  end end end
  DebugEnd
} bind def
%    \end{macrocode}
% \end{macro}
%
% \begin{macro}{LoadCurveIntersectionPoints}
%   Load the intersection points. This loads the same intersection point
%   from both curves, and chooses the one with the lowest error.
% \begin{pssyntax}
% \PSarray{curveA} \PSarray{tA} \PSArray{curveB} \PSarray{tB} 
%  \PSop{LoadCurveIntersectionPoints}
% \PSarray{I0.x I0.y \ldots IN.x YN.x}
% \end{pssyntax}
%    \begin{macrocode}
/LoadCurveIntersectionPoints {
  (LoadCurveIntersectionPoints) DebugBegin
  2 {
    4 2 roll
    [ exch { aload pop } forall ]
    exch [ exch { aload pop } forall ]
  } repeat
  % [A0.x A0.y ... AM.x AM.y] [tA0a tA0b ... tAMa tAMb] [tB0a tB0b ... tBNa tBNb] [B0.x B0.y ... BN.x BN.y]
  tx@Dict begin tx@FuncDict begin 2 dict begin
    dup length 2 idiv 1 sub /BezierType exch def /Points exch def
      [ exch { GetBezierCoor } forall ]
    3 1 roll 
    dup length 2 idiv 1 sub /BezierType exch def /Points exch def
      [ exch { GetBezierCoor } forall ]
    end
    %2 debug
    % [IB0.xa IB0.ya IB0.xb IB0.yb ... IBM.yb] [IA0.xa IA0.ya IA0.xb IA0.yb ... IAM.yb]
    2 {
      [ exch aload length 4 idiv {
        [ 5 1 roll ] counttomark 1 roll
      } repeat ]
      exch 
    } repeat
    % [[IB0.xa ...] ... [... IBM.yb]] [[IA0.xa IA0.ya IA0.xb IA0.yb] ...[IAM.xa ... IAM.yb]]
    2 {
      dup hulldict /comp get exch quicksort exch
    } repeat
    2 dict begin
      /B exch def /A exch def
      [ 0 1 A length 1 sub {
        dup A exch get exch B exch get % [IAi] [IBi]
        2 copy aload pop VecSub Pyth exch 
        aload pop VecSub Pyth lt { exch } if pop
        aload pop VecAdd 0.5 VecScale
      } for 
      % merge near intersection points
      %counttomark 2 idiv 1 1 3 -1 roll {
      %  pop
      %  counttomark 4 lt { exit } if
      %  4 copy 4 2 roll ToVec 3 1 roll ToVec AreNear {

      %  }
      %} for
      ]
    end
  end end
  DebugEnd
} bind def
%    \end{macrocode}
% \end{macro}
%
% \begin{macro}{LoadIntersectionPoints}
%    \begin{macrocode}
/LoadIntersectionPoints {
  (LoadIntersectionPoints) DebugBegin
  4 copy pop exch pop length 2 eq exch length 2 eq and {
    pop pop LoadLineIntersectionPoints
  }{
    LoadCurveIntersectionPoints
  } ifelse
  DebugEnd
} bind def
%    \end{macrocode}
% \end{macro}
% 
% \begin{macro}{IterateIntersection}
% Iteration procedure to compute all intersections of CurveA and CurveB.
% This contains the
% 
% \begin{pssyntax}
% \PSarray{CurveA} \PSarray{CurveB} \PSarray{intervalA} \PSarray{intervalB}
% \PSop{IterateIntersection} \PSarray{domsA} \PSarray{domsB}
% \end{pssyntax}
%    \begin{macrocode}
/IterateIntersection {
    (IterateIntersection) DebugBegin
    11 dict begin
	/precision MaxPrecision def
%    \end{macrocode}
% in order to limit recursion
%    \begin{macrocode}
        /counter 0 def
	/depth 0 def
	/domsA [] def
	/domsB [] def
	/domsA /domsB 6 2 roll _IterateIntersection
	domsB domsA
    end
    dup length 0 gt {
      TArraysRemoveDup
    } if
    DebugEnd
} bind def
/TArraysRemoveDup {
  4 dict begin
    /tB exch def
    /tA exch def
    /j 0 def
    [ tA 0 get tB 0 get
    1 1 tA length 1 sub {
      /i exch def
      tA j get aload pop tA i get aload pop tx@Dict begin Pyth2 end MaxPrecision gt
      tB j get aload pop tB i get aload pop tx@Dict begin Pyth2 end MaxPrecision gt and {
        % keep the current parameter point
        /j i def
        tB i get tA i get
        counttomark 2 idiv 1 add 1 roll 
      } if
    } for
    counttomark 2 idiv 1 add [ exch 1 roll ] % [ ... [tB]
    counttomark 1 add 1 roll ] exch % [tA] [tB]
  end 
} bind def
%    \end{macrocode}
% \end{macro}
% 
% \begin{macro}{_IterateIntersection}
% This is the iteration part which is called recursively.
%
% \begin{pssyntax}
%   \PSname{domsA} \PSname{domsB} \PSarray{CurveA} \PSarray{CurveB}
%   \PSarray{domA} \PSarray{domB} \PSop{_IterateIntersection}
% \end{pssyntax}
%    \begin{macrocode}
/_IterateIntersection {
    (_IterateIntersection) DebugBegin
    CloneVec /domB exch def
    CloneVec /domA exch def
    CloneCurve /CurveB exch def
    CloneCurve /CurveA exch def
    /iter 0 def
    /depth depth 1 add def
    /dom null def
    /counter counter 1 add def

    CheckIT {
	(>> curve subdivision performed: dom(A) = ) domA CurveToString strcat
	(, dom(B) = ) strcat domB CurveToString strcat ( <<) strcat ==
    } if
    CurveA IsConstant CurveB IsConstant and {
	CurveA MiddlePoint ToVec
	CurveB MiddlePoint ToVec AreNear {
	    domA domB 4 -1 roll exch PutInterval PutInterval
	} {
	    pop pop
	} ifelse
    }{
	counter 100 lt {
%    \end{macrocode}
% Use a loop to simulate some kind of return to exit at different positions.
%    \begin{macrocode}
	    {
		/iter iter 1 add def
		iter 100 lt
		domA Extent precision ge
		domB Extent precision ge or and not {
		    iter 100 ge {
			false 
		    } {
			CurveA MiddlePoint ToVec
			CurveB MiddlePoint ToVec AreNear {
			    domA domB true
			}{
			    false
			} ifelse
		    } ifelse
		    exit
		} if
%    \end{macrocode}
% iter < 100 && (dompA.extent() >= precision || dompB.extent() >= precision)
%    \begin{macrocode}
		CheckIT {
		    (counter: ) counter 20 string cvs strcat
		    (, iter: ) iter 20 string cvs strcat strcat
		    (, depth: ) depth 20 string cvs strcat strcat ==
		} if
	
		CurveA CurveB ClipCurve /dom exch def
	
		CheckIT {(dom : ) dom CurveToString strcat == } if		
		dom IsEmptyInterval {
		    CheckIT { (empty interval, exit) == } if
		    false exit
		} if
%    \end{macrocode}
% dom[0] > dom[1], invalid.
%    \begin{macrocode}
		dom aload pop 2 copy min 3 1 roll max gt {
		    CheckIT {
			(dom[0] > dom[1], invalid!) ==
		    } if
		    false exit
		} if

		domB dom MapTo /domB exch def
		CurveB dom Portion

		CurveB IsConstant CurveA IsConstant and {
		    CheckIT {
          		(both curves are constant: ) ==	
			(C1: [ ) CurveA { CurveToString ( ) strcat strcat } forall (]) strcat ==
			(C2: [ ) CurveB { CurveToString ( ) strcat strcat } forall (]) strcat ==
		    } if
		    CurveA MiddlePoint ToVec
		    CurveB MiddlePoint ToVec AreNear {
			domA domB true
		    } {
			false
		    } ifelse
		    exit
		} if
%    \end{macrocode}
% If we have clipped less than 20%, we need to subdivide the
% curve with the largest domain into two sub-curves.
%    \begin{macrocode} 
		dom Extent MinClippedSizeThreshold gt {
		    CheckIT {
			(clipped less than 20% : ) ==
			(angle(A) = ) CurveA dup length 1 sub get aload pop
				      CurveA 0 get aload pop VecSub
   				      exch 2 copy 0 eq exch 0 eq and {
					  pop pop (NaN)
				      } {
					  atan 20 string cvs
				      } ifelse strcat ==
		        (angle(B) = ) CurveB dup length 1 sub get aload pop
		                      CurveB 0 get aload pop VecSub
				      exch 2 copy 0 eq exch 0 eq and {
					  pop pop (NaN)
				      } {
					  atan 20 string cvs
				      } ifelse strcat ==
		        (dom : ) == dom == (domB :) == domB ==
		    } if
%    \end{macrocode}
% Leave those five values on the stack to revert to the current state after the recursive calls.
%    \begin{macrocode}
		    CurveA CurveB domA domB iter
     		    7 -2 roll 2 copy 9 2 roll 2 copy 
%    \end{macrocode}
% On the stack: /domsA /domsB CurveA CurveB domA domB iter /domsA /domsB /domsA /domsB
%    \begin{macrocode}
		    domA Extent domB Extent gt {
			CurveA CloneCurve dup H1Interval Portion % pC1
			CurveA CloneCurve dup H2Interval Portion % pC2
			domA H1Interval MapTo                    % dompC1
			domA H2Interval MapTo                    % dompC2
%    \end{macrocode}
% Need on the stack: /domsA /domsB pC2 CurveB dompC2 domB   /domsA /domsB pC1 CurveB dompC1 domB
%    \begin{macrocode}
			3 -1 roll exch % /domsA /domsB /domsA /domsB pC1 dompC1 pC2 dompC2
			CurveB exch domB 8 4 roll % /domsA /domsB pC2 CurveB dompC2 domB /domsA /domsB pC1 dompC1
			CurveB exch domB % /domsA /domsB pC2 CurveB dompC2 domB /domsA /domsB pC1 CurveB dompC1 domB
		    } {
			CurveB CloneCurve dup H1Interval Portion % pC1
			CurveB CloneCurve dup H2Interval Portion % pC2
			domB H1Interval MapTo                    % dompC1
			domB H2Interval MapTo                    % dompC2
%    \end{macrocode}
% Need on the stack: /domsB /domsA pC2 CurveA dompC2 domA   /domsB /domsA pC1 CurveA dompC1 domA
%    \begin{macrocode}
			8 -2 roll exch 8 2 roll 6 -2 roll exch 6 2 roll % /domsB /domsA /domsB /domsA pC1 pC2 dompC1 dompC2
			3 -1 roll exch % /domsB /domsA /domsB /domsA pC1 dompC1 pC2 dompC2
			CurveA exch domA 8 4 roll % /domsB /domsA pC2 CurveA dompC2 domA /domsB /domsA pC1 dompC1
			CurveA exch domA          % /domsB /domsA pC2 CurveA dompC2 domA /domsB /domsA pC1 CurveA dompC1 domA
		    } ifelse

		    _IterateIntersection
		    _IterateIntersection
%    \end{macrocode}		    
% Restore the state before the recursive calls.
%    \begin{macrocode}
		    /iter exch def
		    /domB exch def
		    /domA exch def
		    /CurveB exch def
		    /CurveA exch def
		    false exit
		} if
		CurveA CurveB /CurveA exch def /CurveB exch def
		domA domB /domA exch def /domB exch def
%    \end{macrocode}
% exchange /domsA and /domsB on the stack!
%    \begin{macrocode}
		exch
	    } loop	
%    \end{macrocode}
% boolean on stack
%    \begin{macrocode}
	    {
		4 -1 roll exch PutInterval PutInterval
		CheckIT {
		    (found an intersection ============================) ==
		} if
	    } { pop pop } ifelse
	} {
	    pop pop
	} ifelse
    } ifelse
    /depth depth 1 sub def
    DebugEnd
} bind def
%    \end{macrocode}
% \end{macro}
% 
% \begin{macro}{PutInterval}
%   Add a new interval \PSarray{newinterval} to the array stored in
%   \PSname{/Intervals}. The new interval is "cloned" before storing it.
% \begin{pssyntax}
% \PSname{Intervals} \PSarray{newinterval} \PSop{PutInterval}
% \end{pssyntax}
%    \begin{macrocode}
/PutInterval {
    CloneVec [ exch 3 -1 roll dup 4 1 roll load aload pop ] def
} bind def
%    \end{macrocode}
% \end{macro}
% 
% \begin{macro}{IsEmptyInterval}
% Check if an interval is empty, which is represented by a [1 0] interval.
% \begin{pssyntx}
% \PSarray{interval} \PSop{IsEmptyInterval} \PSvar{boolean}
%    \begin{macrocode}
/IsEmptyInterval {
    aload pop 0 eq exch 1 eq and
} bind def
%    \end{macrocode}
% \end{macro}
%
% \begin{macro}{ToUnitInterval}
% Limit an interval \PSvar{a b} to the unit interval \PSarray{0 1}.
% \begin{pssyntax}
% \PSvar{a b} \PSop{ToUnitInterval} \PSarray{a|0 b|1}
% \end{pssyntax}
%    \begin{macrocode}
/ToUnitInterval {
    ToUnitRange exch ToUnitRange 2 copy gt {
	exch
    } if
    ToVec
} bind def
%    \end{macrocode}
% \end{macro}
% \begin{macro}{ToUnitRange}
% Limit a number to the range \PSarray{0 1}.
%    \begin{macrocode}
/ToUnitRange {
    dup 0 lt {
	pop 0
    }{
	dup 1 gt {
	    pop 1
	} if
    } ifelse
} bind def
%    \end{macrocode}
% \end{macro}
% 
% \begin{macro}{CloneCurve}
% Does a deep copy of the array \PSarray{Curve}. This also involved deep copies of the contained point arrays.
% \begin{pssyntax}
% \PSarray{Curve} \PSop{CloneCurve} \PSarray{newCurve}
%    \begin{macrocode}
/CloneCurve {
    [ exch {
	CloneVec
    } forall ]
} bind def
%    \end{macrocode}
% \end{macro}
%
% \begin{macro}{CloneVec}
% Does a deep copy of the vector \PSarray{X Y}
% \begin{pssyntax}
% \PSarray{X Y} \PSop{CloneVec} \PSarray{Xnew Ynew}
% \end{pssyntax}
%    \begin{macrocode}
/CloneVec {
    aload pop ToVec
} bind def
%    \end{macrocode}
% \end{macro}
% 
% \begin{macro}{MapTo}
% Map the sub-interval \PSarray{I} in \PSarray{0 1} into the interval \PSarray{J}. Returns a new array.
% \begin{pssyntax}
% \PSarray{J} \PSarray{I} \PSop{MapTo} \PSarray{Jnew}
% \end{pssyntax}
%    \begin{macrocode}
/MapTo {
    (MapTo) DebugBegin
    exch aload 0 get 3 1 roll exch sub 2 copy % [I] J0 Jextent J0 Jextent
    5 -1 roll aload aload pop % J0 Jextent J0 Jextent I0 I1 I0 I1
    min 4 -1 roll mul % J0 Jextent J0 I0 I1 min(I0,I1)*Jextent
    4 -1 roll add [ exch % J0 Jextent I0 I1 [ J0new
    6 2 roll max mul add ]
    DebugEnd
} bind def
%    \end{macrocode}
% \end{macro}
% 
% \begin{macro}{Portion}
% Compute the portion of the Bezier curve \PSarray{CurveB} wrt the interval \PSarray{I}.
% \begin{pssyntax}
% \PSarray{CurveB} \PSarray{I} \PSop{Portion} \PSarray{CurvePartB}
% \end{pssyntax}
%    \begin{macrocode}
/Portion {
    (Portion) DebugBegin
    dup Min 0 eq { % [CurveB] [I]
	% I.min() == 0
	Max dup 1 eq {% [CurveB] I.max()
	    % I.max() == 1
	    pop pop	    
	} { % [CurveB] I.max()
	    LeftPortion
	} ifelse
    } { % [CurveB] [I]
	2 copy Min % [CurveB] [I] [CurveB] I.min()
	RightPortion
	dup Max 1 eq {
	    % I.max() == 1
	    pop pop
	} {% [CurveB] [I]
	    dup aload pop exch sub 1 3 -1 roll Min sub div % [CurveB] (I1-I0)/(1-I.min())
	    LeftPortion
	} ifelse
    } ifelse
    DebugEnd
} bind def
%    \end{macrocode}
% \end{macro}
% 
% \begin{macro}{LeftPortion}
%   Compute the portion of the Bezier curve \PSarray{CurveB} wrt the
%   interval \PSarray{0 t}.
% \begin{pssyntax}
% \PSarray{CurveB} \PSvar{t} \PSop{LeftPortion} \PSarray{CurvePartB}
% \end{pssyntax}
%    \begin{macrocode}
/LeftPortion {
    (LeftPortion) DebugBegin
    exch dup length 1 sub dup 4 1 roll % L-1 t [CurveB] L-1
    1 1 3 -1 roll { % L-1 t [CurveB] i
	4 -1 roll dup 5 1 roll % L-1 t [CurveB] i L-1
	-1 3 -1 roll % L-1 t [CurveB] L-1 -1 i
	{ % L-1 t [CurveB] j
	    2 copy 5 copy % L-1 t [CurveB] j [CurveB] j t [CurveB] j [CurveB] j 
	    1 sub get 3 1 roll get % L-1 t [CurveB] j [CurveB] j t B[j-1] B[j]
	    Lerp put pop % L-1 t [CurveB]
	} for
    } for
    pop pop pop
    DebugEnd
} bind def
%    \end{macrocode}
% \end{macro}
%
% \begin{macro}{RightPortion}
%   Compute the portion of the Bezier curve \PSarray{CurveB} wrt the
%   interval \PSarray{t 1}.
% \begin{pssyntax}
% \PSarray{CurveB} \PSvar{t} \PSop{RightPortion} \PSarray{CurvePartB}
% \end{pssyntax}
%    \begin{macrocode}
/RightPortion {
    (RightPortion) DebugBegin
    exch dup length 1 sub dup 4 1 roll % L-1 t [CurveB] L-1
    1 1 3 -1 roll {% L-1 t [CurveB] i
	4 -1 roll dup 5 1 roll % L-1 t [CurveB] i L-1
	exch sub 0 1 3 -1 roll  % L-1 t [CurveB] 0 1 L-i-1
	{% L-1 t [CurveB] j
	    2 copy 5 copy
	    get 3 1 roll 1 add get Lerp put pop
	} for
    } for
    pop pop pop
    DebugEnd
} bind def
%    \end{macrocode}
% \end{macro}
% 
% \begin{macro}{Lerp}
% Given two points and a parameter \PSvar{t} $\in$ \PSarray{0 1}, return a point
% proportionally from \PSarray{A} to \PSarray{B} by \PSvar{t}. Akin to 1 degree Bezier.
% \begin{pssyntax}
% \PSvar{t} \PSarray{A} \PSarray{B} \PSop{Lerp} \PSarray{newpoint}
% \end{pssyntax}
%    \begin{macrocode}
/Lerp {
    (Lerp) DebugBegin
    3 -1 roll dup 1 exch sub 3 1 roll % [A] (1-t) [B] t
    exch aload pop 3 -1 roll VecScale % [A] (1-t) B.x*t B.y*t
    4 2 roll
    exch aload pop 3 -1 roll VecScale VecAdd ToVec % [A.x*(1-t)+B.x*t A.y*(1-t)+B.y*t]
    DebugEnd
} bind def
%    \end{macrocode}
% \end{macro}
% 
% \begin{macro}{IsConstant}
% Test if all points of a curve are near to each other. This is used as termination criterium for the intersection procedure.
% \begin{pssyntax}
% \PSarray{Curve} \PSop{IsConstant} \PSvar{boolean}
% \end{pssyntax}
%    \begin{macrocode}
/IsConstant {
    aload length [ exch 1 roll ] true 3 1 roll
    {
	exch dup 4 1 roll
	AreNear and exch
    } forall
    pop
} bind def
%    \end{macrocode}
% \end{macro}
% \begin{macro}{AreNear}
% Test if two points are near to each other.
% \begin{pssyntax}
% \PSarray{P1} \PSarray{P2} \PSop{AreNear} \PSvar{boolean}
% \end{pssyntax}
%    \begin{macrocode}
/AreNear {
    (AreNear) DebugBegin
    aload pop 3 -1 roll aload pop
    4 copy abs 3 { exch abs max } repeat Epsilon mul
    dup 6 2 roll VecSub abs 4 -1 roll lt exch abs 3 -1 roll lt and
    DebugEnd
} bind def
%    \end{macrocode}
% \end{macro}
% 
% \begin{macro}{Min}
% Get the minimum value of the vector \PSarray{P}.
% \begin{pssyntax}
% \PSarray{P} \PSop{Min} \PSvar{minimum}
% \end{pssyntax}
%    \begin{macrocode}
/Min {
    aload pop min
} bind def
%    \end{macrocode}
% \end{macro}
% \begin{macro}{Min}
% Get the maximum value of the vector \PSarray{P}.
% \begin{pssyntax}
% \PSarray{P} \PSop{Max} \PSvar{maximum}
% \end{pssyntax}
%    \begin{macrocode}
/Max {
    aload pop max
} bind def
%    \end{macrocode}
% \end{macro}
% \begin{macro}{Min}
% Get the extent of the interval \PSarray{I}.
% \begin{pssyntax}
% \PSarray{I} \PSop{Extent} \PSvar{I1-I0}
% \end{pssyntax}
%    \begin{macrocode}
/Extent {
    aload pop exch sub
} bind def
%    \end{macrocode}
% \end{macro}
%
% \begin{macro}{MiddlePoint}
% Compute the middle point of the first and last point of \PSarray{Curve}.
% \begin{pssyntax}
% \PSarray{Curve} \PSop{MiddlePoint} \PSvar{X Y}
% \end{pssyntax}
%    \begin{macrocode}
/MiddlePoint {
    dup dup length 1 sub get aload pop
    3 -1 roll 0 get aload pop
    VecAdd 0.5 VecScale
} bind def
%    \end{macrocode}
% \end{macro}
% 
% \begin{macro}{OrthogonalOrientationLine}
% \begin{pssyntax}
% \PSvar{MiddlePointA} \PSarray{CurveB} \PSop{OrthogonalOrientationLine} \PSvar{A B C}
% \end{pssyntax}
%    \begin{macrocode}
/OrthogonalOrientationLine {
    (OrthogonalOrientationLine) DebugBegin
    dup dup length 1 sub get aload pop 3 -1 roll 0 get aload pop VecSub
%    \end{macrocode}
% rotate by +90 degrees
%    \begin{macrocode}
    neg exch
    4 2 roll 2 copy 6 2 roll VecAdd
    ImplicitLine
    DebugEnd
} bind def
%    \end{macrocode}
% \end{macro}
% 
% \begin{macro}{PickOrientationLine}
%   Pick an orientation line for a Bezier curve. This uses the first
%   point and the lastmost point, which is not near to it.
% \begin{pssyntax}
% \PSarray{Curve} \PSop{PickOrientationLine} \PSvar{A B C}
% \end{pssyntax}
%    \begin{macrocode}
/PickOrientationLine {
    (PickOrientationLine) DebugBegin
    dup dup length 1 sub exch 0 get% [Curve] L-1 P0
    exch -1 1 {% [Curve] P0 i
	3 -1 roll dup 4 1 roll exch get % [Curve] P0 Pi
	2 copy AreNear {
	    pop
	} {
	    exit
	} ifelse
    } for
    3 -1 roll pop
    exch aload pop 3 -1 roll aload pop ImplicitLine
    DebugEnd
} bind def
%    \end{macrocode}
% \end{macro}
% 
% \begin{macro}{ImplicitLine}
% Compute the coefficients \PSvar{A}, \PSvar{B}, \PSvar{C} of the normalized implicit equation
% of the line which goes through the points \PSarray{Xi Yi} and \PSarray{Xj Yj}.
%
% \begin{pssyntax}
% \PSvar{Xi Yi Xj Yj} \PSop{ImplicitLine} \PSvar{A B C}
% \end{pssyntax}
%    \begin{macrocode}
/ImplicitLine {
    4 copy % Xi Yi Xj Yj Xi Yi Xj Yj
    3 -1 roll sub 7 1 roll sub 5 1 roll % Yj-Yi Xi-Xj Xi Yi Xj Yj
    % Yi*Xj - Xi*Yj
    4 -1 roll mul neg % Yj-Yi Xi-Xj Yi Xj -Yj*Xi
    3 1 roll mul add % Yj-Yi Xi-Xj Yi*Xj-Yj*Xi | l0 l1 l2
    3 1 roll 2 copy tx@Dict begin Pyth end dup dup % l2 l0 l1 L L L
    5 -1 roll exch % l2 l1 L L l0 L
    div 5 1 roll % l0/L l2 l1 L L
    3 1 roll div % l0/L l2 L l1/L
    3 1 roll div % l0/L l1/L l2/L
} bind def
%    \end{macrocode}
% \end{macro}
% 
% \begin{macro}{distance}
% Compute the distance of point \PSarray{X Y} from the implicit line given
% by $Ax + By + C = 0,\quad (A^2+B^2 = 1)$.
% \begin{pssyntax}
% \PSvar{X Y A B C} \PSop{distance} \PSvar{d}
% \end{pssyntax}
%    \begin{macrocode}
/distance {
    5 1 roll 3 -1 roll mul 3 1 roll mul add add
} bind def
%    \end{macrocode}
% \end{macro}
%
% \begin{macro}{ArrayToPointArray}
% \begin{pssyntax}
% \PSarray{A.x A.y ... N.x N.y} \PSop{ArrayToPointArray} \PSarray{\PSarray{A.x A.y} \ldots \PSarray{N.x N.y}}
% \end{pssyntax}
%    \begin{macrocode}
/ArrayToPointArray {
    aload length dup 2 idiv {
	3 1 roll [ 3 1 roll ] exch
	dup 1 sub 3 1 roll 1 roll
    } repeat 1 add [ exch 1 roll ]
} bind def
%    \end{macrocode}
% \end{macro}
%
% \begin{macro}{PointArrayToArray}
% \begin{pssyntax}
% \PSarray{\PSarray{A.x A.y} \ldots \PSarray{N.x N.y}} \PSop{PointArrayToArray} \PSarray{A.x A.y ... N.x N.y}
% \end{pssyntax}
%    \begin{macrocode}
/PointArrayToArray {
    aload length dup {
	1 add dup 3 -1 roll aload pop 4 -1 roll 1 add 2 roll
    } repeat 1 add [ exch 1 roll ]
} bind def
%    \end{macrocode}
% \end{macro}
% 
% \begin{macro}{ClipCurve}
% Clip the Bezier curve B with respect to the Bezier curve A for
% individuating intersection points. The new parameter interval for the
% clipped curve is pushed on the stack.
% \begin{pssyntax}
% \PSarray{CurveA} \PSarray{CurveB} \PSop{ClipCurve} \PSarray{newinterval}
% \end{pssyntax}
%    \begin{macrocode}
/ClipCurve {
    (ClipCurve) DebugBegin
    4 dict begin 
    /CurveB exch def /CurveA exch def
    CurveA IsConstant {
    	CurveA MiddlePoint CurveB OrthogonalOrientationLine
    } {
	CurveA PickOrientationLine
    } ifelse
    CheckIT {
	3 copy exch 3 -1 roll (OrientationLine : )
	3 { exch 20 string cvs ( ) strcat strcat } repeat ==
    } if
    CurveA FatLineBounds
    CheckIT { dup (FatLineBounds : ) exch aload pop exch 20 string cvs (, ) strcat exch 20 string cvs strcat strcat == } if
    CurveB ClipCurveInterval
    end
    DebugEnd
} bind def
%    \end{macrocode}
% \end{macro}
% 
% \begin{macro}{FatLineBounds}
% Compute the boundary of the fat line given by \PSvar{A B C}
% \begin{pssyntax}
% \PSvar{A B C} \PSarray{Curve} \PSop{FatLineBounds} \PSvar{A B C} \PSarray{dmin dmax}
% \end{pssyntax}
%    \begin{macrocode}
/FatLineBounds {
    (FatLineBounds) DebugBegin
    /dmin 0 def /dmax 0 def
    { 
	4 copy aload pop 5 2 roll distance
	dup dmin lt { dup /dmin exch def } if
	dup dmax gt { dup /dmax exch def } if
	pop pop
    } forall
    [dmin dmax]
    DebugEnd
} bind def
%    \end{macrocode}
% \end{macro}
% 
% \begin{macro}{ClipCurveInterval}
%   Clip the Bezier curve wrt the fat line defined by the orientation
%   line (given by \PSvar{A B C}) and the interval range
%   \PSarray{bound}. The new parameter interval \PSarray{newinterval}
%   for the clipped curve is pushed on the stack.
% \begin{pssyntax}
% \PSvar{A B C} \PSarray{bound} \PSarray{curve} \PSop{ClipCurveInterval} \PSarray{newinterval}
% \end{pssyntax}
%    \begin{macrocode}
/ClipCurveInterval {
    (ClipCurveInterval) DebugBegin
    15 dict begin
    /curve exch def
    aload pop 2 copy min /boundMin exch def max /boundMax exch def
    [ 4 1 roll ] cvx /fatline exch def
    % number of sub-intervals
    /n curve length 1 sub def
    % distance curve control points
    /D n 1 add array def
    0 1 n { % i
	dup curve exch get aload pop % i Pi.x Pi.y
	fatline distance % distance d of Point i from the orientation line, on stack; i d
	exch dup n div % d i i/n
	[ exch 4 -1 roll ] % i [ i/n d ]
	D 3 1 roll put 
    } for
    D ConvexHull /D exch def
%    \end{macrocode}
% get the x-coordinate of the i-th point, i getX -> D[i][X]
%    \begin{macrocode}
    /getX { D exch get 0 get } def
%    \end{macrocode}
% get the y-coordinate of the i-th point, i getY -> D[i][Y]
%    \begin{macrocode}
    /getY { D exch get 1 get } def
    /tmin 1 def /tmax 0 def
    0 getY dup
    boundMin lt /plower exch def
    boundMax gt /phigher exch def
    plower phigher or not {
%    \end{macrocode}
% inside the fat line
%    \begin{macrocode}
	tmin 0 getX gt { /tmin 0 getX def } if
	tmax 0 getX lt { /tmax 0 getX def } if	
    } if
    1 1 D length 1 sub {
	/i exch def
	/clower i getY boundMin lt def
	/chigher i getY boundMax gt def
	clower chigher or not {
%    \end{macrocode}
% inside the fat line
%    \begin{macrocode}
	    tmin i getX gt { /tmin i getX def } if
	    tmax i getX lt { /tmax i getX def } if
	} if
	clower plower eq not {
%    \end{macrocode}
% cross the lower bound
%    \begin{macrocode}
	    boundMin i 1 sub i D Intersect % t on stack
	    dup tmin lt { dup /tmin exch def } if
	    dup tmax gt { dup /tmax exch def } if
	    pop 
	    /plower clower def
	} if
	chigher phigher eq not {
%    \end{macrocode}
% cross the upper bound
%    \begin{macrocode}
	    boundMax i 1 sub i D Intersect
	    dup tmin lt { dup /tmin exch def } if
	    dup tmax gt { dup /tmax exch def } if
	    pop 
	    /phigher chigher def
	} if
    } for
%    \end{macrocode}
% we have to test the closing segment for intersection
%    \begin{macrocode}
    /i D length 1 sub def
    /clower 0 getY boundMin lt def
    /chigher 0 getY boundMax gt def
    clower plower eq not {
%    \end{macrocode}
% cross the lower bound
%    \begin{macrocode}
	boundMin i 0 D Intersect
	dup tmin lt { dup /tmin exch def } if
	dup tmax gt { dup /tmax exch def } if
	pop
    } if
    chigher phigher eq not {
%    \end{macrocode}
% cross the lower bound
%    \begin{macrocode}
	boundMax i 0 D Intersect
	dup tmin lt { dup /tmin exch def } if
	dup tmax gt { dup /tmax exch def } if
	pop
    } if
    [tmin tmax]
    end
    DebugEnd
} bind def
%    \end{macrocode}
% \end{macro}
% 
% \begin{macro}{Intersect}
%   Get the x component of the intersection point between the line
%   passing through $i$-th and $j$-th points of \PSarray{Curve} and the
%   horizonal line through \PSvar{y}.
% \begin{pssyntax}
% \PSvar{y i j} \PSarray{Curve} \PSop{Intersect} \PSvar{Xisect}
% \end{pssyntax}
%    \begin{macrocode}
/Intersect {
    dup 4 -1 roll get aload pop
    4 2 roll exch get aload pop
%    \end{macrocode}
% On the stack: \PSvar{y Xi Yi Xj Yj}, Compute (Xj - Xi) * (y - Yi)/(Yj - Yi) + Xi
%
% We are sure, that Yi != Yj, because this procedure is called only
% when the lower or upper bound is crossed.
%    \begin{macrocode}
    4 2 roll 2 copy 6 2 roll VecSub
    5 2 roll
    neg 3 -1 roll add
    3 -1 roll div
    3 -1 roll mul add
} bind def
%    \end{macrocode}
% \end{macro}
%
% \begin{macro}{IsPath}
%   Check if an array is a path. A path is represented as array, which
%   contains other arrays which represent native Postscript
%   operations. Those can be \PSarray{X Y /@m}, \PSarray{X Y /@l}, or
%   \PSarray{X1 Y1 X2 Y2 X3 Y3 /@c}.
%
%   \begin{pssyntax}
%     \PSarray{array} \PSop{IsPath} \PSvar{boolean}
%   \end{pssyntax}
%    \begin{macrocode}
/IsPath {
  dup length 1 sub get type /nametype eq { true } { false } ifelse
} bind def
%    \end{macrocode}
% \end{macro}
%
% \begin{macro}{ShowFullPath}
%    \begin{macrocode}
/ShowFullPath {
  4 dict begin
  /movetype /moveto load def
  /linetype /lineto load def
  /curvetype /curveto load def
  mark exch aload pop
  {
    counttomark 0 eq { exit } if
    load exec
  } loop
  pop 
  end
} bind def
/ShowPathPortion {
  6 dict begin
  exch dup 0 lt { pop 0 } if /tstart exch def
  % if tstart is < 0, use the array length
  dup 0 lt { pop dup length } if /tstop exch def
  tstop tstart gt {
    /movetype /moveto load def
    /linetype /lineto load def
    /curvetype /curveto load def
    /n 0 def
    mark exch aload pop
    {
      counttomark 0 eq { pop exit } if
      dup /movetype eq not { /n n 1 add def } if
      % current section is after tstart
      n tstop sub 1 ge { cleartomark exit } if

      dup /movetype eq {
        load exec
      } {
        tstart n gt {
          % current path section is before tstart
          /curvetype eq { 6 2 roll 4 { pop } repeat } if
          movetype
        } {
          tstart n 1 sub gt tstop n lt or {
            % draw a truncated section
            tstart n sub 1 add tstop n sub 1 add
            ToUnitInterval exch
            /linetype eq {
              3 1 roll ToVec currentpoint ToVec exch ToVec
              dup 3 -1 roll Portion
              aload pop exch 
              tstart n 1 sub gt { aload pop moveto } { pop } ifelse
              aload pop lineto
            } {
              7 1 roll [ currentpoint 9 3 roll ] ArrayToPointArray
              dup 3 -1 roll Portion 
              { aload pop } forall
              8 -2 roll
              tstart n 1 sub gt { moveto } { pop pop } ifelse
              curveto
            } ifelse
          }{
            load exec
          } ifelse
        } ifelse
      } ifelse
    } loop
  } if
  end
} bind def
%    \end{macrocode}
% \end{macro}
%
%    \begin{macrocode}
 % Graham Scal algorithm to compute the convex hull of a set of
 % points. Code written by Bill Casselman,
 %  http://www.math.ubc.ca/~cass/graphics/text/www/
 %
 % [[X1 Y1] [X2 Y2] ... [Xn Yn]] hull -> [[...] ... [...]]
 %
/hulldict 32 dict def
hulldict begin

 % u - v 
/vsub { 2 dict begin
/v exch def
/u exch def
[ 
  u 0 get v 0 get sub
  u 1 get v 1 get sub
]
end } def

 % u - v rotated 90 degrees
/vperp { 2 dict begin
/v exch def
/u exch def
[ 
  v 1 get u 1 get sub
  u 0 get v 0 get sub
]
end } def

/dot { 2 dict begin
/v exch def
/u exch def
  v 0 get u 0 get mul
  v 1 get u 1 get mul
  add
end } def 

 % P Q
 % tests whether P < Q in lexicographic order
 % i.e xP < xQ, or yP < yQ if xP = yP
/comp { 2 dict begin
/Q exch def
/P exch def
P 0 get Q 0 get lt 
  P 0 get Q 0 get eq
  P 1 get Q 1 get lt 
  and 
or 
end } def

end

 % args: an array of points C
 % effect: returns the array of points on the boundary of
 %     the convex hull of C, in clockwise order 

/ConvexHull {
(ConvexHull) DebugBegin
hulldict begin
/C exch def
/comp C quicksort
/n C length def
 % Q might circle around to the start
/Q n 1 add array def
Q 0 C 0 get put
Q 1 C 1 get put
/i 2 def
/k 2 def
 % i is next point in C to be looked at
 % k is next point in Q to be added
 % [ Q[0] Q[1] ... ]
 % scan the points to make the top hull
n 2 sub {
  % P is the current point at right
  /P C i get def
  /i i 1 add def
  {
    % if k = 1 then just add P 
    k 2 lt { exit } if
    % now k is 2 or more
    % look at Q[k-2] Q[k-1] P: a left turn (or in a line)?
    % yes if (P - Q[k-1])*(Q[k-1] - Q[k-2])^perp >= 0
    P Q k 1 sub get vsub 
    Q k 1 sub get Q k 2 sub get vperp 
    dot 0 lt {
      % not a left turn
      exit
    } if
    /k k 1 sub def
  } loop
  Q k P put
  /k k 1 add def
} repeat

 % done with top half
 % K is where the right hand point is
/K k 1 sub def

/i n 2 sub def
Q k C i get put
/i i 1 sub def
/k k 1 add def
n 2 sub {
  % P is the current point at right
  /P C i get def
  /i i 1 sub def
  {
    % in this pass k is always 2 or more
    k K 2 add lt { exit } if
    % look at Q[k-2] Q[k-1] P: a left turn (or in a line)?
    % yes if (P - Q[k-1])*(Q[k-1] - Q[k-2])^perp >= 0
    P Q k 1 sub get vsub 
    Q k 1 sub get Q k 2 sub get vperp 
    dot 0 lt {
      % not a left turn
      exit
    } if
    /k k 1 sub def
  } loop
  Q k P put
  /k k 1 add def
} repeat

 % strip Q down to [ Q[0] Q[1] ... Q[k-2] ]
 % excluding the doubled initial point
[ 0 1 k 2 sub {
  Q exch get
} for ] 
end
DebugEnd
} def

/qsortdict 8 dict def

qsortdict begin

 % args: /comp a L R x
 % effect: effects a partition into two pieces [L j] [i R]
 %     leaves i j on stack

/partition { 8 dict begin
/x exch def
/j exch def
/i exch def
/a exch def
dup type /nametype eq { load } if /comp exch def
{
  {
    a i get x comp exec not {
      exit
    } if
    /i i 1 add def
  } loop
  {
    x a j get comp exec not {
      exit
    } if
    /j j 1 sub def
  } loop
  
  i j le {
    % swap a[i] a[j]
    a j a i get
    a i a j get 
    put put
    /i i 1 add def
    /j j 1 sub def
  } if
  i j gt {
    exit
  } if
} loop
i j
end } def

 % args: /comp a L R
 % effect: sorts a[L .. R] according to comp
/subsort {
 % /c a L R
[ 3 1 roll ] 3 copy
 % /c a [L R] /c a [L R]
aload aload pop 
 % /c a [L R] /c a L R L R
add 2 idiv
 % /c a [L R] /c a L R (L+R)/2
3 index exch get
 % /c a [L R] /c a L R x
partition
 % /c a [L R] i j
 % if j > L subsort(a, L, j)
dup 
 % /c a [L R] i j j
3 index 0 get gt {
  % /c a [L R] i j
  5 copy 
  % /c a [L R] i j /c a [L R] i j
  exch pop
  % /c a [L R] i j /c a [L R] j
  exch 0 get exch
  % ... /c a L j 
  subsort
} if
 % /c a [L R] i j
pop dup
 % /c a [L R] i i
 % if i < R subsort(a, i, R)
2 index 1 get lt {
  % /c a [L R] i
  exch 1 get 
  % /c a i R
  subsort
}{
  4 { pop } repeat
} ifelse
} def

end % qsortdict

 % args: /comp a
 % effect: sorts the array a 
 % comp returns truth of x < y for entries in a

/quicksort { qsortdict begin
dup length 1 gt {
 % /comp a
dup 
 % /comp a a 
length 1 sub 
 % /comp a n-1
0 exch subsort
} {
pop pop
} ifelse
end } def
%    \end{macrocode}
% 
% Debugging stuff
%    \begin{macrocode}
/debug {
    dup 1 add copy {==} repeat pop
} bind def
/DebugIT false def
/CheckIT false def
/DebugDepth 0 def
/DebugBegin {
  DebugIT {
    /DebugProcName exch def
    DebugDepth 2 mul string
    0 1 DebugDepth 2 mul 1 sub {
      dup 2 mod 0 eq { (|) }{( )} ifelse
      3 -1 roll dup 4 2 roll
      putinterval
    } for
    DebugProcName strcat ==
    /DebugDepth DebugDepth 1 add def
  }{
    pop
  } ifelse
} bind def
/DebugEnd {
  DebugIT {
    /DebugDepth DebugDepth 1 sub def
    DebugDepth 2 mul 2 add string
    0 1 DebugDepth 2 mul 1 sub {
      dup 2 mod 0 eq { (|) }{ ( ) } ifelse
      3 -1 roll dup 4 2 roll
      putinterval
    } for
    dup DebugDepth 2 mul (+-) putinterval
    ( done) strcat ==
  } if
} bind def
/strcat {
    exch 2 copy
    length exch length add
    string dup dup 5 2 roll
    copy length exch
    putinterval
} bind def
/nametostr {
    dup length string cvs
} bind def
/ShowCurve {
    { aload pop } forall
    8 -2 roll moveto curveto
} bind def
/CurveToString {
    (CurveToString) DebugBegin
    aload pop ([) 3 -1 roll 20 string cvs strcat (, ) strcat exch 20 string cvs strcat (]) strcat
    DebugEnd
} bind def
end % tx@IntersectDict
%    \end{macrocode}
%</prolog> 
% \Finale
% \endinput

\IfFileExists{pst-intersect.pro}{%
    \ProvidesFile{pst-intersect.pro}
      [2014/02/06 PostScript prologue file]
      \@addtofilelist{pst-intersect.pro}}{}%
%    \end{macrocode}
%</stylefile>
%
% \chapter{The \TeX\ implementation}
%
%<*texfile>
%    \begin{macrocode}
\csname PSTintersectLoaded\endcsname
\let\PSTintersectLoaded\endinput

\ifx\PSTricksLoaded\endinput\else\input pstricks.tex \fi
\ifx\PSTXKeyLoaded\endinput\else \input pst-xkey.tex \fi
\ifx\PSTnodesLoaded\endinput\else\input pst-node.tex \fi
\ifx\PSTfuncLoaded\endinput\else \input pst-func.tex \fi

\edef\PstAtCode{\the\catcode`\@} \catcode`\@=11\relax

\pst@addfams{intersect}
\pstheader{pst-intersect.pro}

\def\pst@intersectdict{tx@IntersectDict begin }
\def\PIT@dict#1{\pst@intersectdict #1 end}
\def\PIT@Verb#1{\pst@Verb{\PIT@dict{#1} }}%

\def\savebezier{\pst@object{savebezier}}
\def\savebezier@i#1{%
  \begin@OpenObj
    \PIT@checkname{#1}%
    \addto@pscode{ /\PIT@name{#1} }%
    \pst@getcoors[\savebezier@ii%]
}%
\def\savebezier@ii{%
  \addto@pscode{%
    % only 10 points allowed, remove the rest
    counttomark 20 gt { counttomark 20 sub { pop } repeat } if
    % reverse the point order
    counttomark -2 4 { 2 roll } for
    %\addto@pscode{%
    %counttomark 2 add -1 roll pop % remove the path name
    counttomark 2 idiv 1 sub dup 9 gt { pop 9 } if
    \psk@plotpoints\space exch
    \txFunc@BezierCurve
    \ifshowpoints \txFunc@BezierShowPoints \else pop \fi
    tx@FuncDict begin Points aload pop end
  }%
  \let\use@pscode\PIT@use@pscode
  \end@OpenObj
  \PIT@Verb{%
    [ count 1 sub 1 roll ] ArrayToPointArray def 
  }%
\ignorespaces}%
\define@key[psset]{intersect}{tstart}{%
  \pst@checknum{#1}\PIT@key@tstart
}
\define@key[psset]{intersect}{tstop}{%
  \pst@checknum{#1}\PIT@key@tstop
}
\define@key[psset]{intersect}{istart}{%
  \pst@checknum{#1}\PIT@key@istart
}
\define@key[psset]{intersect}{istop}{%
  \pst@checknum{#1}\PIT@key@istop
}
\define@key[psset]{intersect}{name}{%
  \def\PIT@key@name{#1}%
}%
\newif\PIT@saveintersections
\define@boolkey[psset]{intersect}[PIT@]{saveintersections}[true]{}
\psset[intersect]{%
  tstart=-1,
  tstop=-1,
  istart=-1,
  istop=-1,
  name={},
  saveintersections
}%
\def\PIT@use@pscode{%
  \pstverb{%
    \pst@dict
    \tx@STP
    \pst@newpath
    \psk@origin
    \psk@swapaxes
    \pst@code
    end
    count /ocount exch def
  }%
  \gdef\pst@code{}%
}%
\let\PIT@pst@stroke@orig\pst@stroke
\def\PIT@save@path{%
  \PIT@pst@stroke@orig
  \addto@pscode{%
    clear mark
    { /movetype counttomark 3 roll }
    { /linetype counttomark 3 roll }
    { /curvetype counttomark 7 roll }{} pathforall 
    counttomark 1 add -1 roll pop count }%
}%
\def\PIT@name@default{@tmp}%
\def\PIT@name#1{PIT@#1}%
\def\PIT@checkname#1{%
  \ifx\@empty#1\@empty
    \@pstrickserr{Unexpected empty argument!}\@ehpb
  \fi
}%
\def\savepath{\pst@object{savepath}}%
\long\def\savepath@i#1#2{%
  \begin@SpecialObj
    \PIT@checkname{#1}%
    \let\pst@stroke\PIT@save@path
    \let\use@pscode\PIT@use@pscode
    \pscustom{#2}%
    \PIT@Verb{%
      /\PIT@name{#1}
      [ 3 -1 roll 2 add 2 roll ] def }%
  \end@SpecialObj
}%
\def\psshowcurve{\pst@object{psshowcurve}}%
\def\psshowcurve@i#1{%
  \addbefore@par{plotpoints=200}%
   \@ifnextchar\bgroup
     {\PIT@traceintersection{#1}}%
     {\PIT@tracecurve{#1}}%
}%
\def\PIT@tracecurve#1{%
  \PIT@checkname{#1}%
  \begin@OpenObj
    \addto@pscode{%
      \pst@intersectdict
        \PIT@name{#1} dup IsPath {
          \PIT@key@tstart\space\PIT@key@tstop\space
          ShowPathPortion
        }{
          [exch dup
          \PIT@key@tstart\space\PIT@key@tstop\space 
          dup 0 lt { pop 1 } if
          ToUnitInterval Portion 
          { aload pop } forall
          counttomark 2 sub 2 idiv
          \psk@plotpoints
          exch
          \txFunc@BezierCurve
          \ifshowpoints \txFunc@BezierShowPoints \else pop \fi
        } ifelse
      end
    }%
  \end@OpenObj
}%
\def\PIT@traceintersection#1#2{%
  \PIT@checkname{#2}%
  \begin@OpenObj
    \addto@pscode{%
      \pst@intersectdict
        \ifx\\#1\\%
          /\PIT@name{\PIT@name@default}
        \else
          /\PIT@name{#1}
        \fi 
        dup currentdict exch known not {
          \ifx\\#1\\%
            (You haven't defined an intersection!) ==
          \else
            (You haven't defined the intersection '#1') ==
          \fi
        } if 
        load
        dup dup type /dicttype eq exch /\PIT@name{#2} known and not {
          (You haven't defined the intersection '#2') ==
        } if
        dup /\PIT@name{#2} get
        exch /\PIT@name{#2}@t get
        dup length \PIT@key@istart\space ge 0 \PIT@key@istart\space lt and {
          dup \PIT@key@istart\space cvi 1 sub get
        } {
          \PIT@key@tstart
        } ifelse
        exch % [curve] t_istart|tstart [ts]
        %
        dup length \PIT@key@istop\space ge 0 \PIT@key@istop\space lt and {
          \PIT@key@istop\space cvi 1 sub get
        } {
          pop \PIT@key@tstop
        } ifelse
        2 copy gt { exch } if 
        3 -1 roll dup
        IsPath {
          3 1 roll
          ShowPathPortion
        }{
          [ exch dup 5 -2 roll
          dup 0 lt { pop 1 } if
          ToUnitInterval Portion 
          { aload pop } forall
          counttomark 2 sub 2 idiv
          \psk@plotpoints
          exch
          \txFunc@BezierCurve
          \ifshowpoints \txFunc@BezierShowPoints \else pop \fi
        } ifelse
      end
    }%
  \end@OpenObj
}%
%
% \begin{macro}{\psintersect}
%    \begin{macrocode}
\def\psintersect{\pst@object{psintersect}}
\def\psintersect@i#1#2{%
  \PIT@checkname{#1}%
  \PIT@checkname{#2}%
  \begin@SpecialObj
  \def\PIT@@name{%
    \ifx\PIT@key@name\@empty
      \PIT@name{\PIT@name@default}
    \else
      \PIT@name{\PIT@key@name} 
    \fi}%
  \PIT@Verb{%
    currentdict /\PIT@name{#1} known not {
      (You haven't defined the curve or path '#1') ==
    } if
    currentdict /\PIT@name{#2} known not {
      (You haven't defined the curve or path '#2') ==
    } if
    \PIT@name{#1} \PIT@name{#2}
    \PIT@name{#1} IsPath {
      \PIT@name{#2} IsPath {
        IntersectPaths
      }{
        IntersectPathCurve
      } ifelse
    }{
      \PIT@name{#2} IsPath {
        IntersectCurvePath
      }{
        IntersectBeziers
        4 copy LoadIntersectionPoints 5 1 roll
      } ifelse
    } ifelse
    /\PIT@@name\space /\PIT@name{#1} /\PIT@name{#2} 8 3 roll 
    SaveIntersection
  }%
  \ifPIT@saveintersections
    \pst@Verb{%
      \pst@intersectdict 
        \PIT@@name\space /Points get 
        ArrayToPointArray
      end
      tx@NodeDict begin 
        dup length 1 1 3 -1 roll {
          2 copy 1 sub get cvx
          false 3 -1 roll (N@\PIT@key@name) exch 20 string cvs 
          \pst@intersectdict strcat end cvn
          10 {InitPnode} /NodeScale {} def NewNode
        } for
      end
      pop
    }%
  \fi
  \ifshowpoints
    \addto@pscode{%
      \pst@intersectdict
        [ \PIT@@name\space /Points get aload pop 
      end 
    }%
    \psdots@ii
  \else
    \end@SpecialObj
  \fi
}%
%    \end{macrocode}
% \end{macro}
% 
%    \begin{macrocode}
\catcode`\@=\PstAtCode\relax
%    \end{macrocode}
%</texfile> 
%
% \chapter{The Postscript header file}
% \makeatletter
%^^A Copied this definition from doc.sty and changed it not to add a
%^^A backslash to the Postscript procedure name in the index.
% \def\SpecialIndex@#1#2{%
%    \@SpecialIndexHelper@#1\@nil
%    \def\@tempb{ }%
%    \ifcat \@tempb\@gtempa
%       \special@index{\quotechar\space\actualchar
%                      \string\verb\quotechar*\verbatimchar
%                      \quotechar\space\verbatimchar#2}%
%    \else
%      \def\@tempb##1##2\relax{\ifx\relax##2\relax
%           \def\@tempc{\special@index{\quotechar##1\actualchar
%                       \string\verb\quotechar*\verbatimchar
%                       \quotechar##1\verbatimchar#2}}%
%         \else
%           \def\@tempc{\special@index{##1##2\actualchar
%                        \string\verb\quotechar*\verbatimchar##1##2\verbatimchar#2}}%
%         \fi}%
%      \expandafter\@tempb\@gtempa\relax
%      \@tempc
%    \fi}
% \makeatother
%
%<*prolog>
%    \begin{macrocode}
/tx@IntersectDict 200 dict def
tx@IntersectDict begin
%    \end{macrocode}
% These are some helper procedures for vector operations.
%
% \begin{macro}{VecAdd}
% Addition of two vectors.
% \begin{pssyntax}
%   \PSvar{Xa Ya Xb Yb} \PSop{VecAdd} \PSvar{Xa+Xb Ya+Yb}
% \end{pssyntax}
%    \begin{macrocode}
/VecAdd {
    3 -1 roll add 3 1 roll add exch
} bind def
%    \end{macrocode}
% \end{macro}
%
% \begin{macro}{VecSub}
% Subtraction of two vectors.
% \begin{pssyntax}
%   \PSvar{Xa Ya Xb Yb} \PSop{VecSub} \PSvar{Xa-Xb Ya-Yb}
% \end{pssyntax}
%    \begin{macrocode}
/VecSub {
    neg 3 -1 roll add 3 1 roll neg add exch
} bind def
%    \end{macrocode}
% \end{macro}
% 
% \begin{macro}{VecScale}
% Scale a vector by a factor \PSvar{fac}.
% \begin{pssyntax}
%   \PSvar{Xa Ya fac} \PSop{VecScale} \PSvar{fac}$\cdot$\PSvar{Xa} \PSvar{fac}$\cdot$\PSvar{Ya}
% \end{pssyntax}
%    \begin{macrocode}
/VecScale {
  dup 4 -1 roll mul 3 1 roll mul
} bind def
%    \end{macrocode}
% \end{macro}
%
% \begin{macro}{ToVec}
%   Convert two numbers to a procedure holding the two values. This
%   representation is used to save coordinate values of nodes and vectors.
%   \begin{pssyntax}
%     \PSvar{X Y} \PSop{ToVec} \PSarray{X Y}
%   \end{pssyntax}
%    \begin{macrocode}
/ToVec {
    [ 3 1 roll ]
} bind def
%    \end{macrocode}
% \end{macro}
%
% \PSvar{MaxPrecision} gives the precision of the curve parameter t for the
% intersection. This shouldn't be lower than $10^{-6}$, because
% PostScript uses single precision.
%    \begin{macrocode}
/MaxPrecision 1e-6 def
%    \end{macrocode}
%
% \PSvar{Epsilon} gives the allowed relative error of the intersection point. 
%    \begin{macrocode}
/Epsilon 1e-4 def
%    \end{macrocode}
% 
% The threshold for curve subdivision, see below.
%    \begin{macrocode}
/MinClippedSizeThreshold 0.8 def
%    \end{macrocode}
%
% The predefined intervals for the subdivision of the curves.
%    \begin{macrocode}
/H1Interval [0 0.5] def
/H2Interval [0.5 MaxPrecision add 1] def
%    \end{macrocode}
% 
% \begin{macro}{IntersectBeziers}
%   The main procedure, which computes the intersection of two bezier
%   curves of arbitrary order.  This, and most of the following
%   procedures operate on curves, which are stored as arrays of points,
%   the points are also arrays with two elements -- \PSvar{X} and
%   \PSvar{Y}. A Bezier curve of $n$-th order is then givesn by
%   \PSarray{\PSarray{X0 Y0} \PSarray{X1 Y1} \ldots \PSarray{XN YN}}.
%
% \begin{pssyntax}
%   \PSarray{curveA} \PSarray{curveB} \PSop{IntersectBeziers} 
%   \PSarray{curveA} \PSarray{tA} \PSarray{curveB} \PSarray{tB}
% \end{pssyntax}
%    \begin{macrocode}
/IntersectBeziers {
  2 copy length 2 eq exch length 2 eq and {
    IntersectLines
  }{ 
    2 copy [0 1] [0 1] IterateIntersection
  } ifelse
  3 -1 roll exch
} bind def
%    \end{macrocode}
% \end{macro}
%
% \begin{macro}{IntersectLines}
% 
%   \begin{pssyntax}
%     \PSarray{lineA} \PSarray{lineB} \PSop{IntersectLines}
%     \PSarray{lineA} \PSarray{tA} \PSarray{lineB} \PSarray{tB}
%   \end{pssyntax}
%    \begin{macrocode}
/IntersectLines {
  (IntersectLines) DebugBegin
  2 copy
  exch { aload pop } forall 5 -1 roll { aload pop } forall
  8 -2 roll 2 copy 10 4 roll 4 2 roll 2 copy 6 2 roll 10 2 roll
  VecSub
  6 2 roll 4 2 roll VecSub
  8 4 roll 4 2 roll VecSub % X3-X4 Y3-Y4 X2-X1 Y2-Y1 X3-X1 Y3-Y1 % b1 b2 a1 a2 c1 c2
  6 copy 12 -4 roll 
  neg 4 -1 roll mul 3 1 roll mul add
  dup 0 eq {
    % no intersections
    9 { pop } repeat [] []
  } {
    dup 10 1 roll 5 1 roll
     4 -1 roll mul 3 1 roll mul sub exch div
     6 1 roll 4 -1 roll mul 3 1 roll mul sub exch div
     2 copy 2 copy 0 ge exch 0 ge and 3 1 roll 1 le exch 1 le and and {
       [ exch ] exch [ exch ]
     } {
       pop pop [] []
     } ifelse
  } ifelse
  DebugEnd
} bind def
%    \end{macrocode}
% \end{macro}
% \begin{macro}{IntersectPaths}
%   \begin{pssyntax}
%     \PSarray{pathA} \PSarray{pathB} \PSop{IntersectPaths}
%     \PSarray{intersections} \PSarray{pathA} \PSarray{tA} \PSarray{pathB} \PSarray{tB}
%   \end{pssyntax}
%    \begin{macrocode}
/IntersectPaths {
  (IntersectPaths) DebugBegin
  6 dict begin 
    2 copy exch PreparePath dup length /nA exch def 
    exch PreparePath dup length /nB exch def
    /isect [] def
    /tA [] def /tB [] def
    { % [pathA] [Bi]
      /nB nB 1 sub def
      exch dup 3 1 roll % [pathA] [Bi] [pathA]
      {
        /nA nA 1 sub def
        exch dup 3 1 roll % [pathA] [Bi] [Aj] [Bi]
        IntersectBeziers % [curveA] [tA] [curveB] [tB]
        4 copy LoadIntersectionPoints
        [ exch isect aload pop ] /isect exch def
        exch pop 3 -1 roll pop
        [ tB aload length 2 add -1 roll TArray { nB add } forall ] /tB exch def
        [ tA aload length 2 add -1 roll TArray { nA add } forall ] /tA exch def
      } forall
      pop % remove [Bi]
      dup length /nA exch def
    } forall
    pop % remove [pathA]
    [ isect { aload pop } forall ] 3 1 roll tA exch tB
    % [intersections] [pathA] [tA] [pathB] [tB]
  end
  DebugEnd
} bind def
%    \end{macrocode}
% \end{macro}
%
% \begin{macro}{IntersectCurvePath}
%   \begin{pssyntax}
%     \PSarray{curveA} \PSarray{pathB} \PSop{IntersectCurvePath}
%     \PSarray{intersections} \PSarray{curveA} \PSarray{tA} \PSarray{pathB} \PSarray{tB}
%   \end{pssyntax}
%    \begin{macrocode}
/IntersectCurvePath {
  (IntersectCurvePath) DebugBegin
  6 dict begin 
    2 copy PreparePath dup length /n exch def
    /isect [] def
    /tA [] def /tB [] def
    { % [curveA] [Bi]
      /n n 1 sub def
      exch dup 3 -1 roll % [curveA] [curveA] [Bi] 
      IntersectBeziers 
      4 copy LoadIntersectionPoints % [curveA] [tA] [curveB] [tB]
      [ exch isect aload pop ] /isect exch def
      pop 3 -1 roll pop
      [ tB aload length 2 add -1 roll TArray { n add } forall ] /tB exch def
      [ tA aload length 2 add -1 roll TArray aload pop ] /tA exch def
    } forall
    pop % remove [curveA]
    [ isect { aload pop } forall ] 3 1 roll tA exch tB
    % [intersections] [curveA] [tA] [pathB] [tB]
  end
  DebugEnd
} bind def
/IntersectPathCurve {
  exch IntersectCurvePath 4 2 roll
} bind def
%    \end{macrocode}
% \end{macro}
% 
% \begin{macro}{SaveIntersection}
%   \begin{pssyntax}
%     \PSname{isectname} \PSname{nameA} \PSname{nameB} 
%     \PSarray{intersectionpoints} \PSarray{A} \PSarray{tA} \PSarray{B} \PSarray{tB} 
%     \PSop{SaveIntersection}
%   \end{pssyntax}
%    \begin{macrocode}
/SaveIntersection {
  (SaveIntersection) DebugBegin
  4 dict dup 10 -1 roll exch def
  begin %
    /Points 6 -1 roll def
    5 -1 roll dup 4 -1 roll def % /curveA [curveA] [tA] [tB] /curveB /curveB [curveB] def
    nametostr (@t) strcat cvn exch TArray def % /curveA [curveA] [tA] /curveB@t [tB] def
    3 -1 roll dup 4 -1 roll def
    nametostr (@t) strcat cvn exch TArray def
  end
  DebugEnd
} bind def
%    \end{macrocode}
% \end{macro}
% 
% \begin{macro}{TArray}
%   The curve parameters \PSvar{t} as determined by
%   \PSvar{IntersectBeziers} are given in a special array
%   construct. \PSvar{TArray} creates a simple array with the
%   \PSvar{t}-values given in ascending order.
%
% \begin{pssyntax}
% \PSarray{\PSarray{t0a t0b} \ldots \PSvar{null}\ldots \PSvar{integer}}
% \PSop{TArray} \PSarray{t0 t1 \ldots tN}
% \end{pssyntax}
%    \begin{macrocode}
/TArray {
  (TArray) DebugBegin
  dup length 0 gt {
    dup 0 get type /arraytype eq {
      [ exch
      { %dup type /nulltype eq { pop exit } if
  	aload pop add 0.5 mul
      } forall ]
    } if
    dup /lt exch quicksort
  } if
  DebugEnd %1 debug
} bind def
%    \end{macrocode}
% \end{macro} 
% 
% We can save arbitrary paths using \PSvar{pathforall}. The saved path
% contains the commands \PSname{movetype}, \PSname{linetype} and
% \PSname{curvetype}. By default, these are defined as the respective
% original procedures.
%    \begin{macrocode}
/movetype { /moveto load } bind def
/linetype { /lineto load } bind def
/curvetype { /curveto load } bind def
%    \end{macrocode}
%
% [ ... /movetype ... /linetype .../curvetype ]
%    \begin{macrocode}
/PreparePath {
    [ exch aload pop
    {
	dup type /nametype eq not { exit } if
	dup /movetype eq {
	    pop ToVec /@mycp exch def
	} {
	    dup /linetype eq {
		pop [ @mycp 4 2 roll 2 copy ToVec /@mycp exch def ToVec ]
	    } {
		pop [ @mycp 8 2 roll 2 copy ToVec /@mycp exch def
		ToVec 5 1 roll ToVec 4 1 roll ToVec 3 1 roll ]
	    } ifelse
	    counttomark 1 roll	
	} ifelse
    } loop ]
} bind def
%    \end{macrocode}
%
% \begin{macro}{LoadLineIntersectionPoints}
% Prepare \PSarray{Curve} for use with tx@Func
% \begin{pssyntax}
% \PSarray{curve} \PSarray{t} \PSop{LoadLineIntersectionPoints}
% \PSarray{I0.x I0.y \ldots IN.x YN.x}
% \end{pssyntax}
%    \begin{macrocode}
/LoadLineIntersectionPoints {
  (LoadLineIntersectionPoints) DebugBegin
  exch [ exch { aload pop } forall ]
  tx@Dict begin tx@FuncDict begin 2 dict begin
    dup length 2 idiv 1 sub /BezierType exch def /Points exch def
    [ exch { GetBezierCoor } forall ]
  end end end
  DebugEnd
} bind def
%    \end{macrocode}
% \end{macro}
%
% \begin{macro}{LoadCurveIntersectionPoints}
%   Load the intersection points. This loads the same intersection point
%   from both curves, and chooses the one with the lowest error.
% \begin{pssyntax}
% \PSarray{curveA} \PSarray{tA} \PSArray{curveB} \PSarray{tB} 
%  \PSop{LoadCurveIntersectionPoints}
% \PSarray{I0.x I0.y \ldots IN.x YN.x}
% \end{pssyntax}
%    \begin{macrocode}
/LoadCurveIntersectionPoints {
  (LoadCurveIntersectionPoints) DebugBegin
  2 {
    4 2 roll
    [ exch { aload pop } forall ]
    exch [ exch { aload pop } forall ]
  } repeat
  % [A0.x A0.y ... AM.x AM.y] [tA0a tA0b ... tAMa tAMb] [tB0a tB0b ... tBNa tBNb] [B0.x B0.y ... BN.x BN.y]
  tx@Dict begin tx@FuncDict begin 2 dict begin
    dup length 2 idiv 1 sub /BezierType exch def /Points exch def
      [ exch { GetBezierCoor } forall ]
    3 1 roll 
    dup length 2 idiv 1 sub /BezierType exch def /Points exch def
      [ exch { GetBezierCoor } forall ]
    end
    %2 debug
    % [IB0.xa IB0.ya IB0.xb IB0.yb ... IBM.yb] [IA0.xa IA0.ya IA0.xb IA0.yb ... IAM.yb]
    2 {
      [ exch aload length 4 idiv {
        [ 5 1 roll ] counttomark 1 roll
      } repeat ]
      exch 
    } repeat
    % [[IB0.xa ...] ... [... IBM.yb]] [[IA0.xa IA0.ya IA0.xb IA0.yb] ...[IAM.xa ... IAM.yb]]
    2 {
      dup hulldict /comp get exch quicksort exch
    } repeat
    2 dict begin
      /B exch def /A exch def
      [ 0 1 A length 1 sub {
        dup A exch get exch B exch get % [IAi] [IBi]
        2 copy aload pop VecSub Pyth exch 
        aload pop VecSub Pyth lt { exch } if pop
        aload pop VecAdd 0.5 VecScale
      } for 
      % merge near intersection points
      %counttomark 2 idiv 1 1 3 -1 roll {
      %  pop
      %  counttomark 4 lt { exit } if
      %  4 copy 4 2 roll ToVec 3 1 roll ToVec AreNear {

      %  }
      %} for
      ]
    end
  end end
  DebugEnd
} bind def
%    \end{macrocode}
% \end{macro}
%
% \begin{macro}{LoadIntersectionPoints}
%    \begin{macrocode}
/LoadIntersectionPoints {
  (LoadIntersectionPoints) DebugBegin
  4 copy pop exch pop length 2 eq exch length 2 eq and {
    pop pop LoadLineIntersectionPoints
  }{
    LoadCurveIntersectionPoints
  } ifelse
  DebugEnd
} bind def
%    \end{macrocode}
% \end{macro}
% 
% \begin{macro}{IterateIntersection}
% Iteration procedure to compute all intersections of CurveA and CurveB.
% This contains the
% 
% \begin{pssyntax}
% \PSarray{CurveA} \PSarray{CurveB} \PSarray{intervalA} \PSarray{intervalB}
% \PSop{IterateIntersection} \PSarray{domsA} \PSarray{domsB}
% \end{pssyntax}
%    \begin{macrocode}
/IterateIntersection {
    (IterateIntersection) DebugBegin
    11 dict begin
	/precision MaxPrecision def
%    \end{macrocode}
% in order to limit recursion
%    \begin{macrocode}
        /counter 0 def
	/depth 0 def
	/domsA [] def
	/domsB [] def
	/domsA /domsB 6 2 roll _IterateIntersection
	domsB domsA
    end
    dup length 0 gt {
      TArraysRemoveDup
    } if
    DebugEnd
} bind def
/TArraysRemoveDup {
  4 dict begin
    /tB exch def
    /tA exch def
    /j 0 def
    [ tA 0 get tB 0 get
    1 1 tA length 1 sub {
      /i exch def
      tA j get aload pop tA i get aload pop tx@Dict begin Pyth2 end MaxPrecision gt
      tB j get aload pop tB i get aload pop tx@Dict begin Pyth2 end MaxPrecision gt and {
        % keep the current parameter point
        /j i def
        tB i get tA i get
        counttomark 2 idiv 1 add 1 roll 
      } if
    } for
    counttomark 2 idiv 1 add [ exch 1 roll ] % [ ... [tB]
    counttomark 1 add 1 roll ] exch % [tA] [tB]
  end 
} bind def
%    \end{macrocode}
% \end{macro}
% 
% \begin{macro}{_IterateIntersection}
% This is the iteration part which is called recursively.
%
% \begin{pssyntax}
%   \PSname{domsA} \PSname{domsB} \PSarray{CurveA} \PSarray{CurveB}
%   \PSarray{domA} \PSarray{domB} \PSop{_IterateIntersection}
% \end{pssyntax}
%    \begin{macrocode}
/_IterateIntersection {
    (_IterateIntersection) DebugBegin
    CloneVec /domB exch def
    CloneVec /domA exch def
    CloneCurve /CurveB exch def
    CloneCurve /CurveA exch def
    /iter 0 def
    /depth depth 1 add def
    /dom null def
    /counter counter 1 add def

    CheckIT {
	(>> curve subdivision performed: dom(A) = ) domA CurveToString strcat
	(, dom(B) = ) strcat domB CurveToString strcat ( <<) strcat ==
    } if
    CurveA IsConstant CurveB IsConstant and {
	CurveA MiddlePoint ToVec
	CurveB MiddlePoint ToVec AreNear {
	    domA domB 4 -1 roll exch PutInterval PutInterval
	} {
	    pop pop
	} ifelse
    }{
	counter 100 lt {
%    \end{macrocode}
% Use a loop to simulate some kind of return to exit at different positions.
%    \begin{macrocode}
	    {
		/iter iter 1 add def
		iter 100 lt
		domA Extent precision ge
		domB Extent precision ge or and not {
		    iter 100 ge {
			false 
		    } {
			CurveA MiddlePoint ToVec
			CurveB MiddlePoint ToVec AreNear {
			    domA domB true
			}{
			    false
			} ifelse
		    } ifelse
		    exit
		} if
%    \end{macrocode}
% iter < 100 && (dompA.extent() >= precision || dompB.extent() >= precision)
%    \begin{macrocode}
		CheckIT {
		    (counter: ) counter 20 string cvs strcat
		    (, iter: ) iter 20 string cvs strcat strcat
		    (, depth: ) depth 20 string cvs strcat strcat ==
		} if
	
		CurveA CurveB ClipCurve /dom exch def
	
		CheckIT {(dom : ) dom CurveToString strcat == } if		
		dom IsEmptyInterval {
		    CheckIT { (empty interval, exit) == } if
		    false exit
		} if
%    \end{macrocode}
% dom[0] > dom[1], invalid.
%    \begin{macrocode}
		dom aload pop 2 copy min 3 1 roll max gt {
		    CheckIT {
			(dom[0] > dom[1], invalid!) ==
		    } if
		    false exit
		} if

		domB dom MapTo /domB exch def
		CurveB dom Portion

		CurveB IsConstant CurveA IsConstant and {
		    CheckIT {
          		(both curves are constant: ) ==	
			(C1: [ ) CurveA { CurveToString ( ) strcat strcat } forall (]) strcat ==
			(C2: [ ) CurveB { CurveToString ( ) strcat strcat } forall (]) strcat ==
		    } if
		    CurveA MiddlePoint ToVec
		    CurveB MiddlePoint ToVec AreNear {
			domA domB true
		    } {
			false
		    } ifelse
		    exit
		} if
%    \end{macrocode}
% If we have clipped less than 20%, we need to subdivide the
% curve with the largest domain into two sub-curves.
%    \begin{macrocode} 
		dom Extent MinClippedSizeThreshold gt {
		    CheckIT {
			(clipped less than 20% : ) ==
			(angle(A) = ) CurveA dup length 1 sub get aload pop
				      CurveA 0 get aload pop VecSub
   				      exch 2 copy 0 eq exch 0 eq and {
					  pop pop (NaN)
				      } {
					  atan 20 string cvs
				      } ifelse strcat ==
		        (angle(B) = ) CurveB dup length 1 sub get aload pop
		                      CurveB 0 get aload pop VecSub
				      exch 2 copy 0 eq exch 0 eq and {
					  pop pop (NaN)
				      } {
					  atan 20 string cvs
				      } ifelse strcat ==
		        (dom : ) == dom == (domB :) == domB ==
		    } if
%    \end{macrocode}
% Leave those five values on the stack to revert to the current state after the recursive calls.
%    \begin{macrocode}
		    CurveA CurveB domA domB iter
     		    7 -2 roll 2 copy 9 2 roll 2 copy 
%    \end{macrocode}
% On the stack: /domsA /domsB CurveA CurveB domA domB iter /domsA /domsB /domsA /domsB
%    \begin{macrocode}
		    domA Extent domB Extent gt {
			CurveA CloneCurve dup H1Interval Portion % pC1
			CurveA CloneCurve dup H2Interval Portion % pC2
			domA H1Interval MapTo                    % dompC1
			domA H2Interval MapTo                    % dompC2
%    \end{macrocode}
% Need on the stack: /domsA /domsB pC2 CurveB dompC2 domB   /domsA /domsB pC1 CurveB dompC1 domB
%    \begin{macrocode}
			3 -1 roll exch % /domsA /domsB /domsA /domsB pC1 dompC1 pC2 dompC2
			CurveB exch domB 8 4 roll % /domsA /domsB pC2 CurveB dompC2 domB /domsA /domsB pC1 dompC1
			CurveB exch domB % /domsA /domsB pC2 CurveB dompC2 domB /domsA /domsB pC1 CurveB dompC1 domB
		    } {
			CurveB CloneCurve dup H1Interval Portion % pC1
			CurveB CloneCurve dup H2Interval Portion % pC2
			domB H1Interval MapTo                    % dompC1
			domB H2Interval MapTo                    % dompC2
%    \end{macrocode}
% Need on the stack: /domsB /domsA pC2 CurveA dompC2 domA   /domsB /domsA pC1 CurveA dompC1 domA
%    \begin{macrocode}
			8 -2 roll exch 8 2 roll 6 -2 roll exch 6 2 roll % /domsB /domsA /domsB /domsA pC1 pC2 dompC1 dompC2
			3 -1 roll exch % /domsB /domsA /domsB /domsA pC1 dompC1 pC2 dompC2
			CurveA exch domA 8 4 roll % /domsB /domsA pC2 CurveA dompC2 domA /domsB /domsA pC1 dompC1
			CurveA exch domA          % /domsB /domsA pC2 CurveA dompC2 domA /domsB /domsA pC1 CurveA dompC1 domA
		    } ifelse

		    _IterateIntersection
		    _IterateIntersection
%    \end{macrocode}		    
% Restore the state before the recursive calls.
%    \begin{macrocode}
		    /iter exch def
		    /domB exch def
		    /domA exch def
		    /CurveB exch def
		    /CurveA exch def
		    false exit
		} if
		CurveA CurveB /CurveA exch def /CurveB exch def
		domA domB /domA exch def /domB exch def
%    \end{macrocode}
% exchange /domsA and /domsB on the stack!
%    \begin{macrocode}
		exch
	    } loop	
%    \end{macrocode}
% boolean on stack
%    \begin{macrocode}
	    {
		4 -1 roll exch PutInterval PutInterval
		CheckIT {
		    (found an intersection ============================) ==
		} if
	    } { pop pop } ifelse
	} {
	    pop pop
	} ifelse
    } ifelse
    /depth depth 1 sub def
    DebugEnd
} bind def
%    \end{macrocode}
% \end{macro}
% 
% \begin{macro}{PutInterval}
%   Add a new interval \PSarray{newinterval} to the array stored in
%   \PSname{/Intervals}. The new interval is "cloned" before storing it.
% \begin{pssyntax}
% \PSname{Intervals} \PSarray{newinterval} \PSop{PutInterval}
% \end{pssyntax}
%    \begin{macrocode}
/PutInterval {
    CloneVec [ exch 3 -1 roll dup 4 1 roll load aload pop ] def
} bind def
%    \end{macrocode}
% \end{macro}
% 
% \begin{macro}{IsEmptyInterval}
% Check if an interval is empty, which is represented by a [1 0] interval.
% \begin{pssyntx}
% \PSarray{interval} \PSop{IsEmptyInterval} \PSvar{boolean}
%    \begin{macrocode}
/IsEmptyInterval {
    aload pop 0 eq exch 1 eq and
} bind def
%    \end{macrocode}
% \end{macro}
%
% \begin{macro}{ToUnitInterval}
% Limit an interval \PSvar{a b} to the unit interval \PSarray{0 1}.
% \begin{pssyntax}
% \PSvar{a b} \PSop{ToUnitInterval} \PSarray{a|0 b|1}
% \end{pssyntax}
%    \begin{macrocode}
/ToUnitInterval {
    ToUnitRange exch ToUnitRange 2 copy gt {
	exch
    } if
    ToVec
} bind def
%    \end{macrocode}
% \end{macro}
% \begin{macro}{ToUnitRange}
% Limit a number to the range \PSarray{0 1}.
%    \begin{macrocode}
/ToUnitRange {
    dup 0 lt {
	pop 0
    }{
	dup 1 gt {
	    pop 1
	} if
    } ifelse
} bind def
%    \end{macrocode}
% \end{macro}
% 
% \begin{macro}{CloneCurve}
% Does a deep copy of the array \PSarray{Curve}. This also involved deep copies of the contained point arrays.
% \begin{pssyntax}
% \PSarray{Curve} \PSop{CloneCurve} \PSarray{newCurve}
%    \begin{macrocode}
/CloneCurve {
    [ exch {
	CloneVec
    } forall ]
} bind def
%    \end{macrocode}
% \end{macro}
%
% \begin{macro}{CloneVec}
% Does a deep copy of the vector \PSarray{X Y}
% \begin{pssyntax}
% \PSarray{X Y} \PSop{CloneVec} \PSarray{Xnew Ynew}
% \end{pssyntax}
%    \begin{macrocode}
/CloneVec {
    aload pop ToVec
} bind def
%    \end{macrocode}
% \end{macro}
% 
% \begin{macro}{MapTo}
% Map the sub-interval \PSarray{I} in \PSarray{0 1} into the interval \PSarray{J}. Returns a new array.
% \begin{pssyntax}
% \PSarray{J} \PSarray{I} \PSop{MapTo} \PSarray{Jnew}
% \end{pssyntax}
%    \begin{macrocode}
/MapTo {
    (MapTo) DebugBegin
    exch aload 0 get 3 1 roll exch sub 2 copy % [I] J0 Jextent J0 Jextent
    5 -1 roll aload aload pop % J0 Jextent J0 Jextent I0 I1 I0 I1
    min 4 -1 roll mul % J0 Jextent J0 I0 I1 min(I0,I1)*Jextent
    4 -1 roll add [ exch % J0 Jextent I0 I1 [ J0new
    6 2 roll max mul add ]
    DebugEnd
} bind def
%    \end{macrocode}
% \end{macro}
% 
% \begin{macro}{Portion}
% Compute the portion of the Bezier curve \PSarray{CurveB} wrt the interval \PSarray{I}.
% \begin{pssyntax}
% \PSarray{CurveB} \PSarray{I} \PSop{Portion} \PSarray{CurvePartB}
% \end{pssyntax}
%    \begin{macrocode}
/Portion {
    (Portion) DebugBegin
    dup Min 0 eq { % [CurveB] [I]
	% I.min() == 0
	Max dup 1 eq {% [CurveB] I.max()
	    % I.max() == 1
	    pop pop	    
	} { % [CurveB] I.max()
	    LeftPortion
	} ifelse
    } { % [CurveB] [I]
	2 copy Min % [CurveB] [I] [CurveB] I.min()
	RightPortion
	dup Max 1 eq {
	    % I.max() == 1
	    pop pop
	} {% [CurveB] [I]
	    dup aload pop exch sub 1 3 -1 roll Min sub div % [CurveB] (I1-I0)/(1-I.min())
	    LeftPortion
	} ifelse
    } ifelse
    DebugEnd
} bind def
%    \end{macrocode}
% \end{macro}
% 
% \begin{macro}{LeftPortion}
%   Compute the portion of the Bezier curve \PSarray{CurveB} wrt the
%   interval \PSarray{0 t}.
% \begin{pssyntax}
% \PSarray{CurveB} \PSvar{t} \PSop{LeftPortion} \PSarray{CurvePartB}
% \end{pssyntax}
%    \begin{macrocode}
/LeftPortion {
    (LeftPortion) DebugBegin
    exch dup length 1 sub dup 4 1 roll % L-1 t [CurveB] L-1
    1 1 3 -1 roll { % L-1 t [CurveB] i
	4 -1 roll dup 5 1 roll % L-1 t [CurveB] i L-1
	-1 3 -1 roll % L-1 t [CurveB] L-1 -1 i
	{ % L-1 t [CurveB] j
	    2 copy 5 copy % L-1 t [CurveB] j [CurveB] j t [CurveB] j [CurveB] j 
	    1 sub get 3 1 roll get % L-1 t [CurveB] j [CurveB] j t B[j-1] B[j]
	    Lerp put pop % L-1 t [CurveB]
	} for
    } for
    pop pop pop
    DebugEnd
} bind def
%    \end{macrocode}
% \end{macro}
%
% \begin{macro}{RightPortion}
%   Compute the portion of the Bezier curve \PSarray{CurveB} wrt the
%   interval \PSarray{t 1}.
% \begin{pssyntax}
% \PSarray{CurveB} \PSvar{t} \PSop{RightPortion} \PSarray{CurvePartB}
% \end{pssyntax}
%    \begin{macrocode}
/RightPortion {
    (RightPortion) DebugBegin
    exch dup length 1 sub dup 4 1 roll % L-1 t [CurveB] L-1
    1 1 3 -1 roll {% L-1 t [CurveB] i
	4 -1 roll dup 5 1 roll % L-1 t [CurveB] i L-1
	exch sub 0 1 3 -1 roll  % L-1 t [CurveB] 0 1 L-i-1
	{% L-1 t [CurveB] j
	    2 copy 5 copy
	    get 3 1 roll 1 add get Lerp put pop
	} for
    } for
    pop pop pop
    DebugEnd
} bind def
%    \end{macrocode}
% \end{macro}
% 
% \begin{macro}{Lerp}
% Given two points and a parameter \PSvar{t} $\in$ \PSarray{0 1}, return a point
% proportionally from \PSarray{A} to \PSarray{B} by \PSvar{t}. Akin to 1 degree Bezier.
% \begin{pssyntax}
% \PSvar{t} \PSarray{A} \PSarray{B} \PSop{Lerp} \PSarray{newpoint}
% \end{pssyntax}
%    \begin{macrocode}
/Lerp {
    (Lerp) DebugBegin
    3 -1 roll dup 1 exch sub 3 1 roll % [A] (1-t) [B] t
    exch aload pop 3 -1 roll VecScale % [A] (1-t) B.x*t B.y*t
    4 2 roll
    exch aload pop 3 -1 roll VecScale VecAdd ToVec % [A.x*(1-t)+B.x*t A.y*(1-t)+B.y*t]
    DebugEnd
} bind def
%    \end{macrocode}
% \end{macro}
% 
% \begin{macro}{IsConstant}
% Test if all points of a curve are near to each other. This is used as termination criterium for the intersection procedure.
% \begin{pssyntax}
% \PSarray{Curve} \PSop{IsConstant} \PSvar{boolean}
% \end{pssyntax}
%    \begin{macrocode}
/IsConstant {
    aload length [ exch 1 roll ] true 3 1 roll
    {
	exch dup 4 1 roll
	AreNear and exch
    } forall
    pop
} bind def
%    \end{macrocode}
% \end{macro}
% \begin{macro}{AreNear}
% Test if two points are near to each other.
% \begin{pssyntax}
% \PSarray{P1} \PSarray{P2} \PSop{AreNear} \PSvar{boolean}
% \end{pssyntax}
%    \begin{macrocode}
/AreNear {
    (AreNear) DebugBegin
    aload pop 3 -1 roll aload pop
    4 copy abs 3 { exch abs max } repeat Epsilon mul
    dup 6 2 roll VecSub abs 4 -1 roll lt exch abs 3 -1 roll lt and
    DebugEnd
} bind def
%    \end{macrocode}
% \end{macro}
% 
% \begin{macro}{Min}
% Get the minimum value of the vector \PSarray{P}.
% \begin{pssyntax}
% \PSarray{P} \PSop{Min} \PSvar{minimum}
% \end{pssyntax}
%    \begin{macrocode}
/Min {
    aload pop min
} bind def
%    \end{macrocode}
% \end{macro}
% \begin{macro}{Min}
% Get the maximum value of the vector \PSarray{P}.
% \begin{pssyntax}
% \PSarray{P} \PSop{Max} \PSvar{maximum}
% \end{pssyntax}
%    \begin{macrocode}
/Max {
    aload pop max
} bind def
%    \end{macrocode}
% \end{macro}
% \begin{macro}{Min}
% Get the extent of the interval \PSarray{I}.
% \begin{pssyntax}
% \PSarray{I} \PSop{Extent} \PSvar{I1-I0}
% \end{pssyntax}
%    \begin{macrocode}
/Extent {
    aload pop exch sub
} bind def
%    \end{macrocode}
% \end{macro}
%
% \begin{macro}{MiddlePoint}
% Compute the middle point of the first and last point of \PSarray{Curve}.
% \begin{pssyntax}
% \PSarray{Curve} \PSop{MiddlePoint} \PSvar{X Y}
% \end{pssyntax}
%    \begin{macrocode}
/MiddlePoint {
    dup dup length 1 sub get aload pop
    3 -1 roll 0 get aload pop
    VecAdd 0.5 VecScale
} bind def
%    \end{macrocode}
% \end{macro}
% 
% \begin{macro}{OrthogonalOrientationLine}
% \begin{pssyntax}
% \PSvar{MiddlePointA} \PSarray{CurveB} \PSop{OrthogonalOrientationLine} \PSvar{A B C}
% \end{pssyntax}
%    \begin{macrocode}
/OrthogonalOrientationLine {
    (OrthogonalOrientationLine) DebugBegin
    dup dup length 1 sub get aload pop 3 -1 roll 0 get aload pop VecSub
%    \end{macrocode}
% rotate by +90 degrees
%    \begin{macrocode}
    neg exch
    4 2 roll 2 copy 6 2 roll VecAdd
    ImplicitLine
    DebugEnd
} bind def
%    \end{macrocode}
% \end{macro}
% 
% \begin{macro}{PickOrientationLine}
%   Pick an orientation line for a Bezier curve. This uses the first
%   point and the lastmost point, which is not near to it.
% \begin{pssyntax}
% \PSarray{Curve} \PSop{PickOrientationLine} \PSvar{A B C}
% \end{pssyntax}
%    \begin{macrocode}
/PickOrientationLine {
    (PickOrientationLine) DebugBegin
    dup dup length 1 sub exch 0 get% [Curve] L-1 P0
    exch -1 1 {% [Curve] P0 i
	3 -1 roll dup 4 1 roll exch get % [Curve] P0 Pi
	2 copy AreNear {
	    pop
	} {
	    exit
	} ifelse
    } for
    3 -1 roll pop
    exch aload pop 3 -1 roll aload pop ImplicitLine
    DebugEnd
} bind def
%    \end{macrocode}
% \end{macro}
% 
% \begin{macro}{ImplicitLine}
% Compute the coefficients \PSvar{A}, \PSvar{B}, \PSvar{C} of the normalized implicit equation
% of the line which goes through the points \PSarray{Xi Yi} and \PSarray{Xj Yj}.
%
% \begin{pssyntax}
% \PSvar{Xi Yi Xj Yj} \PSop{ImplicitLine} \PSvar{A B C}
% \end{pssyntax}
%    \begin{macrocode}
/ImplicitLine {
    4 copy % Xi Yi Xj Yj Xi Yi Xj Yj
    3 -1 roll sub 7 1 roll sub 5 1 roll % Yj-Yi Xi-Xj Xi Yi Xj Yj
    % Yi*Xj - Xi*Yj
    4 -1 roll mul neg % Yj-Yi Xi-Xj Yi Xj -Yj*Xi
    3 1 roll mul add % Yj-Yi Xi-Xj Yi*Xj-Yj*Xi | l0 l1 l2
    3 1 roll 2 copy tx@Dict begin Pyth end dup dup % l2 l0 l1 L L L
    5 -1 roll exch % l2 l1 L L l0 L
    div 5 1 roll % l0/L l2 l1 L L
    3 1 roll div % l0/L l2 L l1/L
    3 1 roll div % l0/L l1/L l2/L
} bind def
%    \end{macrocode}
% \end{macro}
% 
% \begin{macro}{distance}
% Compute the distance of point \PSarray{X Y} from the implicit line given
% by $Ax + By + C = 0,\quad (A^2+B^2 = 1)$.
% \begin{pssyntax}
% \PSvar{X Y A B C} \PSop{distance} \PSvar{d}
% \end{pssyntax}
%    \begin{macrocode}
/distance {
    5 1 roll 3 -1 roll mul 3 1 roll mul add add
} bind def
%    \end{macrocode}
% \end{macro}
%
% \begin{macro}{ArrayToPointArray}
% \begin{pssyntax}
% \PSarray{A.x A.y ... N.x N.y} \PSop{ArrayToPointArray} \PSarray{\PSarray{A.x A.y} \ldots \PSarray{N.x N.y}}
% \end{pssyntax}
%    \begin{macrocode}
/ArrayToPointArray {
    aload length dup 2 idiv {
	3 1 roll [ 3 1 roll ] exch
	dup 1 sub 3 1 roll 1 roll
    } repeat 1 add [ exch 1 roll ]
} bind def
%    \end{macrocode}
% \end{macro}
%
% \begin{macro}{PointArrayToArray}
% \begin{pssyntax}
% \PSarray{\PSarray{A.x A.y} \ldots \PSarray{N.x N.y}} \PSop{PointArrayToArray} \PSarray{A.x A.y ... N.x N.y}
% \end{pssyntax}
%    \begin{macrocode}
/PointArrayToArray {
    aload length dup {
	1 add dup 3 -1 roll aload pop 4 -1 roll 1 add 2 roll
    } repeat 1 add [ exch 1 roll ]
} bind def
%    \end{macrocode}
% \end{macro}
% 
% \begin{macro}{ClipCurve}
% Clip the Bezier curve B with respect to the Bezier curve A for
% individuating intersection points. The new parameter interval for the
% clipped curve is pushed on the stack.
% \begin{pssyntax}
% \PSarray{CurveA} \PSarray{CurveB} \PSop{ClipCurve} \PSarray{newinterval}
% \end{pssyntax}
%    \begin{macrocode}
/ClipCurve {
    (ClipCurve) DebugBegin
    4 dict begin 
    /CurveB exch def /CurveA exch def
    CurveA IsConstant {
    	CurveA MiddlePoint CurveB OrthogonalOrientationLine
    } {
	CurveA PickOrientationLine
    } ifelse
    CheckIT {
	3 copy exch 3 -1 roll (OrientationLine : )
	3 { exch 20 string cvs ( ) strcat strcat } repeat ==
    } if
    CurveA FatLineBounds
    CheckIT { dup (FatLineBounds : ) exch aload pop exch 20 string cvs (, ) strcat exch 20 string cvs strcat strcat == } if
    CurveB ClipCurveInterval
    end
    DebugEnd
} bind def
%    \end{macrocode}
% \end{macro}
% 
% \begin{macro}{FatLineBounds}
% Compute the boundary of the fat line given by \PSvar{A B C}
% \begin{pssyntax}
% \PSvar{A B C} \PSarray{Curve} \PSop{FatLineBounds} \PSvar{A B C} \PSarray{dmin dmax}
% \end{pssyntax}
%    \begin{macrocode}
/FatLineBounds {
    (FatLineBounds) DebugBegin
    /dmin 0 def /dmax 0 def
    { 
	4 copy aload pop 5 2 roll distance
	dup dmin lt { dup /dmin exch def } if
	dup dmax gt { dup /dmax exch def } if
	pop pop
    } forall
    [dmin dmax]
    DebugEnd
} bind def
%    \end{macrocode}
% \end{macro}
% 
% \begin{macro}{ClipCurveInterval}
%   Clip the Bezier curve wrt the fat line defined by the orientation
%   line (given by \PSvar{A B C}) and the interval range
%   \PSarray{bound}. The new parameter interval \PSarray{newinterval}
%   for the clipped curve is pushed on the stack.
% \begin{pssyntax}
% \PSvar{A B C} \PSarray{bound} \PSarray{curve} \PSop{ClipCurveInterval} \PSarray{newinterval}
% \end{pssyntax}
%    \begin{macrocode}
/ClipCurveInterval {
    (ClipCurveInterval) DebugBegin
    15 dict begin
    /curve exch def
    aload pop 2 copy min /boundMin exch def max /boundMax exch def
    [ 4 1 roll ] cvx /fatline exch def
    % number of sub-intervals
    /n curve length 1 sub def
    % distance curve control points
    /D n 1 add array def
    0 1 n { % i
	dup curve exch get aload pop % i Pi.x Pi.y
	fatline distance % distance d of Point i from the orientation line, on stack; i d
	exch dup n div % d i i/n
	[ exch 4 -1 roll ] % i [ i/n d ]
	D 3 1 roll put 
    } for
    D ConvexHull /D exch def
%    \end{macrocode}
% get the x-coordinate of the i-th point, i getX -> D[i][X]
%    \begin{macrocode}
    /getX { D exch get 0 get } def
%    \end{macrocode}
% get the y-coordinate of the i-th point, i getY -> D[i][Y]
%    \begin{macrocode}
    /getY { D exch get 1 get } def
    /tmin 1 def /tmax 0 def
    0 getY dup
    boundMin lt /plower exch def
    boundMax gt /phigher exch def
    plower phigher or not {
%    \end{macrocode}
% inside the fat line
%    \begin{macrocode}
	tmin 0 getX gt { /tmin 0 getX def } if
	tmax 0 getX lt { /tmax 0 getX def } if	
    } if
    1 1 D length 1 sub {
	/i exch def
	/clower i getY boundMin lt def
	/chigher i getY boundMax gt def
	clower chigher or not {
%    \end{macrocode}
% inside the fat line
%    \begin{macrocode}
	    tmin i getX gt { /tmin i getX def } if
	    tmax i getX lt { /tmax i getX def } if
	} if
	clower plower eq not {
%    \end{macrocode}
% cross the lower bound
%    \begin{macrocode}
	    boundMin i 1 sub i D Intersect % t on stack
	    dup tmin lt { dup /tmin exch def } if
	    dup tmax gt { dup /tmax exch def } if
	    pop 
	    /plower clower def
	} if
	chigher phigher eq not {
%    \end{macrocode}
% cross the upper bound
%    \begin{macrocode}
	    boundMax i 1 sub i D Intersect
	    dup tmin lt { dup /tmin exch def } if
	    dup tmax gt { dup /tmax exch def } if
	    pop 
	    /phigher chigher def
	} if
    } for
%    \end{macrocode}
% we have to test the closing segment for intersection
%    \begin{macrocode}
    /i D length 1 sub def
    /clower 0 getY boundMin lt def
    /chigher 0 getY boundMax gt def
    clower plower eq not {
%    \end{macrocode}
% cross the lower bound
%    \begin{macrocode}
	boundMin i 0 D Intersect
	dup tmin lt { dup /tmin exch def } if
	dup tmax gt { dup /tmax exch def } if
	pop
    } if
    chigher phigher eq not {
%    \end{macrocode}
% cross the lower bound
%    \begin{macrocode}
	boundMax i 0 D Intersect
	dup tmin lt { dup /tmin exch def } if
	dup tmax gt { dup /tmax exch def } if
	pop
    } if
    [tmin tmax]
    end
    DebugEnd
} bind def
%    \end{macrocode}
% \end{macro}
% 
% \begin{macro}{Intersect}
%   Get the x component of the intersection point between the line
%   passing through $i$-th and $j$-th points of \PSarray{Curve} and the
%   horizonal line through \PSvar{y}.
% \begin{pssyntax}
% \PSvar{y i j} \PSarray{Curve} \PSop{Intersect} \PSvar{Xisect}
% \end{pssyntax}
%    \begin{macrocode}
/Intersect {
    dup 4 -1 roll get aload pop
    4 2 roll exch get aload pop
%    \end{macrocode}
% On the stack: \PSvar{y Xi Yi Xj Yj}, Compute (Xj - Xi) * (y - Yi)/(Yj - Yi) + Xi
%
% We are sure, that Yi != Yj, because this procedure is called only
% when the lower or upper bound is crossed.
%    \begin{macrocode}
    4 2 roll 2 copy 6 2 roll VecSub
    5 2 roll
    neg 3 -1 roll add
    3 -1 roll div
    3 -1 roll mul add
} bind def
%    \end{macrocode}
% \end{macro}
%
% \begin{macro}{IsPath}
%   Check if an array is a path. A path is represented as array, which
%   contains other arrays which represent native Postscript
%   operations. Those can be \PSarray{X Y /@m}, \PSarray{X Y /@l}, or
%   \PSarray{X1 Y1 X2 Y2 X3 Y3 /@c}.
%
%   \begin{pssyntax}
%     \PSarray{array} \PSop{IsPath} \PSvar{boolean}
%   \end{pssyntax}
%    \begin{macrocode}
/IsPath {
  dup length 1 sub get type /nametype eq { true } { false } ifelse
} bind def
%    \end{macrocode}
% \end{macro}
%
% \begin{macro}{ShowFullPath}
%    \begin{macrocode}
/ShowFullPath {
  4 dict begin
  /movetype /moveto load def
  /linetype /lineto load def
  /curvetype /curveto load def
  mark exch aload pop
  {
    counttomark 0 eq { exit } if
    load exec
  } loop
  pop 
  end
} bind def
/ShowPathPortion {
  6 dict begin
  exch dup 0 lt { pop 0 } if /tstart exch def
  % if tstart is < 0, use the array length
  dup 0 lt { pop dup length } if /tstop exch def
  tstop tstart gt {
    /movetype /moveto load def
    /linetype /lineto load def
    /curvetype /curveto load def
    /n 0 def
    mark exch aload pop
    {
      counttomark 0 eq { pop exit } if
      dup /movetype eq not { /n n 1 add def } if
      % current section is after tstart
      n tstop sub 1 ge { cleartomark exit } if

      dup /movetype eq {
        load exec
      } {
        tstart n gt {
          % current path section is before tstart
          /curvetype eq { 6 2 roll 4 { pop } repeat } if
          movetype
        } {
          tstart n 1 sub gt tstop n lt or {
            % draw a truncated section
            tstart n sub 1 add tstop n sub 1 add
            ToUnitInterval exch
            /linetype eq {
              3 1 roll ToVec currentpoint ToVec exch ToVec
              dup 3 -1 roll Portion
              aload pop exch 
              tstart n 1 sub gt { aload pop moveto } { pop } ifelse
              aload pop lineto
            } {
              7 1 roll [ currentpoint 9 3 roll ] ArrayToPointArray
              dup 3 -1 roll Portion 
              { aload pop } forall
              8 -2 roll
              tstart n 1 sub gt { moveto } { pop pop } ifelse
              curveto
            } ifelse
          }{
            load exec
          } ifelse
        } ifelse
      } ifelse
    } loop
  } if
  end
} bind def
%    \end{macrocode}
% \end{macro}
%
%    \begin{macrocode}
 % Graham Scal algorithm to compute the convex hull of a set of
 % points. Code written by Bill Casselman,
 %  http://www.math.ubc.ca/~cass/graphics/text/www/
 %
 % [[X1 Y1] [X2 Y2] ... [Xn Yn]] hull -> [[...] ... [...]]
 %
/hulldict 32 dict def
hulldict begin

 % u - v 
/vsub { 2 dict begin
/v exch def
/u exch def
[ 
  u 0 get v 0 get sub
  u 1 get v 1 get sub
]
end } def

 % u - v rotated 90 degrees
/vperp { 2 dict begin
/v exch def
/u exch def
[ 
  v 1 get u 1 get sub
  u 0 get v 0 get sub
]
end } def

/dot { 2 dict begin
/v exch def
/u exch def
  v 0 get u 0 get mul
  v 1 get u 1 get mul
  add
end } def 

 % P Q
 % tests whether P < Q in lexicographic order
 % i.e xP < xQ, or yP < yQ if xP = yP
/comp { 2 dict begin
/Q exch def
/P exch def
P 0 get Q 0 get lt 
  P 0 get Q 0 get eq
  P 1 get Q 1 get lt 
  and 
or 
end } def

end

 % args: an array of points C
 % effect: returns the array of points on the boundary of
 %     the convex hull of C, in clockwise order 

/ConvexHull {
(ConvexHull) DebugBegin
hulldict begin
/C exch def
/comp C quicksort
/n C length def
 % Q might circle around to the start
/Q n 1 add array def
Q 0 C 0 get put
Q 1 C 1 get put
/i 2 def
/k 2 def
 % i is next point in C to be looked at
 % k is next point in Q to be added
 % [ Q[0] Q[1] ... ]
 % scan the points to make the top hull
n 2 sub {
  % P is the current point at right
  /P C i get def
  /i i 1 add def
  {
    % if k = 1 then just add P 
    k 2 lt { exit } if
    % now k is 2 or more
    % look at Q[k-2] Q[k-1] P: a left turn (or in a line)?
    % yes if (P - Q[k-1])*(Q[k-1] - Q[k-2])^perp >= 0
    P Q k 1 sub get vsub 
    Q k 1 sub get Q k 2 sub get vperp 
    dot 0 lt {
      % not a left turn
      exit
    } if
    /k k 1 sub def
  } loop
  Q k P put
  /k k 1 add def
} repeat

 % done with top half
 % K is where the right hand point is
/K k 1 sub def

/i n 2 sub def
Q k C i get put
/i i 1 sub def
/k k 1 add def
n 2 sub {
  % P is the current point at right
  /P C i get def
  /i i 1 sub def
  {
    % in this pass k is always 2 or more
    k K 2 add lt { exit } if
    % look at Q[k-2] Q[k-1] P: a left turn (or in a line)?
    % yes if (P - Q[k-1])*(Q[k-1] - Q[k-2])^perp >= 0
    P Q k 1 sub get vsub 
    Q k 1 sub get Q k 2 sub get vperp 
    dot 0 lt {
      % not a left turn
      exit
    } if
    /k k 1 sub def
  } loop
  Q k P put
  /k k 1 add def
} repeat

 % strip Q down to [ Q[0] Q[1] ... Q[k-2] ]
 % excluding the doubled initial point
[ 0 1 k 2 sub {
  Q exch get
} for ] 
end
DebugEnd
} def

/qsortdict 8 dict def

qsortdict begin

 % args: /comp a L R x
 % effect: effects a partition into two pieces [L j] [i R]
 %     leaves i j on stack

/partition { 8 dict begin
/x exch def
/j exch def
/i exch def
/a exch def
dup type /nametype eq { load } if /comp exch def
{
  {
    a i get x comp exec not {
      exit
    } if
    /i i 1 add def
  } loop
  {
    x a j get comp exec not {
      exit
    } if
    /j j 1 sub def
  } loop
  
  i j le {
    % swap a[i] a[j]
    a j a i get
    a i a j get 
    put put
    /i i 1 add def
    /j j 1 sub def
  } if
  i j gt {
    exit
  } if
} loop
i j
end } def

 % args: /comp a L R
 % effect: sorts a[L .. R] according to comp
/subsort {
 % /c a L R
[ 3 1 roll ] 3 copy
 % /c a [L R] /c a [L R]
aload aload pop 
 % /c a [L R] /c a L R L R
add 2 idiv
 % /c a [L R] /c a L R (L+R)/2
3 index exch get
 % /c a [L R] /c a L R x
partition
 % /c a [L R] i j
 % if j > L subsort(a, L, j)
dup 
 % /c a [L R] i j j
3 index 0 get gt {
  % /c a [L R] i j
  5 copy 
  % /c a [L R] i j /c a [L R] i j
  exch pop
  % /c a [L R] i j /c a [L R] j
  exch 0 get exch
  % ... /c a L j 
  subsort
} if
 % /c a [L R] i j
pop dup
 % /c a [L R] i i
 % if i < R subsort(a, i, R)
2 index 1 get lt {
  % /c a [L R] i
  exch 1 get 
  % /c a i R
  subsort
}{
  4 { pop } repeat
} ifelse
} def

end % qsortdict

 % args: /comp a
 % effect: sorts the array a 
 % comp returns truth of x < y for entries in a

/quicksort { qsortdict begin
dup length 1 gt {
 % /comp a
dup 
 % /comp a a 
length 1 sub 
 % /comp a n-1
0 exch subsort
} {
pop pop
} ifelse
end } def
%    \end{macrocode}
% 
% Debugging stuff
%    \begin{macrocode}
/debug {
    dup 1 add copy {==} repeat pop
} bind def
/DebugIT false def
/CheckIT false def
/DebugDepth 0 def
/DebugBegin {
  DebugIT {
    /DebugProcName exch def
    DebugDepth 2 mul string
    0 1 DebugDepth 2 mul 1 sub {
      dup 2 mod 0 eq { (|) }{( )} ifelse
      3 -1 roll dup 4 2 roll
      putinterval
    } for
    DebugProcName strcat ==
    /DebugDepth DebugDepth 1 add def
  }{
    pop
  } ifelse
} bind def
/DebugEnd {
  DebugIT {
    /DebugDepth DebugDepth 1 sub def
    DebugDepth 2 mul 2 add string
    0 1 DebugDepth 2 mul 1 sub {
      dup 2 mod 0 eq { (|) }{ ( ) } ifelse
      3 -1 roll dup 4 2 roll
      putinterval
    } for
    dup DebugDepth 2 mul (+-) putinterval
    ( done) strcat ==
  } if
} bind def
/strcat {
    exch 2 copy
    length exch length add
    string dup dup 5 2 roll
    copy length exch
    putinterval
} bind def
/nametostr {
    dup length string cvs
} bind def
/ShowCurve {
    { aload pop } forall
    8 -2 roll moveto curveto
} bind def
/CurveToString {
    (CurveToString) DebugBegin
    aload pop ([) 3 -1 roll 20 string cvs strcat (, ) strcat exch 20 string cvs strcat (]) strcat
    DebugEnd
} bind def
end % tx@IntersectDict
%    \end{macrocode}
%</prolog> 
% \Finale
% \endinput

\IfFileExists{pst-intersect.pro}{%
    \ProvidesFile{pst-intersect.pro}
      [2014/02/06 PostScript prologue file]
      \@addtofilelist{pst-intersect.pro}}{}%
%    \end{macrocode}
%</stylefile>
%
% \chapter{The \TeX\ implementation}
%
%<*texfile>
%    \begin{macrocode}
\csname PSTintersectLoaded\endcsname
\let\PSTintersectLoaded\endinput

\ifx\PSTricksLoaded\endinput\else\input pstricks.tex \fi
\ifx\PSTXKeyLoaded\endinput\else \input pst-xkey.tex \fi
\ifx\PSTnodesLoaded\endinput\else\input pst-node.tex \fi
\ifx\PSTfuncLoaded\endinput\else \input pst-func.tex \fi

\edef\PstAtCode{\the\catcode`\@} \catcode`\@=11\relax

\pst@addfams{intersect}
\pstheader{pst-intersect.pro}

\def\pst@intersectdict{tx@IntersectDict begin }
\def\PIT@dict#1{\pst@intersectdict #1 end}
\def\PIT@Verb#1{\pst@Verb{\PIT@dict{#1} }}%

\def\savebezier{\pst@object{savebezier}}
\def\savebezier@i#1{%
  \begin@OpenObj
    \PIT@checkname{#1}%
    \addto@pscode{ /\PIT@name{#1} }%
    \pst@getcoors[\savebezier@ii%]
}%
\def\savebezier@ii{%
  \addto@pscode{%
    % only 10 points allowed, remove the rest
    counttomark 20 gt { counttomark 20 sub { pop } repeat } if
    % reverse the point order
    counttomark -2 4 { 2 roll } for
    %\addto@pscode{%
    %counttomark 2 add -1 roll pop % remove the path name
    counttomark 2 idiv 1 sub dup 9 gt { pop 9 } if
    \psk@plotpoints\space exch
    \txFunc@BezierCurve
    \ifshowpoints \txFunc@BezierShowPoints \else pop \fi
    tx@FuncDict begin Points aload pop end
  }%
  \let\use@pscode\PIT@use@pscode
  \end@OpenObj
  \PIT@Verb{%
    [ count 1 sub 1 roll ] ArrayToPointArray def 
  }%
\ignorespaces}%
\define@key[psset]{intersect}{tstart}{%
  \pst@checknum{#1}\PIT@key@tstart
}
\define@key[psset]{intersect}{tstop}{%
  \pst@checknum{#1}\PIT@key@tstop
}
\define@key[psset]{intersect}{istart}{%
  \pst@checknum{#1}\PIT@key@istart
}
\define@key[psset]{intersect}{istop}{%
  \pst@checknum{#1}\PIT@key@istop
}
\define@key[psset]{intersect}{name}{%
  \def\PIT@key@name{#1}%
}%
\newif\PIT@saveintersections
\define@boolkey[psset]{intersect}[PIT@]{saveintersections}[true]{}
\psset[intersect]{%
  tstart=-1,
  tstop=-1,
  istart=-1,
  istop=-1,
  name={},
  saveintersections
}%
\def\PIT@use@pscode{%
  \pstverb{%
    \pst@dict
    \tx@STP
    \pst@newpath
    \psk@origin
    \psk@swapaxes
    \pst@code
    end
    count /ocount exch def
  }%
  \gdef\pst@code{}%
}%
\let\PIT@pst@stroke@orig\pst@stroke
\def\PIT@save@path{%
  \PIT@pst@stroke@orig
  \addto@pscode{%
    clear mark
    { /movetype counttomark 3 roll }
    { /linetype counttomark 3 roll }
    { /curvetype counttomark 7 roll }{} pathforall 
    counttomark 1 add -1 roll pop count }%
}%
\def\PIT@name@default{@tmp}%
\def\PIT@name#1{PIT@#1}%
\def\PIT@checkname#1{%
  \ifx\@empty#1\@empty
    \@pstrickserr{Unexpected empty argument!}\@ehpb
  \fi
}%
\def\savepath{\pst@object{savepath}}%
\long\def\savepath@i#1#2{%
  \begin@SpecialObj
    \PIT@checkname{#1}%
    \let\pst@stroke\PIT@save@path
    \let\use@pscode\PIT@use@pscode
    \pscustom{#2}%
    \PIT@Verb{%
      /\PIT@name{#1}
      [ 3 -1 roll 2 add 2 roll ] def }%
  \end@SpecialObj
}%
\def\psshowcurve{\pst@object{psshowcurve}}%
\def\psshowcurve@i#1{%
  \addbefore@par{plotpoints=200}%
   \@ifnextchar\bgroup
     {\PIT@traceintersection{#1}}%
     {\PIT@tracecurve{#1}}%
}%
\def\PIT@tracecurve#1{%
  \PIT@checkname{#1}%
  \begin@OpenObj
    \addto@pscode{%
      \pst@intersectdict
        \PIT@name{#1} dup IsPath {
          \PIT@key@tstart\space\PIT@key@tstop\space
          ShowPathPortion
        }{
          [exch dup
          \PIT@key@tstart\space\PIT@key@tstop\space 
          dup 0 lt { pop 1 } if
          ToUnitInterval Portion 
          { aload pop } forall
          counttomark 2 sub 2 idiv
          \psk@plotpoints
          exch
          \txFunc@BezierCurve
          \ifshowpoints \txFunc@BezierShowPoints \else pop \fi
        } ifelse
      end
    }%
  \end@OpenObj
}%
\def\PIT@traceintersection#1#2{%
  \PIT@checkname{#2}%
  \begin@OpenObj
    \addto@pscode{%
      \pst@intersectdict
        \ifx\\#1\\%
          /\PIT@name{\PIT@name@default}
        \else
          /\PIT@name{#1}
        \fi 
        dup currentdict exch known not {
          \ifx\\#1\\%
            (You haven't defined an intersection!) ==
          \else
            (You haven't defined the intersection '#1') ==
          \fi
        } if 
        load
        dup dup type /dicttype eq exch /\PIT@name{#2} known and not {
          (You haven't defined the intersection '#2') ==
        } if
        dup /\PIT@name{#2} get
        exch /\PIT@name{#2}@t get
        dup length \PIT@key@istart\space ge 0 \PIT@key@istart\space lt and {
          dup \PIT@key@istart\space cvi 1 sub get
        } {
          \PIT@key@tstart
        } ifelse
        exch % [curve] t_istart|tstart [ts]
        %
        dup length \PIT@key@istop\space ge 0 \PIT@key@istop\space lt and {
          \PIT@key@istop\space cvi 1 sub get
        } {
          pop \PIT@key@tstop
        } ifelse
        2 copy gt { exch } if 
        3 -1 roll dup
        IsPath {
          3 1 roll
          ShowPathPortion
        }{
          [ exch dup 5 -2 roll
          dup 0 lt { pop 1 } if
          ToUnitInterval Portion 
          { aload pop } forall
          counttomark 2 sub 2 idiv
          \psk@plotpoints
          exch
          \txFunc@BezierCurve
          \ifshowpoints \txFunc@BezierShowPoints \else pop \fi
        } ifelse
      end
    }%
  \end@OpenObj
}%
%
% \begin{macro}{\psintersect}
%    \begin{macrocode}
\def\psintersect{\pst@object{psintersect}}
\def\psintersect@i#1#2{%
  \PIT@checkname{#1}%
  \PIT@checkname{#2}%
  \begin@SpecialObj
  \def\PIT@@name{%
    \ifx\PIT@key@name\@empty
      \PIT@name{\PIT@name@default}
    \else
      \PIT@name{\PIT@key@name} 
    \fi}%
  \PIT@Verb{%
    currentdict /\PIT@name{#1} known not {
      (You haven't defined the curve or path '#1') ==
    } if
    currentdict /\PIT@name{#2} known not {
      (You haven't defined the curve or path '#2') ==
    } if
    \PIT@name{#1} \PIT@name{#2}
    \PIT@name{#1} IsPath {
      \PIT@name{#2} IsPath {
        IntersectPaths
      }{
        IntersectPathCurve
      } ifelse
    }{
      \PIT@name{#2} IsPath {
        IntersectCurvePath
      }{
        IntersectBeziers
        4 copy LoadIntersectionPoints 5 1 roll
      } ifelse
    } ifelse
    /\PIT@@name\space /\PIT@name{#1} /\PIT@name{#2} 8 3 roll 
    SaveIntersection
  }%
  \ifPIT@saveintersections
    \pst@Verb{%
      \pst@intersectdict 
        \PIT@@name\space /Points get 
        ArrayToPointArray
      end
      tx@NodeDict begin 
        dup length 1 1 3 -1 roll {
          2 copy 1 sub get cvx
          false 3 -1 roll (N@\PIT@key@name) exch 20 string cvs 
          \pst@intersectdict strcat end cvn
          10 {InitPnode} /NodeScale {} def NewNode
        } for
      end
      pop
    }%
  \fi
  \ifshowpoints
    \addto@pscode{%
      \pst@intersectdict
        [ \PIT@@name\space /Points get aload pop 
      end 
    }%
    \psdots@ii
  \else
    \end@SpecialObj
  \fi
}%
%    \end{macrocode}
% \end{macro}
% 
%    \begin{macrocode}
\catcode`\@=\PstAtCode\relax
%    \end{macrocode}
%</texfile> 
%
% \chapter{The Postscript header file}
% \makeatletter
%^^A Copied this definition from doc.sty and changed it not to add a
%^^A backslash to the Postscript procedure name in the index.
% \def\SpecialIndex@#1#2{%
%    \@SpecialIndexHelper@#1\@nil
%    \def\@tempb{ }%
%    \ifcat \@tempb\@gtempa
%       \special@index{\quotechar\space\actualchar
%                      \string\verb\quotechar*\verbatimchar
%                      \quotechar\space\verbatimchar#2}%
%    \else
%      \def\@tempb##1##2\relax{\ifx\relax##2\relax
%           \def\@tempc{\special@index{\quotechar##1\actualchar
%                       \string\verb\quotechar*\verbatimchar
%                       \quotechar##1\verbatimchar#2}}%
%         \else
%           \def\@tempc{\special@index{##1##2\actualchar
%                        \string\verb\quotechar*\verbatimchar##1##2\verbatimchar#2}}%
%         \fi}%
%      \expandafter\@tempb\@gtempa\relax
%      \@tempc
%    \fi}
% \makeatother
%
%<*prolog>
%    \begin{macrocode}
/tx@IntersectDict 200 dict def
tx@IntersectDict begin
%    \end{macrocode}
% These are some helper procedures for vector operations.
%
% \begin{macro}{VecAdd}
% Addition of two vectors.
% \begin{pssyntax}
%   \PSvar{Xa Ya Xb Yb} \PSop{VecAdd} \PSvar{Xa+Xb Ya+Yb}
% \end{pssyntax}
%    \begin{macrocode}
/VecAdd {
    3 -1 roll add 3 1 roll add exch
} bind def
%    \end{macrocode}
% \end{macro}
%
% \begin{macro}{VecSub}
% Subtraction of two vectors.
% \begin{pssyntax}
%   \PSvar{Xa Ya Xb Yb} \PSop{VecSub} \PSvar{Xa-Xb Ya-Yb}
% \end{pssyntax}
%    \begin{macrocode}
/VecSub {
    neg 3 -1 roll add 3 1 roll neg add exch
} bind def
%    \end{macrocode}
% \end{macro}
% 
% \begin{macro}{VecScale}
% Scale a vector by a factor \PSvar{fac}.
% \begin{pssyntax}
%   \PSvar{Xa Ya fac} \PSop{VecScale} \PSvar{fac}$\cdot$\PSvar{Xa} \PSvar{fac}$\cdot$\PSvar{Ya}
% \end{pssyntax}
%    \begin{macrocode}
/VecScale {
  dup 4 -1 roll mul 3 1 roll mul
} bind def
%    \end{macrocode}
% \end{macro}
%
% \begin{macro}{ToVec}
%   Convert two numbers to a procedure holding the two values. This
%   representation is used to save coordinate values of nodes and vectors.
%   \begin{pssyntax}
%     \PSvar{X Y} \PSop{ToVec} \PSarray{X Y}
%   \end{pssyntax}
%    \begin{macrocode}
/ToVec {
    [ 3 1 roll ]
} bind def
%    \end{macrocode}
% \end{macro}
%
% \PSvar{MaxPrecision} gives the precision of the curve parameter t for the
% intersection. This shouldn't be lower than $10^{-6}$, because
% PostScript uses single precision.
%    \begin{macrocode}
/MaxPrecision 1e-6 def
%    \end{macrocode}
%
% \PSvar{Epsilon} gives the allowed relative error of the intersection point. 
%    \begin{macrocode}
/Epsilon 1e-4 def
%    \end{macrocode}
% 
% The threshold for curve subdivision, see below.
%    \begin{macrocode}
/MinClippedSizeThreshold 0.8 def
%    \end{macrocode}
%
% The predefined intervals for the subdivision of the curves.
%    \begin{macrocode}
/H1Interval [0 0.5] def
/H2Interval [0.5 MaxPrecision add 1] def
%    \end{macrocode}
% 
% \begin{macro}{IntersectBeziers}
%   The main procedure, which computes the intersection of two bezier
%   curves of arbitrary order.  This, and most of the following
%   procedures operate on curves, which are stored as arrays of points,
%   the points are also arrays with two elements -- \PSvar{X} and
%   \PSvar{Y}. A Bezier curve of $n$-th order is then givesn by
%   \PSarray{\PSarray{X0 Y0} \PSarray{X1 Y1} \ldots \PSarray{XN YN}}.
%
% \begin{pssyntax}
%   \PSarray{curveA} \PSarray{curveB} \PSop{IntersectBeziers} 
%   \PSarray{curveA} \PSarray{tA} \PSarray{curveB} \PSarray{tB}
% \end{pssyntax}
%    \begin{macrocode}
/IntersectBeziers {
  2 copy length 2 eq exch length 2 eq and {
    IntersectLines
  }{ 
    2 copy [0 1] [0 1] IterateIntersection
  } ifelse
  3 -1 roll exch
} bind def
%    \end{macrocode}
% \end{macro}
%
% \begin{macro}{IntersectLines}
% 
%   \begin{pssyntax}
%     \PSarray{lineA} \PSarray{lineB} \PSop{IntersectLines}
%     \PSarray{lineA} \PSarray{tA} \PSarray{lineB} \PSarray{tB}
%   \end{pssyntax}
%    \begin{macrocode}
/IntersectLines {
  (IntersectLines) DebugBegin
  2 copy
  exch { aload pop } forall 5 -1 roll { aload pop } forall
  8 -2 roll 2 copy 10 4 roll 4 2 roll 2 copy 6 2 roll 10 2 roll
  VecSub
  6 2 roll 4 2 roll VecSub
  8 4 roll 4 2 roll VecSub % X3-X4 Y3-Y4 X2-X1 Y2-Y1 X3-X1 Y3-Y1 % b1 b2 a1 a2 c1 c2
  6 copy 12 -4 roll 
  neg 4 -1 roll mul 3 1 roll mul add
  dup 0 eq {
    % no intersections
    9 { pop } repeat [] []
  } {
    dup 10 1 roll 5 1 roll
     4 -1 roll mul 3 1 roll mul sub exch div
     6 1 roll 4 -1 roll mul 3 1 roll mul sub exch div
     2 copy 2 copy 0 ge exch 0 ge and 3 1 roll 1 le exch 1 le and and {
       [ exch ] exch [ exch ]
     } {
       pop pop [] []
     } ifelse
  } ifelse
  DebugEnd
} bind def
%    \end{macrocode}
% \end{macro}
% \begin{macro}{IntersectPaths}
%   \begin{pssyntax}
%     \PSarray{pathA} \PSarray{pathB} \PSop{IntersectPaths}
%     \PSarray{intersections} \PSarray{pathA} \PSarray{tA} \PSarray{pathB} \PSarray{tB}
%   \end{pssyntax}
%    \begin{macrocode}
/IntersectPaths {
  (IntersectPaths) DebugBegin
  6 dict begin 
    2 copy exch PreparePath dup length /nA exch def 
    exch PreparePath dup length /nB exch def
    /isect [] def
    /tA [] def /tB [] def
    { % [pathA] [Bi]
      /nB nB 1 sub def
      exch dup 3 1 roll % [pathA] [Bi] [pathA]
      {
        /nA nA 1 sub def
        exch dup 3 1 roll % [pathA] [Bi] [Aj] [Bi]
        IntersectBeziers % [curveA] [tA] [curveB] [tB]
        4 copy LoadIntersectionPoints
        [ exch isect aload pop ] /isect exch def
        exch pop 3 -1 roll pop
        [ tB aload length 2 add -1 roll TArray { nB add } forall ] /tB exch def
        [ tA aload length 2 add -1 roll TArray { nA add } forall ] /tA exch def
      } forall
      pop % remove [Bi]
      dup length /nA exch def
    } forall
    pop % remove [pathA]
    [ isect { aload pop } forall ] 3 1 roll tA exch tB
    % [intersections] [pathA] [tA] [pathB] [tB]
  end
  DebugEnd
} bind def
%    \end{macrocode}
% \end{macro}
%
% \begin{macro}{IntersectCurvePath}
%   \begin{pssyntax}
%     \PSarray{curveA} \PSarray{pathB} \PSop{IntersectCurvePath}
%     \PSarray{intersections} \PSarray{curveA} \PSarray{tA} \PSarray{pathB} \PSarray{tB}
%   \end{pssyntax}
%    \begin{macrocode}
/IntersectCurvePath {
  (IntersectCurvePath) DebugBegin
  6 dict begin 
    2 copy PreparePath dup length /n exch def
    /isect [] def
    /tA [] def /tB [] def
    { % [curveA] [Bi]
      /n n 1 sub def
      exch dup 3 -1 roll % [curveA] [curveA] [Bi] 
      IntersectBeziers 
      4 copy LoadIntersectionPoints % [curveA] [tA] [curveB] [tB]
      [ exch isect aload pop ] /isect exch def
      pop 3 -1 roll pop
      [ tB aload length 2 add -1 roll TArray { n add } forall ] /tB exch def
      [ tA aload length 2 add -1 roll TArray aload pop ] /tA exch def
    } forall
    pop % remove [curveA]
    [ isect { aload pop } forall ] 3 1 roll tA exch tB
    % [intersections] [curveA] [tA] [pathB] [tB]
  end
  DebugEnd
} bind def
/IntersectPathCurve {
  exch IntersectCurvePath 4 2 roll
} bind def
%    \end{macrocode}
% \end{macro}
% 
% \begin{macro}{SaveIntersection}
%   \begin{pssyntax}
%     \PSname{isectname} \PSname{nameA} \PSname{nameB} 
%     \PSarray{intersectionpoints} \PSarray{A} \PSarray{tA} \PSarray{B} \PSarray{tB} 
%     \PSop{SaveIntersection}
%   \end{pssyntax}
%    \begin{macrocode}
/SaveIntersection {
  (SaveIntersection) DebugBegin
  4 dict dup 10 -1 roll exch def
  begin %
    /Points 6 -1 roll def
    5 -1 roll dup 4 -1 roll def % /curveA [curveA] [tA] [tB] /curveB /curveB [curveB] def
    nametostr (@t) strcat cvn exch TArray def % /curveA [curveA] [tA] /curveB@t [tB] def
    3 -1 roll dup 4 -1 roll def
    nametostr (@t) strcat cvn exch TArray def
  end
  DebugEnd
} bind def
%    \end{macrocode}
% \end{macro}
% 
% \begin{macro}{TArray}
%   The curve parameters \PSvar{t} as determined by
%   \PSvar{IntersectBeziers} are given in a special array
%   construct. \PSvar{TArray} creates a simple array with the
%   \PSvar{t}-values given in ascending order.
%
% \begin{pssyntax}
% \PSarray{\PSarray{t0a t0b} \ldots \PSvar{null}\ldots \PSvar{integer}}
% \PSop{TArray} \PSarray{t0 t1 \ldots tN}
% \end{pssyntax}
%    \begin{macrocode}
/TArray {
  (TArray) DebugBegin
  dup length 0 gt {
    dup 0 get type /arraytype eq {
      [ exch
      { %dup type /nulltype eq { pop exit } if
  	aload pop add 0.5 mul
      } forall ]
    } if
    dup /lt exch quicksort
  } if
  DebugEnd %1 debug
} bind def
%    \end{macrocode}
% \end{macro} 
% 
% We can save arbitrary paths using \PSvar{pathforall}. The saved path
% contains the commands \PSname{movetype}, \PSname{linetype} and
% \PSname{curvetype}. By default, these are defined as the respective
% original procedures.
%    \begin{macrocode}
/movetype { /moveto load } bind def
/linetype { /lineto load } bind def
/curvetype { /curveto load } bind def
%    \end{macrocode}
%
% [ ... /movetype ... /linetype .../curvetype ]
%    \begin{macrocode}
/PreparePath {
    [ exch aload pop
    {
	dup type /nametype eq not { exit } if
	dup /movetype eq {
	    pop ToVec /@mycp exch def
	} {
	    dup /linetype eq {
		pop [ @mycp 4 2 roll 2 copy ToVec /@mycp exch def ToVec ]
	    } {
		pop [ @mycp 8 2 roll 2 copy ToVec /@mycp exch def
		ToVec 5 1 roll ToVec 4 1 roll ToVec 3 1 roll ]
	    } ifelse
	    counttomark 1 roll	
	} ifelse
    } loop ]
} bind def
%    \end{macrocode}
%
% \begin{macro}{LoadLineIntersectionPoints}
% Prepare \PSarray{Curve} for use with tx@Func
% \begin{pssyntax}
% \PSarray{curve} \PSarray{t} \PSop{LoadLineIntersectionPoints}
% \PSarray{I0.x I0.y \ldots IN.x YN.x}
% \end{pssyntax}
%    \begin{macrocode}
/LoadLineIntersectionPoints {
  (LoadLineIntersectionPoints) DebugBegin
  exch [ exch { aload pop } forall ]
  tx@Dict begin tx@FuncDict begin 2 dict begin
    dup length 2 idiv 1 sub /BezierType exch def /Points exch def
    [ exch { GetBezierCoor } forall ]
  end end end
  DebugEnd
} bind def
%    \end{macrocode}
% \end{macro}
%
% \begin{macro}{LoadCurveIntersectionPoints}
%   Load the intersection points. This loads the same intersection point
%   from both curves, and chooses the one with the lowest error.
% \begin{pssyntax}
% \PSarray{curveA} \PSarray{tA} \PSArray{curveB} \PSarray{tB} 
%  \PSop{LoadCurveIntersectionPoints}
% \PSarray{I0.x I0.y \ldots IN.x YN.x}
% \end{pssyntax}
%    \begin{macrocode}
/LoadCurveIntersectionPoints {
  (LoadCurveIntersectionPoints) DebugBegin
  2 {
    4 2 roll
    [ exch { aload pop } forall ]
    exch [ exch { aload pop } forall ]
  } repeat
  % [A0.x A0.y ... AM.x AM.y] [tA0a tA0b ... tAMa tAMb] [tB0a tB0b ... tBNa tBNb] [B0.x B0.y ... BN.x BN.y]
  tx@Dict begin tx@FuncDict begin 2 dict begin
    dup length 2 idiv 1 sub /BezierType exch def /Points exch def
      [ exch { GetBezierCoor } forall ]
    3 1 roll 
    dup length 2 idiv 1 sub /BezierType exch def /Points exch def
      [ exch { GetBezierCoor } forall ]
    end
    %2 debug
    % [IB0.xa IB0.ya IB0.xb IB0.yb ... IBM.yb] [IA0.xa IA0.ya IA0.xb IA0.yb ... IAM.yb]
    2 {
      [ exch aload length 4 idiv {
        [ 5 1 roll ] counttomark 1 roll
      } repeat ]
      exch 
    } repeat
    % [[IB0.xa ...] ... [... IBM.yb]] [[IA0.xa IA0.ya IA0.xb IA0.yb] ...[IAM.xa ... IAM.yb]]
    2 {
      dup hulldict /comp get exch quicksort exch
    } repeat
    2 dict begin
      /B exch def /A exch def
      [ 0 1 A length 1 sub {
        dup A exch get exch B exch get % [IAi] [IBi]
        2 copy aload pop VecSub Pyth exch 
        aload pop VecSub Pyth lt { exch } if pop
        aload pop VecAdd 0.5 VecScale
      } for 
      % merge near intersection points
      %counttomark 2 idiv 1 1 3 -1 roll {
      %  pop
      %  counttomark 4 lt { exit } if
      %  4 copy 4 2 roll ToVec 3 1 roll ToVec AreNear {

      %  }
      %} for
      ]
    end
  end end
  DebugEnd
} bind def
%    \end{macrocode}
% \end{macro}
%
% \begin{macro}{LoadIntersectionPoints}
%    \begin{macrocode}
/LoadIntersectionPoints {
  (LoadIntersectionPoints) DebugBegin
  4 copy pop exch pop length 2 eq exch length 2 eq and {
    pop pop LoadLineIntersectionPoints
  }{
    LoadCurveIntersectionPoints
  } ifelse
  DebugEnd
} bind def
%    \end{macrocode}
% \end{macro}
% 
% \begin{macro}{IterateIntersection}
% Iteration procedure to compute all intersections of CurveA and CurveB.
% This contains the
% 
% \begin{pssyntax}
% \PSarray{CurveA} \PSarray{CurveB} \PSarray{intervalA} \PSarray{intervalB}
% \PSop{IterateIntersection} \PSarray{domsA} \PSarray{domsB}
% \end{pssyntax}
%    \begin{macrocode}
/IterateIntersection {
    (IterateIntersection) DebugBegin
    11 dict begin
	/precision MaxPrecision def
%    \end{macrocode}
% in order to limit recursion
%    \begin{macrocode}
        /counter 0 def
	/depth 0 def
	/domsA [] def
	/domsB [] def
	/domsA /domsB 6 2 roll _IterateIntersection
	domsB domsA
    end
    dup length 0 gt {
      TArraysRemoveDup
    } if
    DebugEnd
} bind def
/TArraysRemoveDup {
  4 dict begin
    /tB exch def
    /tA exch def
    /j 0 def
    [ tA 0 get tB 0 get
    1 1 tA length 1 sub {
      /i exch def
      tA j get aload pop tA i get aload pop tx@Dict begin Pyth2 end MaxPrecision gt
      tB j get aload pop tB i get aload pop tx@Dict begin Pyth2 end MaxPrecision gt and {
        % keep the current parameter point
        /j i def
        tB i get tA i get
        counttomark 2 idiv 1 add 1 roll 
      } if
    } for
    counttomark 2 idiv 1 add [ exch 1 roll ] % [ ... [tB]
    counttomark 1 add 1 roll ] exch % [tA] [tB]
  end 
} bind def
%    \end{macrocode}
% \end{macro}
% 
% \begin{macro}{_IterateIntersection}
% This is the iteration part which is called recursively.
%
% \begin{pssyntax}
%   \PSname{domsA} \PSname{domsB} \PSarray{CurveA} \PSarray{CurveB}
%   \PSarray{domA} \PSarray{domB} \PSop{_IterateIntersection}
% \end{pssyntax}
%    \begin{macrocode}
/_IterateIntersection {
    (_IterateIntersection) DebugBegin
    CloneVec /domB exch def
    CloneVec /domA exch def
    CloneCurve /CurveB exch def
    CloneCurve /CurveA exch def
    /iter 0 def
    /depth depth 1 add def
    /dom null def
    /counter counter 1 add def

    CheckIT {
	(>> curve subdivision performed: dom(A) = ) domA CurveToString strcat
	(, dom(B) = ) strcat domB CurveToString strcat ( <<) strcat ==
    } if
    CurveA IsConstant CurveB IsConstant and {
	CurveA MiddlePoint ToVec
	CurveB MiddlePoint ToVec AreNear {
	    domA domB 4 -1 roll exch PutInterval PutInterval
	} {
	    pop pop
	} ifelse
    }{
	counter 100 lt {
%    \end{macrocode}
% Use a loop to simulate some kind of return to exit at different positions.
%    \begin{macrocode}
	    {
		/iter iter 1 add def
		iter 100 lt
		domA Extent precision ge
		domB Extent precision ge or and not {
		    iter 100 ge {
			false 
		    } {
			CurveA MiddlePoint ToVec
			CurveB MiddlePoint ToVec AreNear {
			    domA domB true
			}{
			    false
			} ifelse
		    } ifelse
		    exit
		} if
%    \end{macrocode}
% iter < 100 && (dompA.extent() >= precision || dompB.extent() >= precision)
%    \begin{macrocode}
		CheckIT {
		    (counter: ) counter 20 string cvs strcat
		    (, iter: ) iter 20 string cvs strcat strcat
		    (, depth: ) depth 20 string cvs strcat strcat ==
		} if
	
		CurveA CurveB ClipCurve /dom exch def
	
		CheckIT {(dom : ) dom CurveToString strcat == } if		
		dom IsEmptyInterval {
		    CheckIT { (empty interval, exit) == } if
		    false exit
		} if
%    \end{macrocode}
% dom[0] > dom[1], invalid.
%    \begin{macrocode}
		dom aload pop 2 copy min 3 1 roll max gt {
		    CheckIT {
			(dom[0] > dom[1], invalid!) ==
		    } if
		    false exit
		} if

		domB dom MapTo /domB exch def
		CurveB dom Portion

		CurveB IsConstant CurveA IsConstant and {
		    CheckIT {
          		(both curves are constant: ) ==	
			(C1: [ ) CurveA { CurveToString ( ) strcat strcat } forall (]) strcat ==
			(C2: [ ) CurveB { CurveToString ( ) strcat strcat } forall (]) strcat ==
		    } if
		    CurveA MiddlePoint ToVec
		    CurveB MiddlePoint ToVec AreNear {
			domA domB true
		    } {
			false
		    } ifelse
		    exit
		} if
%    \end{macrocode}
% If we have clipped less than 20%, we need to subdivide the
% curve with the largest domain into two sub-curves.
%    \begin{macrocode} 
		dom Extent MinClippedSizeThreshold gt {
		    CheckIT {
			(clipped less than 20% : ) ==
			(angle(A) = ) CurveA dup length 1 sub get aload pop
				      CurveA 0 get aload pop VecSub
   				      exch 2 copy 0 eq exch 0 eq and {
					  pop pop (NaN)
				      } {
					  atan 20 string cvs
				      } ifelse strcat ==
		        (angle(B) = ) CurveB dup length 1 sub get aload pop
		                      CurveB 0 get aload pop VecSub
				      exch 2 copy 0 eq exch 0 eq and {
					  pop pop (NaN)
				      } {
					  atan 20 string cvs
				      } ifelse strcat ==
		        (dom : ) == dom == (domB :) == domB ==
		    } if
%    \end{macrocode}
% Leave those five values on the stack to revert to the current state after the recursive calls.
%    \begin{macrocode}
		    CurveA CurveB domA domB iter
     		    7 -2 roll 2 copy 9 2 roll 2 copy 
%    \end{macrocode}
% On the stack: /domsA /domsB CurveA CurveB domA domB iter /domsA /domsB /domsA /domsB
%    \begin{macrocode}
		    domA Extent domB Extent gt {
			CurveA CloneCurve dup H1Interval Portion % pC1
			CurveA CloneCurve dup H2Interval Portion % pC2
			domA H1Interval MapTo                    % dompC1
			domA H2Interval MapTo                    % dompC2
%    \end{macrocode}
% Need on the stack: /domsA /domsB pC2 CurveB dompC2 domB   /domsA /domsB pC1 CurveB dompC1 domB
%    \begin{macrocode}
			3 -1 roll exch % /domsA /domsB /domsA /domsB pC1 dompC1 pC2 dompC2
			CurveB exch domB 8 4 roll % /domsA /domsB pC2 CurveB dompC2 domB /domsA /domsB pC1 dompC1
			CurveB exch domB % /domsA /domsB pC2 CurveB dompC2 domB /domsA /domsB pC1 CurveB dompC1 domB
		    } {
			CurveB CloneCurve dup H1Interval Portion % pC1
			CurveB CloneCurve dup H2Interval Portion % pC2
			domB H1Interval MapTo                    % dompC1
			domB H2Interval MapTo                    % dompC2
%    \end{macrocode}
% Need on the stack: /domsB /domsA pC2 CurveA dompC2 domA   /domsB /domsA pC1 CurveA dompC1 domA
%    \begin{macrocode}
			8 -2 roll exch 8 2 roll 6 -2 roll exch 6 2 roll % /domsB /domsA /domsB /domsA pC1 pC2 dompC1 dompC2
			3 -1 roll exch % /domsB /domsA /domsB /domsA pC1 dompC1 pC2 dompC2
			CurveA exch domA 8 4 roll % /domsB /domsA pC2 CurveA dompC2 domA /domsB /domsA pC1 dompC1
			CurveA exch domA          % /domsB /domsA pC2 CurveA dompC2 domA /domsB /domsA pC1 CurveA dompC1 domA
		    } ifelse

		    _IterateIntersection
		    _IterateIntersection
%    \end{macrocode}		    
% Restore the state before the recursive calls.
%    \begin{macrocode}
		    /iter exch def
		    /domB exch def
		    /domA exch def
		    /CurveB exch def
		    /CurveA exch def
		    false exit
		} if
		CurveA CurveB /CurveA exch def /CurveB exch def
		domA domB /domA exch def /domB exch def
%    \end{macrocode}
% exchange /domsA and /domsB on the stack!
%    \begin{macrocode}
		exch
	    } loop	
%    \end{macrocode}
% boolean on stack
%    \begin{macrocode}
	    {
		4 -1 roll exch PutInterval PutInterval
		CheckIT {
		    (found an intersection ============================) ==
		} if
	    } { pop pop } ifelse
	} {
	    pop pop
	} ifelse
    } ifelse
    /depth depth 1 sub def
    DebugEnd
} bind def
%    \end{macrocode}
% \end{macro}
% 
% \begin{macro}{PutInterval}
%   Add a new interval \PSarray{newinterval} to the array stored in
%   \PSname{/Intervals}. The new interval is "cloned" before storing it.
% \begin{pssyntax}
% \PSname{Intervals} \PSarray{newinterval} \PSop{PutInterval}
% \end{pssyntax}
%    \begin{macrocode}
/PutInterval {
    CloneVec [ exch 3 -1 roll dup 4 1 roll load aload pop ] def
} bind def
%    \end{macrocode}
% \end{macro}
% 
% \begin{macro}{IsEmptyInterval}
% Check if an interval is empty, which is represented by a [1 0] interval.
% \begin{pssyntx}
% \PSarray{interval} \PSop{IsEmptyInterval} \PSvar{boolean}
%    \begin{macrocode}
/IsEmptyInterval {
    aload pop 0 eq exch 1 eq and
} bind def
%    \end{macrocode}
% \end{macro}
%
% \begin{macro}{ToUnitInterval}
% Limit an interval \PSvar{a b} to the unit interval \PSarray{0 1}.
% \begin{pssyntax}
% \PSvar{a b} \PSop{ToUnitInterval} \PSarray{a|0 b|1}
% \end{pssyntax}
%    \begin{macrocode}
/ToUnitInterval {
    ToUnitRange exch ToUnitRange 2 copy gt {
	exch
    } if
    ToVec
} bind def
%    \end{macrocode}
% \end{macro}
% \begin{macro}{ToUnitRange}
% Limit a number to the range \PSarray{0 1}.
%    \begin{macrocode}
/ToUnitRange {
    dup 0 lt {
	pop 0
    }{
	dup 1 gt {
	    pop 1
	} if
    } ifelse
} bind def
%    \end{macrocode}
% \end{macro}
% 
% \begin{macro}{CloneCurve}
% Does a deep copy of the array \PSarray{Curve}. This also involved deep copies of the contained point arrays.
% \begin{pssyntax}
% \PSarray{Curve} \PSop{CloneCurve} \PSarray{newCurve}
%    \begin{macrocode}
/CloneCurve {
    [ exch {
	CloneVec
    } forall ]
} bind def
%    \end{macrocode}
% \end{macro}
%
% \begin{macro}{CloneVec}
% Does a deep copy of the vector \PSarray{X Y}
% \begin{pssyntax}
% \PSarray{X Y} \PSop{CloneVec} \PSarray{Xnew Ynew}
% \end{pssyntax}
%    \begin{macrocode}
/CloneVec {
    aload pop ToVec
} bind def
%    \end{macrocode}
% \end{macro}
% 
% \begin{macro}{MapTo}
% Map the sub-interval \PSarray{I} in \PSarray{0 1} into the interval \PSarray{J}. Returns a new array.
% \begin{pssyntax}
% \PSarray{J} \PSarray{I} \PSop{MapTo} \PSarray{Jnew}
% \end{pssyntax}
%    \begin{macrocode}
/MapTo {
    (MapTo) DebugBegin
    exch aload 0 get 3 1 roll exch sub 2 copy % [I] J0 Jextent J0 Jextent
    5 -1 roll aload aload pop % J0 Jextent J0 Jextent I0 I1 I0 I1
    min 4 -1 roll mul % J0 Jextent J0 I0 I1 min(I0,I1)*Jextent
    4 -1 roll add [ exch % J0 Jextent I0 I1 [ J0new
    6 2 roll max mul add ]
    DebugEnd
} bind def
%    \end{macrocode}
% \end{macro}
% 
% \begin{macro}{Portion}
% Compute the portion of the Bezier curve \PSarray{CurveB} wrt the interval \PSarray{I}.
% \begin{pssyntax}
% \PSarray{CurveB} \PSarray{I} \PSop{Portion} \PSarray{CurvePartB}
% \end{pssyntax}
%    \begin{macrocode}
/Portion {
    (Portion) DebugBegin
    dup Min 0 eq { % [CurveB] [I]
	% I.min() == 0
	Max dup 1 eq {% [CurveB] I.max()
	    % I.max() == 1
	    pop pop	    
	} { % [CurveB] I.max()
	    LeftPortion
	} ifelse
    } { % [CurveB] [I]
	2 copy Min % [CurveB] [I] [CurveB] I.min()
	RightPortion
	dup Max 1 eq {
	    % I.max() == 1
	    pop pop
	} {% [CurveB] [I]
	    dup aload pop exch sub 1 3 -1 roll Min sub div % [CurveB] (I1-I0)/(1-I.min())
	    LeftPortion
	} ifelse
    } ifelse
    DebugEnd
} bind def
%    \end{macrocode}
% \end{macro}
% 
% \begin{macro}{LeftPortion}
%   Compute the portion of the Bezier curve \PSarray{CurveB} wrt the
%   interval \PSarray{0 t}.
% \begin{pssyntax}
% \PSarray{CurveB} \PSvar{t} \PSop{LeftPortion} \PSarray{CurvePartB}
% \end{pssyntax}
%    \begin{macrocode}
/LeftPortion {
    (LeftPortion) DebugBegin
    exch dup length 1 sub dup 4 1 roll % L-1 t [CurveB] L-1
    1 1 3 -1 roll { % L-1 t [CurveB] i
	4 -1 roll dup 5 1 roll % L-1 t [CurveB] i L-1
	-1 3 -1 roll % L-1 t [CurveB] L-1 -1 i
	{ % L-1 t [CurveB] j
	    2 copy 5 copy % L-1 t [CurveB] j [CurveB] j t [CurveB] j [CurveB] j 
	    1 sub get 3 1 roll get % L-1 t [CurveB] j [CurveB] j t B[j-1] B[j]
	    Lerp put pop % L-1 t [CurveB]
	} for
    } for
    pop pop pop
    DebugEnd
} bind def
%    \end{macrocode}
% \end{macro}
%
% \begin{macro}{RightPortion}
%   Compute the portion of the Bezier curve \PSarray{CurveB} wrt the
%   interval \PSarray{t 1}.
% \begin{pssyntax}
% \PSarray{CurveB} \PSvar{t} \PSop{RightPortion} \PSarray{CurvePartB}
% \end{pssyntax}
%    \begin{macrocode}
/RightPortion {
    (RightPortion) DebugBegin
    exch dup length 1 sub dup 4 1 roll % L-1 t [CurveB] L-1
    1 1 3 -1 roll {% L-1 t [CurveB] i
	4 -1 roll dup 5 1 roll % L-1 t [CurveB] i L-1
	exch sub 0 1 3 -1 roll  % L-1 t [CurveB] 0 1 L-i-1
	{% L-1 t [CurveB] j
	    2 copy 5 copy
	    get 3 1 roll 1 add get Lerp put pop
	} for
    } for
    pop pop pop
    DebugEnd
} bind def
%    \end{macrocode}
% \end{macro}
% 
% \begin{macro}{Lerp}
% Given two points and a parameter \PSvar{t} $\in$ \PSarray{0 1}, return a point
% proportionally from \PSarray{A} to \PSarray{B} by \PSvar{t}. Akin to 1 degree Bezier.
% \begin{pssyntax}
% \PSvar{t} \PSarray{A} \PSarray{B} \PSop{Lerp} \PSarray{newpoint}
% \end{pssyntax}
%    \begin{macrocode}
/Lerp {
    (Lerp) DebugBegin
    3 -1 roll dup 1 exch sub 3 1 roll % [A] (1-t) [B] t
    exch aload pop 3 -1 roll VecScale % [A] (1-t) B.x*t B.y*t
    4 2 roll
    exch aload pop 3 -1 roll VecScale VecAdd ToVec % [A.x*(1-t)+B.x*t A.y*(1-t)+B.y*t]
    DebugEnd
} bind def
%    \end{macrocode}
% \end{macro}
% 
% \begin{macro}{IsConstant}
% Test if all points of a curve are near to each other. This is used as termination criterium for the intersection procedure.
% \begin{pssyntax}
% \PSarray{Curve} \PSop{IsConstant} \PSvar{boolean}
% \end{pssyntax}
%    \begin{macrocode}
/IsConstant {
    aload length [ exch 1 roll ] true 3 1 roll
    {
	exch dup 4 1 roll
	AreNear and exch
    } forall
    pop
} bind def
%    \end{macrocode}
% \end{macro}
% \begin{macro}{AreNear}
% Test if two points are near to each other.
% \begin{pssyntax}
% \PSarray{P1} \PSarray{P2} \PSop{AreNear} \PSvar{boolean}
% \end{pssyntax}
%    \begin{macrocode}
/AreNear {
    (AreNear) DebugBegin
    aload pop 3 -1 roll aload pop
    4 copy abs 3 { exch abs max } repeat Epsilon mul
    dup 6 2 roll VecSub abs 4 -1 roll lt exch abs 3 -1 roll lt and
    DebugEnd
} bind def
%    \end{macrocode}
% \end{macro}
% 
% \begin{macro}{Min}
% Get the minimum value of the vector \PSarray{P}.
% \begin{pssyntax}
% \PSarray{P} \PSop{Min} \PSvar{minimum}
% \end{pssyntax}
%    \begin{macrocode}
/Min {
    aload pop min
} bind def
%    \end{macrocode}
% \end{macro}
% \begin{macro}{Min}
% Get the maximum value of the vector \PSarray{P}.
% \begin{pssyntax}
% \PSarray{P} \PSop{Max} \PSvar{maximum}
% \end{pssyntax}
%    \begin{macrocode}
/Max {
    aload pop max
} bind def
%    \end{macrocode}
% \end{macro}
% \begin{macro}{Min}
% Get the extent of the interval \PSarray{I}.
% \begin{pssyntax}
% \PSarray{I} \PSop{Extent} \PSvar{I1-I0}
% \end{pssyntax}
%    \begin{macrocode}
/Extent {
    aload pop exch sub
} bind def
%    \end{macrocode}
% \end{macro}
%
% \begin{macro}{MiddlePoint}
% Compute the middle point of the first and last point of \PSarray{Curve}.
% \begin{pssyntax}
% \PSarray{Curve} \PSop{MiddlePoint} \PSvar{X Y}
% \end{pssyntax}
%    \begin{macrocode}
/MiddlePoint {
    dup dup length 1 sub get aload pop
    3 -1 roll 0 get aload pop
    VecAdd 0.5 VecScale
} bind def
%    \end{macrocode}
% \end{macro}
% 
% \begin{macro}{OrthogonalOrientationLine}
% \begin{pssyntax}
% \PSvar{MiddlePointA} \PSarray{CurveB} \PSop{OrthogonalOrientationLine} \PSvar{A B C}
% \end{pssyntax}
%    \begin{macrocode}
/OrthogonalOrientationLine {
    (OrthogonalOrientationLine) DebugBegin
    dup dup length 1 sub get aload pop 3 -1 roll 0 get aload pop VecSub
%    \end{macrocode}
% rotate by +90 degrees
%    \begin{macrocode}
    neg exch
    4 2 roll 2 copy 6 2 roll VecAdd
    ImplicitLine
    DebugEnd
} bind def
%    \end{macrocode}
% \end{macro}
% 
% \begin{macro}{PickOrientationLine}
%   Pick an orientation line for a Bezier curve. This uses the first
%   point and the lastmost point, which is not near to it.
% \begin{pssyntax}
% \PSarray{Curve} \PSop{PickOrientationLine} \PSvar{A B C}
% \end{pssyntax}
%    \begin{macrocode}
/PickOrientationLine {
    (PickOrientationLine) DebugBegin
    dup dup length 1 sub exch 0 get% [Curve] L-1 P0
    exch -1 1 {% [Curve] P0 i
	3 -1 roll dup 4 1 roll exch get % [Curve] P0 Pi
	2 copy AreNear {
	    pop
	} {
	    exit
	} ifelse
    } for
    3 -1 roll pop
    exch aload pop 3 -1 roll aload pop ImplicitLine
    DebugEnd
} bind def
%    \end{macrocode}
% \end{macro}
% 
% \begin{macro}{ImplicitLine}
% Compute the coefficients \PSvar{A}, \PSvar{B}, \PSvar{C} of the normalized implicit equation
% of the line which goes through the points \PSarray{Xi Yi} and \PSarray{Xj Yj}.
%
% \begin{pssyntax}
% \PSvar{Xi Yi Xj Yj} \PSop{ImplicitLine} \PSvar{A B C}
% \end{pssyntax}
%    \begin{macrocode}
/ImplicitLine {
    4 copy % Xi Yi Xj Yj Xi Yi Xj Yj
    3 -1 roll sub 7 1 roll sub 5 1 roll % Yj-Yi Xi-Xj Xi Yi Xj Yj
    % Yi*Xj - Xi*Yj
    4 -1 roll mul neg % Yj-Yi Xi-Xj Yi Xj -Yj*Xi
    3 1 roll mul add % Yj-Yi Xi-Xj Yi*Xj-Yj*Xi | l0 l1 l2
    3 1 roll 2 copy tx@Dict begin Pyth end dup dup % l2 l0 l1 L L L
    5 -1 roll exch % l2 l1 L L l0 L
    div 5 1 roll % l0/L l2 l1 L L
    3 1 roll div % l0/L l2 L l1/L
    3 1 roll div % l0/L l1/L l2/L
} bind def
%    \end{macrocode}
% \end{macro}
% 
% \begin{macro}{distance}
% Compute the distance of point \PSarray{X Y} from the implicit line given
% by $Ax + By + C = 0,\quad (A^2+B^2 = 1)$.
% \begin{pssyntax}
% \PSvar{X Y A B C} \PSop{distance} \PSvar{d}
% \end{pssyntax}
%    \begin{macrocode}
/distance {
    5 1 roll 3 -1 roll mul 3 1 roll mul add add
} bind def
%    \end{macrocode}
% \end{macro}
%
% \begin{macro}{ArrayToPointArray}
% \begin{pssyntax}
% \PSarray{A.x A.y ... N.x N.y} \PSop{ArrayToPointArray} \PSarray{\PSarray{A.x A.y} \ldots \PSarray{N.x N.y}}
% \end{pssyntax}
%    \begin{macrocode}
/ArrayToPointArray {
    aload length dup 2 idiv {
	3 1 roll [ 3 1 roll ] exch
	dup 1 sub 3 1 roll 1 roll
    } repeat 1 add [ exch 1 roll ]
} bind def
%    \end{macrocode}
% \end{macro}
%
% \begin{macro}{PointArrayToArray}
% \begin{pssyntax}
% \PSarray{\PSarray{A.x A.y} \ldots \PSarray{N.x N.y}} \PSop{PointArrayToArray} \PSarray{A.x A.y ... N.x N.y}
% \end{pssyntax}
%    \begin{macrocode}
/PointArrayToArray {
    aload length dup {
	1 add dup 3 -1 roll aload pop 4 -1 roll 1 add 2 roll
    } repeat 1 add [ exch 1 roll ]
} bind def
%    \end{macrocode}
% \end{macro}
% 
% \begin{macro}{ClipCurve}
% Clip the Bezier curve B with respect to the Bezier curve A for
% individuating intersection points. The new parameter interval for the
% clipped curve is pushed on the stack.
% \begin{pssyntax}
% \PSarray{CurveA} \PSarray{CurveB} \PSop{ClipCurve} \PSarray{newinterval}
% \end{pssyntax}
%    \begin{macrocode}
/ClipCurve {
    (ClipCurve) DebugBegin
    4 dict begin 
    /CurveB exch def /CurveA exch def
    CurveA IsConstant {
    	CurveA MiddlePoint CurveB OrthogonalOrientationLine
    } {
	CurveA PickOrientationLine
    } ifelse
    CheckIT {
	3 copy exch 3 -1 roll (OrientationLine : )
	3 { exch 20 string cvs ( ) strcat strcat } repeat ==
    } if
    CurveA FatLineBounds
    CheckIT { dup (FatLineBounds : ) exch aload pop exch 20 string cvs (, ) strcat exch 20 string cvs strcat strcat == } if
    CurveB ClipCurveInterval
    end
    DebugEnd
} bind def
%    \end{macrocode}
% \end{macro}
% 
% \begin{macro}{FatLineBounds}
% Compute the boundary of the fat line given by \PSvar{A B C}
% \begin{pssyntax}
% \PSvar{A B C} \PSarray{Curve} \PSop{FatLineBounds} \PSvar{A B C} \PSarray{dmin dmax}
% \end{pssyntax}
%    \begin{macrocode}
/FatLineBounds {
    (FatLineBounds) DebugBegin
    /dmin 0 def /dmax 0 def
    { 
	4 copy aload pop 5 2 roll distance
	dup dmin lt { dup /dmin exch def } if
	dup dmax gt { dup /dmax exch def } if
	pop pop
    } forall
    [dmin dmax]
    DebugEnd
} bind def
%    \end{macrocode}
% \end{macro}
% 
% \begin{macro}{ClipCurveInterval}
%   Clip the Bezier curve wrt the fat line defined by the orientation
%   line (given by \PSvar{A B C}) and the interval range
%   \PSarray{bound}. The new parameter interval \PSarray{newinterval}
%   for the clipped curve is pushed on the stack.
% \begin{pssyntax}
% \PSvar{A B C} \PSarray{bound} \PSarray{curve} \PSop{ClipCurveInterval} \PSarray{newinterval}
% \end{pssyntax}
%    \begin{macrocode}
/ClipCurveInterval {
    (ClipCurveInterval) DebugBegin
    15 dict begin
    /curve exch def
    aload pop 2 copy min /boundMin exch def max /boundMax exch def
    [ 4 1 roll ] cvx /fatline exch def
    % number of sub-intervals
    /n curve length 1 sub def
    % distance curve control points
    /D n 1 add array def
    0 1 n { % i
	dup curve exch get aload pop % i Pi.x Pi.y
	fatline distance % distance d of Point i from the orientation line, on stack; i d
	exch dup n div % d i i/n
	[ exch 4 -1 roll ] % i [ i/n d ]
	D 3 1 roll put 
    } for
    D ConvexHull /D exch def
%    \end{macrocode}
% get the x-coordinate of the i-th point, i getX -> D[i][X]
%    \begin{macrocode}
    /getX { D exch get 0 get } def
%    \end{macrocode}
% get the y-coordinate of the i-th point, i getY -> D[i][Y]
%    \begin{macrocode}
    /getY { D exch get 1 get } def
    /tmin 1 def /tmax 0 def
    0 getY dup
    boundMin lt /plower exch def
    boundMax gt /phigher exch def
    plower phigher or not {
%    \end{macrocode}
% inside the fat line
%    \begin{macrocode}
	tmin 0 getX gt { /tmin 0 getX def } if
	tmax 0 getX lt { /tmax 0 getX def } if	
    } if
    1 1 D length 1 sub {
	/i exch def
	/clower i getY boundMin lt def
	/chigher i getY boundMax gt def
	clower chigher or not {
%    \end{macrocode}
% inside the fat line
%    \begin{macrocode}
	    tmin i getX gt { /tmin i getX def } if
	    tmax i getX lt { /tmax i getX def } if
	} if
	clower plower eq not {
%    \end{macrocode}
% cross the lower bound
%    \begin{macrocode}
	    boundMin i 1 sub i D Intersect % t on stack
	    dup tmin lt { dup /tmin exch def } if
	    dup tmax gt { dup /tmax exch def } if
	    pop 
	    /plower clower def
	} if
	chigher phigher eq not {
%    \end{macrocode}
% cross the upper bound
%    \begin{macrocode}
	    boundMax i 1 sub i D Intersect
	    dup tmin lt { dup /tmin exch def } if
	    dup tmax gt { dup /tmax exch def } if
	    pop 
	    /phigher chigher def
	} if
    } for
%    \end{macrocode}
% we have to test the closing segment for intersection
%    \begin{macrocode}
    /i D length 1 sub def
    /clower 0 getY boundMin lt def
    /chigher 0 getY boundMax gt def
    clower plower eq not {
%    \end{macrocode}
% cross the lower bound
%    \begin{macrocode}
	boundMin i 0 D Intersect
	dup tmin lt { dup /tmin exch def } if
	dup tmax gt { dup /tmax exch def } if
	pop
    } if
    chigher phigher eq not {
%    \end{macrocode}
% cross the lower bound
%    \begin{macrocode}
	boundMax i 0 D Intersect
	dup tmin lt { dup /tmin exch def } if
	dup tmax gt { dup /tmax exch def } if
	pop
    } if
    [tmin tmax]
    end
    DebugEnd
} bind def
%    \end{macrocode}
% \end{macro}
% 
% \begin{macro}{Intersect}
%   Get the x component of the intersection point between the line
%   passing through $i$-th and $j$-th points of \PSarray{Curve} and the
%   horizonal line through \PSvar{y}.
% \begin{pssyntax}
% \PSvar{y i j} \PSarray{Curve} \PSop{Intersect} \PSvar{Xisect}
% \end{pssyntax}
%    \begin{macrocode}
/Intersect {
    dup 4 -1 roll get aload pop
    4 2 roll exch get aload pop
%    \end{macrocode}
% On the stack: \PSvar{y Xi Yi Xj Yj}, Compute (Xj - Xi) * (y - Yi)/(Yj - Yi) + Xi
%
% We are sure, that Yi != Yj, because this procedure is called only
% when the lower or upper bound is crossed.
%    \begin{macrocode}
    4 2 roll 2 copy 6 2 roll VecSub
    5 2 roll
    neg 3 -1 roll add
    3 -1 roll div
    3 -1 roll mul add
} bind def
%    \end{macrocode}
% \end{macro}
%
% \begin{macro}{IsPath}
%   Check if an array is a path. A path is represented as array, which
%   contains other arrays which represent native Postscript
%   operations. Those can be \PSarray{X Y /@m}, \PSarray{X Y /@l}, or
%   \PSarray{X1 Y1 X2 Y2 X3 Y3 /@c}.
%
%   \begin{pssyntax}
%     \PSarray{array} \PSop{IsPath} \PSvar{boolean}
%   \end{pssyntax}
%    \begin{macrocode}
/IsPath {
  dup length 1 sub get type /nametype eq { true } { false } ifelse
} bind def
%    \end{macrocode}
% \end{macro}
%
% \begin{macro}{ShowFullPath}
%    \begin{macrocode}
/ShowFullPath {
  4 dict begin
  /movetype /moveto load def
  /linetype /lineto load def
  /curvetype /curveto load def
  mark exch aload pop
  {
    counttomark 0 eq { exit } if
    load exec
  } loop
  pop 
  end
} bind def
/ShowPathPortion {
  6 dict begin
  exch dup 0 lt { pop 0 } if /tstart exch def
  % if tstart is < 0, use the array length
  dup 0 lt { pop dup length } if /tstop exch def
  tstop tstart gt {
    /movetype /moveto load def
    /linetype /lineto load def
    /curvetype /curveto load def
    /n 0 def
    mark exch aload pop
    {
      counttomark 0 eq { pop exit } if
      dup /movetype eq not { /n n 1 add def } if
      % current section is after tstart
      n tstop sub 1 ge { cleartomark exit } if

      dup /movetype eq {
        load exec
      } {
        tstart n gt {
          % current path section is before tstart
          /curvetype eq { 6 2 roll 4 { pop } repeat } if
          movetype
        } {
          tstart n 1 sub gt tstop n lt or {
            % draw a truncated section
            tstart n sub 1 add tstop n sub 1 add
            ToUnitInterval exch
            /linetype eq {
              3 1 roll ToVec currentpoint ToVec exch ToVec
              dup 3 -1 roll Portion
              aload pop exch 
              tstart n 1 sub gt { aload pop moveto } { pop } ifelse
              aload pop lineto
            } {
              7 1 roll [ currentpoint 9 3 roll ] ArrayToPointArray
              dup 3 -1 roll Portion 
              { aload pop } forall
              8 -2 roll
              tstart n 1 sub gt { moveto } { pop pop } ifelse
              curveto
            } ifelse
          }{
            load exec
          } ifelse
        } ifelse
      } ifelse
    } loop
  } if
  end
} bind def
%    \end{macrocode}
% \end{macro}
%
%    \begin{macrocode}
 % Graham Scal algorithm to compute the convex hull of a set of
 % points. Code written by Bill Casselman,
 %  http://www.math.ubc.ca/~cass/graphics/text/www/
 %
 % [[X1 Y1] [X2 Y2] ... [Xn Yn]] hull -> [[...] ... [...]]
 %
/hulldict 32 dict def
hulldict begin

 % u - v 
/vsub { 2 dict begin
/v exch def
/u exch def
[ 
  u 0 get v 0 get sub
  u 1 get v 1 get sub
]
end } def

 % u - v rotated 90 degrees
/vperp { 2 dict begin
/v exch def
/u exch def
[ 
  v 1 get u 1 get sub
  u 0 get v 0 get sub
]
end } def

/dot { 2 dict begin
/v exch def
/u exch def
  v 0 get u 0 get mul
  v 1 get u 1 get mul
  add
end } def 

 % P Q
 % tests whether P < Q in lexicographic order
 % i.e xP < xQ, or yP < yQ if xP = yP
/comp { 2 dict begin
/Q exch def
/P exch def
P 0 get Q 0 get lt 
  P 0 get Q 0 get eq
  P 1 get Q 1 get lt 
  and 
or 
end } def

end

 % args: an array of points C
 % effect: returns the array of points on the boundary of
 %     the convex hull of C, in clockwise order 

/ConvexHull {
(ConvexHull) DebugBegin
hulldict begin
/C exch def
/comp C quicksort
/n C length def
 % Q might circle around to the start
/Q n 1 add array def
Q 0 C 0 get put
Q 1 C 1 get put
/i 2 def
/k 2 def
 % i is next point in C to be looked at
 % k is next point in Q to be added
 % [ Q[0] Q[1] ... ]
 % scan the points to make the top hull
n 2 sub {
  % P is the current point at right
  /P C i get def
  /i i 1 add def
  {
    % if k = 1 then just add P 
    k 2 lt { exit } if
    % now k is 2 or more
    % look at Q[k-2] Q[k-1] P: a left turn (or in a line)?
    % yes if (P - Q[k-1])*(Q[k-1] - Q[k-2])^perp >= 0
    P Q k 1 sub get vsub 
    Q k 1 sub get Q k 2 sub get vperp 
    dot 0 lt {
      % not a left turn
      exit
    } if
    /k k 1 sub def
  } loop
  Q k P put
  /k k 1 add def
} repeat

 % done with top half
 % K is where the right hand point is
/K k 1 sub def

/i n 2 sub def
Q k C i get put
/i i 1 sub def
/k k 1 add def
n 2 sub {
  % P is the current point at right
  /P C i get def
  /i i 1 sub def
  {
    % in this pass k is always 2 or more
    k K 2 add lt { exit } if
    % look at Q[k-2] Q[k-1] P: a left turn (or in a line)?
    % yes if (P - Q[k-1])*(Q[k-1] - Q[k-2])^perp >= 0
    P Q k 1 sub get vsub 
    Q k 1 sub get Q k 2 sub get vperp 
    dot 0 lt {
      % not a left turn
      exit
    } if
    /k k 1 sub def
  } loop
  Q k P put
  /k k 1 add def
} repeat

 % strip Q down to [ Q[0] Q[1] ... Q[k-2] ]
 % excluding the doubled initial point
[ 0 1 k 2 sub {
  Q exch get
} for ] 
end
DebugEnd
} def

/qsortdict 8 dict def

qsortdict begin

 % args: /comp a L R x
 % effect: effects a partition into two pieces [L j] [i R]
 %     leaves i j on stack

/partition { 8 dict begin
/x exch def
/j exch def
/i exch def
/a exch def
dup type /nametype eq { load } if /comp exch def
{
  {
    a i get x comp exec not {
      exit
    } if
    /i i 1 add def
  } loop
  {
    x a j get comp exec not {
      exit
    } if
    /j j 1 sub def
  } loop
  
  i j le {
    % swap a[i] a[j]
    a j a i get
    a i a j get 
    put put
    /i i 1 add def
    /j j 1 sub def
  } if
  i j gt {
    exit
  } if
} loop
i j
end } def

 % args: /comp a L R
 % effect: sorts a[L .. R] according to comp
/subsort {
 % /c a L R
[ 3 1 roll ] 3 copy
 % /c a [L R] /c a [L R]
aload aload pop 
 % /c a [L R] /c a L R L R
add 2 idiv
 % /c a [L R] /c a L R (L+R)/2
3 index exch get
 % /c a [L R] /c a L R x
partition
 % /c a [L R] i j
 % if j > L subsort(a, L, j)
dup 
 % /c a [L R] i j j
3 index 0 get gt {
  % /c a [L R] i j
  5 copy 
  % /c a [L R] i j /c a [L R] i j
  exch pop
  % /c a [L R] i j /c a [L R] j
  exch 0 get exch
  % ... /c a L j 
  subsort
} if
 % /c a [L R] i j
pop dup
 % /c a [L R] i i
 % if i < R subsort(a, i, R)
2 index 1 get lt {
  % /c a [L R] i
  exch 1 get 
  % /c a i R
  subsort
}{
  4 { pop } repeat
} ifelse
} def

end % qsortdict

 % args: /comp a
 % effect: sorts the array a 
 % comp returns truth of x < y for entries in a

/quicksort { qsortdict begin
dup length 1 gt {
 % /comp a
dup 
 % /comp a a 
length 1 sub 
 % /comp a n-1
0 exch subsort
} {
pop pop
} ifelse
end } def
%    \end{macrocode}
% 
% Debugging stuff
%    \begin{macrocode}
/debug {
    dup 1 add copy {==} repeat pop
} bind def
/DebugIT false def
/CheckIT false def
/DebugDepth 0 def
/DebugBegin {
  DebugIT {
    /DebugProcName exch def
    DebugDepth 2 mul string
    0 1 DebugDepth 2 mul 1 sub {
      dup 2 mod 0 eq { (|) }{( )} ifelse
      3 -1 roll dup 4 2 roll
      putinterval
    } for
    DebugProcName strcat ==
    /DebugDepth DebugDepth 1 add def
  }{
    pop
  } ifelse
} bind def
/DebugEnd {
  DebugIT {
    /DebugDepth DebugDepth 1 sub def
    DebugDepth 2 mul 2 add string
    0 1 DebugDepth 2 mul 1 sub {
      dup 2 mod 0 eq { (|) }{ ( ) } ifelse
      3 -1 roll dup 4 2 roll
      putinterval
    } for
    dup DebugDepth 2 mul (+-) putinterval
    ( done) strcat ==
  } if
} bind def
/strcat {
    exch 2 copy
    length exch length add
    string dup dup 5 2 roll
    copy length exch
    putinterval
} bind def
/nametostr {
    dup length string cvs
} bind def
/ShowCurve {
    { aload pop } forall
    8 -2 roll moveto curveto
} bind def
/CurveToString {
    (CurveToString) DebugBegin
    aload pop ([) 3 -1 roll 20 string cvs strcat (, ) strcat exch 20 string cvs strcat (]) strcat
    DebugEnd
} bind def
end % tx@IntersectDict
%    \end{macrocode}
%</prolog> 
% \Finale
% \endinput

\IfFileExists{pst-intersect.pro}{%
    \ProvidesFile{pst-intersect.pro}
      [2014/02/22 PostScript prologue file]
      \@addtofilelist{pst-intersect.pro}}{}%
%    \end{macrocode}
%</stylefile>
%
% \chapter{The \TeX\ implementation}
%
%<*texfile>
%    \begin{macrocode}
\csname PSTintersectLoaded\endcsname
\let\PSTintersectLoaded\endinput

\ifx\PSTricksLoaded\endinput\else\input pstricks.tex \fi
\ifx\PSTXKeyLoaded\endinput\else \input pst-xkey.tex \fi
\ifx\PSTnodesLoaded\endinput\else\input pst-node.tex \fi
\ifx\PSTfuncLoaded\endinput\else \input pst-func.tex \fi

\edef\PstAtCode{\the\catcode`\@} \catcode`\@=11\relax

\pst@addfams{intersect}
\pstheader{pst-intersect.pro}

\def\pst@intersectdict{tx@IntersectDict begin }
\def\PIT@dict#1{\pst@intersectdict #1 end}
\def\PIT@Verb#1{\pst@Verb{\PIT@dict{#1} }}%

\def\pssavebezier{\pst@object{pssavebezier}}
\def\pssavebezier@i#1{%
  \begin@OpenObj
    \PIT@checkname{#1}%
    \addto@pscode{ /\PIT@name{#1} }%
    \pst@getcoors[\pssavebezier@ii%]
}%
\def\pssavebezier@ii{%
  \addto@pscode{%
    % only 10 points allowed, remove the rest
    counttomark 20 gt { counttomark 20 sub { pop } repeat } if
    counttomark 2 idiv 1 sub
    \psk@plotpoints\space exch
    \txFunc@BezierCurve
    \ifshowpoints \txFunc@BezierShowPoints \else pop \fi
    tx@FuncDict begin Points aload pop end
  }%
  \let\use@pscode\PIT@use@pscode
  \end@OpenObj
  \PIT@Verb{%
    [ count 1 sub 1 roll ] ArrayToPointArray def 
  }%
\ignorespaces}%
\define@key[psset]{intersect}{tstart}{%
  \pst@checknum{#1}\PIT@key@tstart
}
\define@key[psset]{intersect}{tstop}{%
  \pst@checknum{#1}\PIT@key@tstop
}
\define@key[psset]{intersect}{istart}{%
  \pst@checknum{#1}\PIT@key@istart
}
\define@key[psset]{intersect}{istop}{%
  \pst@checknum{#1}\PIT@key@istop
}
\define@key[psset]{intersect}{name}{%
  \def\PIT@key@name{#1}%
}%
\newif\PIT@saveintersections
\newif\PIT@reversepath
\define@boolkey[psset]{intersect}[PIT@]{saveintersections}[true]{}
\define@boolkey[psset]{intersect}[PIT@]{reversepath}[true]{}
\psset[intersect]{%
  tstart=-1,
  tstop=-1,
  istart=-1,
  istop=-1,
  name={},
  saveintersections,
  reversepath=false
}%
\def\PIT@use@pscode{%
  \pstverb{%
    \pst@dict
    \tx@STP
    \pst@newpath
    \psk@origin
    \psk@swapaxes
    \pst@code
    end
    count /ocount exch def
  }%
  \gdef\pst@code{}%
}%
\let\PIT@pst@stroke@orig\pst@stroke
\def\PIT@save@path{%
  \PIT@pst@stroke@orig
  \addto@pscode{%
    clear mark
    \pst@intersectdict GetFullPath end
    counttomark 1 add -1 roll pop count }%
}%
\def\PIT@name@default{@tmp}%
\def\PIT@name#1{PIT@#1}%
\def\PIT@checkname#1{%
  \ifx\@empty#1\@empty
    \@pstrickserr{Unexpected empty argument!}\@ehpb
  \fi
}%
\def\pssavepath{\pst@object{pssavepath}}%
\long\def\pssavepath@i#1#2{%
  \begin@SpecialObj
    \PIT@checkname{#1}%
    \let\pst@stroke\PIT@save@path
    \let\use@pscode\PIT@use@pscode
    \pscustom{#2}%
    \PIT@Verb{%
      /\PIT@name{#1}
      [ 3 -1 roll 2 add 2 roll ] def }%
  \end@SpecialObj
}%
\def\pstracecurve{\pst@object{pstracecurve}}%
\def\pstracecurve@i#1{%
  \addbefore@par{plotpoints=200}%
  \begin@OpenObj
   \@ifnextchar\bgroup
     {\PIT@traceintersection{#1}}%
     {\PIT@tracecurve{#1}}%
}%
\def\PIT@tracecurve#1{%
  \PIT@checkname{#1}%
  \addto@pscode{%
    \pst@intersectdict
    \PIT@key@tstop\space dup 0 lt { pop dup GetSegmentCount } if
    \PIT@key@tstart\space dup 0 lt { pop 0 } if exch
    \PIT@name{#1} GetSegmentCount 
    2 copy exch dup 3 1 roll
    lt exch 0 lt or { exch } if pop
    \PIT@name{#1} dup IsPath {
      3 1 roll 2 copy gt \ifPIT@reversepath true \else false \fi xor {
        2 copy gt { exch } if
        3 -1 roll dup 4 1 roll GetSegmentCount
        dup 4 1 roll exch sub 3 1 roll sub exch
        % reverse the path, draw everything and resave the path
        gsave newpath
          [ 4 -1 roll aload pop InitTracing
          { counttomark 0 eq { exit } if
            load exec
          } loop
          reversepath
          GetFullPath ] 3 1 roll        
        grestore
      } if
      ShowPathPortion
    }{
      [ exch dup 5 -2 roll
      2 copy gt \ifPIT@reversepath true \else false \fi xor {
        2 copy gt { exch } if
        [ 5 -2 roll pop PointArrayToArray aload pop
        counttomark -2 4 { 2 roll } for ] ArrayToPointArray dup
        4 2 roll
      } if
      ToUnitInterval Portion
      { aload pop } forall
      % reverse the point order
      counttomark -2 4 { 2 roll } for
      counttomark 2 sub 2 idiv
      \psk@plotpoints
      exch
      \txFunc@BezierCurve
      \ifshowpoints \txFunc@BezierShowPoints \else pop \fi
    } ifelse
    end
  }%
  \end@OpenObj
}%
\def\PIT@traceintersection#1#2{%
  \PIT@checkname{#2}%
  \addto@pscode{%
    \pst@intersectdict
    \ifx\\#1\\%
    /\PIT@name{\PIT@name@default}
    \else
    /\PIT@name{#1}
    \fi 
    dup currentdict exch known not {
      \ifx\\#1\\%
      (You haven't defined an intersection!) ==
      \else
      (You haven't defined the intersection '#1') ==
      \fi
    } if 
    load
    dup dup type /dicttype eq exch /\PIT@name{#2} known and not {
      (You haven't defined the intersection '#2') ==
    } if
    dup /\PIT@name{#2} get
    exch /\PIT@name{#2}@t get
    dup length \PIT@key@istart\space ge 0 \PIT@key@istart\space lt and {
      dup \PIT@key@istart\space cvi 1 sub get
    } {
      \PIT@key@tstart
    } ifelse
    dup 0 lt { pop 0 } if
    exch % [curve] t_istart|tstart [ts]
    % 
    dup length \PIT@key@istop\space ge 0 \PIT@key@istop\space lt and {
      \PIT@key@istop\space cvi 1 sub get
    } {
      pop \PIT@key@tstop
    } ifelse
    3 -1 roll dup 4 1 roll GetSegmentCount 
    2 copy exch dup 3 1 roll % [curve] tstart tstop cnt tstop cnt tstop 
    lt exch 0 lt or { exch } if pop % (tstop < 0 | cnt < tstop)
    3 -1 roll dup
    IsPath {
      3 1 roll
      2 copy gt \ifPIT@reversepath true \else false \fi xor {
        2 copy gt { exch } if
        3 -1 roll dup 4 1 roll GetSegmentCount
        dup 4 1 roll exch sub 3 1 roll sub exch
        % reverse the path, draw everything and resave the path
        gsave newpath
          [ 4 -1 roll aload pop InitTracing
          { counttomark 0 eq { exit } if
            load exec
          } loop
          reversepath
          GetFullPath ] 3 1 roll        
        grestore
      } if
      ShowPathPortion
    }{
      [ exch dup 5 -2 roll
      exch dup 0 lt { pop 0 } if exch
      dup 0 lt { pop 1 } if
      2 copy gt \ifPIT@reversepath true \else false \fi xor {
        2 copy gt { exch } if
        [ 5 -2 roll pop PointArrayToArray aload pop
        counttomark -2 4 { 2 roll } for ] ArrayToPointArray dup
        4 2 roll
      } if
      ToUnitInterval Portion 
      { aload pop } forall
      counttomark 2 sub 2 idiv
      \psk@plotpoints
      exch
      \txFunc@BezierCurve
      \ifshowpoints \txFunc@BezierShowPoints \else pop \fi
    } ifelse
    end
  }%
  \end@OpenObj
}%
%
% \begin{macro}{\psintersect}
%    \begin{macrocode}
\def\psintersect{\pst@object{psintersect}}
\def\psintersect@i#1#2{%
  \PIT@checkname{#1}%
  \PIT@checkname{#2}%
  \begin@SpecialObj
  \def\PIT@@name{%
    \ifx\PIT@key@name\@empty
      \PIT@name{\PIT@name@default}
    \else
      \PIT@name{\PIT@key@name} 
    \fi}%
  \PIT@Verb{%
    currentdict /\PIT@name{#1} known not {
      (You haven't defined the curve or path '#1') ==
    } if
    currentdict /\PIT@name{#2} known not {
      (You haven't defined the curve or path '#2') ==
    } if
    \PIT@name{#1} \PIT@name{#2}
    \PIT@name{#1} IsPath {
      \PIT@name{#2} IsPath {
        IntersectPaths
      }{
        IntersectPathCurve
      } ifelse
    }{
      \PIT@name{#2} IsPath {
        IntersectCurvePath
      }{
        IntersectBeziers
        4 copy LoadIntersectionPoints 5 1 roll
      } ifelse
    } ifelse
    /\PIT@@name\space /\PIT@name{#1} /\PIT@name{#2} 8 3 roll 
    SaveIntersection
  }%
  \ifPIT@saveintersections
    \pst@Verb{%
      \pst@intersectdict 
        \PIT@@name\space /Points get 
        ArrayToPointArray
      end
      tx@NodeDict begin 
        dup length 1 1 3 -1 roll {
          2 copy 1 sub get cvx
          false 3 -1 roll (N@\PIT@key@name) exch 20 string cvs 
          \pst@intersectdict strcat end cvn
          10 {InitPnode} /NodeScale {} def NewNode
        } for
      end
      pop
    }%
  \fi
  \ifshowpoints
    \addto@pscode{%
      \pst@intersectdict
        [ \PIT@@name\space /Points get aload pop 
      end 
    }%
    \psdots@ii
  \else
    \end@SpecialObj
  \fi
}%
%    \end{macrocode}
% \end{macro}
% 
%    \begin{macrocode}
\catcode`\@=\PstAtCode\relax
%    \end{macrocode}
%</texfile> 
%
% \chapter{The Postscript header file}
% \makeatletter
%^^A Copied this definition from doc.sty and changed it not to add a
%^^A backslash to the Postscript procedure name in the index.
% \def\SpecialIndex@#1#2{%
%    \@SpecialIndexHelper@#1\@nil
%    \def\@tempb{ }%
%    \ifcat \@tempb\@gtempa
%       \special@index{\quotechar\space\actualchar
%                      \string\verb\quotechar*\verbatimchar
%                      \quotechar\space\verbatimchar#2}%
%    \else
%      \def\@tempb##1##2\relax{\ifx\relax##2\relax
%           \def\@tempc{\special@index{\quotechar##1\actualchar
%                       \string\verb\quotechar*\verbatimchar
%                       \quotechar##1\verbatimchar#2}}%
%         \else
%           \def\@tempc{\special@index{##1##2\actualchar
%                        \string\verb\quotechar*\verbatimchar##1##2\verbatimchar#2}}%
%         \fi}%
%      \expandafter\@tempb\@gtempa\relax
%      \@tempc
%    \fi}
% \makeatother
%
%<*prolog>
%    \begin{macrocode}
/tx@IntersectDict 200 dict def
tx@IntersectDict begin
%    \end{macrocode}
% These are some helper procedures for vector operations.
%
% \begin{macro}{VecAdd}
% Addition of two vectors.
% \begin{pssyntax}
%   \PSvar{Xa Ya Xb Yb} \PSop{VecAdd} \PSvar{Xa+Xb Ya+Yb}
% \end{pssyntax}
%    \begin{macrocode}
/VecAdd {
    3 -1 roll add 3 1 roll add exch
} bind def
%    \end{macrocode}
% \end{macro}
%
% \begin{macro}{VecSub}
% Subtraction of two vectors.
% \begin{pssyntax}
%   \PSvar{Xa Ya Xb Yb} \PSop{VecSub} \PSvar{Xa-Xb Ya-Yb}
% \end{pssyntax}
%    \begin{macrocode}
/VecSub {
    neg 3 -1 roll add 3 1 roll neg add exch
} bind def
%    \end{macrocode}
% \end{macro}
% 
% \begin{macro}{VecScale}
% Scale a vector by a factor \PSvar{fac}.
% \begin{pssyntax}
%   \PSvar{Xa Ya fac} \PSop{VecScale} \PSvar{fac}$\cdot$\PSvar{Xa} \PSvar{fac}$\cdot$\PSvar{Ya}
% \end{pssyntax}
%    \begin{macrocode}
/VecScale {
  dup 4 -1 roll mul 3 1 roll mul
} bind def
%    \end{macrocode}
% \end{macro}
%
% \begin{macro}{ToVec}
%   Convert two numbers to a procedure holding the two values. This
%   representation is used to save coordinate values of nodes and vectors.
%   \begin{pssyntax}
%     \PSvar{X Y} \PSop{ToVec} \PSarray{X Y}
%   \end{pssyntax}
%    \begin{macrocode}
/ToVec {
    [ 3 1 roll ]
} bind def
%    \end{macrocode}
% \end{macro}
%
% \PSvar{MaxPrecision} gives the precision of the curve parameter t for the
% intersection. This shouldn't be lower than $10^{-6}$, because
% PostScript uses single precision.
%    \begin{macrocode}
/MaxPrecision 1e-6 def
%    \end{macrocode}
%
% \PSvar{Epsilon} gives the allowed relative error of the intersection point. 
%    \begin{macrocode}
/Epsilon 1e-4 def
%    \end{macrocode}
% 
% The threshold for curve subdivision, see below.
%    \begin{macrocode}
/MinClippedSizeThreshold 0.8 def
%    \end{macrocode}
%
% The predefined intervals for the subdivision of the curves.
%    \begin{macrocode}
/H1Interval [0 0.5] def
/H2Interval [0.5 MaxPrecision add 1] def
%    \end{macrocode}
% 
% \begin{macro}{IntersectBeziers}
%   The main procedure, which computes the intersection of two bezier
%   curves of arbitrary order.  This, and most of the following
%   procedures operate on curves, which are stored as arrays of points,
%   the points are also arrays with two elements -- \PSvar{X} and
%   \PSvar{Y}. A Bezier curve of $n$-th order is then givesn by
%   \PSarray{\PSarray{X0 Y0} \PSarray{X1 Y1} \ldots \PSarray{XN YN}}.
%
% \begin{pssyntax}
%   \PSarray{curveA} \PSarray{curveB} \PSop{IntersectBeziers} 
%   \PSarray{curveA} \PSarray{tA} \PSarray{curveB} \PSarray{tB}
% \end{pssyntax}
%    \begin{macrocode}
/IntersectBeziers {
  2 copy length 2 eq exch length 2 eq and {
    IntersectLines
  }{ 
    2 copy [0 1] [0 1] IterateIntersection
  } ifelse
  3 -1 roll exch
} bind def
%    \end{macrocode}
% \end{macro}
%
% \begin{macro}{IntersectLines}
% 
%   \begin{pssyntax}
%     \PSarray{lineA} \PSarray{lineB} \PSop{IntersectLines}
%     \PSarray{lineA} \PSarray{tA} \PSarray{lineB} \PSarray{tB}
%   \end{pssyntax}
%    \begin{macrocode}
/IntersectLines {
  (IntersectLines) DebugBegin
  2 copy
  exch { aload pop } forall 5 -1 roll { aload pop } forall
  8 -2 roll 2 copy 10 4 roll 4 2 roll 2 copy 6 2 roll 10 2 roll
  VecSub
  6 2 roll 4 2 roll VecSub
  8 4 roll 4 2 roll VecSub % X3-X4 Y3-Y4 X2-X1 Y2-Y1 X3-X1 Y3-Y1 % b1 b2 a1 a2 c1 c2
  6 copy 12 -4 roll 
  neg 4 -1 roll mul 3 1 roll mul add
  dup 0 eq {
    % no intersections
    9 { pop } repeat [] []
  } {
    dup 10 1 roll 5 1 roll
     4 -1 roll mul 3 1 roll mul sub exch div
     6 1 roll 4 -1 roll mul 3 1 roll mul sub exch div
     2 copy 2 copy 0 ge exch 0 ge and 3 1 roll 1 le exch 1 le and and {
       [ exch ] exch [ exch ]
     } {
       pop pop [] []
     } ifelse
  } ifelse
  DebugEnd
} bind def
%    \end{macrocode}
% \end{macro}
% \begin{macro}{IntersectPaths}
%   \begin{pssyntax}
%     \PSarray{pathA} \PSarray{pathB} \PSop{IntersectPaths}
%     \PSarray{intersections} \PSarray{pathA} \PSarray{tA} \PSarray{pathB} \PSarray{tB}
%   \end{pssyntax}
%    \begin{macrocode}
/IntersectPaths {
  (IntersectPaths) DebugBegin
  6 dict begin 
    2 copy exch PreparePath dup length /nA exch def 
    exch PreparePath dup length /nB exch def
    /isect [] def
    /tA [] def /tB [] def
    { % [pathA] [Bi]
      /nB nB 1 sub def
      exch dup 3 1 roll % [pathA] [Bi] [pathA]
      {
        /nA nA 1 sub def
        exch dup 3 1 roll % [pathA] [Bi] [Aj] [Bi]
        IntersectBeziers % [curveA] [tA] [curveB] [tB]
        4 copy LoadIntersectionPoints
        [ exch isect aload pop ] /isect exch def
        exch pop 3 -1 roll pop
        [ tB aload length 2 add -1 roll TArray { nB add } forall ] /tB exch def
        [ tA aload length 2 add -1 roll TArray { nA add } forall ] /tA exch def
      } forall
      pop % remove [Bi]
      dup length /nA exch def
    } forall
    pop % remove [pathA]
    [ isect { aload pop } forall ] 3 1 roll tA exch tB
    % [intersections] [pathA] [tA] [pathB] [tB]
  end
  DebugEnd
} bind def
%    \end{macrocode}
% \end{macro}
%
% \begin{macro}{IntersectCurvePath}
%   \begin{pssyntax}
%     \PSarray{curveA} \PSarray{pathB} \PSop{IntersectCurvePath}
%     \PSarray{intersections} \PSarray{curveA} \PSarray{tA} \PSarray{pathB} \PSarray{tB}
%   \end{pssyntax}
%    \begin{macrocode}
/IntersectCurvePath {
  (IntersectCurvePath) DebugBegin
  6 dict begin 
    2 copy PreparePath dup length /n exch def
    /isect [] def
    /tA [] def /tB [] def
    { % [curveA] [Bi]
      /n n 1 sub def
      exch dup 3 -1 roll % [curveA] [curveA] [Bi] 
      IntersectBeziers 
      4 copy LoadIntersectionPoints % [curveA] [tA] [curveB] [tB]
      [ exch isect aload pop ] /isect exch def
      pop 3 -1 roll pop
      [ tB aload length 2 add -1 roll TArray { n add } forall ] /tB exch def
      [ tA aload length 2 add -1 roll TArray aload pop ] /tA exch def
    } forall
    pop % remove [curveA]
    [ isect { aload pop } forall ] 3 1 roll tA exch tB
    % [intersections] [curveA] [tA] [pathB] [tB]
  end
  DebugEnd
} bind def
/IntersectPathCurve {
  exch IntersectCurvePath 4 2 roll
} bind def
%    \end{macrocode}
% \end{macro}
% 
% \begin{macro}{SaveIntersection}
%   \begin{pssyntax}
%     \PSname{isectname} \PSname{nameA} \PSname{nameB} 
%     \PSarray{intersectionpoints} \PSarray{A} \PSarray{tA} \PSarray{B} \PSarray{tB} 
%     \PSop{SaveIntersection}
%   \end{pssyntax}
%    \begin{macrocode}
/SaveIntersection {
  (SaveIntersection) DebugBegin
  4 dict dup 10 -1 roll exch def
  begin %
    /Points 6 -1 roll def
    5 -1 roll dup 4 -1 roll def % /curveA [curveA] [tA] [tB] /curveB /curveB [curveB] def
    nametostr (@t) strcat cvn exch TArray def % /curveA [curveA] [tA] /curveB@t [tB] def
    3 -1 roll dup 4 -1 roll def
    nametostr (@t) strcat cvn exch TArray def
  end
  DebugEnd
} bind def
%    \end{macrocode}
% \end{macro}
% 
% \begin{macro}{TArray}
%   The curve parameters \PSvar{t} as determined by
%   \PSvar{IntersectBeziers} are given in a special array
%   construct. \PSvar{TArray} creates a simple array with the
%   \PSvar{t}-values given in ascending order.
%
% \begin{pssyntax}
% \PSarray{\PSarray{t0a t0b} \ldots \PSvar{null}\ldots \PSvar{integer}}
% \PSop{TArray} \PSarray{t0 t1 \ldots tN}
% \end{pssyntax}
%    \begin{macrocode}
/TArray {
  (TArray) DebugBegin
  dup length 0 gt {
    dup 0 get type /arraytype eq {
      [ exch
      { %dup type /nulltype eq { pop exit } if
  	aload pop add 0.5 mul
      } forall ]
    } if
    dup /lt exch quicksort
  } if
  DebugEnd %1 debug
} bind def
%    \end{macrocode}
% \end{macro} 
% 
% We can save arbitrary paths using \PSvar{pathforall}. The saved path
% contains the commands \PSname{movetype}, \PSname{linetype} and
% \PSname{curvetype}. By default, these are defined as the respective
% original procedures.
%    \begin{macrocode}
/InitTracing {
  /movetype /moveto load def
  /linetype /lineto load def
  /curvetype /curveto load def
} bind def
InitTracing
/GetFullPath {
  { /movetype counttomark 3 roll }
  { /linetype counttomark 3 roll }
  { /curvetype counttomark 7 roll }{} pathforall 
} bind def
%    \end{macrocode}
%
% \begin{macro}{PreparePath}
% [ ... /movetype ... /linetype .../curvetype ]
%    \begin{macrocode}
/PreparePath {
  [ exch aload pop
  {
    dup type /nametype eq not { exit } if
    dup /movetype eq {
      pop ToVec /@mycp exch def
    } {
      dup /linetype eq {
        pop [ @mycp 4 2 roll 2 copy ToVec /@mycp exch def ToVec ]
      } {
        pop [ @mycp 8 2 roll 2 copy ToVec /@mycp exch def
        ToVec 5 1 roll ToVec 4 1 roll ToVec 3 1 roll ]
      } ifelse
      counttomark 1 roll	
    } ifelse
  } loop ]
} bind def
%    \end{macrocode}
% \end{macro}
%
% \begin{macro}{GetSegmentCount}
% \PSarray{CurveOrPath} \PSop{GetSegmentCount} -> number of /linetype and /curvetypes
%    \begin{macrocode}
/GetSegmentCount {
  dup IsPath {
    [ exch aload pop 0
    {
      counttomark 1 eq { exit } if
      exch 
      dup /movetype eq {
        pop 3 1 roll pop pop
      }{
        dup /linetype eq {
          pop 1 add 3 1 roll pop pop
        }{
          pop 1 add 7 1 roll 6 { pop } repeat
        } ifelse
      } ifelse
    } loop
    exch pop
  } {
    % a Bezier curve is a single segment
    pop 1
  } ifelse
} bind def
%    \end{macrocode}
% \end{macro}
% 
% \begin{macro}{LoadLineIntersectionPoints}
% Prepare \PSarray{Curve} for use with tx@Func
% \begin{pssyntax}
% \PSarray{curve} \PSarray{t} \PSop{LoadLineIntersectionPoints}
% \PSarray{I0.x I0.y \ldots IN.x YN.x}
% \end{pssyntax}
%    \begin{macrocode}
/LoadLineIntersectionPoints {
  (LoadLineIntersectionPoints) DebugBegin
  exch [ exch { aload pop } forall ]
  tx@Dict begin tx@FuncDict begin 2 dict begin
    dup length 2 idiv 1 sub /BezierType exch def /Points exch def
    [ exch { GetBezierCoor } forall ]
  end end end
  DebugEnd
} bind def
%    \end{macrocode}
% \end{macro}
%
% \begin{macro}{LoadCurveIntersectionPoints}
%   Load the intersection points. This loads the same intersection point
%   from both curves, and chooses the one with the lowest error.
% \begin{pssyntax}
% \PSarray{curveA} \PSarray{tA} \PSArray{curveB} \PSarray{tB} 
%  \PSop{LoadCurveIntersectionPoints}
% \PSarray{I0.x I0.y \ldots IN.x YN.x}
% \end{pssyntax}
%    \begin{macrocode}
/LoadCurveIntersectionPoints {
  (LoadCurveIntersectionPoints) DebugBegin
  2 {
    4 2 roll
    [ exch { aload pop } forall ]
    exch [ exch { aload pop } forall ]
  } repeat
  % [A0.x A0.y ... AM.x AM.y] [tA0a tA0b ... tAMa tAMb] [tB0a tB0b ... tBNa tBNb] [B0.x B0.y ... BN.x BN.y]
  tx@Dict begin tx@FuncDict begin 2 dict begin
    dup length 2 idiv 1 sub /BezierType exch def /Points exch def
      [ exch { GetBezierCoor } forall ]
    3 1 roll 
    dup length 2 idiv 1 sub /BezierType exch def /Points exch def
      [ exch { GetBezierCoor } forall ]
    end
    %2 debug
    % [IB0.xa IB0.ya IB0.xb IB0.yb ... IBM.yb] [IA0.xa IA0.ya IA0.xb IA0.yb ... IAM.yb]
    2 {
      [ exch aload length 4 idiv {
        [ 5 1 roll ] counttomark 1 roll
      } repeat ]
      exch 
    } repeat
    % [[IB0.xa ...] ... [... IBM.yb]] [[IA0.xa IA0.ya IA0.xb IA0.yb] ...[IAM.xa ... IAM.yb]]
    2 {
      dup hulldict /comp get exch quicksort exch
    } repeat
    2 dict begin
      /B exch def /A exch def
      [ 0 1 A length 1 sub {
        dup A exch get exch B exch get % [IAi] [IBi]
        2 copy aload pop VecSub Pyth exch 
        aload pop VecSub Pyth lt { exch } if pop
        aload pop VecAdd 0.5 VecScale
      } for 
      % merge near intersection points
      %counttomark 2 idiv 1 1 3 -1 roll {
      %  pop
      %  counttomark 4 lt { exit } if
      %  4 copy 4 2 roll ToVec 3 1 roll ToVec AreNear {

      %  }
      %} for
      ]
    end
  end end
  DebugEnd
} bind def
%    \end{macrocode}
% \end{macro}
%
% \begin{macro}{LoadIntersectionPoints}
%    \begin{macrocode}
/LoadIntersectionPoints {
  (LoadIntersectionPoints) DebugBegin
  4 copy pop exch pop length 2 eq exch length 2 eq and {
    pop pop LoadLineIntersectionPoints
  }{
    LoadCurveIntersectionPoints
  } ifelse
  DebugEnd
} bind def
%    \end{macrocode}
% \end{macro}
% 
% \begin{macro}{IterateIntersection}
% Iteration procedure to compute all intersections of CurveA and CurveB.
% This contains the
% 
% \begin{pssyntax}
% \PSarray{CurveA} \PSarray{CurveB} \PSarray{intervalA} \PSarray{intervalB}
% \PSop{IterateIntersection} \PSarray{domsA} \PSarray{domsB}
% \end{pssyntax}
%    \begin{macrocode}
/IterateIntersection {
    (IterateIntersection) DebugBegin
    12 dict begin
	/precision MaxPrecision def
        4 2 roll 2 copy 6 2 roll
        dup IsPath not { PointArrayToArray } if
        0 exch { dup type /nametype eq { pop }{ abs max} ifelse } forall
        exch dup IsPath not { PointArrayToArray } if
        { dup type /nametype eq { pop }{ abs max} ifelse } forall
        Epsilon mul /epsilon exch def
%    \end{macrocode}
% in order to limit recursion
%    \begin{macrocode}
        /counter 0 def
	/depth 0 def
	/domsA [] def
	/domsB [] def
	/domsA /domsB 6 2 roll _IterateIntersection
	domsB domsA
    end
    dup length 0 gt {
      TArraysRemoveDup
    } if
    DebugEnd
} bind def
/TArraysRemoveDup {
  4 dict begin
    /tB exch def
    /tA exch def
    /j 0 def
    [ tA 0 get tB 0 get
    1 1 tA length 1 sub {
      /i exch def
      tA j get aload pop tA i get aload pop tx@Dict begin Pyth2 end MaxPrecision gt
      tB j get aload pop tB i get aload pop tx@Dict begin Pyth2 end MaxPrecision gt and {
        % keep the current parameter point
        /j i def
        tB i get tA i get
        counttomark 2 idiv 1 add 1 roll 
      } if
    } for
    counttomark 2 idiv 1 add [ exch 1 roll ] % [ ... [tB]
    counttomark 1 add 1 roll ] exch % [tA] [tB]
  end 
} bind def
%    \end{macrocode}
% \end{macro}
% 
% \begin{macro}{_IterateIntersection}
% This is the iteration part which is called recursively.
%
% \begin{pssyntax}
%   \PSname{domsA} \PSname{domsB} \PSarray{CurveA} \PSarray{CurveB}
%   \PSarray{domA} \PSarray{domB} \PSop{_IterateIntersection}
% \end{pssyntax}
%    \begin{macrocode}
/_IterateIntersection {
    (_IterateIntersection) DebugBegin
    CloneVec /domB exch def
    CloneVec /domA exch def
    CloneCurve /CurveB exch def
    CloneCurve /CurveA exch def
    /iter 0 def
    /depth depth 1 add def
    /dom null def
    /counter counter 1 add def

    CheckIT {
	(>> curve subdivision performed: dom(A) = ) domA CurveToString strcat
	(, dom(B) = ) strcat domB CurveToString strcat ( <<) strcat ==
    } if
    CurveA IsConstant CurveB IsConstant and {
	CurveA MiddlePoint ToVec
	CurveB MiddlePoint ToVec AreNear {
	    domA domB 4 -1 roll exch PutInterval PutInterval
	} {
	    pop pop
	} ifelse
    }{
	counter 100 lt {
%    \end{macrocode}
% Use a loop to simulate some kind of return to exit at different positions.
%    \begin{macrocode}
	    {
		/iter iter 1 add def
		iter 100 lt
		domA Extent precision ge
		domB Extent precision ge or and not {
		    iter 100 ge {
			false 
		    } {
			CurveA MiddlePoint ToVec
			CurveB MiddlePoint ToVec AreNear {
			    domA domB true
			}{
			    false
			} ifelse
		    } ifelse
		    exit
		} if
%    \end{macrocode}
% iter < 100 && (dompA.extent() >= precision || dompB.extent() >= precision)
%    \begin{macrocode}
		CheckIT {
		    (counter: ) counter 20 string cvs strcat
		    (, iter: ) iter 20 string cvs strcat strcat
		    (, depth: ) depth 20 string cvs strcat strcat ==
		} if
	
		CurveA CurveB ClipCurve /dom exch def
	
		CheckIT {(dom : ) dom CurveToString strcat == } if		
		dom IsEmptyInterval {
		    CheckIT { (empty interval, exit) == } if
		    false exit
		} if
%    \end{macrocode}
% dom[0] > dom[1], invalid.
%    \begin{macrocode}
		dom aload pop 2 copy min 3 1 roll max gt {
		    CheckIT {
			(dom[0] > dom[1], invalid!) ==
		    } if
		    false exit
		} if

		domB dom MapTo /domB exch def
		CurveB dom Portion

		CurveB IsConstant CurveA IsConstant and {
		    CheckIT {
          		(both curves are constant: ) ==	
			(C1: [ ) CurveA { CurveToString ( ) strcat strcat } forall (]) strcat ==
			(C2: [ ) CurveB { CurveToString ( ) strcat strcat } forall (]) strcat ==
		    } if
		    CurveA MiddlePoint ToVec
		    CurveB MiddlePoint ToVec AreNear {
			domA domB true
		    } {
			false
		    } ifelse
		    exit
		} if
%    \end{macrocode}
% If we have clipped less than 20%, we need to subdivide the
% curve with the largest domain into two sub-curves.
%    \begin{macrocode} 
		dom Extent MinClippedSizeThreshold gt {
		    CheckIT {
			(clipped less than 20% : ) ==
			(angle(A) = ) CurveA dup length 1 sub get aload pop
				      CurveA 0 get aload pop VecSub
   				      exch 2 copy 0 eq exch 0 eq and {
					  pop pop (NaN)
				      } {
					  atan 20 string cvs
				      } ifelse strcat ==
		        (angle(B) = ) CurveB dup length 1 sub get aload pop
		                      CurveB 0 get aload pop VecSub
				      exch 2 copy 0 eq exch 0 eq and {
					  pop pop (NaN)
				      } {
					  atan 20 string cvs
				      } ifelse strcat ==
		        (dom : ) == dom == (domB :) == domB ==
		    } if
%    \end{macrocode}
% Leave those five values on the stack to revert to the current state after the recursive calls.
%    \begin{macrocode}
		    CurveA CurveB domA domB iter
     		    7 -2 roll 2 copy 9 2 roll 2 copy 
%    \end{macrocode}
% On the stack: /domsA /domsB CurveA CurveB domA domB iter /domsA /domsB /domsA /domsB
%    \begin{macrocode}
		    domA Extent domB Extent gt {
			CurveA CloneCurve dup H1Interval Portion % pC1
			CurveA CloneCurve dup H2Interval Portion % pC2
			domA H1Interval MapTo                    % dompC1
			domA H2Interval MapTo                    % dompC2
%    \end{macrocode}
% Need on the stack: /domsA /domsB pC2 CurveB dompC2 domB   /domsA /domsB pC1 CurveB dompC1 domB
%    \begin{macrocode}
			3 -1 roll exch % /domsA /domsB /domsA /domsB pC1 dompC1 pC2 dompC2
			CurveB exch domB 8 4 roll % /domsA /domsB pC2 CurveB dompC2 domB /domsA /domsB pC1 dompC1
			CurveB exch domB % /domsA /domsB pC2 CurveB dompC2 domB /domsA /domsB pC1 CurveB dompC1 domB
		    } {
			CurveB CloneCurve dup H1Interval Portion % pC1
			CurveB CloneCurve dup H2Interval Portion % pC2
			domB H1Interval MapTo                    % dompC1
			domB H2Interval MapTo                    % dompC2
%    \end{macrocode}
% Need on the stack: /domsB /domsA pC2 CurveA dompC2 domA   /domsB /domsA pC1 CurveA dompC1 domA
%    \begin{macrocode}
			8 -2 roll exch 8 2 roll 6 -2 roll exch 6 2 roll % /domsB /domsA /domsB /domsA pC1 pC2 dompC1 dompC2
			3 -1 roll exch % /domsB /domsA /domsB /domsA pC1 dompC1 pC2 dompC2
			CurveA exch domA 8 4 roll % /domsB /domsA pC2 CurveA dompC2 domA /domsB /domsA pC1 dompC1
			CurveA exch domA          % /domsB /domsA pC2 CurveA dompC2 domA /domsB /domsA pC1 CurveA dompC1 domA
		    } ifelse

		    _IterateIntersection
		    _IterateIntersection
%    \end{macrocode}		    
% Restore the state before the recursive calls.
%    \begin{macrocode}
		    /iter exch def
		    /domB exch def
		    /domA exch def
		    /CurveB exch def
		    /CurveA exch def
		    false exit
		} if
		CurveA CurveB /CurveA exch def /CurveB exch def
		domA domB /domA exch def /domB exch def
%    \end{macrocode}
% exchange /domsA and /domsB on the stack!
%    \begin{macrocode}
		exch
	    } loop	
%    \end{macrocode}
% boolean on stack
%    \begin{macrocode}
	    {
		4 -1 roll exch PutInterval PutInterval
		CheckIT {
		    (found an intersection ============================) ==
		} if
	    } { pop pop } ifelse
	} {
	    pop pop
	} ifelse
    } ifelse
    /depth depth 1 sub def
    DebugEnd
} bind def
%    \end{macrocode}
% \end{macro}
% 
% \begin{macro}{PutInterval}
%   Add a new interval \PSarray{newinterval} to the array stored in
%   \PSname{/Intervals}. The new interval is "cloned" before storing it.
% \begin{pssyntax}
% \PSname{Intervals} \PSarray{newinterval} \PSop{PutInterval}
% \end{pssyntax}
%    \begin{macrocode}
/PutInterval {
    CloneVec [ exch 3 -1 roll dup 4 1 roll load aload pop ] def
} bind def
%    \end{macrocode}
% \end{macro}
% 
% \begin{macro}{IsEmptyInterval}
% Check if an interval is empty, which is represented by a [1 0] interval.
% \begin{pssyntx}
% \PSarray{interval} \PSop{IsEmptyInterval} \PSvar{boolean}
%    \begin{macrocode}
/IsEmptyInterval {
    aload pop 0 eq exch 1 eq and
} bind def
%    \end{macrocode}
% \end{macro}
%
% \begin{macro}{ToUnitInterval}
% Limit an interval \PSvar{a b} to the unit interval \PSarray{0 1}.
% \begin{pssyntax}
% \PSvar{a b} \PSop{ToUnitInterval} \PSarray{a|0 b|1}
% \end{pssyntax}
%    \begin{macrocode}
/ToUnitInterval {
    ToUnitRange exch ToUnitRange 2 copy gt {
	exch
    } if
    ToVec
} bind def
%    \end{macrocode}
% \end{macro}
% \begin{macro}{ToUnitRange}
% Limit a number to the range \PSarray{0 1}.
%    \begin{macrocode}
/ToUnitRange {
    dup 0 lt {
	pop 0
    }{
	dup 1 gt {
	    pop 1
	} if
    } ifelse
} bind def
%    \end{macrocode}
% \end{macro}
% 
% \begin{macro}{CloneCurve}
% Does a deep copy of the array \PSarray{Curve}. This also involved deep copies of the contained point arrays.
% \begin{pssyntax}
% \PSarray{Curve} \PSop{CloneCurve} \PSarray{newCurve}
%    \begin{macrocode}
/CloneCurve {
    [ exch {
	CloneVec
    } forall ]
} bind def
%    \end{macrocode}
% \end{macro}
%
% \begin{macro}{CloneVec}
% Does a deep copy of the vector \PSarray{X Y}
% \begin{pssyntax}
% \PSarray{X Y} \PSop{CloneVec} \PSarray{Xnew Ynew}
% \end{pssyntax}
%    \begin{macrocode}
/CloneVec {
    aload pop ToVec
} bind def
%    \end{macrocode}
% \end{macro}
% 
% \begin{macro}{MapTo}
% Map the sub-interval \PSarray{I} in \PSarray{0 1} into the interval \PSarray{J}. Returns a new array.
% \begin{pssyntax}
% \PSarray{J} \PSarray{I} \PSop{MapTo} \PSarray{Jnew}
% \end{pssyntax}
%    \begin{macrocode}
/MapTo {
    (MapTo) DebugBegin
    exch aload 0 get 3 1 roll exch sub 2 copy % [I] J0 Jextent J0 Jextent
    5 -1 roll aload aload pop % J0 Jextent J0 Jextent I0 I1 I0 I1
    min 4 -1 roll mul % J0 Jextent J0 I0 I1 min(I0,I1)*Jextent
    4 -1 roll add [ exch % J0 Jextent I0 I1 [ J0new
    6 2 roll max mul add ]
    DebugEnd
} bind def
%    \end{macrocode}
% \end{macro}
% 
% \begin{macro}{Portion}
% Compute the portion of the Bezier curve \PSarray{CurveB} wrt the interval \PSarray{I}.
% \begin{pssyntax}
% \PSarray{CurveB} \PSarray{I} \PSop{Portion} \PSarray{CurvePartB}
% \end{pssyntax}
%    \begin{macrocode}
/Portion {
    (Portion) DebugBegin
    dup Min 0 eq { % [CurveB] [I]
	% I.min() == 0
	Max dup 1 eq {% [CurveB] I.max()
	    % I.max() == 1
	    pop pop	    
	} { % [CurveB] I.max()
	    LeftPortion
	} ifelse
    } { % [CurveB] [I]
	2 copy Min % [CurveB] [I] [CurveB] I.min()
	RightPortion
	dup Max 1 eq {
	    % I.max() == 1
	    pop pop
	} {% [CurveB] [I]
	    dup aload pop exch sub 1 3 -1 roll Min sub div % [CurveB] (I1-I0)/(1-I.min())
	    LeftPortion
	} ifelse
    } ifelse
    DebugEnd
} bind def
%    \end{macrocode}
% \end{macro}
% 
% \begin{macro}{LeftPortion}
%   Compute the portion of the Bezier curve \PSarray{CurveB} wrt the
%   interval \PSarray{0 t}.
% \begin{pssyntax}
% \PSarray{CurveB} \PSvar{t} \PSop{LeftPortion} \PSarray{CurvePartB}
% \end{pssyntax}
%    \begin{macrocode}
/LeftPortion {
    (LeftPortion) DebugBegin
    exch dup length 1 sub dup 4 1 roll % L-1 t [CurveB] L-1
    1 1 3 -1 roll { % L-1 t [CurveB] i
	4 -1 roll dup 5 1 roll % L-1 t [CurveB] i L-1
	-1 3 -1 roll % L-1 t [CurveB] L-1 -1 i
	{ % L-1 t [CurveB] j
	    2 copy 5 copy % L-1 t [CurveB] j [CurveB] j t [CurveB] j [CurveB] j 
	    1 sub get 3 1 roll get % L-1 t [CurveB] j [CurveB] j t B[j-1] B[j]
	    Lerp put pop % L-1 t [CurveB]
	} for
    } for
    pop pop pop
    DebugEnd
} bind def
%    \end{macrocode}
% \end{macro}
%
% \begin{macro}{RightPortion}
%   Compute the portion of the Bezier curve \PSarray{CurveB} wrt the
%   interval \PSarray{t 1}.
% \begin{pssyntax}
% \PSarray{CurveB} \PSvar{t} \PSop{RightPortion} \PSarray{CurvePartB}
% \end{pssyntax}
%    \begin{macrocode}
/RightPortion {
    (RightPortion) DebugBegin
    exch dup length 1 sub dup 4 1 roll % L-1 t [CurveB] L-1
    1 1 3 -1 roll {% L-1 t [CurveB] i
	4 -1 roll dup 5 1 roll % L-1 t [CurveB] i L-1
	exch sub 0 1 3 -1 roll  % L-1 t [CurveB] 0 1 L-i-1
	{% L-1 t [CurveB] j
	    2 copy 5 copy
	    get 3 1 roll 1 add get Lerp put pop
	} for
    } for
    pop pop pop
    DebugEnd
} bind def
%    \end{macrocode}
% \end{macro}
% 
% \begin{macro}{Lerp}
% Given two points and a parameter \PSvar{t} $\in$ \PSarray{0 1}, return a point
% proportionally from \PSarray{A} to \PSarray{B} by \PSvar{t}. Akin to 1 degree Bezier.
% \begin{pssyntax}
% \PSvar{t} \PSarray{A} \PSarray{B} \PSop{Lerp} \PSarray{newpoint}
% \end{pssyntax}
%    \begin{macrocode}
/Lerp {
    (Lerp) DebugBegin
    3 -1 roll dup 1 exch sub 3 1 roll % [A] (1-t) [B] t
    exch aload pop 3 -1 roll VecScale % [A] (1-t) B.x*t B.y*t
    4 2 roll
    exch aload pop 3 -1 roll VecScale VecAdd ToVec % [A.x*(1-t)+B.x*t A.y*(1-t)+B.y*t]
    DebugEnd
} bind def
%    \end{macrocode}
% \end{macro}
% 
% \begin{macro}{IsConstant}
% Test if all points of a curve are near to each other. This is used as termination criterium for the intersection procedure.
% \begin{pssyntax}
% \PSarray{Curve} \PSop{IsConstant} \PSvar{boolean}
% \end{pssyntax}
%    \begin{macrocode}
/IsConstant {
    aload length [ exch 1 roll ] true 3 1 roll
    {
	exch dup 4 1 roll
	AreNear and exch
    } forall
    pop
} bind def
%    \end{macrocode}
% \end{macro}
% \begin{macro}{AreNear}
% Test if two points are near to each other.
% \begin{pssyntax}
% \PSarray{P1} \PSarray{P2} \PSop{AreNear} \PSvar{boolean}
% \end{pssyntax}
%    \begin{macrocode}
/AreNear {
    (AreNear) DebugBegin
    aload pop 3 -1 roll aload pop
    VecSub abs epsilon lt exch abs epsilon lt and
    DebugEnd
} bind def
%    \end{macrocode}
% \end{macro}
% 
% \begin{macro}{Min}
% Get the minimum value of the vector \PSarray{P}.
% \begin{pssyntax}
% \PSarray{P} \PSop{Min} \PSvar{minimum}
% \end{pssyntax}
%    \begin{macrocode}
/Min {
    aload pop min
} bind def
%    \end{macrocode}
% \end{macro}
% \begin{macro}{Min}
% Get the maximum value of the vector \PSarray{P}.
% \begin{pssyntax}
% \PSarray{P} \PSop{Max} \PSvar{maximum}
% \end{pssyntax}
%    \begin{macrocode}
/Max {
    aload pop max
} bind def
%    \end{macrocode}
% \end{macro}
% \begin{macro}{Min}
% Get the extent of the interval \PSarray{I}.
% \begin{pssyntax}
% \PSarray{I} \PSop{Extent} \PSvar{I1-I0}
% \end{pssyntax}
%    \begin{macrocode}
/Extent {
    aload pop exch sub
} bind def
%    \end{macrocode}
% \end{macro}
%
% \begin{macro}{MiddlePoint}
% Compute the middle point of the first and last point of \PSarray{Curve}.
% \begin{pssyntax}
% \PSarray{Curve} \PSop{MiddlePoint} \PSvar{X Y}
% \end{pssyntax}
%    \begin{macrocode}
/MiddlePoint {
    dup dup length 1 sub get aload pop
    3 -1 roll 0 get aload pop
    VecAdd 0.5 VecScale
} bind def
%    \end{macrocode}
% \end{macro}
% 
% \begin{macro}{OrthogonalOrientationLine}
% \begin{pssyntax}
% \PSvar{MiddlePointA} \PSarray{CurveB} \PSop{OrthogonalOrientationLine} \PSvar{A B C}
% \end{pssyntax}
%    \begin{macrocode}
/OrthogonalOrientationLine {
    (OrthogonalOrientationLine) DebugBegin
    dup dup length 1 sub get aload pop 3 -1 roll 0 get aload pop VecSub
%    \end{macrocode}
% rotate by +90 degrees
%    \begin{macrocode}
    neg exch
    4 2 roll 2 copy 6 2 roll VecAdd
    ImplicitLine
    DebugEnd
} bind def
%    \end{macrocode}
% \end{macro}
% 
% \begin{macro}{PickOrientationLine}
%   Pick an orientation line for a Bezier curve. This uses the first
%   point and the lastmost point, which is not near to it.
% \begin{pssyntax}
% \PSarray{Curve} \PSop{PickOrientationLine} \PSvar{A B C}
% \end{pssyntax}
%    \begin{macrocode}
/PickOrientationLine {
    (PickOrientationLine) DebugBegin
    dup dup length 1 sub exch 0 get% [Curve] L-1 P0
    exch -1 1 {% [Curve] P0 i
	3 -1 roll dup 4 1 roll exch get % [Curve] P0 Pi
	2 copy AreNear {
	    pop
	} {
	    exit
	} ifelse
    } for
    3 -1 roll pop
    exch aload pop 3 -1 roll aload pop ImplicitLine
    DebugEnd
} bind def
%    \end{macrocode}
% \end{macro}
% 
% \begin{macro}{ImplicitLine}
% Compute the coefficients \PSvar{A}, \PSvar{B}, \PSvar{C} of the normalized implicit equation
% of the line which goes through the points \PSarray{Xi Yi} and \PSarray{Xj Yj}.
%
% \begin{pssyntax}
% \PSvar{Xi Yi Xj Yj} \PSop{ImplicitLine} \PSvar{A B C}
% \end{pssyntax}
%    \begin{macrocode}
/ImplicitLine {
    4 copy % Xi Yi Xj Yj Xi Yi Xj Yj
    3 -1 roll sub 7 1 roll sub 5 1 roll % Yj-Yi Xi-Xj Xi Yi Xj Yj
    % Yi*Xj - Xi*Yj
    4 -1 roll mul neg % Yj-Yi Xi-Xj Yi Xj -Yj*Xi
    3 1 roll mul add % Yj-Yi Xi-Xj Yi*Xj-Yj*Xi | l0 l1 l2
    3 1 roll 2 copy tx@Dict begin Pyth end dup dup % l2 l0 l1 L L L
    5 -1 roll exch % l2 l1 L L l0 L
    div 5 1 roll % l0/L l2 l1 L L
    3 1 roll div % l0/L l2 L l1/L
    3 1 roll div % l0/L l1/L l2/L
} bind def
%    \end{macrocode}
% \end{macro}
% 
% \begin{macro}{distance}
% Compute the distance of point \PSarray{X Y} from the implicit line given
% by $Ax + By + C = 0,\quad (A^2+B^2 = 1)$.
% \begin{pssyntax}
% \PSvar{X Y A B C} \PSop{distance} \PSvar{d}
% \end{pssyntax}
%    \begin{macrocode}
/distance {
    5 1 roll 3 -1 roll mul 3 1 roll mul add add
} bind def
%    \end{macrocode}
% \end{macro}
%
% \begin{macro}{ArrayToPointArray}
% \begin{pssyntax}
% \PSarray{A.x A.y ... N.x N.y} \PSop{ArrayToPointArray} \PSarray{\PSarray{A.x A.y} \ldots \PSarray{N.x N.y}}
% \end{pssyntax}
%    \begin{macrocode}
/ArrayToPointArray {
    aload length dup 2 idiv {
	3 1 roll [ 3 1 roll ] exch
	dup 1 sub 3 1 roll 1 roll
    } repeat 1 add [ exch 1 roll ]
} bind def
%    \end{macrocode}
% \end{macro}
%
% \begin{macro}{PointArrayToArray}
% \begin{pssyntax}
% \PSarray{\PSarray{A.x A.y} \ldots \PSarray{N.x N.y}} \PSop{PointArrayToArray} \PSarray{A.x A.y ... N.x N.y}
% \end{pssyntax}
%    \begin{macrocode}
/PointArrayToArray {
    aload length dup {
	1 add dup 3 -1 roll aload pop 4 -1 roll 1 add 2 roll
    } repeat 1 add [ exch 1 roll ]
} bind def
%    \end{macrocode}
% \end{macro}
% 
% \begin{macro}{ClipCurve}
% Clip the Bezier curve B with respect to the Bezier curve A for
% individuating intersection points. The new parameter interval for the
% clipped curve is pushed on the stack.
% \begin{pssyntax}
% \PSarray{CurveA} \PSarray{CurveB} \PSop{ClipCurve} \PSarray{newinterval}
% \end{pssyntax}
%    \begin{macrocode}
/ClipCurve {
    (ClipCurve) DebugBegin
    4 dict begin 
    /CurveB exch def /CurveA exch def
    CurveA IsConstant {
    	CurveA MiddlePoint CurveB OrthogonalOrientationLine
    } {
	CurveA PickOrientationLine
    } ifelse
    CheckIT {
	3 copy exch 3 -1 roll (OrientationLine : )
	3 { exch 20 string cvs ( ) strcat strcat } repeat ==
    } if
    CurveA FatLineBounds
    CheckIT { dup (FatLineBounds : ) exch aload pop exch 20 string cvs (, ) strcat exch 20 string cvs strcat strcat == } if
    CurveB ClipCurveInterval
    end
    DebugEnd
} bind def
%    \end{macrocode}
% \end{macro}
% 
% \begin{macro}{FatLineBounds}
% Compute the boundary of the fat line given by \PSvar{A B C}
% \begin{pssyntax}
% \PSvar{A B C} \PSarray{Curve} \PSop{FatLineBounds} \PSvar{A B C} \PSarray{dmin dmax}
% \end{pssyntax}
%    \begin{macrocode}
/FatLineBounds {
    (FatLineBounds) DebugBegin
    /dmin 0 def /dmax 0 def
    { 
	4 copy aload pop 5 2 roll distance
	dup dmin lt { dup /dmin exch def } if
	dup dmax gt { dup /dmax exch def } if
	pop pop
    } forall
    [dmin dmax]
    DebugEnd
} bind def
%    \end{macrocode}
% \end{macro}
% 
% \begin{macro}{ClipCurveInterval}
%   Clip the Bezier curve wrt the fat line defined by the orientation
%   line (given by \PSvar{A B C}) and the interval range
%   \PSarray{bound}. The new parameter interval \PSarray{newinterval}
%   for the clipped curve is pushed on the stack.
% \begin{pssyntax}
% \PSvar{A B C} \PSarray{bound} \PSarray{curve} \PSop{ClipCurveInterval} \PSarray{newinterval}
% \end{pssyntax}
%    \begin{macrocode}
/ClipCurveInterval {
    (ClipCurveInterval) DebugBegin
    15 dict begin
    /curve exch def
    aload pop 2 copy min /boundMin exch def max /boundMax exch def
    [ 4 1 roll ] cvx /fatline exch def
    % number of sub-intervals
    /n curve length 1 sub def
    % distance curve control points
    /D n 1 add array def
    0 1 n { % i
	dup curve exch get aload pop % i Pi.x Pi.y
	fatline distance % distance d of Point i from the orientation line, on stack; i d
	exch dup n div % d i i/n
	[ exch 4 -1 roll ] % i [ i/n d ]
	D 3 1 roll put 
    } for
    D ConvexHull /D exch def
%    \end{macrocode}
% get the x-coordinate of the i-th point, i getX -> D[i][X]
%    \begin{macrocode}
    /getX { D exch get 0 get } def
%    \end{macrocode}
% get the y-coordinate of the i-th point, i getY -> D[i][Y]
%    \begin{macrocode}
    /getY { D exch get 1 get } def
    /tmin 1 def /tmax 0 def
    0 getY dup
    boundMin lt /plower exch def
    boundMax gt /phigher exch def
    plower phigher or not {
%    \end{macrocode}
% inside the fat line
%    \begin{macrocode}
	tmin 0 getX gt { /tmin 0 getX def } if
	tmax 0 getX lt { /tmax 0 getX def } if	
    } if
    1 1 D length 1 sub {
	/i exch def
	/clower i getY boundMin lt def
	/chigher i getY boundMax gt def
	clower chigher or not {
%    \end{macrocode}
% inside the fat line
%    \begin{macrocode}
	    tmin i getX gt { /tmin i getX def } if
	    tmax i getX lt { /tmax i getX def } if
	} if
	clower plower eq not {
%    \end{macrocode}
% cross the lower bound
%    \begin{macrocode}
	    boundMin i 1 sub i D Intersect % t on stack
	    dup tmin lt { dup /tmin exch def } if
	    dup tmax gt { dup /tmax exch def } if
	    pop 
	    /plower clower def
	} if
	chigher phigher eq not {
%    \end{macrocode}
% cross the upper bound
%    \begin{macrocode}
	    boundMax i 1 sub i D Intersect
	    dup tmin lt { dup /tmin exch def } if
	    dup tmax gt { dup /tmax exch def } if
	    pop 
	    /phigher chigher def
	} if
    } for
%    \end{macrocode}
% we have to test the closing segment for intersection
%    \begin{macrocode}
    /i D length 1 sub def
    /clower 0 getY boundMin lt def
    /chigher 0 getY boundMax gt def
    clower plower eq not {
%    \end{macrocode}
% cross the lower bound
%    \begin{macrocode}
	boundMin i 0 D Intersect
	dup tmin lt { dup /tmin exch def } if
	dup tmax gt { dup /tmax exch def } if
	pop
    } if
    chigher phigher eq not {
%    \end{macrocode}
% cross the lower bound
%    \begin{macrocode}
	boundMax i 0 D Intersect
	dup tmin lt { dup /tmin exch def } if
	dup tmax gt { dup /tmax exch def } if
	pop
    } if
    [tmin tmax]
    end
    DebugEnd
} bind def
%    \end{macrocode}
% \end{macro}
% 
% \begin{macro}{Intersect}
%   Get the x component of the intersection point between the line
%   passing through $i$-th and $j$-th points of \PSarray{Curve} and the
%   horizonal line through \PSvar{y}.
% \begin{pssyntax}
% \PSvar{y i j} \PSarray{Curve} \PSop{Intersect} \PSvar{Xisect}
% \end{pssyntax}
%    \begin{macrocode}
/Intersect {
    dup 4 -1 roll get aload pop
    4 2 roll exch get aload pop
%    \end{macrocode}
% On the stack: \PSvar{y Xi Yi Xj Yj}, Compute (Xj - Xi) * (y - Yi)/(Yj - Yi) + Xi
%
% We are sure, that Yi != Yj, because this procedure is called only
% when the lower or upper bound is crossed.
%    \begin{macrocode}
    4 2 roll 2 copy 6 2 roll VecSub
    5 2 roll
    neg 3 -1 roll add
    3 -1 roll div
    3 -1 roll mul add
} bind def
%    \end{macrocode}
% \end{macro}
%
% \begin{macro}{IsPath}
%   Check if an array is a path. A path is represented as array, which
%   contains other arrays which represent native Postscript
%   operations. Those can be \PSarray{X Y /@m}, \PSarray{X Y /@l}, or
%   \PSarray{X1 Y1 X2 Y2 X3 Y3 /@c}.
%
%   \begin{pssyntax}
%     \PSarray{array} \PSop{IsPath} \PSvar{boolean}
%   \end{pssyntax}
%    \begin{macrocode}
/IsPath {
  dup length 1 sub get type /nametype eq { true } { false } ifelse
} bind def
%    \end{macrocode}
% \end{macro}
%
% \begin{macro}{ShowPathPortion}
%   \begin{pssyntax}
%     \PSarray{path} \PSvar{tstart tstop} \PSop{ShowPathPortion}
%   \end{pssyntax}
%    \begin{macrocode}
/ShowPathPortion {
  6 dict begin
  /tstop exch def
  /tstart exch def
  InitTracing
  /n 0 def
  mark exch aload pop
  {
    counttomark 0 eq n tstop ge or { pop exit } if
    dup /movetype eq not { /n n 1 add def } if
    
    dup /movetype eq {
      load exec
    } {
      tstart n gt {
%    \end{macrocode}      
% current path section is before tstop
%    \begin{macrocode}
        /curvetype eq { 6 2 roll 4 { pop } repeat } if
        movetype
      } {
        tstart n 1 sub gt tstop n lt or {
%    \end{macrocode}
% draw a truncated segment
%    \begin{macrocode}
          tstart n sub 1 add tstop n sub 1 add
          ToUnitInterval exch
          /linetype eq {
            3 1 roll ToVec currentpoint ToVec exch ToVec
            dup 3 -1 roll Portion
            aload pop exch 
            tstart n 1 sub gt { 
%    \end{macrocode}
% This is the start segment, move to the starting point, as it lies in the middle of the segment.
%    \begin{macrocode}
              exch aload pop 3 -1 roll aload pop ArrowA 
              tstop n le { 
%    \end{macrocode}
% only a single segment, draw also the ending arrow 
%    \begin{macrocode}
                currentpoint 4 2 roll ArrowB lineto pop pop
              } {
%    \end{macrocode}
% other segments to follow
%    \begin{macrocode}
                lineto
              } ifelse
            } { 
%    \end{macrocode}
% this is the last segment
%    \begin{macrocode}
              pop aload pop currentpoint 4 2 roll ArrowB lineto pop pop
            } ifelse
          } {
            7 1 roll [ currentpoint 9 3 roll ] ArrayToPointArray
            dup 3 -1 roll Portion 
            { aload pop } forall
            8 -2 roll
            tstart n 1 sub gt { moveto } { pop pop } ifelse
            6 -2 roll currentpoint ArrowA 6 2 roll
            tstop n le {
%    \end{macrocode}              
% only a single segment
%    \begin{macrocode}
              ArrowB
            } if
            curveto
          } ifelse
        }{
%    \end{macrocode}
% full segment
%    \begin{macrocode}
          tstart n 1 sub eq {
%    \end{macrocode}
% the first segment
%    \begin{macrocode}
            /linetype eq {
              currentpoint ArrowA
              tstop n eq {
%    \end{macrocode}
% a single, full segment
%    \begin{macrocode}
                currentpoint 4 2 roll ArrowB lineto pop pop
              } {
%    \end{macrocode}
% not the last segment
%    \begin{macrocode}
                lineto
              } ifelse
            } {
              6 -2 roll currentpoint ArrowA 6 2 roll
              tstop n eq {
%    \end{macrocode}
% single, full curve segment
%   \begin{macrocode}
                ArrowB
              } if
              curveto
            } ifelse
          } {
%    \end{macrocode}
% not the first segment
%    \begin{macrocode}
            /linetype eq {
              tstop n eq {
%    \end{macrocode}
% last segment but not a single one
%    \begin{macrocode}
                currentpoint 4 2 roll ArrowB lineto pop pop
              }{
%    \end{macrocode}
% full middle segment
%    \begin{macrocode}
                lineto
              } ifelse
            } {
              tstop n eq {
%    \end{macrocode}
% last curveto segment, not a single one
%    \begin{macrocode}
                ArrowB
              } if
              curveto
            } ifelse
          } ifelse
        } ifelse
      } ifelse
    } ifelse
  } loop
  end
} bind def
%    \end{macrocode}
% \end{macro}
%
%    \begin{macrocode}
 % Graham Scal algorithm to compute the convex hull of a set of
 % points. Code written by Bill Casselman,
 %  http://www.math.ubc.ca/~cass/graphics/text/www/
 %
 % [[X1 Y1] [X2 Y2] ... [Xn Yn]] hull -> [[...] ... [...]]
 %
/hulldict 32 dict def
hulldict begin

 % u - v 
/vsub { 2 dict begin
/v exch def
/u exch def
[ 
  u 0 get v 0 get sub
  u 1 get v 1 get sub
]
end } def

 % u - v rotated 90 degrees
/vperp { 2 dict begin
/v exch def
/u exch def
[ 
  v 1 get u 1 get sub
  u 0 get v 0 get sub
]
end } def

/dot { 2 dict begin
/v exch def
/u exch def
  v 0 get u 0 get mul
  v 1 get u 1 get mul
  add
end } def 

 % P Q
 % tests whether P < Q in lexicographic order
 % i.e xP < xQ, or yP < yQ if xP = yP
/comp { 2 dict begin
/Q exch def
/P exch def
P 0 get Q 0 get lt 
  P 0 get Q 0 get eq
  P 1 get Q 1 get lt 
  and 
or 
end } def

end

 % args: an array of points C
 % effect: returns the array of points on the boundary of
 %     the convex hull of C, in clockwise order 

/ConvexHull {
(ConvexHull) DebugBegin
hulldict begin
/C exch def
/comp C quicksort
/n C length def
 % Q might circle around to the start
/Q n 1 add array def
Q 0 C 0 get put
Q 1 C 1 get put
/i 2 def
/k 2 def
 % i is next point in C to be looked at
 % k is next point in Q to be added
 % [ Q[0] Q[1] ... ]
 % scan the points to make the top hull
n 2 sub {
  % P is the current point at right
  /P C i get def
  /i i 1 add def
  {
    % if k = 1 then just add P 
    k 2 lt { exit } if
    % now k is 2 or more
    % look at Q[k-2] Q[k-1] P: a left turn (or in a line)?
    % yes if (P - Q[k-1])*(Q[k-1] - Q[k-2])^perp >= 0
    P Q k 1 sub get vsub 
    Q k 1 sub get Q k 2 sub get vperp 
    dot 0 lt {
      % not a left turn
      exit
    } if
    /k k 1 sub def
  } loop
  Q k P put
  /k k 1 add def
} repeat

 % done with top half
 % K is where the right hand point is
/K k 1 sub def

/i n 2 sub def
Q k C i get put
/i i 1 sub def
/k k 1 add def
n 2 sub {
  % P is the current point at right
  /P C i get def
  /i i 1 sub def
  {
    % in this pass k is always 2 or more
    k K 2 add lt { exit } if
    % look at Q[k-2] Q[k-1] P: a left turn (or in a line)?
    % yes if (P - Q[k-1])*(Q[k-1] - Q[k-2])^perp >= 0
    P Q k 1 sub get vsub 
    Q k 1 sub get Q k 2 sub get vperp 
    dot 0 lt {
      % not a left turn
      exit
    } if
    /k k 1 sub def
  } loop
  Q k P put
  /k k 1 add def
} repeat

 % strip Q down to [ Q[0] Q[1] ... Q[k-2] ]
 % excluding the doubled initial point
[ 0 1 k 2 sub {
  Q exch get
} for ] 
end
DebugEnd
} def

/qsortdict 8 dict def

qsortdict begin

 % args: /comp a L R x
 % effect: effects a partition into two pieces [L j] [i R]
 %     leaves i j on stack

/partition { 8 dict begin
/x exch def
/j exch def
/i exch def
/a exch def
dup type /nametype eq { load } if /comp exch def
{
  {
    a i get x comp exec not {
      exit
    } if
    /i i 1 add def
  } loop
  {
    x a j get comp exec not {
      exit
    } if
    /j j 1 sub def
  } loop
  
  i j le {
    % swap a[i] a[j]
    a j a i get
    a i a j get 
    put put
    /i i 1 add def
    /j j 1 sub def
  } if
  i j gt {
    exit
  } if
} loop
i j
end } def

 % args: /comp a L R
 % effect: sorts a[L .. R] according to comp
/subsort {
 % /c a L R
[ 3 1 roll ] 3 copy
 % /c a [L R] /c a [L R]
aload aload pop 
 % /c a [L R] /c a L R L R
add 2 idiv
 % /c a [L R] /c a L R (L+R)/2
3 index exch get
 % /c a [L R] /c a L R x
partition
 % /c a [L R] i j
 % if j > L subsort(a, L, j)
dup 
 % /c a [L R] i j j
3 index 0 get gt {
  % /c a [L R] i j
  5 copy 
  % /c a [L R] i j /c a [L R] i j
  exch pop
  % /c a [L R] i j /c a [L R] j
  exch 0 get exch
  % ... /c a L j 
  subsort
} if
 % /c a [L R] i j
pop dup
 % /c a [L R] i i
 % if i < R subsort(a, i, R)
2 index 1 get lt {
  % /c a [L R] i
  exch 1 get 
  % /c a i R
  subsort
}{
  4 { pop } repeat
} ifelse
} def

end % qsortdict

 % args: /comp a
 % effect: sorts the array a 
 % comp returns truth of x < y for entries in a

/quicksort { qsortdict begin
dup length 1 gt {
 % /comp a
dup 
 % /comp a a 
length 1 sub 
 % /comp a n-1
0 exch subsort
} {
pop pop
} ifelse
end } def
%    \end{macrocode}
% 
% Debugging stuff
%    \begin{macrocode}
/debug {
    dup 1 add copy {==} repeat pop
} bind def
/DebugIT false def
/CheckIT false def
/DebugDepth 0 def
/DebugBegin {
  DebugIT {
    /DebugProcName exch def
    DebugDepth 2 mul string
    0 1 DebugDepth 2 mul 1 sub {
      dup 2 mod 0 eq { (|) }{( )} ifelse
      3 -1 roll dup 4 2 roll
      putinterval
    } for
    DebugProcName strcat ==
    /DebugDepth DebugDepth 1 add def
  }{
    pop
  } ifelse
} bind def
/DebugEnd {
  DebugIT {
    /DebugDepth DebugDepth 1 sub def
    DebugDepth 2 mul 2 add string
    0 1 DebugDepth 2 mul 1 sub {
      dup 2 mod 0 eq { (|) }{ ( ) } ifelse
      3 -1 roll dup 4 2 roll
      putinterval
    } for
    dup DebugDepth 2 mul (+-) putinterval
    ( done) strcat ==
  } if
} bind def
/strcat {
    exch 2 copy
    length exch length add
    string dup dup 5 2 roll
    copy length exch
    putinterval
} bind def
/nametostr {
    dup length string cvs
} bind def
/ShowCurve {
    { aload pop } forall
    8 -2 roll moveto curveto
} bind def
/CurveToString {
    (CurveToString) DebugBegin
    aload pop ([) 3 -1 roll 20 string cvs strcat (, ) strcat exch 20 string cvs strcat (]) strcat
    DebugEnd
} bind def
end % tx@IntersectDict
%    \end{macrocode}
%</prolog> 
% \Finale
% \endinput
